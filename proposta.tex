% proposal.tex
% Based on http://www.latextemplates.com/template/simple-sectioned-essay
\documentclass[12pt]{article}

\usepackage[utf8]{inputenc}
\usepackage[T1]{fontenc}
\usepackage[a4paper, margin=2cm]{geometry}
\usepackage[brazil]{babel}

\usepackage{graphicx} % resizebox
\usepackage{multirow}
\usepackage{tabularx}

\linespread{1.2} % Line spacing

\newcommand{\Title}[1]{\textbf{\MakeUppercase{#1}}}

\begin{document}

\begin{titlepage}
    \center

    \Title{Universidade Federal de Santa Catarina - UFSC}

    \Title{Ciência Da Computação}

    \vspace*{\stretch{1}}

    Tiago Royer

    \vspace*{\stretch{2}}

    \Title{Complexidade de Circuitos}\\[3cm]

    \begin{flushright}
        \begin{minipage}{0.5\textwidth}
            Proposta de Trabalho de Conclusão de Curso,
            a ser submetido ao Curso de Ciências da Computação
            para a obtenção do Grau de
            Bacharel em Ciências da Computação.

            Orientadora: Profa. Dra. Jerusa Marchi
        \end{minipage}
    \end{flushright}

    \vspace*{\stretch{3}}

    Florianópolis, \today.

    Entrega do TCC: 2º semestre de 2015.
\end{titlepage}


\begin{abstract}
    Complexidade de Circuitos é uma subárea da Teoria da Computação
    que estuda a taxa de crescimento de circuitos booleanos.
    Em particular, um dos objetivos desta área é demonstrar que certas
    funções booleanas exigem circuitos que crescem exponencialmente
    conforme cresce o tamanho das palavras --- e, como consequência
    disso, concluir que $P \neq NP$ \cite{Sipser2006}.

    \textbf{Palavras-chave:} Complexidade de Circuitos, Complexidade Computacional
\end{abstract}

\newpage
\tableofcontents
\newpage

\section{Introdução}

    O mais famoso problema da área de complexidade computacional é o $P$ versus $NP$.
    Os problemas da classe $P$ são os que podem ser resolvidos em tempo polinomial
    por uma Máquina de Turing determinística, enquanto que a classe $NP$ contempla
    os problemas que podem ser resolvidos em tempo polinomial por Máquinas
    não-determinísticas.
    Os problemas considerados são problemas de decisão; isto é, determinar se uma
    palavra está ou não numa determinada linguagem. $P$ e $NP$ são conjuntos de
    linguagens que correspondem a problemas de decisão. Sabemos que $P \subseteq NP$;
    o problema $P$ versus $NP$ pergunta se esta inclusão é estrita ou não \cite{Sipser2006}.
    A maior parte dos pesquisadores acredita que $P \neq NP$,
    embora ainda não tenhamos demonstrado nada.

    Uma das ``frentes de ataque'' consiste em tentar ``enfraquecer'' a máquina de Turing.
    A ideia é trabalhar com um modelo de computação muito mais restritivo,
    provar um equivalente a $P \neq NP$ para este modelo de computação,
    e transpor esta demonstração para o modelo de Turing.
    Circuitos booleanos são exemplos de modelos de computação mais simples.
    Como eles não têm ``partes móveis'' ou múltiplos estados,
    intuitivamente uma demonstração de que $P \neq NP$ deve ser
    mais simples de se obter neste dispositivo computacional. \cite{Hastad1987}.

    As duas principais medidas de complexidade de circuitos são a quantidade de
    nodos (quantidade de portas lógicas) e a profundidade do circuito
    (maior caminho entre uma entrada e uma saída).
    Existem várias classes de complexidades de circuitos
    que exploram estas duas métricas;
    neste trabalho, pretende-se relacionar estas classes de complexidade
    com as classes de complexidade computacional.

\section{Objetivos}

    Estabelecer relações entre classes de complexidade computacional
    e complexidade de circuitos.

\subsection{Objetivos específicos}

    \begin{enumerate}
        \item Estudar as classes de complexidade computacional
            e complexidade de circuitos.
        \item Demonstrar a relação entre complexidade de circuitos
            e o problema $P$ versus $NP$.
        \item Correlacionar as principais classes de complexidade computacional
            às correspondentes classes de complexidade de circuitos
        \item Estabelecer limites inferiores no tamanho e profundidade dos circuitos
            que computam certas funções recursivas.
    \end{enumerate}

\section{Cronograma}

    \begin{tabularx}{\linewidth}{|X|*{10}{c|}}
        \hline
        \multicolumn{1}{|c|}{\multirow{2}{*}{Etapas}} & \multicolumn{10}{|c|}{Meses}\\ \cline{2-11}
        & jan & fev & mar & abr & mai & jun & jul & ago & set & out \\ \hline

        Estudar classes de complexidade computacional
        &  x  &  x  &  x  &     &     &     &     &     &     &     \\ \hline

        Estudar classes de complexidade de circuitos
        &     &  x  &  x  &  x  &     &     &     &     &     &     \\ \hline

        Estabelecer relações de equivalência entre as classes de complexidade
        &     &  x  &  x  &  x  &  x  &  x  &     &     &     &     \\ \hline

        Rever demonstrações de limites inferiores para certas funções recursivas
        &     &     &     &     &  x  &  x  &  x  &  x  &     &     \\ \hline

        Explorar a relação com o problema $P$ versus $NP$
        &     &  x  &     &     &     &     &     &  x  &     &     \\ \hline

        Documentar o aprendido
        &  x  &  x  &  x  &  x  &  x  &  x  &  x  &  x  &  x  &  x  \\ \hline

    \end{tabularx}

\section{Método de pesquisa}

    Este trabalho possui forte viés teórico e está enraizado na Matemática.
    Será feita uma pesquisa explicativa centrará em estudos e discussões sobre o tema,
    além da escrita do trabalho final.
    O trabalho estará baseado em pesquisas anteriores,
    disponíveis em livros, artigos, teses e dissertações.

\section{Custos}
    Por ser um trabalho de cunho teórico, não foram identificados custos para o trabalho.

\section{Recursos Humanos}
    \begin{tabular}{l l}
        \hline
        Nome            & Função \\
        \hline
        Tiago Royer     & Aluno \\
        Jerusa Marchi   & Orientadora \\
        Renato Cislaghi & Professor de Projetos \\
        \hline
    \end{tabular}
    \\
    \\
    A banca ainda não foi definida.

\section{Comunicação}
    \begin{tabular}{l l l l}
        \hline
        O quê                   & De quem       & Para Quem         & Como \\
        \hline
        Proposta de TCC         & Tiago Royer   & Renato Cislaghi   & Site de projetos \\
        Relatório de TCC I      & Tiago Royer   & Renato Cislaghi   & Site de projetos \\
        Prévia do TCC, em TCC I & Tiago Royer   & Banca             & Email \\
        Defesa do TCC           & Tiago Royer   & Banca             & Pessoalmente \\
        Reunião de Orientação   & Jerusa Marchi & Tiago Royer       & Pessoalmente \\
        \hline
    \end{tabular}

\section{Riscos}
    Como este trabalho possui um viés teórico, não foram identificados
    riscos específicos a ele.

\bibliographystyle{abnt-num}
\bibliography{bib/bibliography}

\end{document}
