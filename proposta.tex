% proposal.tex
% Based on http://www.latextemplates.com/template/simple-sectioned-essay
\documentclass[12pt]{article}

\usepackage[utf8]{inputenc}
\usepackage[T1]{fontenc}
\usepackage[a4paper, margin=2cm]{geometry}
\usepackage[brazil]{babel}

\usepackage{blindtext}

\usepackage{graphicx} % resizebox
\usepackage{multirow}

\linespread{1.2} % Line spacing

\newcommand{\Title}[1]{\textbf{\MakeUppercase{#1}}}

\begin{document}

\begin{titlepage}
    \center

    \Title{Universidade Federal de Santa Catarina - UFSC}

    \Title{Ciência Da Computação}

    \vspace*{\stretch{1}}

    Tiago Royer

    \vspace*{\stretch{2}}

    \Title{Complexidade de Circuitos}\\[3cm]

    \begin{flushright}
        \begin{minipage}{0.5\textwidth}
            Proposta de Trabalho de Conclusão de Curso,
            a ser submetido ao Curso de Ciências da Computação
            para a obtenção do Grau de
            Bacharel em Ciências da Computação.

            Orientadora: Profa. Dra. Jerusa Marchi
        \end{minipage}
    \end{flushright}

    \vspace*{\stretch{3}}

    Florianópolis, \today.

    Entrega do TCC: 2º semestre de 2015.
\end{titlepage}


\begin{abstract}
    Complexidade de Circuitos é uma sub-área da Teoria da Computação
    que estuda a taxa de crescimento de circuitos booleanos.
    Em particular, um dos objetivos desta área é demostrar que certas
    funções booleanas exigem circuitos que crescem exponencialmente
    conforme cresce o tamanho das palavras --- e, como consequências
    disso, concluir que $P \neq NP$ \cite{Sipser2006}.

    \textbf{Palavras-chave:} Complexidade de Circuitos, Complexidade Computacional
\end{abstract}

\newpage
\tableofcontents
\newpage

\section{Introdução}

    O mais famoso problema da área de complexidade computacional é o $P$ versus $NP$.
    Os problemas da classe $P$ são os que podem ser resolvidos em tempo polinomial
    por uma Máquina de Turing determinística, enquanto que a classe $NP$ contempla
    os problemas que podem ser resolvidos em tempo polinomial por Máquinas
    não-determinísticas.
    Os problemas considerados são problemas de decisão; isto é, determinar se uma
    palavra está ou não numa determinada linguagem. $P$ e $NP$ são conjuntos de
    linguagens que correspondem a problemas de decisão. Sabemos que $P \subseteq NP$;
    o problema $P$ versus $NP$ pergunta se esta inclusão é estrita ou não \cite{Sipser2006}.
    A maior parte dos pesquisadores acredita que $P \neq NP$,
    embora ainda não tenhamos demonstrado nada.

    Uma das ``frentes de pesquisa'' consiste em tentar ``enfraquecer'' a máquina de Turing.
    A ideia é trabalhar com um modelo de computação muito
    mais restritivo do que o que Turing propôs,
    provar um equivalente a $P \neq NP$ para este modelo de computação,
    e transpôr esta demonstração para o modelo de Turing.
    Circuitos booleanos são exemplos de modelos de computação mais simples.
    Como eles não têm ``partes móveis'' ou múltiplos estados,
    intuitivamente uma demonstração de que $P \neq NP$ deve ser
    mais simples de se obter neste dispositivo computacional. \cite{Hastad1987}.

    As duas principais medidas de complexidade de circuitos são a quantidade de
    nodos (portas lógicas) e a profundidade do circuito.
    Existem várias classes de complexidades de circuitos
    que exploram estas duas métricas;
    neste trabalho, pretende-se relacionar estas classes de complexidade
    com as classes de complexidade computacional.

\section{Objetivos}

    Estabelecer relações entre classes de complexidade computacional
    e complexidade de circuitos.

\subsection{Objetivos especificos}

    \blindlist{enumerate}[5]

\section{Cronograma}

    \begin{tabular}{|c|c|c|c|c|c|c|c|c|c|c|c|c|}
        \hline
        \multirow{2}{*}{Etapas} & \multicolumn{12}{|c|}{Meses}\\ \cline{2-13}
                & dez & jan & fev & mar & abr & mai & jun & jul & ago & set & out & nov \\ \hline
        Etapa 1 &  x  &  x  &  x  &     &     &     &     &     &     &     &     &     \\ \hline
        Etapa 2 &     &     &  x  &  x  &  x  &  x  &     &     &     &     &     &     \\ \hline
        Etapa 3 &     &     &     &     &     &  x  &  x  &     &     &     &     &     \\ \hline
        Etapa 4 &     &     &     &     &     &  x  &  x  &  x  &  x  &  x  &  x  &  x  \\ \hline
        Etapa 5 &     &     &     &     &     &     &     &     &     &     &     &  x  \\ \hline
    \end{tabular}

\section{Método de pesquisa}
    \blindtext[3]

\section{Custos}
    \blindtext[1]

\section{Recursos Humanos}
    \blindtext[1]

\section{Comunicação}
    \blindtext[1]

\section{Riscos}
    \blindtext[1]

\bibliographystyle{abnt-num}
\bibliography{bibliography}

\end{document}
