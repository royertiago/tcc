\section{ENUMERAÇÕES DE GÖDEL ACEITÁVEIS}
\label{sec:acceptable_godel_numbering}

De posse dos teoremas da máquina universal e $S_{mn}$,
podemos definir as restrições sob as quais definiremos os axiomas de Blum.

\begin{definition}
    Seja $\mathcal P$ o conjunto de todas as funções recursivas parciais.
    Uma \emph{enumeração de Gödel aceitável}
    é uma função $\phi: \{0, 1\}^* \to \mathcal P$, sobrejetora,
    que satisfaz às conclusões do teorema da máquina universal
    e do teorema $S_{mn}$.
    (\citeauthor{Rogers1987}, \citeyear{Rogers1987}, p.~41;
    \citeauthor{Blum1967}, \citeyear{Blum1967}, p.~324)
    % TODO: Descobrir como citar dois autores separadamente,
    % incluindo os números de páginas para ambos.
\end{definition}

Uma enumeração de Gödel aceitável é uma função $\phi$ que associa
cada cadeia de $\{0, 1\}^*$ a uma função recursiva parcial.
Como $f$ é sobrejetora,
o contrário também acontece:
toda função recursiva parcial é ``enumerada'' por $\phi$.
Denotaremos $\phi(w)$ por $\phi_w$.

Por ``satisfazer às conclusões do teorema da máquina universal''
quer"=se dizer que,
para esta enumeração $\phi$,
existe alguma função recursiva parcial $g$, de duas variáveis,
tal que
\begin{equation*}
    g(x, y) \simeq \phi_x(y).
\end{equation*}
Similarmente, por ``satisfazer ao teorema $S_{mn}$'' queremos dizer que
existe alguma função recursiva total $\sigma$ de duas variáveis tal que
\begin{equation*}
    \phi_{\sigma(x, y)}(z) \simeq \phi_x(\langle x, y \rangle).
\end{equation*}

\begin{example}
    O mapeamento $w \mapsto f_w$,
    definido na seção~\ref{sec:definition_enumeration_of_recursive_functions},
    é uma enumeração de Gödel aceitável,
    pois enumera todas as funções recursivas (por definição)
    e satisfaz aos teoremas da máquina universal e $S_{mn}$.
\end{example}

Podemos pensar nas enumerações de Gödel aceitáveis
como mapeamentos semânticos;
são regras que atribuem a cada cadeia de $\{0, 1\}^*$
uma função de $\mathcal P$ --- isto é,
a função $\phi_x$ é o que a cadeia $x$ significa.
Todos os modelos de computação equivalentes à máquina de Turing
induzem enumerações de Gödel aceitáveis.
Em particular,
o formalismo das máquinas de Turing com oráculos,
que definiremos na seção~\ref{sec:oracle_definition},
pode ser traduzido numa enumeração de Gödel aceitável.

\begin{example}
    Toda linguagem de programação Turing"=completa pode,
    de certa forma,
    ser vista como uma enumeração de Gödel aceitável.
    Para isso,
    precisamos restringir os programas a serem não"=interativos.
    Por exemplo,
    se restringirmos os programas da linguagem \verb"C"
    a se comunicar com o mundo externo
    usando apenas \verb"scanf", \verb"printf" e similares
    (isto é, sem arquivos, sockets, prompts interativos, etc.),
    definimos o mapeamento
    \begin{equation*}
        \texttt{file.c} \mapsto C_{\texttt{file.c}},
    \end{equation*}
    em que $C_{\texttt{file.c}}$ é a função calculada pelo arquivo \verb"file.c"
    depois de compilado.

    (De certa forma, o mapeamento $C$ é um ``compilador''.)
\end{example}

Computacionalmente,
não podemos trabalhar diretamente com as enumerações,
pois são estruturas infinitas;
as restrições adicionais
(possuir uma função universal e satisfazer ao teorema $S_{mn}$)
nos permite trabalhar com estas enumerações indiretamente.
