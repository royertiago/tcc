\section{Teorema da Máquina Universal}
\label{sec:universal_turing_machine}

Durante a demonstração da indecidibilidade do problema da parada,
é construída uma máquina de Turing universal,
que é capaz de simular qualquer outra máquina de Turing,
bastando apenas lhe ser fornecida uma representação da máquina de Turing simulada.
(De certa forma,
os computadores modernos são máquinas de Turing universais.)
Podemos sintetizar esta construção através do mapeamento de funções recursivas
definido na seção anterior.

\begin{theorem}
    Existe uma função recursiva parcial $g$ de duas variáveis tal que
    \begin{equation*}
        g(x, y) \simeq f_x(y).
    \end{equation*}
\end{theorem}

\begin{proof}
    A função $g$ é a função computada por uma máquina de Turing universal.
    Em nosso caso,
    a máquina universal $U$,
    na entrada $\langle x, y \rangle$,
    usará a tabela de transições disponível em $x$
    para executar as etapas de computação de alguma máquina $M$
    que satisfaz $\langle M \rangle = x$.
    $x$ é a codificação binária de $M$;
    portanto,
    ao simular as etapas de computação de $M$ em $y$,
    $U$ estará, efetivamente,
    calculando o valor de $f_x$ (que é a função computada por $M$).

    $f_x(y)$ só não está definida se $M$ nunca parar ao processar $y$;
    neste caso, $U$ também não parará ao processar $\langle x, y \rangle$.
    Caso contrário,
    após a simulação chegar a um estado final,
    basta que $U$ escreva a ``resposta'' de $M$ na saída.
    Como a resposta de $M$ é $M(y) = f_x(y)$,
    obtemos a igualdade $g(x, y) = f_x(y)$.
\end{proof}
