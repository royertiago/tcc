\section{Enumeração de Funções Recursivas}
\label{sec:definition_enumeration_of_recursive_functions}

A existência de uma codificação em binário de máquinas de Turing
permite, de certa forma,
enumerar todas as funções recursivas parciais
--- basta listar todos os códigos possíveis para máquinas de Turing.
Estritamente falando,
não é as funções que estão sendo listadas
(isso seria impossível pois funções são objetos matemáticos infinitos),
mas sim códigos em binário para máquinas que as computam.

A construção da codificação das máquinas de Turing não é ``injetora'';
isto é, existem máquinas $M$ e $N$ diferentes
tais que
\begin{equation*}
    \langle M \rangle = \langle N \rangle.
\end{equation*}
Entretanto,
a construção foi feita de forma que
as funções computadas por $M$ e $N$ coincidam
sempre que a equação acima for satisfeita.
Isto é,
se $\langle M \rangle = \langle N \rangle$,
então, para todo $x$,
\begin{equation*}
    M(x) \simeq N(x).
\end{equation*}

Ou seja,
$M$ e $N$ computam exatamente a mesma função parcial.
Portanto,
dada uma cadeia $w \in \{0, 1\}^*$,
existe uma única função recursiva parcial
que é computada pelas máquinas cuja codificação em binário é $w$.
Chamaremos esta função de $f_w$.

\begin{notation}
    \simbolo{$f_w$}{Função computada pela máquina codificada pela palavra $w$}
    Se $w = \langle M \rangle$, então $f_w$ é a função parcial computada por $M$.
\end{notation}

Toda máquina de Turing é representada por uma palavra de $\{0, 1\}^*$,
portanto o mapeamento $w \mapsto f_w$
enumera todas as funções recursivas.
Além disso,
toda palavra de $\{0, 1\}^*$ representa alguma máquina;
assim, $f_w$ sempre está bem definido.
As seções \ref{sec:universal_turing_machine} e~\ref{sec:s_m_n_theorem}
contém teoremas que nos permitem trabalhar com as funções $f_w$
e a seção~\ref{sec:acceptable_godel_numbering}
define o conceito de ``Enumeração de Gödel aceitável'',
que generaliza este mapeamento.
