\section{MODELO DE MÁQUINA UTILIZADO}

Para os propósitos deste trabalho,
assumiremos que todas as máquinas de Turing possuem três fitas.
\begin{itemize}
    \item A primeira fita será utilizada apenas para a entrada da máquina;
        ela sempre conterá uma palavra de $\{0, 1\}^*$,
        e o seu conteúdo não poderá ser alterado.
    \item A segunda fita será a fita de trabalho da máquina.
        A máquina poderá ler e escrever símbolos arbitrários nesta fita infinita,
        que começará totalmente em branco.
    \item A terceira fita será utilizada apenas para saída.
        O cabeçote desta fita poderá realizar apenas três movimentos:
        \begin{itemize}
            \item Escrever um $0$ e mover"=se para a direita;
            \item Escrever um $1$ e mover"=se para a direita;
            \item Manter"=se parada.
        \end{itemize}
\end{itemize}

A função de transição da máquina poderá depender apenas das duas primeiras fitas.
Como estaremos interessados no cálculo de funções,
não precisaremos de estados de aceitação ou rejeição;
basta não haver transição partindo do estado atual para que a máquina pare.
(Podemos codificar ``aceitação'' como sendo a escrita de um único $1$ da fita de saída
e ``rejeição'' como a escrita de um único $0$.)

Observe que não há perda de generalidade
ao fazer com que a fita de saída seja somente"=escrita,
pois a máquina poderia construir a saída toda na máquina de trabalho,
e escrevê"=la de uma só vez na fita de saída.

Em relação à definição usual de ``máquina de Turing'',
nossa versão não possui a definição de $\Sigma$,
o alfabeto de entrada.
Em todo o TCC,
assumiremos que $\Sigma = \{0, 1\}$;
isto é, todas as máquinas trabalharão com o sistema binário.
Observe que isto não é uma restrição significativa,
pois podemos codificar outros alfabetos usando o sistema binário,
sem grandes perdas de eficiência.

Este modelo precisará ser ajustado no capítulo~\ref{ch:oracles},
quando adicionaremos mais uma fita à máquina,
que será usada para comunicação com o oráculo.

O apêndice~\ref{app:turing_machines}
contém a definição formal deste modelo.
