A teoria de complexidade axiomática proposta por \citeonline[p.~324]{Blum1967},
que estudaremos na seção~\ref{sec:blum_axioms},
é \emph{independente de máquina};
isto é,
ela é aplicável a qualquer modelo de máquina ``bem comportado''.
Nesta seção,
veremos como generalizar a noção de ``modelo de máquina''
e estudaremos as condições para que ele seja ``bem comportado'';
isto é,
que propriedades o modelo de máquina precisa satisfazer
para que possamos definir os axiomas de Blum.

Entenderemos um modelo de máquina como
uma maneira de atribuir um significado a cada cadeia binária.%
\footnote{
    Estamos identificando uma máquina com sua representação em binário.
}
A ideia é que cada cadeia represente o ``código"=fonte''
de alguma função recursiva;
o modelo de máquina, então,
dirá como devemos interpretar este código"=fonte.

Precisaremos, antes,
definir a noção de \emph{função recursiva}.
Para isso,
usaremos um modelo específico de máquina de Turing,
que será definido na seção~\ref{sec:machine_model}.

A definição de \emph{função recursiva} está na seção~\ref{sec:recursive_functions}.
As seções \ref{sec:universal_turing_machine} e~\ref{sec:s_m_n_theorem}
contém teoremas $S_{mn}$ e da máquina universal,
que são as características que exigiremos de um modelo de máquina
para ``merecer'' o título de ``bem comportado''.
Esta amarração está na seção~\ref{sec:acceptable_godel_numbering}.
