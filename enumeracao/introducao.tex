A teoria de complexidade axiomática proposta por \citeonline[p.~324]{Blum1967},
que estudaremos na seção~\ref{sec:blum_axioms},
é \emph{independente de máquina};
isto é,
ela é aplicável a qualquer modelo de máquina ``bem comportado''.
Nesta seção,
veremos como generalizar a noção de ``modelo de máquina''
e estudaremos as condições para que ele seja ``bem comportado'';
isto é,
que propriedades o modelo de máquina precisa satisfazer
para que possamos definir os axiomas de Blum.

Entenderemos um modelo de máquina como
uma maneira de atribuir um significado a cada cadeia binária.%
\footnote{
    Estamos identificando uma máquina com sua representação em binário.
}
A ideia é que cada cadeia represente o ``código"=fonte''
de alguma função recursiva;
o modelo de máquina, então,
dirá como devemos interpretar este código"=fonte.

Precisaremos, antes,
definir a noção de \emph{função recursiva}.
Para isso,
usaremos um modelo específico de máquina de Turing,
que será definido na seção~\ref{sec:machine_model}.

A definição de \emph{função recursiva} está na seção~\ref{sec:recursive_functions}.
As seções \ref{sec:universal_turing_machine} e~\ref{sec:s_m_n_theorem}
contém teoremas $S_{mn}$ e da máquina universal,
que são as características que exigiremos de um modelo de máquina
para ``merecer'' o título de ``bem comportado''.
Esta amarração está na seção~\ref{sec:acceptable_godel_numbering}.

Definiremos agora uma maneira de identificar duas funções
que não estão definidas no mesmo domínio;
esta notação simplificará a notação no restante deste capítulo.

\begin{notation}
    \simbolo{$f(x) \simeq g(x)$}{$f$ concorda com $g$ em $x$}
    Dadas duas funções $f, g: A \to B$,
    escreveremos
    \begin{equation*}
        f(x) \simeq g(y)
    \end{equation*}
    para expressar o fato de que
    $f$ está definida em $x$ se, e somente se, $g$ está definida em $y$;
    e, caso ambas estejam definidas,
    $f(x) = g(y)$.
    (Esta notação foi extraída de \citeonline[p.~124]{EpsteinCarnielli2008}.
    Nos casos em que $x = y$,
    podemos ler esta notação como ``$f$ e $g$ concordam em $x$''.)
\end{notation}

\begin{example}
    Defina $f, g: \mathbb R \to \mathbb R$, dadas por
    \begin{align*}
        f(x) &= \sqrt{x^2} & g(x) = {(\sqrt x)}^2.
    \end{align*}
    Então, para $x \geq 0$,
    \begin{equation*}
        f(x) \simeq g(x),
    \end{equation*}
    mas isso não é válido para $x < 0$.
    Além disso, por exemplo,
    \begin{equation*}
        g(-1) \simeq g(-3)
    \end{equation*}
    pois tanto $g(-1)$ quanto $g(-3)$ não existem.
\end{example}
