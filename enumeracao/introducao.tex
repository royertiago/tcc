\begin{notation}
    \simbolo{$f(x) \simeq g(x)$}{$f$ concorda com $g$ em $x$}
    Dadas duas funções $f, g: A \to B$,
    escreveremos
    \begin{equation*}
        f(x) \simeq g(y)
    \end{equation*}
    para expressar o fato de que
    $f$ está definida em $x$ se, e somente se, $g$ está definida em $y$;
    e, caso ambas estejam definidas,
    $f(x) = g(y)$.
    (Esta notação foi extraída de \citeonline[p.~124]{EpsteinCarnielli2008}.
    Nos casos em que $x = y$,
    podemos ler esta notação como ``$f$ e $g$ concordam em $x$''.)
\end{notation}

\begin{example}
    Defina $f, g: \mathbb R \to \mathbb R$, dadas por
    \begin{align*}
        f(x) &= \sqrt{x^2} & g(x) = {(\sqrt x)}^2.
    \end{align*}
    Então, para $x \geq 0$,
    \begin{equation*}
        f(x) \simeq g(x),
    \end{equation*}
    mas isso não é válido para $x < 0$.
    Além disso, por exemplo,
    \begin{equation*}
        g(-1) \simeq g(-3)
    \end{equation*}
    pois tanto $g(-1)$ quanto $g(-3)$ não existem.
\end{example}

Esta notação simplificará o enunciado de alguns teoremas.
