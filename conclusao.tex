\chapter{CONSIDERAÇÕES FINAIS}

Neste trabalho,
desenvolvemos uma teoria de funções não"=determinísticas,
guiados pelo objetivo de poder definir sua complexidade
em termos dos axiomas de Blum.
Conseguimos generalizar a classe $\NP$ para problemas funcionais
(a classe $\FNP$)
da mesma forma como a classe $\FP$ generaliza a classe $\P$;
problemas de $\NP$,
como $\SAT$,
possuem uma generalização natural em nossa classe $\FNP$,
como encontrar uma atribuição satisfazível, no caso de $\SAT$.

O objetivo de manter compatibilidade com os axiomas de Blum foi atingido.
É interessante notar que, dos trabalhos analisados neste capítulo,
apenas Hopcroft se preocupa com os axiomas de Blum.
Além disso,
nenhum dos autores tenta interpretar as funções não"=determinísticas
como uma enumeração de Gödel,
que é um passo necessário para que uma medida de complexidade
satisfaça aos axiomas de Blum.

Também pudemos construir uma ``hierarquia polinomial funcional'',
simplesmente adaptando as definições da hierarquia polinomial ``tradicional''.
O fato de estarmos trabalhando com funções
cujo domínio e contradomínio coincidem
nos permitiu compô"=las,
possibilitando resultados como o teorema~\ref{thm:compositive_closure},
que nos deu uma interpretação de $\Delta_n^f$
em termos da composição de funções de $\Sigma_n^f$.

Dentre os trabalhos analisados,
o único que contém definições parecidas com o que foi desenvolvido neste TCC
é o de \cite[p.~3]{Krentel1988},
que foi encontrado apenas no final do desenvolvimento.
Com isso,
as referências bibliográficas acabaram todas ``presas''
no século passado.
Seria interessante refazer a pesquisa bibliográfica,
mas direcionando para trabalhos que lidem com problemas de otimização
(como a classe OptP, de Krentel).

Os dois principais problemas que ficaram em aberto
estão relacionados à hierarquia polinomial funcional desenvolvida.
A demonstração do teorema~\ref{thm:compositive_closure}
utilizou o problema completo
\begin{equation*}
    \HaltFNP^{\FNP^\mathcal A},
\end{equation*}
retirado do teorema~\ref{thm:polinomially_complete_problems}.
A formulação deste problema é baseada no problema da parada.
Como observado por \citeonline[p.~255]{Papadimitriou1994},
esse tipo de classe de complexidade\footnote{
    As \emph{classes de complexidade sintáticas}
    são aquelas em que a pertinência à classe
    depende apenas do modelo de máquina utilizado
    \cite[p.~255]{Papadimitriou1994}.
    (Note que esta não é uma definição formal.)
    Por exemplo,
    classes como $\P$, $\NP$, $\FNP$ etc.\ são classes de complexidade sintáticas;
    enquanto que a classe dos problemas $\NP$"=completos não é
    --- esta é uma classe \emph{semântica}.
}
sempre possuirá um problema completo da forma como é $\HaltFNP$.
Fica em aberto determinar se existem problemas mais naturais
que sejam completos para os demais níveis da hierarquia polinomial funcional.

Outro problema é o mencionado no final do capítulo~\ref{ch:nondeterministic_functions},
que é a possibilidade de termos ``exagerado na dose''
ao definir o valor de $M(x)$ como sendo o máximo dos ramos de computação.
Dessa forma,
com um simples pós"=processamento
(conforme o teorema~\ref{thm:strong_compositive_closure}),
podemos simular várias chamadas a um mesmo oráculo.
Seria interessante tentar ``enfraquecer'' esta definição,
não permitindo mais que uma única chamada
dê informação equivalente a várias,
criando uma hierarquia mais densa de problemas.
