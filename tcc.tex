\documentclass{ufsctex/ufsctex}

\usepackage{amsmath}
\usepackage{amssymb}

\usepackage{amsthm}
\newtheorem{theorem}{Teorema}[chapter]
\newtheorem{lemma}[theorem]{Lema}
\newtheorem{corollary}[theorem]{Corolário}
\newtheorem{proposition}[theorem]{Proposição}
\theoremstyle{definition}
\newtheorem{definition}[theorem]{Definição}
\newtheorem{example}[theorem]{Exemplo}

\usepackage{tikz}
\usepackage{hyphenat}
\usepackage{enumitem}
\usepackage{accessdate}

% ufsctex.cls se baseia em abntex.cls que carrega o pacote mathptmx.
% Este pacote altera as fontes para Times, inclusive \mathcal.
% Este código "mágico" restaura-a para seu formato padrão.
\DeclareMathAlphabet{\mathcal}{OMS}{cmsy}{m}{n}

\begin{document}

% Capa e folha de rosto

\instituicao[a]{Universidade Federal de Santa Catarina} % Opcional
\departamento[a]{Biblioteca Universitária}
\curso[o]{Programa de ...}
\documento[a]{Tese} % [o] para dissertação [a] para tese
\titulo{Título}
\subtitulo{Subtítulo (se houver)} % Opcional
\autor{Nome completo do autor}
\grau{...}
\local{Florianópolis} % Opcional (Florianópolis é o padrão)
\data{04}{junho}{2013}
\orientador[Orientador\\Universidade ...]{Prof. Dr.}
\coorientador[Coorientador\\Universidade ...]{Prof. Dr.}
\coordenador[Coordenador\\Universidade ...]{Prof. Dr. }

\capa
\folhaderosto[comficha] % Se nao quiser imprimir a ficha, é só não usar o parâmetro

%\numerodemembrosnabanca{3}
\bancaMembroA{
    Profa. Dra. Karina Girardi Roggia \\
    Universidade do Estado de Santa Catarina
}
\bancaMembroB{
    Prof. Dr. Ricardo Azambuja Silveira \\
    Universidade Federal de Santa Catarina
}
\bancaMembroC{
    Prof. Dr. Rosvelter João Coelho da Costa \\
    Universidade Federal de Santa Catarina
}

\folhaaprovacao

%% Dedicatória

\dedicatoria{
    À minha mãe, Venida
}

\paginadedicatoria

%% Agradecimentos

\agradecimento{
    Agradeço muito minha família, em especial à minha mãe,
    por sempre terem me apoiado, guiado e auxiliado
    ao longo dos últimos 20 anos.

    Agradeço minha orientadora, Jerusa,
    que me acolheu logo no início do curso,
    por todas as conversas e orientações,
    que muitas vezes extrapolaram os limites da Academia.

    Agradeço aos professores do Programa Avançado de Matemática
    (Melissa, Gilles, Fernando),
    que me ajudaram a construir a maturidade matemática
    que me foi extremamente útil nos quatro anos de graduação.

    Agradeço as discussões com Santana, Schultz, Gabriel e Abouhatem
    sobre os algoritmos da Maratona de Programação.

    Agradeço todas as discussões produtivas (e improdutivas)
    com o pessoal do laboratório IATE.

    E, por último, agradeço o companheirismo dos meus amigos
    ao longo desta jornada.
}

\paginaagradecimento

%% Epígrafe

\epigrafe{
    Even if one works basically all week trying to prove $\P \neq \NP$,
    one should set aside Friday afternoon for trying to prove $\P = \NP$.
}
{
    (John Edward Hopcroft)
}
\paginaepigrafe

% Resumo

\textoResumo{
    O texto do resumo deve ser digitado,
    em um único bloco,
    sem espaço de parágrafo.
    O resumo deve ser significativo,
    composto de uma sequência de frases concisas,
    afirmativas e não de uma enumeração de tópicos.
    Não deve conter citações.
    Deve usar o verbo na voz passiva.
    Abaixo do resumo,
    deve-se informar as palavras-chave
    (palavras ou expressões significativas retiradas do texto)
    ou, termos retirados de thesaurus da área.
}
\palavrasChave{
    Palavra-chave 1.
    Palavra-chave 2.
    Palavra-chave 3.
}

\paginaresumo

%% Resumo, em inglês.

\textAbstract{
    Resumo traduzido para outros idiomas,
    neste caso,
    inglês.
    Segue o formato do resumo
    feito na língua vernácula.
    As palavras-chave traduzidas,
    versão em língua estrangeira,
    são colocadas abaixo do texto
    precedidas pela expressão ``Keywords'',
    separadas por ponto.
}
\keywords{
    Keyword 1.
    Keyword 2.
    Keyword 3.
}

\paginaabstract

% Sumário e listas usadas no documento.
% As listas dependem da necessidade do usuário

\listadefiguras
\listadetabelas
\listadeabreviaturas
\listadesimbolos
\sumario

\sumario

\emph{Complexidade} é a quantidade de recursos
que uma máquina de Turing gasta
para computar determinada função
ou para decidir pertinência a uma linguagem
\cite[p.~285]{HopcroftUllman1979}.
Os recursos mais importantes para a teoria de complexidade computacional
são o espaço e o tempo,
em máquinas de Turing determinísticas e não"=determinísticas.

Os axiomas de Blum definem a noção de complexidade computacional
não apenas para máquinas de Turing,
mas sim para qualquer modelo de computação que satisfaça algumas restrições.

A seção~\ref{sec:enumeration_of_recursive_functions}
contém a maquinaria matemática necessária para expressar estas restrições.
Os axiomas de Blum aparecem na seção~\ref{sec:blum_axioms}.
Por fim, a seção~\ref{sec:default_measures}
define as medidas de complexidade padrão.

\section{Complexidade Funcional}

De posse do formalismo de funções não"=determinísticas,
finalmente,
concluímos a definição $\PhiNT$ e $\PhiNS$ para funções.
$\FDTIME$, $\FDSPACE$, $\FNTIME$ e $\FNSPACE$
são definidos para funções de maneira análoga às definições para decisores.
Por simetria, também definiremos $\coFNTIME$ e $\coFNSPACE$;
aqui, ao invés de tomarmos o máximo dentre os ramos,
pegaremos o mínimo para ser o valor da função.
(São as funções ``co"=não"=determinísticas''.)
Nos concentraremos nas classes de complexidade de tempo,
onde o uso de não"=determinismo aparenta ter mais impacto.

\begin{definition}
    \begin{align*}
        \FP &= \bigcup_{k > 0} \FDTIME(n^k) \\
        \FNP &= \bigcup_{k > 0} \FNTIME(n^k) \\
        \coFNP &= \bigcup_{k > 0} \coFNTIME(n^k)
    \end{align*}
    \simbolo{$\FP$}{Funções determinísticas polinomiais}
    \simbolo{$\FNP$}{Funções não"=determinísticas polinomiais}
    \simbolo{$\coFNP$}{Funções co"=não"=determinísticas polinomiais}
\end{definition}

$\FP$ são as funções que podem ser calculadas
em tempo polinomial por máquinas determinísticas,
e $\FNP$, por máquinas não"=determinísticas.\footnotemark
$\coFNP$ é o análogo a $\coNP$.
\footnotetext{
    Nossa definição de $\FNP$ diverge da definição da literatura.
    A definição de \citeonline[p.~229]{Papadimitriou1994},
    por exemplo,
    é baseada em linguagens polinomialmente equilibradas
    (mencionadas na nota de rodapé~\ref{foot:polinomially_balanced},
    na página~\pageref{foot:polinomially_balanced}).

    Tome um problema de decisão $L \in \NP$
    e construa uma linguagem polinomialmente equilibrada $R_L$ associada a $L$.
    $R_L$ será um conjunto de pares $(x, y)$,
    em que $x \in L$ e $y$ é um ``certificado'' de que $x \in L$.

    O \emph{problema funcional associado} a $L$, $\class{F}L$,
    consiste em encontrar algum $y$ tal que $(x, y) \in R_L$,
    ou determinar que tal $y$ não existe;
    então,
    \citeonline[p.~229]{Papadimitriou1994} define $\FNP$
    como o conjunto de todos os $\class{F}L$
    para $L \in \NP$.

    Estes problemas são por vezes chamados de \emph{problemas de busca}
    (do inglês \emph{search problem}),
    em oposição a \emph{problemas de decisão}.

    A diferença crucial é que \citeauthoronline{Papadimitriou1994}
    pede apenas que encontremos algum $y$,
    enquanto que nossa definição exige que encontremos o maior deles.
}

\begin{example}
    Todos os algoritmos polinomiais
    discutidos em cursos de projeto e análise de algoritmos
    correspondem a funções em $\FP$.
    Por exemplo,
    temos cálculo de determinantes,
    programação linear\footnote{
        Programação linear é resolvível em tempo polinomial
        quando o domínio do problema são os números racionais.
        Se restringirmos o domínio da solução a números inteiros,
        o problema é $\NP$"=completo.

        Programação linear nos inteiros geralmente é chamada de
        \emph{programação inteira}
        (do inglês \emph{integer programming}).
    }
    e parsing de linguagens livres de contexto.
\end{example}

\begin{uproposition}
    Todas as funções características de problemas em $\NP$ estão em $\FNP$,
    e todas as funções características de problemas em $\coNP$ estão em $\coFNP$.
\end{uproposition}

\begin{example}
    Os problemas em $\NP$ podem ser facilmente generalizados para funções em $\FNP$.
    Por exemplo,
    podemos pegar uma máquina não"=determinística para $\SAT$,
    e modificá"=la para que,
    após achar uma atribuição de valores"=verdade,
    a escreva na fita.
    Caso contrário,
    escreva a palavra vazia.
    A função não"=determinística computada por esta máquina
    é a maior atribuição de valores"=verdade,
    lexicograficamente.
    Esta função está em $\FNP$.

    Similarmente,
    podemos encontrar o maior clique num grafo,
    achar fatores de um número,
    e encontrar um isomorfismo entre dois grafos
    usando o poder computacional de $\FNP$.

    Já problemas de minimização
    são mais facilmente interpretados como funções de $\coFNP$.
    Por exemplo,
    encontrar o caminho hamiltoniano de menor custo,
    e calcular o número cromático.
\end{example}

O contrário também é possível;
isto é,
podemos transformar problemas de $\FNP$ em problemas de $\NP$.

\begin{theorem}
    Se $f$ é uma função de $\FNP$,
    então a linguagem
    \begin{equation*}
        L_f = \{ \langle x, y \rangle \mid f(x) \geq y \},
    \end{equation*}
    em que $\geq$ significa comparação lexicográfica,
    pertence a $\NP$.
    \label{thm:fnp_to_np_conversion}
\end{theorem}

\begin{proof}
    A ideia é fazer com que a máquina não"=determinística $M$,
    que reconhecerá $L_f$,
    compute $f$ usando seu próprio não"=determinismo,
    e aceitará a entrada caso ache algum ramo de computação
    que produza um valor maior ou igual a $y$.

    Mais formalmente,
    seja $N$ uma máquina não"=determinística que computa $f$ em tempo polinomial.
    Considere que a entrada é $\langle x, y \rangle$.
    A primeira coisa que $M$ fará é executar $N$ em $x$.
    Sabemos que isso é possível pois ambas as máquinas são não"=determinísticas.

    Após $N$ parar
    (o que ocorrerá em tempo polinomial),
    $N$ deixará na fita um ``candidato'' a $f(x)$.
    De fato,
    todas as folhas da computação de $N$ estarão disponíveis nos ramos de $M$.
    Como $N$ computa $f$,
    apenas o maior destes valores será $f(x)$;
    portanto,
    se algum ramo encontrar algum candidato que seja maior ou igual a $y$,
    a entrada $\langle x, y \rangle$ pode ser aceita.

    O contrário também ocorre.
    Se todos os ramos de computação gerarem candidatos a $f(x)$
    que são menores que $y$,
    o próprio valor de $f(x)$ será menor que $y$,
    portanto a entrada pode ser rejeitada.

    Em outras palavras,
    a máquina $M$ deve aceitar exatamente quando encontrar
    alguma folha da computação de $N(x)$ que é maior que $y$.
\end{proof}

De certa forma,
este teorema nos permite fazer uma ``busca binária''
atrás do valor de $f(x)$
usando apenas algum dispositivo que reconheça $L_f$.
Voltaremos a esta ideia na seção~\ref{sec:functional_oracles}.

\chapter{COMPLEXIDADE DE CIRCUITOS}

\section{DEFINIÇÕES}

\section{RELAÇÃO COM O PROBLEMA P vs NP}

\section{LIMITES INFERIORES}

%\chapter{Comparação com outros trabalhos}

Este capítulo discute brevemente outros trabalhos que,
de certa forma,
tentam capturar a noção de ``função não"=determinística''.

\section{Problemas de busca vs problemas de decisão (Papadimitriou)}
\label{sec:papadimitriou_comparison}

\citeonline[p.~229]{Papadimitriou1994}
define $\FNP$ diferente de nossa definição
(na seção~\ref{sec:functional_complexity}).

\begin{definition}
    Uma linguagem $L$ é \emph{polinomialmente equilibrada}
    se existir algum polinômio $p$ tal que
    todos os elementos de $L$ são pares ordenados da forma $(x, y)$
    em que $|y| \leq p(|x|)$.
\end{definition}
Isto é, $L$ é constituída de pares,
de forma que o segundo elemento do par não é muito maior que o primeiro elemento.

Linguagens de $\NP$ e linguagens polinomialmente equilibradas
estão relacionadas por certificados de pertinência.
Por exemplo,
para a linguagem $\SAT$,
podemos provar eficientemente
(isto é, em tempo polinomial)
que determinada instância $\varphi$ é satisfazível
se fornecermos uma atribuição de valores"=verdade que satisfaz a instância;
esta atribuição é uma \emph{testemunha} ou \emph{certificado}
para a pertinência de $\varphi$ a $\SAT$.

Para cada linguagem $L \in \NP$,
podemos sistematicamente prover certificados de pertinência para $L$:
como sabemos que existe uma máquina de Turing não"=determinística que decide $L$,
podemos fornecer a sequência de transições não"=determinísticas
como um certificado de pertinência.
Como esta máquina opera em tempo polinomial,
para cada $x \in L$,
o par
\begin{equation*}
    (x, y),
\end{equation*}
em que $y$ é esta sequência de transições,
satisfará $|y| \leq p(|x|)$,
para algum polinômio $p$
--- no caso, $p$ é o próprio limite de tempo da máquina não"=determinística
que reconhece $L$.

Dessa forma,
podemos, sistematicamente,
associar uma linguagem $L \in \NP$
a uma linguagem polinomialmente equilibrada $R_L \in \P$.
Assim,
construiremos o conjunto $\FNP$ definido por \citeonline[p.~229]{Papadimitriou1994}.

\begin{definition}
    Se $L \in \NP$, chame de $R_L$
    uma linguagem polinomialmente equilibrada associada com $L$.
    Então defina $\FNP$ por
    \begin{equation*}
        \FNP = \{R_L \mid L \in \NP\}
    \end{equation*}
    \cite[p.~229]{Papadimitriou1994}
\end{definition}

Estes problemas são por vezes chamados de \emph{problemas de busca}
(do inglês \emph{search problem}),
em oposição a \emph{problemas de decisão}.

Do ponto de vista computacional,
a definição de Papadimitriou captura a ideia de
encontrar alguma solução para o problema $L$,
potencialmente descartando uma quantidade exponencial de outras soluções.

Esta definição contorna o problema de definir ``função não"=determinística''
trabalhando diretamente com os certificados de pertinência.
Entretanto,
a ausência do conceito de função
dificulta a formalização do conceito de oráculo
(pois não há mais uma única resposta ``certa'',
mas sim várias)
e impossibilita a análise do efeito da composição de funções,
como a feita na seção~\ref{sec:function_composition}.

\section{Conjunto potência como contradomínio (Andreev)}

\citeonline[p.~3]{Andreev1994}
apresenta uma abordagem diferente.
Em vez de tentar amarrar algum dos ramos de computação,
a função retornará todos eles.

\begin{definition}
    Sejam $\mathcal A$ e $\mathcal B$ conjuntos finitos,
    e $2^\mathcal B$ o conjunto potência de $\mathcal B$.
    Uma \emph{função não"=determinística de $\mathcal A$ em $\mathcal B$}
    é uma função da forma
    \begin{equation*}
        f : \mathcal A \to 2^\mathcal B.
    \end{equation*}
    \cite[p.~3]{Andreev1994}
\end{definition}

Os conjuntos $\mathcal A$ e $\mathcal B$ usados por \citeonline[p.~4]{Andreev1994}
são da forma $\{0, 1\}^k$.
A ideia é que $\mathcal A$ represente as possíveis entradas para um circuito booleano,
e que $\mathcal B$ represente as saídas desse circuito.
Dado um circuito computacional $S$,
com $k$ entradas e $l$ saídas,
dizemos que $S$ computa a função não"=determinística $F$ de $\{0, 1\}^k$ em $\{0, 1\}^l$
se,
para todo $a \in \{0, 1\}^k$,
$S(a) \in F(a)$.
Isto é,
todas as saídas possíveis de $S$ estão previstas em $F$.
Esta definição é similar à definição de \citeonline[p.~229]{Papadimitriou1994},
no sentido de que basta retornar um valor de $F(a)$.
\citeonline[p.~4]{Andreev1994}
define, então,
a \emph{complexidade} de uma função não"=determinística $F$
como sendo o tamanho do menor circuito que computa $F$.

O fato de Andreev se restringir a domínio e contradomínio finitos
mostra que esta definição não ``combina'' corretamente
a noção de função determinística com máquinas não"=determinísticas,
pois as funções operam sobre o domínio $\{0, 1\}^*$.

Além disso,
como o domínio e o contradomínio das funções não"=determinísticas de Andreev
são diferentes,
seria necessário um malabarismo adicional
para desenvolver a noção de composição de funções,
assim como na definição de Papadimitriou.

\section{Definição de Hopcroft e Ullman}

A definição de \citeonline[p.~313]{HopcroftUllman1979},
mencionada na introdução,
é a que motivou este trabalho.
\begin{definition}
    Seja $M$ uma máquina de Turing não"=determinística.
    Dizemos que $M(x) = y$ se
    algum dos ramos de computação de $M$ em $x$ produz $y$,
    e nenhum outro ramo de computação que se encerre
    produz um valor diferente de $y$.
    \cite[p.~313]{HopcroftUllman1979}
\end{definition}

O problema é que esta definição não enumera corretamente as funções recursivas.
O grande vilão está na permissão que damos à máquina $M$
não parar em todos os ramos de computação.

Por exemplo,
construa uma máquina $N$ que,
na entrada $\langle M, x \rangle$,
assuma dois ramos de computação:
no primeiro,
$N$ incondicionalmente escreve $1$ na fita de saída e para.
No outro, $N$ simula a máquina determinística $M$ em $x$ e,
caso $M$ pare ao computar $x$,
$N$ escreve $0$ na fita.

Se $M$ não para em $x$,
então $N$ possui um único ramo de computação que para,
e este ramo produz $1$ na fita;
portanto, $N(\langle M, x \rangle) = 1$.
Caso $M$ pare ao computar $x$,
$N(\langle M, x \rangle)$ estará indefinido,
pois dois ramos de computação de $N$ escrevem coisas diferentes na fita.

Ou seja,
esta máquina computa
\begin{equation*}
    f(\langle M, x \rangle) = 1, \text{ se $M$ não parar ao computar $x$,}
\end{equation*}
que é exatamente o complemento do problema da parada.
Portanto,
pela definição de Hopcroft e Ullman,
conseguimos enumerar funções não"=recursivas.

Mas,
se impormos à definição de Hopcroft e Ullman a restrição adicional
de que todos os ramos de $M$ devem parar para que $M(x)$ exista,
então não há a necessidade de haver vários ramos de computação,
pois todos eles retornam o mesmo resultado.
Basta fixar uma das possíveis transições não"=determinísticas,
tornando a máquina \emph{determinística}.
Neste caso,
perdemos o aparente ganho de tempo exponencial ao usar não"=determinismo.


\bibliographystyle{abnt-alf}
\bibliography{bib/bibliography}

\apendice
\chapter{Máquinas de Turing}
\label{app:turing_machines}

Este apêndice contêm uma formalização de máquinas de Turing
e seu uso como decisores e computadores de funções recursivas.

\section{Definição Matemática}

\subsection{Decisores e Enumeradores}

\subsection{Funções de Inteiros}

\section{Máquinas Multifitas}

\subsection{Fitas de Entrada e Saída}

\section{Codificação de Máquinas de Turing}

\subsection{Máquinas de Turing Universais}


\end{document}
