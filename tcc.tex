\documentclass{ufsctex/ufsctex}

\usepackage{amsmath}
\usepackage{amssymb}

\usepackage[amsthm, amsmath, thmmarks]{ntheorem}
\theoremstyle{plain}
\newtheorem{theorem}{Teorema}[chapter]
\newtheorem{lemma}[theorem]{Lema}
\newtheorem{corollary}[theorem]{Corolário}
\newtheorem{proposition}[theorem]{Proposição}

\theoremsymbol{$\square$}
\newtheorem{utheorem}[theorem]{Teorema} % Unproved theorem
\newtheorem{ucorollary}[theorem]{Corolário} % Unproved corollary
\newtheorem{uproposition}[theorem]{Proposição} % Unproved proposition
\theoremsymbol{}

\theoremstyle{definition}
\newtheorem{definition}[theorem]{Definição}
\newtheorem{provisionaldefinition}[theorem]{Definição provisória}

\theoremsymbol{$\square$}
\newtheorem{example}[theorem]{Exemplo}
\newtheorem{counterexample}[theorem]{Contraexemplo}
\theoremsymbol{}

\theoremstyle{nonumberremark}
\newtheorem{notation}{Notação}

\renewcommand{\proofSymbol}{$\blacksquare$}

\usepackage{tikz}
\usetikzlibrary{calc}

\usepackage[portuguese,algochapter,linesnumbered]{algorithm2e}
\usepackage{complexity}
\usepackage{hyphenshorthand}
\usepackage{enumitem}
\usepackage{accessdate}

% ufsctex.cls se baseia em abntex.cls que carrega o pacote mathptmx.
% Este pacote altera as fontes para Times, inclusive \mathcal.
% Este código "mágico" restaura-a para seu formato padrão.
\DeclareMathAlphabet{\mathcal}{OMS}{cmsy}{m}{n}

\begin{document}

% Capa e folha de rosto

\instituicao[a]{Universidade Federal de Santa Catarina} % Opcional
\departamento[a]{Biblioteca Universitária}
\curso[o]{Programa de ...}
\documento[a]{Tese} % [o] para dissertação [a] para tese
\titulo{Título}
\subtitulo{Subtítulo (se houver)} % Opcional
\autor{Nome completo do autor}
\grau{...}
\local{Florianópolis} % Opcional (Florianópolis é o padrão)
\data{04}{junho}{2013}
\orientador[Orientador\\Universidade ...]{Prof. Dr.}
\coorientador[Coorientador\\Universidade ...]{Prof. Dr.}
\coordenador[Coordenador\\Universidade ...]{Prof. Dr. }

\capa
\folhaderosto[comficha] % Se nao quiser imprimir a ficha, é só não usar o parâmetro

\numerodemembrosnabanca{3}
\bancaMembroA{
    Profa. Dra. Karina Girardi Roggia \\
    Universidade do Estado de Santa Catarina
}
\bancaMembroB{
    Prof. Dr. Ricardo Azambuja Silveira \\
    Universidade Federal de Santa Catarina
}
\bancaMembroC{
    Prof. Dr. Rosvelter João Coelho da Costa \\
    Universidade Federal de Santa Catarina
}

\folhaaprovacao

%% Dedicatória

\dedicatoria{
    À minha mãe, Venida
}

\paginadedicatoria

%% Agradecimentos

\agradecimento{
    Agradeço muito minha família, em especial à minha mãe,
    por sempre terem me apoiado, guiado e auxiliado
    ao longo dos últimos 20 anos.

    Agradeço minha orientadora, Jerusa,
    que me acolheu logo no início do curso,
    por todas as conversas e orientações,
    que muitas vezes extrapolaram os limites da Academia.

    Agradeço aos professores do Programa Avançado de Matemática
    (Melissa, Gilles, Fernando),
    que me ajudaram a construir a maturidade matemática
    que me foi extremamente útil nos quatro anos de graduação.

    Agradeço as discussões com Santana, Schultz, Gabriel e Abouhatem
    sobre os algoritmos da Maratona de Programação.

    Agradeço todas as discussões produtivas (e improdutivas)
    com o pessoal do laboratório IATE.

    E, por último, agradeço o companheirismo dos meus amigos
    ao longo desta jornada.
}

\paginaagradecimento

% Epígrafe

\epigrafe{
    Even if one works basically all week trying to prove $\P \neq \NP$,
    one should set aside Friday afternoon for trying to prove $\P = \NP$.
}
{
    (John Edward Hopcroft)
}
\paginaepigrafe

% Resumo

\textoResumo{
    O texto do resumo deve ser digitado,
    em um único bloco,
    sem espaço de parágrafo.
    O resumo deve ser significativo,
    composto de uma sequência de frases concisas,
    afirmativas e não de uma enumeração de tópicos.
    Não deve conter citações.
    Deve usar o verbo na voz passiva.
    Abaixo do resumo,
    deve-se informar as palavras-chave
    (palavras ou expressões significativas retiradas do texto)
    ou, termos retirados de thesaurus da área.
}
\palavrasChave{
    Palavra-chave 1.
    Palavra-chave 2.
    Palavra-chave 3.
}

\paginaresumo

%% Resumo, em inglês.

\textAbstract{
    Resumo traduzido para outros idiomas,
    neste caso,
    inglês.
    Segue o formato do resumo
    feito na língua vernácula.
    As palavras-chave traduzidas,
    versão em língua estrangeira,
    são colocadas abaixo do texto
    precedidas pela expressão ``Keywords'',
    separadas por ponto.
}
\keywords{
    Keyword 1.
    Keyword 2.
    Keyword 3.
}

\paginaabstract

% Sumário e listas usadas no documento.
% As listas dependem da necessidade do usuário

\listadefiguras
\listadetabelas
\listadeabreviaturas
\listadesimbolos
\sumario

\sumario

\emph{Complexidade} é a quantidade de recursos
que uma máquina de Turing gasta
para computar determinada função
ou para decidir pertinência a uma linguagem
\cite[p.~285]{HopcroftUllman1979}.
Os recursos mais importantes para a teoria de complexidade computacional
são o espaço e o tempo,
em máquinas de Turing determinísticas e não"=determinísticas.

Os axiomas de Blum definem a noção de complexidade computacional
não apenas para máquinas de Turing,
mas sim para qualquer modelo de computação que satisfaça algumas restrições.

A seção~\ref{sec:enumeration_of_recursive_functions}
contém a maquinaria matemática necessária para expressar estas restrições.
Os axiomas de Blum aparecem na seção~\ref{sec:blum_axioms}.
Por fim, a seção~\ref{sec:default_measures}
define as medidas de complexidade padrão.

\section{Enumeração de Funções Recursivas}
\label{sec:definition_enumeration_of_recursive_functions}

A existência de uma codificação em binário de máquinas de Turing
permite, de certa forma,
enumerar todas as funções recursivas parciais
--- basta listar todos os códigos possíveis para máquinas de Turing.
Estritamente falando,
não é as funções que estão sendo listadas
(isso seria impossível pois funções são objetos matemáticos infinitos),
mas sim códigos em binário para máquinas que as computam.

A construção da codificação das máquinas de Turing não é ``injetora'';
isto é, existem máquinas $M$ e $N$ diferentes
tais que
\begin{equation*}
    \langle M \rangle = \langle N \rangle.
\end{equation*}
Entretanto,
a construção foi feita de forma que
as funções computadas por $M$ e $N$ coincidam
sempre que a equação acima for satisfeita.
Portanto,
dada uma cadeia $x \in \{0, 1\}^*$,
existe uma única função recursiva parcial
que é computada pelas máquinas cuja codificação em binário é $x$.
Chamaremos esta função de $f_x$.
Observe que o mapeamento $x \mapsto f_x$
enumera todas as funções recursivas;
as seções \ref{sec:universal_turing_machine} e~\ref{sec:s_m_n_theorem}
contém teoremas que nos permitem trabalhar com as funções $f_x$
e a seção~\ref{sec:acceptable_godel_numbering}
define o conceito de ``Enumeração de Gödel aceitável'',
que generaliza este mapeamento.
\simbolo{$f_x$}{Função computada pela máquina codificada por $x$}

\section{Complexidade Funcional}

De posse do formalismo de funções não"=determinísticas,
finalmente,
concluímos a definição $\PhiNT$ e $\PhiNS$ para funções.
$\FDTIME$, $\FDSPACE$, $\FNTIME$ e $\FNSPACE$
são definidos para funções de maneira análoga às definições para decisores.
Por simetria, também definiremos $\coFNTIME$ e $\coFNSPACE$;
aqui, ao invés de tomarmos o máximo dentre os ramos,
pegaremos o mínimo para ser o valor da função.
(São as funções ``co"=não"=determinísticas''.)
Nos concentraremos nas classes de complexidade de tempo,
onde o uso de não"=determinismo aparenta ter mais impacto.

\begin{definition}
    \begin{align*}
        \FP &= \bigcup_{k > 0} \FDTIME(n^k) \\
        \FNP &= \bigcup_{k > 0} \FNTIME(n^k) \\
        \coFNP &= \bigcup_{k > 0} \coFNTIME(n^k)
    \end{align*}
    \simbolo{$\FP$}{Funções determinísticas polinomiais}
    \simbolo{$\FNP$}{Funções não"=determinísticas polinomiais}
    \simbolo{$\coFNP$}{Funções co"=não"=determinísticas polinomiais}
\end{definition}

$\FP$ são as funções que podem ser calculadas
em tempo polinomial por máquinas determinísticas,
e $\FNP$, por máquinas não"=determinísticas.\footnotemark
$\coFNP$ é o análogo a $\coNP$.
\footnotetext{
    Nossa definição de $\FNP$ diverge da definição da literatura.
    A definição de \citeonline[p.~229]{Papadimitriou1994},
    por exemplo,
    é baseada em linguagens polinomialmente equilibradas
    (mencionadas na nota de rodapé~\ref{foot:polinomially_balanced},
    na página~\pageref{foot:polinomially_balanced}).

    Tome um problema de decisão $L \in \NP$
    e construa uma linguagem polinomialmente equilibrada $R_L$ associada a $L$.
    $R_L$ será um conjunto de pares $(x, y)$,
    em que $x \in L$ e $y$ é um ``certificado'' de que $x \in L$.

    O \emph{problema funcional associado} a $L$, $\class{F}L$,
    consiste em encontrar algum $y$ tal que $(x, y) \in R_L$,
    ou determinar que tal $y$ não existe;
    então,
    \citeonline[p.~229]{Papadimitriou1994} define $\FNP$
    como o conjunto de todos os $\class{F}L$
    para $L \in \NP$.

    Estes problemas são por vezes chamados de \emph{problemas de busca}
    (do inglês \emph{search problem}),
    em oposição a \emph{problemas de decisão}.

    A diferença crucial é que \citeauthoronline{Papadimitriou1994}
    pede apenas que encontremos algum $y$,
    enquanto que nossa definição exige que encontremos o maior deles.
}

\begin{example}
    Todos os algoritmos polinomiais
    discutidos em cursos de projeto e análise de algoritmos
    correspondem a funções em $\FP$.
    Por exemplo,
    temos cálculo de determinantes,
    programação linear\footnote{
        Programação linear é resolvível em tempo polinomial
        quando o domínio do problema são os números racionais.
        Se restringirmos o domínio da solução a números inteiros,
        o problema é $\NP$"=completo.

        Programação linear nos inteiros geralmente é chamada de
        \emph{programação inteira}
        (do inglês \emph{integer programming}).
    }
    e parsing de linguagens livres de contexto.
\end{example}

\begin{uproposition}
    Todas as funções características de problemas em $\NP$ estão em $\FNP$,
    e todas as funções características de problemas em $\coNP$ estão em $\coFNP$.
\end{uproposition}

\begin{example}
    Os problemas em $\NP$ podem ser facilmente generalizados para funções em $\FNP$.
    Por exemplo,
    podemos pegar uma máquina não"=determinística para $\SAT$,
    e modificá"=la para que,
    após achar uma atribuição de valores"=verdade,
    a escreva na fita.
    Caso contrário,
    escreva a palavra vazia.
    A função não"=determinística computada por esta máquina
    é a maior atribuição de valores"=verdade,
    lexicograficamente.
    Esta função está em $\FNP$.

    Similarmente,
    podemos encontrar o maior clique num grafo,
    achar fatores de um número,
    e encontrar um isomorfismo entre dois grafos
    usando o poder computacional de $\FNP$.

    Já problemas de minimização
    são mais facilmente interpretados como funções de $\coFNP$.
    Por exemplo,
    encontrar o caminho hamiltoniano de menor custo,
    e calcular o número cromático.
\end{example}

O contrário também é possível;
isto é,
podemos transformar problemas de $\FNP$ em problemas de $\NP$.

\begin{theorem}
    Se $f$ é uma função de $\FNP$,
    então a linguagem
    \begin{equation*}
        L_f = \{ \langle x, y \rangle \mid f(x) \geq y \},
    \end{equation*}
    em que $\geq$ significa comparação lexicográfica,
    pertence a $\NP$.
    \label{thm:fnp_to_np_conversion}
\end{theorem}

\begin{proof}
    A ideia é fazer com que a máquina não"=determinística $M$,
    que reconhecerá $L_f$,
    compute $f$ usando seu próprio não"=determinismo,
    e aceitará a entrada caso ache algum ramo de computação
    que produza um valor maior ou igual a $y$.

    Mais formalmente,
    seja $N$ uma máquina não"=determinística que computa $f$ em tempo polinomial.
    Considere que a entrada é $\langle x, y \rangle$.
    A primeira coisa que $M$ fará é executar $N$ em $x$.
    Sabemos que isso é possível pois ambas as máquinas são não"=determinísticas.

    Após $N$ parar
    (o que ocorrerá em tempo polinomial),
    $N$ deixará na fita um ``candidato'' a $f(x)$.
    De fato,
    todas as folhas da computação de $N$ estarão disponíveis nos ramos de $M$.
    Como $N$ computa $f$,
    apenas o maior destes valores será $f(x)$;
    portanto,
    se algum ramo encontrar algum candidato que seja maior ou igual a $y$,
    a entrada $\langle x, y \rangle$ pode ser aceita.

    O contrário também ocorre.
    Se todos os ramos de computação gerarem candidatos a $f(x)$
    que são menores que $y$,
    o próprio valor de $f(x)$ será menor que $y$,
    portanto a entrada pode ser rejeitada.

    Em outras palavras,
    a máquina $M$ deve aceitar exatamente quando encontrar
    alguma folha da computação de $N(x)$ que é maior que $y$.
\end{proof}

De certa forma,
este teorema nos permite fazer uma ``busca binária''
atrás do valor de $f(x)$
usando apenas algum dispositivo que reconheça $L_f$.
Voltaremos a esta ideia na seção~\ref{sec:functional_oracles}.

\chapter{Oráculos}

Oráculos computacionais são um mecanismo bastante poderoso
para definir novas classes de complexidade computacional.
Com ele,
podemos formalizar perguntas como
"Que linguagens conseguiríamos reconhecer
se desse para resolver $\SAT$ gratuitamente?",
ou "E se o problema da parada fosse decidível?".

Na seção~\ref{sec:oracle_definition} será apresentada a definição formal
de ``oráculo computacional''.
A seção~\ref{sec:undecidable_problems_as_oracles}
discute a interpretação de utilizar um problema indecidível
(como o problema da parada)
como oráculo.
Ainda no contexto da indecidibilidade,
a seção~\ref{sec:oracle_equivalence} define o que significa
dois oráculos serem equivalentes,
e a seção~\ref{sec:undecidable_hierarchy}
constroi uma hierarquia de problemas indecidíveis.
Por fim,
a seção~\ref{sec:oracles_and_reductions}
interpreta o conceito de redução em termos de oráculos.

\section{Definição}
\label{sec:oracle_definition}

\simbolo{$M^A$}{Máquina de Turing $M$ com oráculo $A$}
\simbolo{$L^A(M)$}{Linguagem da máquina $M$, quando equipada com o oráculo $A$}
\begin{definition}
    Seja $A$ uma linguagem qualquer.
    Uma \emph{máquina de Turing com oráculo $A$}
    é uma máquina de Turing $M^A$ que possui uma fita especial
    e três estados adicionais: $q_?$, $q_y$ e $q_n$.
    Transitar para $q_?$ significa consultar o oráculo;
    ao fazer esta transição,
    caso a palavra nesta fita pertença à linguagem $A$,
    no próximo estado da computação $M$ transitará para $q_y$
    (a resposta foi positiva);
    caso contrário, $M$ transitará para $q_n$.%
    \footnote{
        Observe que a cabeça da fita não se mexe durante a consulta.
        Portanto, a máquina pode escrever seu estado atual na fita
        antes de transitar para $q_?$ e recuperá"=lo depois.
    }

    A definição de aceitação não é alterada.
    Chamaremos de $L^A(M)$ o conjunto das palavras aceitas por $M^A$.
\end{definition}

Intuitivamente, o oráculo é um dispositivo computacional
acoplado à máquina de Turing $M$.
É como se a máquina delegasse parte da computação
a outra máquina de Turing;
uma ``chamada de função''.

Observe que a única influência que $A$ possui em $M^A$
são as transições após $M$ ir para o estado $q_?$.
Ou seja, $A$ não faz parte de $M$;
de fato, podemos ``acoplar'' várias linguagens diferentes
numa mesma máquina de Turing $M$
e obter diferentes $L^A(M)$ com isso.%
\footnote{
    É exatamente por causa disso que,
    na notação do conjunto das palavras aceitas por $M^A$,
    o $A$ sobrescrito está junto de $L$, não de $M$.
}

Como a linguagem $A$ não faz parte da máquina $M^A$,
do ponto de vista formal,
$M^A$ continua sendo, essencialmente,
uma tabela de transições;
portanto,
podemos estender de maneira natural a codificação $\langle \cdot \rangle$
para máquinas com oráculos.
Da mesma forma, conceitos como $M^A(x)$ e função associada
podem ser definidos de maneira análoga.

Assim,
podemos definir uma enumeração de Gödel aceitável usando oráculos.

\begin{definition}
    \simbolo{$\phi^A$}{Enumeração das funções recursivas em $A$}
    Seja $A$ uma linguagem.
    A enumeração de funções recursivas em $A$,
    denotada por $\phi^A$,
    é o mapeamento
    \begin{equation*}
        \langle M^A \rangle \mapsto \phi^A_{\langle M \rangle},
    \end{equation*}
    em que $\phi^A_{\langle M \rangle}$
    é a função definida por
    \begin{equation*}
        \phi^A_{\langle M \rangle}(x) \simeq M^A(x).
    \end{equation*}
\end{definition}

Nem sempre $\phi^A$ será uma enumeração de Gödel aceitável.
O próximo teorema delimita as situações em que isso não acontece.

\begin{theorem}
    $\phi^A$ é uma enumeração de Gödel aceitável se,
    e somente se,
    $A$ for uma linguagem decidível.
\end{theorem}

\begin{proof}
    Suponha que $A$ não seja decidível.
    Defina a função $f: \{0, 1\}^* \to \{0, 1\}^*$ por
    \begin{equation*}
        f(x) =
        \begin{cases}
            1,& \text{ se } x \in A; \\
            0,& \text{ se } x \notin A.
        \end{cases}
    \end{equation*}
    Observe que $f$ não é uma função recursiva
    (pois $A$ não é decidível),
    mas ela pode ser computada facilmente usando $A$ como oráculo.
    Portanto, $\phi^A$ enumera uma função que não é recursiva,
    fazendo com que não seja uma enumeração de Gödel.

    Agora, suponha que $A$ seja decidível.
    Primeiro,
    observe que qualquer máquina de Turing sem oráculos $M$
    pode ser transformada numa máquina de Turing com oráculo $M^A$,
    que simplesmente ignora seu próprio oráculo.
    Portanto,
    todas as funções recursivas parciais são enumeradas por $\phi^A$.

    Agora,
    como $A$ é uma linguagem decidível,
    existe uma máquina de Turing que sempre para que decide pertinência a $A$.
    Dada uma máquina $M^A$,
    podemos transformá"=la numa máquina sem oráculo $M$
    equivalente --- isto é,
    \begin{equation*}
        M^A(x) \simeq M(x);
    \end{equation*}
    basta trocar as chamadas ao oráculo $A$
    pelo algoritmo que a decide.
    Portanto, $\phi^A$ só enumera funções recursivas parciais.

    Concluímos que $\phi^A$ é uma função sobrejetora
    cuja imagem é o conjunto das funções recursivas parciais.
    A construção da máquina universal
    e do teorema $S_{mn}$ é análoga às máquinas sem oráculos.%
    \footnote{
        Observe que construir a máquina universal
        e o teorema $S_{mn}$ pode ser feito mesmo que $A$ não seja decidível;
        o problema é, justamente,
        a possibilidade de enumerar funções que não são recursivas.
    }
\end{proof}

\section{Equivalência entre oráculos}
\label{sec:oracle_equivalence}

Na subseção anterior,
mostramos que $L_u^1$ é decidível usando um oráculo para $S_1$
(isto é, $L_u^1$ é \emph{recursiva em} $S_1$).
O oposto também ocorre:

Suponha que dispomos de um oráculo para $L_u^1$.
Na entrada $\langle M \rangle$,
construa uma máquina $M'$ que ignorará sua própria entrada
e enumerar os pares $(i, j)$.
Para cada par enumerado,
$M'$ executará $M$ na $i$"=ésima palavra por $j$ movimentos.
Caso $M$ aceite, $M'$ aceita a entrada, qualquer que seja.
Caso contrário, $M'$ enumera o próximo par, e tente de novo.
De posse desta máquina, peça ao oráculo se ela aceita $\varepsilon$
(ou qualquer palavra).
Se sim, significa que, em algum par $(i, j)$,
a $M$ aceitou a entrada $i$; portanto, $L(M) \neq \emptyset$.
Caso contrário, significa que $M$ nunca aceitou palavra alguma;
portanto, $L(M) = \emptyset$.

\begin{definition}
    Dois oráculos $A$ e $B$ são ditos equivalentes
    se existem máquinas $M$ e $N$ tais que
    $L^A(M) = B$ e $L^B(N) = A$.
\end{definition}

Isto é, usando uma linguagem como oráculo, conseguimos decidir a outra,
e vice"=versa.
O que nós mostramos no parágrafo anterior é que
$L_u^1$ e $S_1$ são equivalentes.

A intuição por trás da equivalência é que,
ao ``programar'' uma máquina de Turing com oráculo $S_1$ (por exemplo),
podemos fingir que também podemos usar $L_u^1$ como oráculo,
e fazer uma ``chamada'' a este oráculo;
como eles são equivalentes,
há um algoritmo que traduz de um para outro,
portanto, podemos usar este algoritmo de tradução
para simular uma chamada a $L_u^1$.

\section{Hierarquia de problemas indecidíveis}
\label{sec:undecidable_hierarchy}

Muitos outros problemas podem ser demonstrados equivalentes
(como oráculos)
ao problema da parada,
como o problema da correspondência de Post \cite[p.~214]{HopcroftUllman1979}.
Entretanto,
nem todos os problemas podem ser resolvidos com oráculos para a parada.
De fato, a mesma técnica que mostra que a parada é indecidível
pode ser usado para gerar um problema indecidível
para máquinas que usam o problema da parada como oráculo.

Defina
\simbolo{$L_u^n$}{Problema da parada, generalizado via oráculos}
\simbolo{$S_n$}{Problema da vacuidade, generalizado via oráculos}
\begin{align*}
    L_u^{n+1} &= \{ \langle M^{L_u^n}, x \rangle \mid M^{L_u^n} \text{ aceita } x \}, \\
    S_{n+1} &= \{ \langle M^{S_n} \rangle \mid L^{S_n}(M) = \emptyset \}.
\end{align*}
$L_u^{n+1}$ é o problema da parada
para máquinas de Turing que usam $L_u^n$ como oráculo;
$S_{n+1}$ é o problema da vacuidade
para máquinas de Turing que usam $S_n$ como oráculo.%
\footnote{
    Observe que,
    se definirmos $L_u^0 = S_0 = \emptyset$
    (ou algum conjunto recursivo qualquer),
    a fórmula é válida também para $n = 1$.
}
Podemos demonstrar por indução que $S_n$ é equivalente a $L_u^n$;
e, imitando o teorema da parada,
podemos mostrar que $L_u^n$ é indecidível para máquinas que usam $L_u^n$ como oráculo.

Defina
\simbolo{$\R_n$}{Linguagens decidíveis usando um oráculo para $L_u^n$}
\begin{equation*}
    \R_n = \{ L \mid L = L^{L_u^n}(M) \text{ para alguma máquina $M^{L_u^n}$ } \}.
\end{equation*}
Ou seja, $\R_n$ é o conjunto das linguagens
que são decidíveis usando $L_u^n$ como oráculo.%
\footnote{
    Se usarmos a convenção de que $L_u^0 = \emptyset$,
    o conjunto $\R_0$ é exatamente o conjunto das linguagens recursivas
    (geralmente denotado por \R).
}
O parágrafo anterior mostra que $L_u^{n+1}, S_{n+1} \notin \R_n$.

Podemos demonstrar (por indução) que $\R_n \subset \R_{n+1}$.
Desta forma,
criamos uma hierarquia de problemas indecidíveis.%
\footnote{
    Em outras palavras,
    existem problemas ``mais indecidíveis'' que outros;
    é uma situação análoga à existência de infinitos de tamanhos diferentes.
}

Estes conjuntos também podem ser denotados por $\R^{L_u^n}$;
as ``linguagens recursivas que usam $L_u^n$ como oráculo''.
São as linguagens que são decididas
pelo mesmo tipo de máquina usado para definir $\R$,
mas agora equipadas com um oráculo para $L_u^n$.

Observe que,
de posse de um oráculo para $S_n$ (por exemplo),
ao ``programar'' uma máquina de Turing com esse oráculo,
podemos ``fingir'' que podemos usar qualquer linguagem de $\R_n$ como oráculo
--- como esta linguagem possui um algoritmo que a resolve
usando o nosso oráculo $S_n$,
podemos substituir todas as chamadas àquele oráculo
por este algoritmo.
Abusando um pouco da notação,
podemos escrever
\begin{equation}
    \begin{aligned}
        \R_0 &= \R \\
        \R_{n+1} &= \R^{\R_n}.
    \end{aligned}
    \label{eq:oracle_notation_abuse}
\end{equation}
A interpretação é que $\R_{n+1}$ é o conjunto das linguagens
decididas pelas mesmas máquinas de $\R$,
mas equipadas com um oráculo qualquer de $\R_n$.
Como $S_n \in \R_n$, esta nova definição inclui a definição anterior;
Como todas as linguagens de $\R_n$ podem ser decididas usando um oráculo para $S_n$,
qualquer linguagem de $\R_{n+1}$
pode ser decidida usando apenas $S_n$
(basta fazer aquela troca de chamadas ao oráculo);
portanto as duas definições são equivalentes.
Usaremos esta notação novamente na seção~\ref{sec:polynomial_hierarchy},
mas indo direto à esta construção alternativa,
sem passar pela definição via $S_n$ primeiro.

\section{REDUÇÕES E ORÁCULOS}
\label{sec:oracles_and_reductions}

\begin{definition}[$\NP$"=completude]
    Uma linguagem $L$ é $\NP$"=completa se
    \begin{itemize}
        \item $L \in \NP$; e
        \item Para toda linguagem $L' \in \NP$ existe alguma função $f$,
            computável em tempo polinomial,
            tal que
            \begin{equation*}
                x \in L' \iff f(x) \in L.
            \end{equation*}
    \end{itemize}
\end{definition}

Esta é a definição usual de $\NP$"=completude,
encontrada, por exemplo,
nos livros didáticos de \citeonline[p.~288]{Sipser2006}
e \citeonline[p.~42]{AroraBarak2009}.
\citeonline[p.~324]{HopcroftUllman1979} e \citeonline[p.~174]{Papadimitriou1994}
apresentam uma definição um pouco mais restritiva:
a função $f$ precisa ser computada em espaço logarítmico
(o que implica tempo polinomial).

O termo ``$\NP$"=completude'' tenta formalizar a noção de
``ser um problema difícil da classe $\NP$''.
Com a definição acima,
podemos provar que,
se acharmos algum algoritmo polinomial para um problema $\NP$"=completo,
podemos resolver \emph{todos} os problemas da classe $\NP$ em tempo polinomial.%
\footnote{
    Na vida real,
    após provar que certo problema é $\NP$"=completo,
    nós desistimos de achar algoritmos polinomiais para ele
    e tentamos desenvolver heurísticas
    (como, por exemplo, limitar o espaço de busca ou programação dinâmica),
    usar técnicas baseadas em inteligência artificial,
    mapear para problemas NP"=completos mais bem estudados
    (como $\SAT$ e programação linear)
    ou nos contentar com soluções aproximadas.
}
Isto é,
resolver um significa resolver todos;
portanto,
intuitivamente,
estes problemas devem ser os ``mais difíceis'' da classe $\NP$.
Por exemplo,
deve ser mais fácil achar um algoritmo polinomial para fatoração de inteiros,
que ainda não foi nem provado estar em $\P$
nem ser $\NP$"=completo \cite[p.~120]{DuKo2014},
do que achar um algoritmo polinomial para $\SAT$,
o primeiro problema natural provado $\NP$"=completo \cite[p.~80]{DuKo2014}.

Podemos interpretar a redução de forma algorítmica,
usando o conceito de oráculo.
Usaremos como exemplo os problemas $\SAT$ e $3\SAT$.

Sabemos que podemos reduzir $\SAT$ para $3\SAT$.
Dada uma instância $x$ de $\SAT$,
execute o algoritmo de redução para obter uma instância $y$ de $3\SAT$.
$x$ é satisfazível se, e somente se, $y$ o for.
Portanto, executar um algoritmo para $\SAT$ em $x$
retornará a mesma resposta
que executar um algoritmo para $3\SAT$ em $y$.
Ou seja, com um oráculo para $3\SAT$,
podemos resolver $\SAT$ em tempo polinomial:
o algoritmo de redução opera em tempo polinomial
e a chamada ao oráculo gasta apenas uma transição
(de $q_?$ para $q_y$ ou $q_n$).
Esta interpretação está sumarizada pelo algoritmo~\ref{algo:reduction}.

\begin{algorithm}[h]
    Leia a entrada $x$\;
    Execute o algoritmo de redução $f$ em $x$ para obter $y$\;
    Retorne \texttt{Oráculo}($y$)\;
    \caption{
        Interpretação algorítmica da noção de redução.
    }
    \label{algo:reduction}
\end{algorithm}

Esta interpretação via oráculos mostra que
a noção de redução possui uma restrição bastante forte:
nós podemos chamar o oráculo apenas uma vez,
e esta chamada precisa ser a última coisa que o algoritmo faz
--- não podemos nem mesmo alterar o valor retornado pelo oráculo.%
\footnote{
    Esta descrição é semelhante às restrições impostas pela recursão caudal
    (\emph{tail recursion}).
}

Embora esta restrição garanta que
um algoritmo polinomial para o oráculo
permita construir um algoritmo polinomial para a outra linguagem,
ela não é necessária.
Caso a função \texttt{Oráculo}, do algoritmo~\ref{algo:reduction},
seja implementada em tempo polinomial,
podemos fazer quantas chamadas quisermos no algoritmo externo
e alterar os valores retornados pelas chamadas à vontade;
se o algoritmo externo em si
(ignorando as chamadas ao oráculo)
for polinomial,
o algoritmo resultante também terá complexidade polinomial.%
\footnote{
    Este conceito também é válido para as hierarquias indecidíveis.
    Se $A$ é recursiva em $B$
    (isto é, com um oráculo para $B$, conseguimos decidir $A$),
    então, caso $B$ seja decidível,
    $A$ também o será,
    mesmo que não tenhamos uma redução de $A$ para $B$.
}

Isto sugere que reduções impõem restrições arbitrárias
e podem ser substituídas pelo uso de oráculos.%
\footnote{
    Alguns autores (como \citeonline[p.~42, p.~65]{AroraBarak2009})
    distinguem várias noções diferentes de ``redução''.
    A redução apresentada na definição de $\NP$"=completude
    é chamada de \emph{redução de Karp},
    \emph{redução por mapeamento} ou \emph{redução muitos"=para"=um}
    (dependendo do autor).
    Já o uso de oráculos, por sua vez, constitui noutro tipo de redução:
    uma \emph{redução de Cook} de $A$ para $B$
    é uma máquina $M^B$ que decide $A$ em tempo polinomial
    (usando $B$ como oráculo).
}

\section{HIERARQUIA POLINOMIAL}
\label{sec:polynomial_hierarchy}

A hierarquia polinomial é uma generalização da classe $\NP$,
baseada no conceito de oráculo.
Para defini"=la,
utilizaremos o abuso de notação mencionado na equação~\ref{eq:oracle_notation_abuse}.
Formalmente:

\simbolo{$\P^\mathcal A$}{Classe $\P$ relativa à $\mathcal A$}
\simbolo{$\NP^\mathcal A$}{Classe $\NP$ relativa à $\mathcal A$}
\simbolo{$\coNP^\mathcal A$}{Classe $\coNP$ relativa à $\mathcal A$}
\begin{definition}
    Seja $\mathcal A$ uma classe computacional.
    Então,
    \begin{itemize}
        \item $\P^\mathcal A$
            é a classe dos problemas que podem ser resolvidos em tempo polinomial
            por uma máquina de Turing determinística,
            usando como oráculo alguma linguagem de $\mathcal A$;
        \item $\NP^\mathcal A$
            é a classe dos problemas que podem ser resolvidos em tempo polinomial
            por uma máquina de Turing não"=determinística,
            usando como oráculo alguma linguagem de $\mathcal A$;
            e
        \item $\coNP^\mathcal A$
            é a classe dos complementos dos problemas em $\NP^\mathcal A$.
    \end{itemize}
\end{definition}

(A única diferença entre $\P^\mathcal A$ e $\NP^\mathcal A$
é o fato de permitirmos não"=determinismo em $\NP^\mathcal A$.)

Intuitivamente,
$\P^\mathcal A$ é a classe de problemas que podem ser resolvidos
se pegarmos as máquinas que resolvem os problemas em $\P$
e darmos oráculos em $\mathcal A$ para elas.%
\footnote{
    Note que, caso $\mathcal A$ possua um problema polinomialmente completo $B$
    (isto é, todos os problemas de $\mathcal A$ se reduzem
    em tempo polinomial a algum problema $B \in \mathcal A$ em questão),
    a classe $\P^\mathcal A$
    é a mesma classe que $\P^B$,
    pois qualquer problema de $\mathcal A$
    pode ser resolvido em tempo polinomial usando $B$ como oráculo;
    basta substituir a chamada ao oráculo de $\mathcal A$
    pelo algoritmo que o resolve usando $B$.
    Embora esta troca possa aumentar o tempo de computação,
    a máquina ainda termina de executar em tempo polinomial.
    Isso justifica a notação $\P^\mathcal A$;
    é como se tivéssemos todos os problemas de $\mathcal A$ à nossa disposição.
}

É importante ressaltar que esta notação
não está associada semanticamente ao conceito de potência;
de fato,
existem oráculos $A$ e $B$ para os quais
$\P ^ A = \NP ^ A$ e $\P ^ B \neq \NP ^ B$
\cite[p.~362]{HopcroftUllman1979},%
\footnote{
    De fato, \citeonline[p.~362]{HopcroftUllman1979}
    argumentam que este é um dos motivos pelos quais
    a questão $\P = \NP$ é tão difícil de ser resolvida
    --- os métodos que conhecemos
    (como, por exemplo, diagonalização)
    são facilmente traduzíveis para máquinas com oráculos.
    Portanto,
    uma prova de que $\P \neq \NP$ (por exemplo)
    precisaria empregar um método que deixaria de funcionar
    ao equipar as máquinas com o oráculo $A$ citado acima.
}
independente de $\P$ ser igual a $\NP$ ou não.

Valendo"=se desta notação,
podemos definir a hierarquia polinomial de maneira muito enxuta.

\simbolo{$\Delta_n^p$}{Generalização de $\P$ na hierarquia polinomial}
\simbolo{$\Sigma_n^p$}{Generalização de $\NP$ na hierarquia polinomial}
\simbolo{$\Pi_n^p$}{Generalização de $\coNP$ na hierarquia polinomial}
\begin{definition}[Hierarquia polinomial\footnotemark]
    \begin{align*}
        \Delta_0^p &= \Sigma_0^p = \Pi_0^p = \P \\
        \Delta_{n+1}^p &= \P^{\Sigma_n^p} \\
        \Sigma_{n+1}^p &= \NP^{\Sigma_n^p} \\
        \Pi_{n+1}^p &= \coNP^{\Sigma_n^p} \\
        \PH &= \bigcup_{n >= 0} \Sigma_n^p
    \end{align*}
\end{definition}
\footnotetext{
    Através de problemas completos para os níveis anteriores,
    \citeonline[p.~102]{AroraBarak2009}
    mostram uma caracterização da hierarquia polinomial
    usando apenas um oráculo por nível.
    Por exemplo, $\Sigma_n^p$ é equivalente a $\NP^{\Sigma_{n-1}\SAT}$,
    em que $\Sigma_{n-1}\SAT$ é um problema completo para $\Sigma_{n-1}^p$.
}

Os primeiros elementos da hierarquia polinomial podem ser expressados diretamente.

Por exemplo, $\Sigma_1^p = \NP^\P$.
Esta classe corresponde às máquinas de Turing não"=determinísticas
com acesso a oráculos em $\P$ que operam em tempo polinomial.
Mas o próprio oráculo pode ser computado em tempo polinomial;
como há uma quantidade polinomial de chamadas a este oráculo,
ao substituí"=lo por um algoritmo ``de verdade'',
a máquina resultante ainda terá complexidade polinomial.
Portanto, todo problema de $\Sigma_1^p$ está em $\NP$.
E vice"=versa: qualquer problema de $\NP$ está em $\NP^\P$
--- basta a máquina não usar seu oráculo.
Portanto, $\Sigma_1^p = \NP$.

O mesmo raciocínio mostra que $\Delta_1^p = \P$
e que $\Pi_1^p = \coNP$.%
\footnote{
    Poderíamos ter escolhido qualquer subconjunto de $\P$
    como início da hierarquia polinomial,
    que o restante seria o mesmo.
    De fato, \citeonline[p.~128]{MeyerStockmeyer1972}
    introduziram a hierarquia polinomial ao mundo
    escolhendo $\Delta_0^p = \Sigma_0^p = \Pi_0^p = \emptyset$.
}

Já $\Delta_2^p$ representa a ideia de generalizar reduções.%
\footnote{
    Esta é uma interpretação minha da hierarquia polinomial.
    Os criadores originais da hierarquia,
    \citeonline[p.~128]{MeyerStockmeyer1972},
    tinham por objetivo achar problemas ``naturais''
    que fossem ``difíceis''.
    As separações entre as classes de complexidade,
    como, por exemplo, as demonstrações por \citeonline[p.~299]{HopcroftUllman1979}
    são baseadas em diagonalização,
    e resultam em linguagens artificiais.
    O que \citeonline[p.~129]{MeyerStockmeyer1972} queriam
    era achar problemas que surgem ``naturalmente''
    ao analisar problemas computacionais,
    que possuíssem as mesmas características
    destes resultantes de diagonalizações
    (isto é, que separassem classes de complexidade
    --- fossem ``difíceis'').
}
Ao contrário de se comportar da maneira limitada,
como no algoritmo~\ref{algo:reduction},
as máquinas polinomiais de $\Delta_2^p$
podem chamar seu oráculo uma quantidade polinomial de vezes,
alterando o valor retornado arbitrariamente.
Com isso, $\Delta_2^p$ engloba tanto $\NP$ quanto $\coNP$
--- acredita"=se que a inclusão seja própria.

As inclusões dentro da hierarquia polinomial podem ser sumarizadas por
\begin{align*}
    \Pi_n^p \cup \Sigma_n^p & \subseteq \Delta_{n+1}^p, \\
    \Delta_{n+1}^p & \subseteq \Sigma_{n+1}^p, \\
    \Delta_{n+1}^p & \subseteq \Pi_{n+1}^p.
\end{align*}
Observe que a estrutura de inclusões da hierarquia polinomial
é muito parecida com a estrutura de inclusões da hierarquia aritmética
(o que justifica o uso dos mesmos símbolos para ambas as hierarquias),
mas, no caso da hierarquia polinomial,
não conseguimos demonstrar que as inclusões são estritas.
(De fato, uma delas, $\Delta_1^p \subseteq \Sigma_1^p$,
é exatamente o problema $\P$ vs $\NP$.)
Esta estrutura está esquematizada na figura~\ref{fig:polynomial_hierarchy}.

\begin{figure}
    \centering
    \begin{tikzpicture}
        \draw let \n1 = {0}, \n2 = {8}, \n3 = {1.5}, \n4 = {0.5} in
            (\n1, -\n4) -- (\n2, -\n4)
            (\n1, 0) -- (\n2, \n3)
            (\n1, \n3) -- (\n2, 0)
            (\n1, \n3) -- (\n2, 2*\n3)
            (\n1, 2*\n3) -- (\n2, \n3)
            (\n1, 2*\n3) -- (\n2, 3*\n3)
            (\n1, 3*\n3) -- (\n2, 2*\n3)
            (\n1, 4*\n3) -- (\n2, 4*\n3)
            (\n1, -\n4) -- (\n1, 4*\n3)
            (\n2, -\n4) -- (\n2, 4*\n3)
            node (a) at ({(\n1+\n2)/2}, 0) {
                $\Sigma_0^p = \Delta_0^p = \Pi_0^p = \P$
            }
            node (a) at (\n1+2*\n4, \n3/2) {$\Sigma_1^p = \NP$}
            node (a) at (\n2-2*\n4, \n3/2) {$\Pi_1^p = \coNP$}
            node (a) at ({\n1+\n2/2}, \n3) {$\Delta_2^p = \P^\NP$}
            node (a) at (\n1+\n4, 3*\n3/2) {$\Sigma_2^p$}
            node (a) at (\n2-\n4, 3*\n3/2) {$\Pi_2^p$}
            node (a) at ({\n1+\n2/2}, 2*\n3) {$\Delta_3^p$}
            node (a) at (\n1+\n4, 5*\n3/2) {$\Sigma_3^p$}
            node (a) at (\n2-\n4, 5*\n3/2) {$\Pi_3^p$}
            node (a) at ({\n1+\n2/2}, 3*\n3) {$\vdots$}
            node (a) at ({\n1+\n2/2}, 7*\n3/2) {$\PH \subseteq \PSPACE$};
    \end{tikzpicture}
    \caption[Estrutura de inclusões da hierarquia polinomial.]{
        Estrutura de inclusões da hierarquia polinomial.

        Um conjunto estar listado abaixo do outro
        (mesmo que na diagonal) denota que o conjunto de baixo
        é um subconjunto do conjunto de cima.
    }
    \label{fig:polynomial_hierarchy}
\end{figure}


\chapter{Funções não"=determinísticas}

% Classes de complexidade funcional
\newclass{\FDTIME}{FDTIME}
\newclass{\FNTIME}{FNTIME}
\newclass{\FDSPACE}{FDSPACE}
\newclass{\FNSPACE}{FNSPACE}
\newclass{\coFNTIME}{coFNTIME}
\newclass{\coFNSPACE}{coFNSPACE}
\newclass{\coFNP}{coFNP}

Na seção~\ref{axiomas_blum},
definimos dois axiomas que compreendem toda a informação necessária
para definir uma medida de complexidade abstrata.
As máquinas de Turing foram interpretadas como
representações de funções de inteiros
(isto é, funções cujo domínio e contradomínio são os números inteiros);
neste contexto,
a medidas de complexidade
se referiam a funções, não a linguagens.

As duas medidas mais importantes,
complexidade de tempo e espaço determinísticas,
foram definidos nos exemplos
\ref{complexidade_tempo} e~\ref{complexidade_espaco},
respectivamente.
O exemplo~\ref{complexidade_nao_deterministica},
porém,
ficou incompleto;
o que exatamente é uma ``função não"=determinística''?
O objetivo deste capítulo é prover uma definição para este termo
e interpretá"=lo em termos da hierarquia polinomial.

\section{Motivação}

No contexto de problemas de decisão,
embora não"=determinismo não acrescente poder computacional
--- isto é,
todo problema que uma máquina de Turing não"=determinística resolve
também pode ser resolvido usando máquinas determinísticas
---
não"=determinismo aparenta reduzir a complexidade de um problema.
Em particular,
máquinas determinísticas aparentam ser exponencialmente mais lentas
do que máquinas não"=determinísticas
para problemas como $\SAT$.

Entretanto,
problemas de decisão apenas retornam uma resposta booleana:
descobrimos se a instância em questão possui solução ou não.
Em problemas do mundo real,
geralmente queremos \emph{a solução},
não apenas saber se ela existe.
Por exemplo,
em vez de apenas descobrir que certa fórmula lógica não é uma tautologia,
queremos saber exatamente qual atribuição de valores"=verdade
torna a fórmula falsa
--- por exemplo,
esta fórmula pode ser gerada durante uma verificador formal de software,
e a atribuição de valores"=verdade que invalida a fórmula
codifica uma entrada para o programa
em que ele não faz o que deveria.
Conhecer esta instância é importante para corrigir o software sendo verificado.

Nós já medimos a complexidade de uma função
calculada por uma máquina de Turing determinística.
Entretanto,
para que possamos aferir a complexidade de uma função não"=determinística,
precisamos especificar como uma máquina não"=determinística pode computar uma função.

\section{Definição}

\begin{definition}[Função não"=determinística]
    \footnote{
        O termo ``função não"=determinística''
        está tecnicamente incorreto.
        Funções não são determinísticas ou não"=determinísticas;
        o que é determinístico ou não
        é o dispositivo computacional que a calcula.

        Toleraremos este abuso de nomenclatura
        para encurtar o texto.
    }
    Seja $M$
    uma máquina de Turing não"=determinística
    e $x$ uma palavra.
    Se a árvore de computação que $M$ gera ao processar $x$ for finita
    (isto é, nenhuma sequência de transições de $M$ leva a um loop infinito),
    definiremos $M(x)$ como sendo o maior valor que a fita atinge
    nas folhas da árvore.
    Caso a árvore não seja finita, deixaremos $M(x)$ indefinido.
\end{definition}

\section{Complexidade Funcional}

De posse do formalismo de funções não"=determinísticas,
finalmente,
concluímos a definição $\PhiNT$ e $\PhiNS$ para funções.
$\FDTIME$, $\FDSPACE$, $\FNTIME$ e $\FNSPACE$
são definidos para funções de maneira análoga às definições para decisores.
Por simetria, também definiremos $\coFNTIME$ e $\coFNSPACE$;
aqui, ao invés de tomarmos o máximo dentre os ramos,
pegaremos o mínimo para ser o valor da função.
(São as funções ``co"=não"=determinísticas''.)
Nos concentraremos nas classes de complexidade de tempo,
onde o uso de não"=determinismo aparenta ter mais impacto.

\begin{definition}
    \begin{align*}
        \FP &= \bigcup_{k > 0} \FDTIME(n^k) \\
        \FNP &= \bigcup_{k > 0} \FNTIME(n^k) \\
        \coFNP &= \bigcup_{k > 0} \coFNTIME(n^k)
    \end{align*}
    \simbolo{$\FP$}{Funções determinísticas polinomiais}
    \simbolo{$\FNP$}{Funções não"=determinísticas polinomiais}
    \simbolo{$\coFNP$}{Funções co"=não"=determinísticas polinomiais}
\end{definition}

$\FP$ são as funções que podem ser calculadas
em tempo polinomial por máquinas determinísticas,
e $\FNP$, por máquinas não"=determinísticas.\footnotemark
$\coFNP$ é o análogo a $\coNP$.
\footnotetext{
    Nossa definição de $\FNP$ diverge da definição da literatura.
    A definição de \citeonline[p.~229]{Papadimitriou1994},
    por exemplo,
    é baseada em linguagens polinomialmente equilibradas
    (mencionadas na nota de rodapé~\ref{foot:polinomially_balanced},
    na página~\pageref{foot:polinomially_balanced}).

    Tome um problema de decisão $L \in \NP$
    e construa uma linguagem polinomialmente equilibrada $R_L$ associada a $L$.
    $R_L$ será um conjunto de pares $(x, y)$,
    em que $x \in L$ e $y$ é um ``certificado'' de que $x \in L$.

    O \emph{problema funcional associado} a $L$, $\class{F}L$,
    consiste em encontrar algum $y$ tal que $(x, y) \in R_L$,
    ou determinar que tal $y$ não existe;
    então,
    \citeonline[p.~229]{Papadimitriou1994} define $\FNP$
    como o conjunto de todos os $\class{F}L$
    para $L \in \NP$.

    Estes problemas são por vezes chamados de \emph{problemas de busca}
    (do inglês \emph{search problem}),
    em oposição a \emph{problemas de decisão}.

    A diferença crucial é que \citeauthoronline{Papadimitriou1994}
    pede apenas que encontremos algum $y$,
    enquanto que nossa definição exige que encontremos o maior deles.
}

\begin{example}
    Todos os algoritmos polinomiais
    discutidos em cursos de projeto e análise de algoritmos
    correspondem a funções em $\FP$.
    Por exemplo,
    temos cálculo de determinantes,
    programação linear\footnote{
        Programação linear é resolvível em tempo polinomial
        quando o domínio do problema são os números racionais.
        Se restringirmos o domínio da solução a números inteiros,
        o problema é $\NP$"=completo.

        Programação linear nos inteiros geralmente é chamada de
        \emph{programação inteira}
        (do inglês \emph{integer programming}).
    }
    e parsing de linguagens livres de contexto.
\end{example}

\begin{uproposition}
    Todas as funções características de problemas em $\NP$ estão em $\FNP$,
    e todas as funções características de problemas em $\coNP$ estão em $\coFNP$.
\end{uproposition}

\begin{example}
    Os problemas em $\NP$ podem ser facilmente generalizados para funções em $\FNP$.
    Por exemplo,
    podemos pegar uma máquina não"=determinística para $\SAT$,
    e modificá"=la para que,
    após achar uma atribuição de valores"=verdade,
    a escreva na fita.
    Caso contrário,
    escreva a palavra vazia.
    A função não"=determinística computada por esta máquina
    é a maior atribuição de valores"=verdade,
    lexicograficamente.
    Esta função está em $\FNP$.

    Similarmente,
    podemos encontrar o maior clique num grafo,
    achar fatores de um número,
    e encontrar um isomorfismo entre dois grafos
    usando o poder computacional de $\FNP$.

    Já problemas de minimização
    são mais facilmente interpretados como funções de $\coFNP$.
    Por exemplo,
    encontrar o caminho hamiltoniano de menor custo,
    e calcular o número cromático.
\end{example}

O contrário também é possível;
isto é,
podemos transformar problemas de $\FNP$ em problemas de $\NP$.

\begin{theorem}
    Se $f$ é uma função de $\FNP$,
    então a linguagem
    \begin{equation*}
        L_f = \{ \langle x, y \rangle \mid f(x) \geq y \},
    \end{equation*}
    em que $\geq$ significa comparação lexicográfica,
    pertence a $\NP$.
    \label{thm:fnp_to_np_conversion}
\end{theorem}

\begin{proof}
    A ideia é fazer com que a máquina não"=determinística $M$,
    que reconhecerá $L_f$,
    compute $f$ usando seu próprio não"=determinismo,
    e aceitará a entrada caso ache algum ramo de computação
    que produza um valor maior ou igual a $y$.

    Mais formalmente,
    seja $N$ uma máquina não"=determinística que computa $f$ em tempo polinomial.
    Considere que a entrada é $\langle x, y \rangle$.
    A primeira coisa que $M$ fará é executar $N$ em $x$.
    Sabemos que isso é possível pois ambas as máquinas são não"=determinísticas.

    Após $N$ parar
    (o que ocorrerá em tempo polinomial),
    $N$ deixará na fita um ``candidato'' a $f(x)$.
    De fato,
    todas as folhas da computação de $N$ estarão disponíveis nos ramos de $M$.
    Como $N$ computa $f$,
    apenas o maior destes valores será $f(x)$;
    portanto,
    se algum ramo encontrar algum candidato que seja maior ou igual a $y$,
    a entrada $\langle x, y \rangle$ pode ser aceita.

    O contrário também ocorre.
    Se todos os ramos de computação gerarem candidatos a $f(x)$
    que são menores que $y$,
    o próprio valor de $f(x)$ será menor que $y$,
    portanto a entrada pode ser rejeitada.

    Em outras palavras,
    a máquina $M$ deve aceitar exatamente quando encontrar
    alguma folha da computação de $N(x)$ que é maior que $y$.
\end{proof}

De certa forma,
este teorema nos permite fazer uma ``busca binária''
atrás do valor de $f(x)$
usando apenas algum dispositivo que reconheça $L_f$.
Voltaremos a esta ideia na seção~\ref{sec:functional_oracles}.

\section{Oráculos funcionais e a hierarquia polinomial}

\chapter{COMPARAÇÃO COM OUTROS TRABALHOS}

\emph{Complexidade} é a quantidade de recursos
que uma máquina de Turing gasta
para computar determinada função
ou para decidir pertinência a uma linguagem
\cite[p.~285]{HopcroftUllman1979}.
Os recursos mais importantes para a teoria de complexidade computacional
são o espaço e o tempo,
em máquinas de Turing determinísticas e não"=determinísticas.

Os axiomas de Blum definem a noção de complexidade computacional
não apenas para máquinas de Turing,
mas sim para qualquer modelo de computação que satisfaça algumas restrições.

A seção~\ref{sec:enumeration_of_recursive_functions}
contém a maquinaria matemática necessária para expressar estas restrições.
Os axiomas de Blum aparecem na seção~\ref{sec:blum_axioms}.
Por fim, a seção~\ref{sec:default_measures}
define as medidas de complexidade padrão.

\input{comparacao/papadimitriou}
\section{CONJUNTO POTÊNCIA COMO CONTRADOMÍNIO (ANDREEV)}

\citeonline[p.~3]{Andreev1994}
apresenta uma abordagem diferente.
Em vez de tentar amarrar algum dos ramos de computação,
a função retornará todos eles.

\begin{definition}
    Sejam $\mathcal A$ e $\mathcal B$ conjuntos finitos,
    e $2^\mathcal B$ o conjunto potência de $\mathcal B$.
    Uma \emph{função não"=determinística de $\mathcal A$ em $\mathcal B$}
    é uma função da forma
    \begin{equation*}
        f : \mathcal A \to 2^\mathcal B.
    \end{equation*}
    \cite[p.~3]{Andreev1994}
\end{definition}

Os conjuntos $\mathcal A$ e $\mathcal B$ usados por \citeonline[p.~4]{Andreev1994}
são da forma $\{0, 1\}^k$.
A ideia é que $\mathcal A$ represente as possíveis entradas para um circuito booleano,
e que $\mathcal B$ represente as saídas desse circuito.
Dado um circuito computacional $S$,
com $k$ entradas e $l$ saídas,
dizemos que $S$ computa a função não"=determinística $F$ de $\{0, 1\}^k$ em $\{0, 1\}^l$
se,
para todo $a \in \{0, 1\}^k$,
$S(a) \in F(a)$.
Isto é,
todas as saídas possíveis de $S$ estão previstas em $F$.
Esta definição é similar à definição de \citeonline[p.~229]{Papadimitriou1994},
no sentido de que basta retornar um valor de $F(a)$.
\citeonline[p.~4]{Andreev1994}
define, então,
a \emph{complexidade} de uma função não"=determinística $F$
como sendo o tamanho do menor circuito que computa $F$.

O fato de Andreev se restringir a domínio e contradomínio finitos
mostra que esta definição não ``combina'' corretamente
a noção de função determinística com máquinas não"=determinísticas,
pois as funções operam sobre o domínio $\{0, 1\}^*$.

Além disso,
como o domínio e o contradomínio das funções não"=determinísticas de Andreev
são diferentes,
seria necessário um malabarismo adicional
para desenvolver a noção de composição de funções,
assim como na definição de Papadimitriou.

\section{DEFINIÇÃO DE HOPCROFT E ULLMAN}

A definição de \citeonline[p.~313]{HopcroftUllman1979},
mencionada na introdução,
é a que motivou este trabalho.
\begin{definition}
    Seja $M$ uma máquina de Turing não"=determinística.
    Dizemos que $M(x) = y$ se
    algum dos ramos de computação de $M$ em $x$ produz $y$,
    e nenhum outro ramo de computação que se encerre
    produz um valor diferente de $y$.
    \cite[p.~313]{HopcroftUllman1979}
\end{definition}

O problema é que esta definição não enumera corretamente as funções recursivas.
O grande vilão está na permissão que damos à máquina $M$
não parar em todos os ramos de computação.

Por exemplo,
construa uma máquina $N$ que,
na entrada $\langle M, x \rangle$,
assuma dois ramos de computação:
no primeiro,
$N$ incondicionalmente escreve $1$ na fita de saída e para.
No outro, $N$ simula a máquina determinística $M$ em $x$ e,
caso $M$ pare ao computar $x$,
$N$ escreve $0$ na fita.

Se $M$ não para em $x$,
então $N$ possui um único ramo de computação que para,
e este ramo produz $1$ na fita;
portanto, $N(\langle M, x \rangle) = 1$.
Caso $M$ pare ao computar $x$,
$N(\langle M, x \rangle)$ estará indefinido,
pois dois ramos de computação de $N$ escrevem coisas diferentes na fita.

Ou seja,
esta máquina computa
\begin{equation*}
    f(\langle M, x \rangle) = 1, \text{ se $M$ não parar ao computar $x$,}
\end{equation*}
que é exatamente o complemento do problema da parada.
Portanto,
pela definição de Hopcroft e Ullman,
conseguimos enumerar funções não"=recursivas.

Mas,
se impormos à definição de Hopcroft e Ullman a restrição adicional
de que todos os ramos de $M$ devem parar para que $M(x)$ exista,
então não há a necessidade de haver vários ramos de computação,
pois todos eles retornam o mesmo resultado.
Basta fixar uma das possíveis transições não"=determinísticas,
tornando a máquina \emph{determinística}.
Neste caso,
perdemos o aparente ganho de tempo exponencial ao usar não"=determinismo.

\section{PROBLEMAS DE OTIMIZAÇÃO (KRENTEL)}

Dentre os trabalhos avaliados,
o artigo de \citeonline[p.~493]{Krentel1988}
é o que mais se aproxima do trabalho desenvolvido neste TCC.

Krentel define a classe $\OptP$,
\simbolo{$\OptP$}{Classe polinomial de problemas de otimização}
os problemas de otimização que rodam em tempo polinomial,
como sendo,
na terminologia da seção~\ref{sec:functional_complexity},
\begin{equation*}
    \OptP = \FNP \cup \coFNP.
\end{equation*}

Entretanto,
a principal diferença é na forma como é definida completude.
\begin{definition}
    Uma função $f \in \OptP$ é completa para $\OptP$
    se, para toda função $g \in \OptP$,
    existir um par de funções $T_1, T_2 \in \P$
    tais que
    \begin{equation*}
        g(x) = T_2( x, f(T_1(x)) ).
    \end{equation*}
\end{definition}

Nossa definição de $\FNP$"=completo
buscava generalizar diretamente a noção de $\NP$"=completude.
Conforme observado na demonstração do teorema~\ref{thm:pi_f_subseteq_delta_f},
podemos, em tempo polinomial determinístico,
inverter a ordenação das palavras, se elas tiverem um tamanho fixo.
Portanto,
usando este pós"=processamento,
podemos tornar uma função que maximiza seus ``valores não"=determinísticos''
(isto é, sua árvore de computação não"=determinística)
numa função que os minimiza.
Como Krentel já embutiu os problemas de maximização e minimização em $\OptP$,
não há problema em permitir o pós"=processamento feito por $T_2$.

Entretanto,
graças a $T_2$,
a classe $\OptP$ acaba ``funcionando'' como nossa classe $\Delta_n^f$,
pois,
como observado no teorema~\ref{thm:strong_compositive_closure},
podemos resolver todos os problemas de $\Delta_n^f$
apenas pós"=processando um resultado de $\FNP$.
Portanto não é possível construir uma ``hierarquia polinomial funcional''
com base em $\OptP$.


\chapter{Comparação com outros trabalhos}

Este capítulo discute brevemente outros trabalhos que,
de certa forma,
tentam capturar a noção de ``função não"=determinística''.

\section{Problemas de busca vs problemas de decisão (Papadimitriou)}
\label{sec:papadimitriou_comparison}

\citeonline[p.~229]{Papadimitriou1994}
define $\FNP$ diferente de nossa definição
(na seção~\ref{sec:functional_complexity}).

\begin{definition}
    Uma linguagem $L$ é \emph{polinomialmente equilibrada}
    se existir algum polinômio $p$ tal que
    todos os elementos de $L$ são pares ordenados da forma $(x, y)$
    em que $|y| \leq p(|x|)$.
\end{definition}
Isto é, $L$ é constituída de pares,
de forma que o segundo elemento do par não é muito maior que o primeiro elemento.

Linguagens de $\NP$ e linguagens polinomialmente equilibradas
estão relacionadas por certificados de pertinência.
Por exemplo,
para a linguagem $\SAT$,
podemos provar eficientemente
(isto é, em tempo polinomial)
que determinada instância $\varphi$ é satisfazível
se fornecermos uma atribuição de valores"=verdade que satisfaz a instância;
esta atribuição é uma \emph{testemunha} ou \emph{certificado}
para a pertinência de $\varphi$ a $\SAT$.

Para cada linguagem $L \in \NP$,
podemos sistematicamente prover certificados de pertinência para $L$:
como sabemos que existe uma máquina de Turing não"=determinística que decide $L$,
podemos fornecer a sequência de transições não"=determinísticas
como um certificado de pertinência.
Como esta máquina opera em tempo polinomial,
para cada $x \in L$,
o par
\begin{equation*}
    (x, y),
\end{equation*}
em que $y$ é esta sequência de transições,
satisfará $|y| \leq p(|x|)$,
para algum polinômio $p$
--- no caso, $p$ é o próprio limite de tempo da máquina não"=determinística
que reconhece $L$.

Dessa forma,
podemos, sistematicamente,
associar uma linguagem $L \in \NP$
a uma linguagem polinomialmente equilibrada $R_L \in \P$.
Assim,
construiremos o conjunto $\FNP$ definido por \citeonline[p.~229]{Papadimitriou1994}.

\begin{definition}
    Se $L \in \NP$, chame de $R_L$
    uma linguagem polinomialmente equilibrada associada com $L$.
    Então defina $\FNP$ por
    \begin{equation*}
        \FNP = \{R_L \mid L \in \NP\}
    \end{equation*}
    \cite[p.~229]{Papadimitriou1994}
\end{definition}

Estes problemas são por vezes chamados de \emph{problemas de busca}
(do inglês \emph{search problem}),
em oposição a \emph{problemas de decisão}.

Do ponto de vista computacional,
a definição de Papadimitriou captura a ideia de
encontrar alguma solução para o problema $L$,
potencialmente descartando uma quantidade exponencial de outras soluções.

Esta definição contorna o problema de definir ``função não"=determinística''
trabalhando diretamente com os certificados de pertinência.
Entretanto,
a ausência do conceito de função
dificulta a formalização do conceito de oráculo
(pois não há mais uma única resposta ``certa'',
mas sim várias)
e impossibilita a análise do efeito da composição de funções,
como a feita na seção~\ref{sec:function_composition}.

\section{Conjunto potência como contradomínio (Andreev)}

\citeonline[p.~3]{Andreev1994}
apresenta uma abordagem diferente.
Em vez de tentar amarrar algum dos ramos de computação,
a função retornará todos eles.

\begin{definition}
    Sejam $\mathcal A$ e $\mathcal B$ conjuntos finitos,
    e $2^\mathcal B$ o conjunto potência de $\mathcal B$.
    Uma \emph{função não"=determinística de $\mathcal A$ em $\mathcal B$}
    é uma função da forma
    \begin{equation*}
        f : \mathcal A \to 2^\mathcal B.
    \end{equation*}
    \cite[p.~3]{Andreev1994}
\end{definition}

Os conjuntos $\mathcal A$ e $\mathcal B$ usados por \citeonline[p.~4]{Andreev1994}
são da forma $\{0, 1\}^k$.
A ideia é que $\mathcal A$ represente as possíveis entradas para um circuito booleano,
e que $\mathcal B$ represente as saídas desse circuito.
Dado um circuito computacional $S$,
com $k$ entradas e $l$ saídas,
dizemos que $S$ computa a função não"=determinística $F$ de $\{0, 1\}^k$ em $\{0, 1\}^l$
se,
para todo $a \in \{0, 1\}^k$,
$S(a) \in F(a)$.
Isto é,
todas as saídas possíveis de $S$ estão previstas em $F$.
Esta definição é similar à definição de \citeonline[p.~229]{Papadimitriou1994},
no sentido de que basta retornar um valor de $F(a)$.
\citeonline[p.~4]{Andreev1994}
define, então,
a \emph{complexidade} de uma função não"=determinística $F$
como sendo o tamanho do menor circuito que computa $F$.

O fato de Andreev se restringir a domínio e contradomínio finitos
mostra que esta definição não ``combina'' corretamente
a noção de função determinística com máquinas não"=determinísticas,
pois as funções operam sobre o domínio $\{0, 1\}^*$.

Além disso,
como o domínio e o contradomínio das funções não"=determinísticas de Andreev
são diferentes,
seria necessário um malabarismo adicional
para desenvolver a noção de composição de funções,
assim como na definição de Papadimitriou.

\section{Definição de Hopcroft e Ullman}

A definição de \citeonline[p.~313]{HopcroftUllman1979},
mencionada na introdução,
é a que motivou este trabalho.
\begin{definition}
    Seja $M$ uma máquina de Turing não"=determinística.
    Dizemos que $M(x) = y$ se
    algum dos ramos de computação de $M$ em $x$ produz $y$,
    e nenhum outro ramo de computação que se encerre
    produz um valor diferente de $y$.
    \cite[p.~313]{HopcroftUllman1979}
\end{definition}

O problema é que esta definição não enumera corretamente as funções recursivas.
O grande vilão está na permissão que damos à máquina $M$
não parar em todos os ramos de computação.

Por exemplo,
construa uma máquina $N$ que,
na entrada $\langle M, x \rangle$,
assuma dois ramos de computação:
no primeiro,
$N$ incondicionalmente escreve $1$ na fita de saída e para.
No outro, $N$ simula a máquina determinística $M$ em $x$ e,
caso $M$ pare ao computar $x$,
$N$ escreve $0$ na fita.

Se $M$ não para em $x$,
então $N$ possui um único ramo de computação que para,
e este ramo produz $1$ na fita;
portanto, $N(\langle M, x \rangle) = 1$.
Caso $M$ pare ao computar $x$,
$N(\langle M, x \rangle)$ estará indefinido,
pois dois ramos de computação de $N$ escrevem coisas diferentes na fita.

Ou seja,
esta máquina computa
\begin{equation*}
    f(\langle M, x \rangle) = 1, \text{ se $M$ não parar ao computar $x$,}
\end{equation*}
que é exatamente o complemento do problema da parada.
Portanto,
pela definição de Hopcroft e Ullman,
conseguimos enumerar funções não"=recursivas.

Mas,
se impormos à definição de Hopcroft e Ullman a restrição adicional
de que todos os ramos de $M$ devem parar para que $M(x)$ exista,
então não há a necessidade de haver vários ramos de computação,
pois todos eles retornam o mesmo resultado.
Basta fixar uma das possíveis transições não"=determinísticas,
tornando a máquina \emph{determinística}.
Neste caso,
perdemos o aparente ganho de tempo exponencial ao usar não"=determinismo.


\bibliographystyle{abnt-alf}
\bibliography{bib/bibliography}

\apendice
\chapter{Máquinas de Turing}
\label{app:turing_machines}

Este apêndice contêm uma formalização de máquinas de Turing
e seu uso como decisores e computadores de funções recursivas.

\section{Definição Matemática}

\subsection{Decisores e Enumeradores}

\subsection{Funções de Inteiros}

\section{Máquinas Multifitas}

\subsection{Fitas de Entrada e Saída}

\section{Codificação de Máquinas de Turing}

\subsection{Máquinas de Turing Universais}


\end{document}
