\documentclass[12pt]{letter}
\usepackage[a4paper,margin=2cm]{geometry}

\usepackage[T1]{fontenc}
\usepackage[utf8]{inputenc}

\usepackage{mathptmx}
\usepackage{tabularx}
\usepackage{multirow}

\begin{document}

\pagestyle{empty}

\begin{centering}

    \textbf{DEPARTAMENTO DE INFORMÁTICA E ESTATÍSTICA -- CTC -- UFSC}

    \textbf{RATIFICAÇÃO DE PLANO DE TRABALHO DO SEMESTRE \\ PARA DESENVOLVIMENTO DE TCC}

\end{centering}


\vspace{1em}
\setlength\extrarowheight{5pt}
\begin{tabular}{l l}
    \textbf{Disciplina:} & TCC 1 \\
    \textbf{Curso:}      & Ciência da Computação \\
    \textbf{Autor:}      & Tiago Royer \\
    \textbf{Título:}     & Complexidade de Circuitos \\
    \textbf{Professora responsável:} & Jerusa Marchi \\
\end{tabular}


\vspace{1em}
{\large \textbf{Objetivos}}
\\

\textbf{Objetivo geral:}
Compreender a relação entre classes de complexidade computacional
e complexidade de circuitos.

\textbf{Objetivos específicos:}
\begin{enumerate}
    \item Estudar as classes de complexidade computacional
        e complexidade de circuitos.
    \item Demonstrar a relação entre complexidade de circuitos
        e o problema $P$ versus $NP$.
    \item Correlacionar as principais classes de complexidade computacional
        às correspondentes classes de complexidade de circuitos
    \item Estabelecer limites inferiores no tamanho e profundidade dos circuitos
        que computam certas funções recursivas.
\end{enumerate}


\vspace{1em}
{\large \textbf{Cronograma}}

\begin{tabularx}{\linewidth}{|X|*{8}{c|}}
    \hline
    \multicolumn{1}{|c|}{\multirow{2}{*}{Etapas}} & \multicolumn{8}{|c|}{Meses}\\ \cline{2-9}
    & mar & abr & mai & jun & jul & ago & set & out \\ \hline

    Estudar classes de complexidade computacional
    &  x  &     &     &     &     &     &     &     \\ \hline

    Estudar classes de complexidade de circuitos
    &  x  &  x  &  x  &     &     &     &     &     \\ \hline

    Estabelecer relações de equivalência entre as classes de complexidade
    &  x  &  x  &  x  &  x  &     &     &     &     \\ \hline

    Rever demonstrações de limites inferiores para \mbox{certas} funções recursivas
    &     &  x  &  x  &  x  &  x  &  x  &     &     \\ \hline

    Explorar a relação com o problema $P$ versus $NP$
    &     &  x  &     &     &     &  x  &     &     \\ \hline

    Documentar o aprendido
    &  x  &  x  &  x  &  x  &  x  &  x  &  x  &  x  \\ \hline

\end{tabularx}

\begin{centering}

    \fbox{\begin{minipage}[c][6em][c]{0.7\textwidth}
        {\center \textbf{Preenchimento pelo Professor responsável pelo TCC}\\[1em]}

        \qquad $(\quad)$ \ Ciente e de acordo.

        \qquad Data: \_\_ / \_\_ / \_\_
    \end{minipage}}

\end{centering}
\end{document}
