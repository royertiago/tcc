\documentclass[utf8,notheorems]{beamer}
\usetheme[compress]{Singapore}

\usepackage[brazil]{babel}
\usepackage{tikz}
\usetikzlibrary{calc}

\newtheorem*{theorem}{Teorema}
\newtheorem*{corollary}{Corolário}
\theoremstyle{definition}
\newtheorem*{definition}{Definição}

\let\C\someundefinedcommand
\let\G\someundefinedcommand
% These two commands are defined by hyperref in file puenc.def,
% and conflict with complexity - which also defines them.
% \C encodes some unicode character (U+030f), \G is textdoublegrave.
% Seems safe to simply undefine them.
\usepackage{complexity}

\begin{document}

\author{Tiago Royer}
\title{Complexidade de Circuitos}
\subtitle{Seminário de andamento}
\date{26 de março de 2015}
\institute{IATE - UFSC}
\begin{frame}
    \titlepage
\end{frame}

\section{Introdução}

\begin{frame}
    \frametitle{Síntese}
    \tableofcontents
\end{frame}

\subsection{Motivação}

\begin{frame}
    \frametitle{Motivação}
    \begin{itemize}
        \item Entscheidungsproblem, 1928
        \item Objetivo: encontrar um método mecânico
            para demonstrar teoremas em lógica de primeira ordem.
        \pause
        \item Dispositivos computacionais
            --- o que caracteriza um ``método mecânico''?
        \pause
        \item Decidível vs Indecidível
        \pause
        \item Tratável vs Intratável
        \pause
        \item $\P$ vs $\NP$
        \pause
        \item Simplificar o modelo computacional
            --- Usar circuitos.
    \end{itemize}
\end{frame}

\section{Circuitos computacionais}

\subsection{Definições}

\newcommand{\placenode}[1]{
    \node (x#1) at (2*#1, 0) {$x_#1$};
    \path (2*#1+0.5, 1) node[draw,circle] (neg#1) {$\neg$};
    \coordinate (p#1) at (2*#1 - 0.5, 2);
    \coordinate (n#1) at (2*#1 + 0.5, 2);
    \draw (x#1) -- (2*#1 - 0.5, 1) -- (p#1) -- (wp);
    \draw (x#1) -- (neg#1) -- (n#1) -- (wn);
}

\begin{frame}
    \frametitle{Circuitos computacionais}
    \centering
    \begin{tikzpicture}
        \path (5.5, 4) node[draw,circle] (wp) {$\wedge$};
        \path (2.5, 4) node[draw,circle] (wn) {$\wedge$};
        \path (4, 6) node[draw,circle] (vee) {$\vee$};
        \draw (wp) -- (vee) -- (wn);
        \draw (vee) -- (4, 7);
        \placenode{1}
        \placenode{2}
        \placenode{3}
    \end{tikzpicture}
\end{frame}

\begin{frame}
    \frametitle{Famílias de circuitos}
    \begin{definition}
        Uma família de circuitos é uma lista indexada
        \begin{equation*}
            \{c_1, c_2, c_3, \dots\}
        \end{equation*}
        em que cada $c_i$ é um circuito com $i$ entradas booleanas.
    \end{definition}

    \pause
    Medidas de complexidade:
    \begin{itemize}
        \item Quantidade de portas lógicas
        \item Profundidade do circuito
    \end{itemize}
\end{frame}

\subsection{Relação com $\P$ vs $\NP$}
\begin{frame}
    \frametitle{Relação com $\P$ vs $\NP$}
    \begin{theorem}
        Se uma máquina de Turing reconhece uma linguagem em tempo $T(n)$,
        então existe uma família de circuitos computacionais
        que reconhece esta mesma linguagem
        com $O(T(n) \log T(n))$ portas lógicas.
    \end{theorem}
    \pause
    \begin{corollary}
        Se algum problema $\NP$
        não possuir circuitos de tamanho polinomial,
        então $\P \neq \NP$.
    \end{corollary}
\end{frame}


\section{Trabalho atual}

\subsection{Complexidade computacional axiomática}
\begin{frame}
    \frametitle{Axiomas de Blum}
    \begin{definition}
        Uma \emph{medida de complexidade}
        é uma função $\Phi$ que satisfaz aos seguintes axiomas:
        \begin{enumerate}
            \item
                $\Phi(M, x)$ está definido
                se, e somente se,
                $M(x)$ está definido.
            \item
                Dados $M$, $x$ e $k$,
                é decidível se $\Phi(M, x) = k$.
        \end{enumerate}
    \end{definition}
\end{frame}

\begin{frame}
    \frametitle{Classes de complexidade}
    \begin{definition}
        Dada uma medida de complexidade $\Phi$
        e uma função recursiva total
        $f: \mathbb N \rightarrow \mathbb N$,
        a \emph{classe de complexidade $f$} com relação a $\Phi$
        é o conjunto
        \begin{equation*}
            \mathcal C_\Phi(f) = \{ L(M) \mid \Phi(M, x) \leq f(|x|)
                \text{ para quase todos os $x$}
            \}.
        \end{equation*}
    \end{definition}
\end{frame}

\subsection{Hierarquias de complexidade}
\begin{frame}
    \frametitle{Hierarquia de classes de complexidade}

    \begin{theorem}
        \begin{equation*}
            \DSPACE(n^c) \subsetneq \DSPACE(n^{c+\varepsilon})
        \end{equation*}
        \begin{equation*}
            \DTIME(n^c) \subsetneq \DTIME(n^{c+\varepsilon})
        \end{equation*}
    \end{theorem}

    \centering
    \begin{tikzpicture}[auto,every node/.style={font=\footnotesize}]
        \node (l1) at (0, 0) {};
        \node (l2) at (0, 0.6) {};
        \node (l3) at (0, 1.2) {};
        \node (l4) at (0, 1.8) {};
        \node (r1) at (10, 0) {};
        \node (r2) at (10, 0.6) {};
        \node (r3) at (10, 1.2) {};
        \node (r4) at (10, 1.8) {};

        \draw (r1) to [out=170, in=10] node {$\vdots$} (l1);
        \draw (r2) to [out=170, in=10] node {$\DTIME(n^c)$} (l2);
        \draw (r3) to [out=170, in=10] node {$\DTIME(n^{c+\varepsilon})$} (l3);
        \draw (r4) to [out=170, in=10] node {$\DTIME(n^{c+2\varepsilon})$}
                      node [swap] {$\vdots$} (l4);
    \end{tikzpicture}

    \pause
    Problema: demonstração usa diagonalização
\end{frame}

\begin{frame}
    \frametitle{Hierarquia polinomial}

    \centering
    \begin{tikzpicture}
        \draw let \n1 = {0}, \n2 = {8}, \n3 = {1.5}, \n4 = {0.5} in
            (\n1, -\n4) -- (\n2, -\n4)
            (\n1, 0) -- (\n2, \n3)
            (\n1, \n3) -- (\n2, 0)
            (\n1, \n3) -- (\n2, 2*\n3)
            (\n1, 2*\n3) -- (\n2, \n3)
            (\n1, 2*\n3) -- (\n2, 3*\n3)
            (\n1, 3*\n3) -- (\n2, 2*\n3)
            (\n1, 4*\n3) -- (\n2, 4*\n3)
            (\n1, -\n4) -- (\n1, 4*\n3)
            (\n2, -\n4) -- (\n2, 4*\n3)
            node (a) at ({(\n1+\n2)/2}, 0) {
                $\Sigma_0^\P = \Delta_0^\P = \Pi_0^\P = \P$
            }
            node (a) at (\n1+2*\n4, \n3/2) {$\Sigma_1^\P = \NP$}
            node (a) at (\n2-2*\n4, \n3/2) {$\Pi_1^\P = \coNP$}
            node (a) at ({\n1+\n2/2}, \n3) {$\Delta_2^\P = \P^\NP$}
            node (a) at (\n1+\n4, 3*\n3/2) {$\Sigma_2^\P$}
            node (a) at (\n2-\n4, 3*\n3/2) {$\Pi_2^\P$}
            node (a) at ({\n1+\n2/2}, 2*\n3) {$\Delta_3^\P$}
            node (a) at (\n1+\n4, 5*\n3/2) {$\Sigma_3^\P$}
            node (a) at (\n2-\n4, 5*\n3/2) {$\Pi_3^\P$}
            node (a) at ({\n1+\n2/2}, 3*\n3) {$\vdots$}
            node (a) at ({\n1+\n2/2}, 7*\n3/2) {$\PH \subseteq \PSPACE$};
    \end{tikzpicture}
\end{frame}

\subsection{Próximos objetivos}
\begin{frame}
    \frametitle{Próximos objetivos}

    \begin{itemize}
        \item Entender a demonstração de Pippenger.
        \item Estudar a hierarquia polinomial.
    \end{itemize}
\end{frame}

\end{document}
