\documentclass[utf8,notheorems]{beamer}
\usetheme[compress]{Singapore}

\usepackage[brazil]{babel}
\usepackage{tikz}

\newtheorem*{theorem}{Teorema}
\newtheorem*{corollary}{Corolário}
\theoremstyle{definition}
\newtheorem*{definition}{Definição}

\let\C\someundefinedcommand
\let\G\someundefinedcommand
% These two commands are defined by hyperref in file puenc.def,
% and conflict with complexity - which also defines them.
% \C encodes some unicode character (U+030f), \G is textdoublegrave.
% Seems safe to simply undefine them.
\usepackage{complexity}

\begin{document}

\author{Tiago Royer}
\title{Complexidade de Circuitos}
\subtitle{Seminário de andamento}
\date{26 de março de 2015}
\institute{IATE - UFSC}
\begin{frame}
    \titlepage
\end{frame}

\section{Introdução}

\begin{frame}
    \frametitle{Síntese}
    \tableofcontents
\end{frame}

\subsection{Motivação}

\begin{frame}
    \frametitle{Motivação}
    \begin{itemize}
        \item Entscheidungsproblem, 1928
        \item Objetivo: enontrar um método mecânico
            para demonstrar teoremas em lógica de primeira ordem.
        \pause
        \item Dispositivos computacionais
            --- o que caracteriza um ``método mecânico''?
        \pause
        \item Decidível vs Indecidível
        \pause
        \item Tratável vs Intratável
        \pause
        \item $\P$ vs $\NP$
        \pause
        \item Simplificar o modelo computacional
            --- Usar circuitos.
    \end{itemize}
\end{frame}

\section{Circuitos computacionais}

\subsection{Definições}

\newcommand{\placenode}[1]{
    \node (x#1) at (2*#1, 0) {$x_#1$};
    \path (2*#1+0.5, 1) node[draw,circle] (neg#1) {$\neg$};
    \coordinate (p#1) at (2*#1 - 0.5, 2);
    \coordinate (n#1) at (2*#1 + 0.5, 2);
    \draw (x#1) -- (2*#1 - 0.5, 1) -- (p#1) -- (wp);
    \draw (x#1) -- (neg#1) -- (n#1) -- (wn);
}

\begin{frame}
    \frametitle{Circuitos computacionais}
    \centering
    \begin{tikzpicture}
        \path (5.5, 4) node[draw,circle] (wp) {$\wedge$};
        \path (2.5, 4) node[draw,circle] (wn) {$\wedge$};
        \path (4, 6) node[draw,circle] (vee) {$\vee$};
        \draw (wp) -- (vee) -- (wn);
        \draw (vee) -- (4, 7);
        \placenode{1}
        \placenode{2}
        \placenode{3}
    \end{tikzpicture}
\end{frame}

\begin{frame}
    \frametitle{Famílias de circuitos}
    \begin{definition}
        Uma família de circuitos é uma lista indexada
        \begin{equation*}
            \{c_1, c_2, c_3, \dots\}
        \end{equation*}
        em que cada $c_i$ é um circuito com $i$ entradas booleanas.
    \end{definition}

    \pause
    Medidas de complexidade:
    \begin{itemize}
        \item Quantidade de portas lógicas
        \item Profundidade do circuito
    \end{itemize}
\end{frame}

\subsection{Relação com $\P$ vs $\NP$}
\begin{frame}
    \frametitle{Relação com $\P$ vs $\NP$}
    \begin{theorem}
        Se uma máquina de Turing reconhece uma linguagem em tempo $T(n)$,
        então existe uma família de circuitos computacionais
        que reconhece esta mesma linguagem
        com $O(T(n) \log T(n))$ portas lógicas.
    \end{theorem}
    \pause
    \begin{corollary}
        Se algum problema $\NP$
        não possuir circuitos de tamanho polinomial,
        então $\P \neq \NP$.
    \end{corollary}
\end{frame}


\section{Trabalho atual}

\subsection{Complexidade computacional axiomática}
\begin{frame}
    \frametitle{Axiomas de Blum}
    \begin{definition}
        Uma \emph{medida de complexidade}
        é uma função $\Phi$ que satisfaz aos seguintes axiomas:
        \begin{enumerate}
            \item
                $\Phi(M, x)$ está definido
                se, e somente se,
                $M(x)$ está definido.
            \item
                Dados $M$, $x$ e $k$,
                é decidível se $\Phi(M, x) = k$.
        \end{enumerate}
    \end{definition}
\end{frame}

\subsection{Hierarquias de complexidade}
\begin{frame}
    \frametitle{Hierarquia de classes de complexidade}

    \begin{theorem}
        \begin{displaymath}
            \DSPACE(n^c) \subsetneq \DSPACE(n^{c+\varepsilon})
        \end{displaymath}
        \begin{displaymath}
            \DTIME(n^c) \subsetneq \DTIME(n^{c+\varepsilon})
        \end{displaymath}
    \end{theorem}
\end{frame}

\begin{frame}
    \frametitle{Hierarquia polinomial}

    \begin{itemize}
        \item Diagonalização
        \item Hierarquia fora de $\P$.
    \end{itemize}
\end{frame}

\subsection{Próximos objetivos}
\begin{frame}
    \frametitle{Próximos objetivos}

    \begin{itemize}
        \item Entender a demonstração de Pippenger.
        \item Estudar a hierarquia polinomial.
    \end{itemize}
\end{frame}

\end{document}
