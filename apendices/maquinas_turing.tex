\chapter{Máquinas de Turing}
\label{app:turing_machines}

Este apêndice contêm uma formalização de máquinas de Turing
e seu uso como decisores e computadores de funções recursivas.
O objetivo deste apêndice é prover uma fundamentação matemática
para algumas afirmações feitas no texto,
como, por exemplo,
como uma máquina de Turing pode simular outra,
porquê as codificações identificam unicamente uma função recursiva,
e porquê não há problemas ao usar problemas indecidíveis como oráculos.

\section{Definição Matemática}

\begin{definition}
    Seja $D = \{\leftarrow, \rightarrow, \cdot\}$.
    Uma \emph{máquina de Turing} $M$ é uma quádrupla
    \begin{equation*}
        M = (Q, \Gamma, \delta, q_0),
    \end{equation*}
    em que
    \begin{itemize}
        \item $Q$ é um conjunto finito de estados.
        \item $\Gamma$ é um conjunto finito de símbolos de fita,
            tais que $0, 1, B \in \Gamma$.
        \item $\delta$ é uma função parcial definida sobre
            \begin{equation*}
                \delta: \{0, 1, B\} \times \Gamma
                    \to D \times D \times \Gamma \times \{0, 1, \varepsilon\}.
            \end{equation*}
            Um par ordenado $(a, b)$ no domínio
            representa os símbolos que a máquina está lendo das fitas
            ($a \in \{0, 1, B\}$ é lido da fita somente"=leitura de entrada,
            e $b \in \Gamma$ é lido da fita de trabalho).
            Uma tupla $(c, d, e, f)$ no domínio
            representa a ação da máquina
            ($c \in D$ é a direção do movimento do cabeçote de leitura,
            $d \in D$ é a do cabeçote de trabalho,
            $e \in \Gamma$ é o que a máquina escreveu na fita de trabalho,
            e $f \in \{0, 1, \varepsilon\}$ é o que a máquina escreveu
            na fita de saída).
        \item $q_0 \in Q$ é o estado inicial.
    \end{itemize}
\end{definition}

$0$ e $1$ são os símbolo do alfabeto binário $\{0, 1\}$,
que é o alfabeto sobre o qual são definidas as funções recursivas.
$B$ é o caractere branco.

Esta definição interpreta máquinas como calculadores de funções.
Em relação às definições usuais de ``máquina de Turing''
(encontrada, por exemplo,
em \cite[p.~148]{HopcroftUllman1979},
\cite[p.~140]{Sipser2006},
e \cite[p.~6]{Kozen2006}),
nossa definição apresenta alguns símbolos ``faltando''.
\begin{itemize}
    \item Omitimos $\Sigma$, que representa o alfabeto de entrada,
        pois usaremos sempre $\Sigma = \{0, 1\}$.
    \item Os estados de aceitação e rejeição são desnecessários,
        pois a máquina não decide uma palavra,
        mas sim, calcula uma função.
    \item O símbolo $B$, o caractere branco,
        foi ``embutido'' diretamente em $\Gamma$.
    \item Não há marcadores de início/fim de fita.
\end{itemize}
