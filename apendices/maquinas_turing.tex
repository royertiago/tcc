\chapter{Máquinas de Turing}
\label{app:turing_machines}

% Definição do símbolo de branco, usável no modo matemático
\newcommand\B{\tikz{\draw (0, 0.5ex) -- (0, 0) -- (0.5em, 0) -- (0.5em, 0.5ex);}}

Este apêndice contêm uma formalização de máquinas de Turing
e seu uso como decisores e computadores de funções recursivas.
O objetivo deste apêndice é prover uma fundamentação matemática
para algumas afirmações feitas no texto,
como, por exemplo,
como uma máquina de Turing pode simular outra,
porquê as codificações identificam unicamente uma função recursiva,
e porquê não há problemas ao usar problemas indecidíveis como oráculos.

\section{Definição Matemática}

\begin{definition}
    Seja $D = \{\leftarrow, \rightarrow, \cdot\}$.
    Uma \emph{máquina de Turing} $M$ é uma quádrupla
    \begin{equation*}
        M = (Q, \Gamma, \delta, q_0),
    \end{equation*}
    em que
    \begin{itemize}
        \setlength{\labelsep}{1ex}
        \item $Q$ é um conjunto finito de estados.
        \item $\Gamma$ é um conjunto finito de símbolos de fita,
            tais que $0, 1, \B \in \Gamma$.
        \item $\delta$ é uma função parcial definida sobre
            \begin{equation*}
                \delta: Q \times \{0, 1, \B\} \times \Gamma
                    \to Q \times D \times D \times \Gamma \times \{0, 1, \varepsilon\}.
            \end{equation*}
            Uma tupla $(q, a, b)$ no domínio
            representa os símbolos que a máquina está lendo das fitas
            ($a \in \{0, 1, \B\}$ é lido da fita somente"=leitura de entrada,
            $b \in \Gamma$ é lido da fita de trabalho,
            e $q$ é o estado atual).
            Uma tupla $(q', d_1, d_2, e, s)$ no domínio
            representa a ação da máquina
            ($q'$ é o estado para o qual a máquina transitará,
            $d_1 \in D$ é a direção do movimento do cabeçote de leitura,
            $d_2 \in D$ é a do cabeçote de trabalho,
            $e \in \Gamma$ é o que a máquina escreveu na fita de trabalho,
            e $s \in \{0, 1, \varepsilon\}$ é o que a máquina escreveu
            na fita de saída).
        \item $q_0 \in Q$ é o estado inicial.
    \end{itemize}
\end{definition}

$0$ e $1$ são os símbolo do alfabeto binário $\{0, 1\}$,
que é o alfabeto sobre o qual são definidas as funções recursivas.
$\B$ é o caractere branco.

Esta definição interpreta máquinas como calculadores de funções.
Em relação às definições usuais de ``máquina de Turing''
(encontrada, por exemplo,
em \cite[p.~148]{HopcroftUllman1979},
\cite[p.~140]{Sipser2006},
e \cite[p.~6]{Kozen2006}),
nossa definição
(mais próxima da definição de \citeonline[p.~12]{AroraBarak2009})
apresenta alguns símbolos ``faltando''.
\begin{itemize}
    \setlength{\labelsep}{1ex}
    \item Omitimos $\Sigma$, que representa o alfabeto de entrada,
        pois usaremos sempre $\Sigma = \{0, 1\}$.
    \item Os estados de aceitação e rejeição são desnecessários,
        pois a máquina não decide uma palavra,
        mas sim, calcula uma função.
    \item O símbolo $\B$, o caractere branco,
        foi ``embutido'' diretamente em $\Gamma$.
    \item Não há marcadores de início/fim de fita.
\end{itemize}

\begin{definition}
    Uma \emph{máquina de Turing não"=determinística} $M$ é uma quádrupla
    \begin{equation*}
        M = (Q, \Gamma, \delta, q_0),
    \end{equation*}
    em que $Q$, $\Gamma$ e $q_0$ possuem a mesma semântica
    das máquinas determinísticas,
    mas a imagem de $\delta$,
    em vez de ser $Q \times D \times D \times \Gamma \times \{0, 1, \varepsilon\}$,
    é o conjunto potência deste conjunto.
    Isto é, $\delta$ é definida sobre
    \begin{equation*}
        \delta: \{0, 1, \B\} \times \Gamma
            \to 2^{Q \times D \times D \times \Gamma \times \{0, 1, \varepsilon\}}.
    \end{equation*}
\end{definition}

\subsection{Estados de computação}

Os estados de computação de uma máquina de Turing
representam a ``descrição instantânea'' de uma etapa do processo de computação.
Eles agregam informações como o estado da máquina e as fitas.
Nossa definição de ``computação'' será um pouco diferente da definição de
\citeonline[p.~149]{HopcroftUllman1979},
visando simplificar a definição da função $T$ da seção~\ref{sec:fnp_complete}.

\begin{definition}
    Seja $M = (Q, \Gamma, \delta, q_0)$.
    Um \emph{estado de computação} $E$ de $M$ é uma tupla
    \begin{equation*}
        E = (x, y, z, q, m, n),
    \end{equation*}
    em que
    \begin{itemize}
        \setlength{\labelsep}{1ex}
        \item $x \in \{0, 1\}^*$ é a fita de entrada.
        \item $y \in \Gamma^*$ é a fita de trabalho.
        \item $z \in \{0, 1\}^*$ é fita de saída.
        \item $q \in Q$ é o estado atual da máquina.
        \item $m \in \mathbb N$ é a posição do cabeçote de leitura.
        \item $n \in \mathbb N$ é a posição do cabeçote de trabalho.
    \end{itemize}
\end{definition}

Construiremos agora o formalismo de transição de estados.

Por conveniência, denote por $w[i]$ a $i$"=ésima letra da palavra $w$;
indexaremos $w$ em $1$,
portanto $w[1]$ é o primeiro símbolo de $w$ e $w[|w|]$ é o último.
Também identifique o conjunto $D = \{\leftarrow, \rightarrow, \dot\}$
com o conjunto $\{-1, 1, 0\}$;
isto é, $\leftarrow = -1$, $\rightarrow = 1$ e $\dot = 0$.
Assim, se $d \in D$ é um deslocamento
e $n \in \mathbb N$ é uma posição do cabeçote,
$n + d$ é o efeito de deslocar $n$ por $d$.

A partir de um estado de computação $E = (x, y, z, q, m, n)$,
para descobrir qual é a próxima ação da máquina $M = (Q, \Gamma, \delta, q_0)$,
precisamos consultar a função $\delta$.
Precisamos passar a $\delta$ uma tupla $(q, a, b)$,
em que $q$ é o próprio estado $q$ de $E$,
$a$ é o que está debaixo do cabeçote da primeira fita
e $b$ é o que está debaixo do cabeçote da segunda fita
(lembre"=se de que não autorizaremos $M$ a consultar a fita de saída).
Se $1 \leq m \leq |x|$,
então $a = x[m]$.
Caso contrário, o cabeçote de leitura da máquina
estará fora da cadeia de entrada.
Interpretaremos a fita como sendo preenchida com brancos ($\B$)
onde não há outros caracteres;
portanto, neste caso, $a = \B$.
Similarmente, se $1 \leq n \leq |y|$, $b = y[n]$; caso contrário, $b = \B$.
Portanto,
a ação da máquina $M$ no estado $E$
será $\delta(q, a, b)$;
denotaremos este valor por $\delta(E)$.%
\footnote{
    Observe que estamos abusando da notação aqui,
    pois nenhum estado de computação está no domínio de $\delta$.
}

Seja $E = (x, y, z, q, m, n)$ um estado de computação
e $A = (q', d_1, d_2, e, s)$ uma ação da máquina de Turing
(pode ser um valor retornado por $\delta$ se a máquina for determinística,
ou um dos elementos do conjunto retornado por $\delta$
se a máquina for não"=determinística).
O \emph{efeito de $A$ em $E$},
que denotaremos por $A(E)$\footnotemark{},
\footnotetext{
    Observe que,
    apesar de usarmos a notação funcional,
    $A$ não é uma função.
}
é a tupla $(x', y', z', q', m', n')$,
dada por
\begin{itemize}
    \setlength{\labelsep}{1ex}
    \item $x' = x$ (isto é, nunca alteraremos $x$).
    \item Se $n = 0$, $y' = y$
        (a máquina não pode escrever no primeiro branco da fita de trabalho);
        se $n = |y| + 1$, $y' = ye$
        (se a máquina estiver imediatamente à direita dos elementos da fita,
        estendemos a representação da fita com o caractere que a máquina escreveu);
        caso contrário,
        $y'$ é $y$ com o $n$"=ésimo simbolo trocado por $e$
        --- $y'[i] = y[i]$ se $i \neq n$, e $y'[n] = e$.
    \item $z' = zs$.
        Se $s$ for $0$ ou $1$,
        isso corresponde a escrever um destes caracteres na fita de saída.
        Se $s$ for $\varepsilon$,
        isso corresponde a manter o cabeçote na fita de saída intacto.
    \item $q'$ é o próprio $q'$ de $A$.
    \item Se $m = 0$ e $d_1 = -1$ (quer dizer, $d_1 = \leftarrow$),
        então $m' = 0$; caso contrário, $m' = m + d_1$.
        Não deixaremos o cabeçote de leitura ler mais de um branco
        à esquerda do início,
        mas permitiremos que ele ``navegue'' livremente à direita.
    \item Se $n = 0$ e $d_2 = -1$, então $n' = 0$; senão, $n' = n + d_2$.
        Também não permitiremos o cabeçote de trabalho ``navegar'' demais à esquerda,
        portanto, efetivamente,
        a fita é apenas infinita à direita.
\end{itemize}
