\chapter{Máquinas de Turing}
\label{app:turing_machines}

% Definição do símbolo de branco, usável no modo matemático
\newcommand\B{\tikz{\draw (0, 0.5ex) -- (0, 0) -- (0.5em, 0) -- (0.5em, 0.5ex);}}

Este apêndice contêm uma formalização de máquinas de Turing
e seu uso como decisores e computadores de funções recursivas.
O objetivo deste apêndice é prover uma fundamentação matemática
para algumas afirmações feitas no texto,
como, por exemplo,
como uma máquina de Turing pode simular outra,
porquê as codificações identificam unicamente uma função recursiva,
e porquê não há problemas ao usar problemas indecidíveis como oráculos.

\section{Definição Matemática}

\begin{definition}
    Seja $D = \{\leftarrow, \rightarrow, \cdot\}$.
    Uma \emph{máquina de Turing} $M$ é uma quádrupla
    \begin{equation*}
        M = (Q, \Gamma, \delta, q_0),
    \end{equation*}
    em que
    \begin{itemize}
        \setlength{\labelsep}{1ex}
        \item $Q$ é um conjunto finito de estados.
        \item $\Gamma$ é um conjunto finito de símbolos de fita,
            tais que $0, 1, \B \in \Gamma$.
        \item $q_0 \in Q$ é o estado inicial.
        \item $\delta$ é uma função parcial definida sobre
            \begin{equation*}
                \delta: Q \times \{0, 1, \B\} \times \Gamma
                    \to Q \times D \times D \times \Gamma \times \{0, 1, \varepsilon\}.
            \end{equation*}
            Uma tupla $(q, a, b)$ no domínio
            representa os símbolos que a máquina está lendo das fitas,
            em que
            \begin{itemize}
                \item $a \in \{0, 1, \B\}$ é lido da fita somente"=leitura de entrada.
                \item $b \in \Gamma$ é lido da fita de trabalho.
                \item e $q$ é o estado atual.
            \end{itemize}
            Uma tupla $(q', d_1, d_2, e, s)$ no domínio representa a ação da máquina.
            \begin{itemize}
                \item $q'$ é o estado para o qual a máquina transitará.
                \item $d_1 \in D$ é a direção do movimento do cabeçote de leitura.
                \item $d_2 \in D$ é a direção do movimento do cabeçote de trabalho.
                \item $e \in \Gamma$ é o que a máquina escreveu na fita de trabalho.
                \item $s \in \{0, 1, \varepsilon\}$ é o que a máquina escreveu
                    na fita de saída.
            \end{itemize}
    \end{itemize}
\end{definition}

$0$ e $1$ são os símbolo do alfabeto binário $\{0, 1\}$,
que é o alfabeto sobre o qual são definidas as funções recursivas.
$\B$ é o caractere branco.

Esta definição interpreta máquinas como calculadores de funções.
Em relação às definições usuais de ``máquina de Turing''
(encontrada, por exemplo,
em \cite[p.~148]{HopcroftUllman1979},
\cite[p.~140]{Sipser2006},
e \cite[p.~6]{Kozen2006}),
nossa definição
(mais próxima da definição de \citeonline[p.~12]{AroraBarak2009})
apresenta alguns símbolos ``faltando''.
\begin{itemize}
    \setlength{\labelsep}{1ex}
    \item Omitimos $\Sigma$, que representa o alfabeto de entrada,
        pois usaremos sempre $\Sigma = \{0, 1\}$.
    \item Os estados de aceitação e rejeição são desnecessários,
        pois a máquina não decide uma palavra,
        mas sim, calcula uma função.
    \item O símbolo $\B$, o caractere branco,
        foi ``embutido'' diretamente em $\Gamma$.
    \item Não há marcadores de início/fim de fita.
\end{itemize}

\begin{definition}
    Uma \emph{máquina de Turing não"=determinística} $M$ é uma quádrupla
    \begin{equation*}
        M = (Q, \Gamma, \delta, q_0),
    \end{equation*}
    em que $Q$, $\Gamma$ e $q_0$ possuem a mesma semântica
    das máquinas determinísticas,
    mas a imagem de $\delta$,
    em vez de ser $Q \times D \times D \times \Gamma \times \{0, 1, \varepsilon\}$,
    é o conjunto potência deste conjunto.
    Isto é, $\delta$ é definida sobre
    \begin{equation*}
        \delta: \{0, 1, \B\} \times \Gamma
            \to 2^{Q \times D \times D \times \Gamma \times \{0, 1, \varepsilon\}}.
    \end{equation*}
\end{definition}

\subsection{Estados de computação}

Os estados de computação de uma máquina de Turing
representam a ``descrição instantânea'' de uma etapa do processo de computação.
Eles agregam informações como o estado da máquina e as fitas.
Nossa definição de ``computação'' será um pouco diferente da definição de
\citeonline[p.~149]{HopcroftUllman1979},
visando simplificar a definição da função $T$ da seção~\ref{sec:fnp_complete}.

\begin{definition}
    Seja $M = (Q, \Gamma, \delta, q_0)$.
    Um \emph{estado de computação} $E$ de $M$ é uma tupla
    \begin{equation*}
        E = (x, y, z, q, m, n),
    \end{equation*}
    em que
    \begin{itemize}
        \setlength{\labelsep}{1ex}
        \item $x \in \{0, 1\}^*$ é a fita de entrada.
        \item $y \in \Gamma^*$ é a fita de trabalho.
        \item $z \in \{0, 1\}^*$ é fita de saída.
        \item $q \in Q$ é o estado atual da máquina.
        \item $m \in \mathbb N$ é a posição do cabeçote de leitura.
        \item $n \in \mathbb N$ é a posição do cabeçote de trabalho.
    \end{itemize}
\end{definition}

Construiremos agora o formalismo de transição de estados.

Por conveniência, denote por $w[i]$ a $i$"=ésima letra da palavra $w$;
indexaremos $w$ em $1$,
portanto $w[1]$ é o primeiro símbolo de $w$ e $w[|w|]$ é o último.
Também identifique o conjunto $D = \{\leftarrow, \rightarrow, \dot\}$
com o conjunto $\{-1, 1, 0\}$;
isto é, $\leftarrow = -1$, $\rightarrow = 1$ e $\dot = 0$.
Assim, se $d \in D$ é um deslocamento
e $n \in \mathbb N$ é uma posição do cabeçote,
$n + d$ é o efeito de deslocar $n$ por $d$.

A partir de um estado de computação $E = (x, y, z, q, m, n)$,
para descobrir qual é a próxima ação da máquina $M = (Q, \Gamma, \delta, q_0)$,
precisamos consultar a função $\delta$.
Precisamos passar a $\delta$ uma tupla $(q, a, b)$,
em que $q$ é o próprio estado $q$ de $E$,
$a$ é o que está debaixo do cabeçote da primeira fita
e $b$ é o que está debaixo do cabeçote da segunda fita
(lembre"=se de que não autorizaremos $M$ a consultar a fita de saída).
Se $1 \leq m \leq |x|$,
então $a = x[m]$.
Caso contrário, o cabeçote de leitura da máquina
estará fora da cadeia de entrada.
Interpretaremos a fita como sendo preenchida com brancos ($\B$)
onde não há outros caracteres;
portanto, neste caso, $a = \B$.
Similarmente, se $1 \leq n \leq |y|$, $b = y[n]$; caso contrário, $b = \B$.
Portanto,
a ação da máquina $M$ no estado $E$
será $\delta(q, a, b)$.
\begin{notation}
    \simbolo{$\delta(E)$}{Ação tomada no estado de computação $E$.}
    Denotaremos a ação da máquina $M = (Q, \Gamma, \delta, q_0)$
    no estado $E$ por $\delta(E)$.%
    \footnote{
        Observe que estamos abusando da notação de chamada de função aqui.
    }
\end{notation}
Note que,
caso $M$ for uma máquina não"=determinística,
$\delta(q, a, b)$ retornará um conjunto;
portanto, $\delta(E)$ será uma lista de ações,
não apenas uma.

Seja $E = (x, y, z, q, m, n)$ um estado de computação
e $A = (q', d_1, d_2, e, s)$ uma ação da máquina de Turing
(pode ser um valor retornado por $\delta$ se a máquina for determinística,
ou um dos elementos do conjunto retornado por $\delta$
se a máquina for não"=determinística).
O \emph{efeito de $A$ em $E$}
é a tupla $(x', y', z', q', m', n')$,
dada por
\begin{itemize}
    \setlength{\labelsep}{1ex}
    \item $x' = x$ (isto é, nunca alteraremos $x$).
    \item Se $n = 0$, $y' = y$
        (a máquina não pode escrever no primeiro branco da fita de trabalho);
        se $n = |y| + 1$, $y' = ye$
        (se a máquina estiver imediatamente à direita dos elementos da fita,
        estendemos a representação da fita com o caractere que a máquina escreveu);
        caso contrário,
        $y'$ é $y$ com o $n$"=ésimo simbolo trocado por $e$
        --- isto é, $y'[i] = y[i]$ se $i \neq n$, e $y'[n] = e$.
    \item $z' = zs$.
        Se $s$ for $0$ ou $1$,
        isso corresponde a escrever um destes caracteres na fita de saída.
        Se $s$ for $\varepsilon$,
        isso corresponde a manter o cabeçote na fita de saída intacto.
    \item $q'$ é o próprio $q'$ de $A$.
    \item Se $m = 0$ e $d_1 = -1$ (quer dizer, $d_1 = \leftarrow$),
        então $m' = 0$; caso contrário, $m' = m + d_1$.
        Não deixaremos o cabeçote de leitura ler mais de um branco
        à esquerda do início,
        mas permitiremos que ele ``navegue'' livremente à direita.
    \item Se $n = 0$ e $d_2 = -1$, então $n' = 0$; senão, $n' = n + d_2$.
        Também não permitiremos o cabeçote de trabalho ``navegar'' demais à esquerda,
        portanto, efetivamente,
        a fita é apenas infinita à direita.
\end{itemize}
\begin{notation}
    \simbolo{$A(E)$}{Efeito da ação $A$ no estado de computação $E$.}
    O efeito de $A$ em $E$ será denotado por $A(E)$.\footnote{
        Outro abuso de notação de chamada de função.
    }
\end{notation}

Usando estas duas noções,
podemos definir como uma máquina de Turing computa sobre uma palavra.

\begin{definition}
    Seja $M = (Q, \Gamma, \delta, q_0)$ uma máquina de Turing determinística
    e $x \in \{0, 1\}^*$ uma palavra.
    O $k$"=ésimo estado de computação de $M$ em $x$,
    denotado por $\mathbb E(M, x, k)$,
    é definido recursivamente por
    \begin{align*}
        \mathbb E(M, x, 0) &= (x, \varepsilon, \varepsilon, q_0, 1, 1); \\
        \mathbb E(M, x, k+1) &= A(\mathbb E(M, x, n)), \quad \text{
            se $A = \delta(\mathbb E(M, x, k))$.
        }
    \end{align*}
    Se, para algum $n$, $\delta(\mathbb E(M, x, k))$ não existir
    --- que é quando a tripla $(q, a, b)$ não pertencer ao domínio de $\delta$,
    na terminologia do parágrafo que define $\delta(E)$ ---
    então defina $\mathbb E(M, x, k+1) = \mathbb E(M, x, k)$.
\end{definition}
De acordo com a definição, quando a máquina para,
todos os estados futuros serão os mesmos.
De certa forma,
o contrário também acontece;
para que dois estados sejam iguais,
nenhum dos cabeçotes da máquina podem se mexer
(nem mesmo o de saída),
e a máquina também não pode escrever na sua fita de trabalho
um símbolo diferente do qual ela está lendo.
Podemos sistematicamente remover todas as transições que causam este ``travamento''
de uma máquina de Turing,
portanto não perdemos generalidade ao assumir que a máquina parou nestes casos.
Então,
podemos, finalmente,
definir formalmente $M(x)$ e $t(M, x, k)$.
\begin{definition}
    Se $\mathbb E(M, x, k) = (x, y, z, q, m, n)$,
    defina $t(M, x, k) = z$.
    Se existir algum $k_0$ tal que
    \begin{equation*}
        \mathbb E(M, x, k) = \mathbb E(M, x, k+1)
    \end{equation*}
    para todo $k \geq k_0$,
    defina $M(x) = t(M, x, k_0)$.
\end{definition}

Para máquinas de Turing não"=determinísticas,
não há apenas uma possível sequência de computação,
portanto, $\mathbb E$ deve ser um conjunto.
Entretanto,
isto cria um problema;
pode ser que hajam dois estados $E$ e $F$, no conjunto $\mathbb E(M, x, k)$,
que se sucedem mutuamente;
isto é, $A(E) = F$ para algum $A \in \delta(E)$,
e $B(F) = E$ para algum $B \in \delta(F)$.
Se $E$ e $F$ forem os únicos elementos de $\mathbb E(M, x, k)$
e $A$ e $B$ forem as únicas ações possíveis de $M$,
então $\mathbb E(M, x, k) = \mathbb E(M, x, k+1)$,
mesmo que $M$ não tenha \emph{parado} de computar em $x$.

Precisamos,
portanto,
ser mais cuidadosos ao definir transição e parada.

\begin{definition}
    Seja $M = (Q, \Gamma, \delta, q_0)$ uma máquina de Turing não"=determinística
    e $x \in \{0, 1\}^*$ uma palavra.
    $\mathbb A(M, x, k)$ é o $k$"=ésimo conjunto de estados
    \emph{ativos} de computação de $M$ em $x$,
    e $\mathbb T(M, x, k)$ é o de estados \emph{terminais} de computação.
    Estes dois conjuntos são mutuamente definidos recursivamente por
    \begin{align*}
        \mathbb A(M, x, 0) &= \{ (x, \varepsilon, \varepsilon, q_0, 1, 1) \}; \\
        \mathbb T(M, x, 0) &= \emptyset; \\
        \mathbb A(M, x, k+1) &= \{
            A(E) \mid E \in \mathbb A(M, x, k) \wedge
            A \in \delta(E)
        \} \\
        \mathbb T(M, x, k+1) &= \mathbb T(M, x, k) \cup \{
            E \in \mathbb A(M, x, k) \mid \text{$\delta(E)$ não está definido}
        \}.
    \end{align*}
\end{definition}

Observe que $\mathbb T(M, x, k+1)$ necessariamente inclui $\mathbb T(M, x, k)$;
portanto, $\mathbb T(M, x, k)$ conterá todos os ramos de computação
que já se encerraram.
Desta forma,
a união
\begin{equation*}
    \mathbb T(M, x, k) \cup \mathbb A(M, x, k)
\end{equation*}
corresponde a $\mathbb E(M, x, k)$, do caso determinístico.
Podemos,
portanto,
definir o símbolo $T(M, x, n)$ usando este conjunto.

\begin{definition}
    \begin{equation*}
        T(M, x, n) = \{
            z \mid (x, y, z, q, m, n) \in \mathbb A(M, x, n) \cup \mathbb T(M, x, n)
        \}
    \end{equation*}
\end{definition}

Tecnicamente,
o conjunto $\mathbb A(M, x, k)$ pode conter estados terminais de computação;
para um estado terminal $E$, $\delta(E)$ não está definido,
então $\mathbb A(M, x, k+1)$ não conterá estados da forma $A(E)$.
Além disso,
este estado $E$ estará em $\mathbb T(M, x, k+1)$;
portanto,
tecnicamente, $\mathbb A(M, x, k)$
contém os estados que estavam ativos no passo anterior,
mas que não necessariamente estarão ativos agora.

Caso todos os ramos de computação de $M$ em $x$ eventualmente se encerrem,
$\mathbb A(M, x, k)$ eventualmente ficará vazio;
diremos,
portanto,
que $M$ para ao computar $x$ se $\mathbb A(M, x, k_0) = \emptyset$
para algum $k_0$.

\begin{definition}
    Seja $M$ uma máquina de Turing não"=determinística.
    Se $\mathbb A(M, x, k_0) = \emptyset$,
    defina $M(x)$ como sendo o maior valor
    (lexicograficamente)
    de $T(M, x, k_0)$.
\end{definition}
(A ordenação lexicográfica usada neste texto
considera que palavras menores precedem palavras maiores,
independente de quais símbolos estejam envolvidos;
por exemplo, $1 < 000$.)

Esta definição é a formalização da definição~\ref{def:nondeterministic_function}.

\section{Codificação de máquinas de Turing}

O objetivo de definir uma codificação em binário para máquinas de Turing
é poder enumerar as funções recursivas que elas computam
e, de alguma forma,
manipulá"=las.
Portanto,
representaremos em nossa codificação apenas os aspectos ``mais importantes''
das máquinas de Turing.

Exclusivamente do ponto de vista de \emph{funções},
a única coisa que importa é o resultado final: $M(x)$.
Entretanto,
existem várias maneiras de se computar uma função,
e nós estamos usando medidas de complexidade para distinguir estas maneiras.
Portanto,
precisaremos ser um pouco mais conservadores antes de descartar alguma informação.
Em particular,
estaremos interessados em,
de alguma forma,
manter os estados de computação das máquinas.

Portanto, a quantidade de estados de uma máquina $M$
(isto é, $|Q|$)
é uma informação relevante;
entretanto,
\emph{quais} estados são utilizados não é.
Poderíamos assumir que $Q$ sempre é um conjunto da forma
\begin{equation*}
    Q = \{1, 2, \dots, n\},
\end{equation*}
e que $q_0 = 1$;
existe um mapeamento direto entre estados de computação para esta máquina
e estados de computação da máquina original,
de forma que a estrutura dos $\mathbb E$, $\mathbb A$ e $\mathbb T$ se mantenha.

Da mesma forma,
exatamente quais símbolos estão em $\Gamma$
não é relevante.
Ficamos apenas com $\delta$,
que é a única informação essencial da quádrupla $(Q, \Gamma, \delta, q_0)$.
É exatamente esta informação que codificaremos em $\langle M \rangle$.
Nossa construção é similar à de \citeonline[p.~182]{HopcroftUllman1979}.

\begin{definition}
    Seja $M = (Q, \Gamma, \delta, q_0)$ uma máquina de Turing.
    Identificaremos os elementos do conjunto $Q$
    com os números $\{1, 2, \dots, |Q|\}$,
    de forma que $q_0$ seja identificado com $1$.
    Da mesma forma,
    identificaremos os elementos de $\Gamma$
    com o conjunto $\{1, 2, \dots, |\Gamma|\}$
    de forma que o símbolo $0$ corresponda ao número $1$,
    o símbolo $1$ corresponda ao número $2$,
    e o símbolo $\B$ corresponda ao número $3$.
    Por fim, identifique $D$ e $\{0, 1, \varepsilon\}$ com $\{1, 2, 3\}$.
    Codifique a transição
    \begin{equation*}
        \delta(q, a, b) = (q', d_1, d_2, e, s)
    \end{equation*}
    por
    \begin{equation*}
        0^q 1 0^a 1 0^b 1 0^{q'} 1 0^{d_1} 0^{d_2} 1 0^e 1 0^s.
    \end{equation*}
    (Nós identificamos os conjuntos $Q$, $\Gamma$ etc.\ com subconjuntos de $\mathbb N$
    para que a notação $0^q$ faça sentido.
    Além disso,
    escolhemos os valores como começando em $1$
    para que sempre haja ao menos um dígito $0$ entre cada par de $1$'s.)

    Agora,
    defina
    \begin{equation*}
        \langle M \rangle = 11 c_1 11 c_2 11 \dots 11 c_n 11,
    \end{equation*}
    em que cada $c_i$ é uma codificação para uma transição,
    e todas as transições são codificadas em algum $c_i$.
    Se a máquina de Turing for não"=determinística,
    basta usar vários $c_i$ com o mesmo $(q, a, b)$.
\end{definition}

Rigorosamente falando,
usar o símbolo de igualdade na definição de $\langle M \rangle$
é um abuso de notação,
pois o lado direito pode se apresentar em várias ordens.
Entretanto,
qualquer que seja a ordem,
a máquina representada é a mesma,
portanto este abuso de notação (e nomenclatura) é tolerável.

Dessa forma,
definimos parte do mapeamento $x \mapsto f_x$,
pois $f_{\langle M \rangle}$ sempre representa a função que $M$ computa
(para as várias representações $\langle M \rangle$ de $M$).
Entretanto,
existem cadeias que não codificam máquina alguma,
de acordo com a definição acima;
para que o mapeamento fique completo,
para estas cadeias $x$,
defina $f_x$ como a função que não está definida para $x$ algum.
