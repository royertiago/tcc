\documentclass[utf8,notheorems]{beamer}
\usetheme[compress]{Singapore}

\usepackage[brazil]{babel}
\let\C\someundefinedcommand
\let\G\someundefinedcommand
% These two commands are defined by hyperref in file puenc.def,
% and conflict with complexity - which also defines them.
\usepackage{complexity}

\theoremstyle{definition}
\newtheorem*{definition}{Definição}

% Gambiarra para que o \Phi possua serifas horizontais,
% como o I em algumas fontes.
\DeclareSymbolFont{Gambiarrra}{U}{zeur}{m}{n}
\DeclareMathSymbol\Phi{\mathalpha}{Gambiarrra}{"08}

\begin{document}

\title{Blum Axioms and Nondeterministic Computation of Functions}
\author{Tiago Royer\inst{1} \and Jerusa Marchi\inst{1}}
\institute[UFSC]{
    \inst{1}
    Universidade Federal de Santa Catarina \\
    Departamento de Informática e Estatística
}
\date[XXXV CTIC]{
    XXXV Concurso de Trabalhos de Iniciação Científica \\
    Congresso da Sociedade Brasileira de Computação --- 2016
}

\begin{frame}
    \titlepage
\end{frame}

\begin{frame}
    \frametitle{Máquinas de Turing e funções}
    \begin{itemize}
        \item Problema de decisão $\rightarrow$ Linguagem
        \item Problema de busca $\rightarrow$ Função
        \item Tipicamente, define-se apenas o reconhecimento de linguagens. \\
    \end{itemize}

    Para provar indecidibilidade/intratabilidade:
    \begin{itemize}
        \item Problema de busca $\rightarrow$ Problema de decisão
            $\rightarrow$ Linguagem
    \end{itemize}
\end{frame}

\begin{frame}
    \frametitle{Cálculo não-determinístico de funções}
    \begin{itemize}
        \item E o cálculo de funções?
        \item Definição direta para máquinas determinísticas
        \item Útil, mesmo do ponto de vista teórico
            \begin{itemize}
                \item Usado na noção de redução
            \end{itemize}
        \item Mas como fazer uma máquina não-determinística \\
            calcular uma função?
    \end{itemize}
\end{frame}

\begin{frame}
    \frametitle{Definição de Hopcroft e Ullman}
    O valor da função computada por uma máquina não-determinística \\
    é $x$, para uma dada entrada, se
    \begin{itemize}
        \item algum ramo de computação devolve $x$, e
        \item nenhum ramo de computação que para devolve $x' \neq x$.
    \end{itemize}
    Caso contrário, deixe a função indefinida nesta entrada.
    \cite[p.~313]{HopcroftUllman1979}
\end{frame}

\begin{frame}
    \frametitle{Definição de Hopcroft e Ullman}
    Problema: podemos computar o complemento do problema da parada.
    \begin{itemize}
        \item Dados uma máquina $M$ e uma entrada $w$, \\
            faça dois ramos de computação:
            \begin{itemize}
                \item O primeiro escreve $1$ e para,
                \item O segundo executa $M$ em $w$ e, se $M$ parar,
                    escreve $0$ e para.
            \end{itemize}
        \item Se $M$ não parar em $w$,
            o resultado da computação é $1$;
        \item Se $M$ parar em $w$,
            o resultado da computação é indefinido.
    \end{itemize}
\end{frame}

\begin{frame}
    \frametitle{Enumerações de Gödel Aceitáveis}
    \begin{definition}
        Seja $\mathcal P$ o conjunto das funções parciais computáveis. \\
        Uma \emph{enumeração de Gödel aceitável} \\
        é uma função sobrejetora $\phi: \{0, 1\}^* \to \mathcal P$ que possui
        \begin{itemize}
            \item uma função computável $U$ tal que $U(w, x) = \phi_w(x)$ \\
                (Teorema da máquina universal); e
            \item uma função computável $\sigma$ tal que
                    $\phi_{\sigma(w, x)}(y) = \phi_w(x, y)$ \\
                (Teorema $S_{mn}$).
        \end{itemize}
        {\cite[p.~41]{Rogers1987}}
    \end{definition}
\end{frame}

\begin{frame}
    \frametitle{Enumerações de Gödel Aceitáveis --- Exemplos}
    \begin{itemize}
        \item Enumerações de máquinas de Turing
        \item Qualquer linguagem de programação moderna
        \item Contraexemplo: Definição de Hopcroft e Ullman
    \end{itemize}
\end{frame}

\begin{frame}
    \frametitle{Definição de Goldreich}
    Seja $\bot$ um símbolo fora do alfabeto. \\
    O valor da função computada por uma máquina não-determinística \\
    é $x$, para uma dada entrada, se
    \begin{itemize}
        \item algum ramo de computação devolve $x$, e
        \item todos os ramos param e devolvem $x$ ou $\bot$.
    \end{itemize}
    Caso contrário, deixe a função indefinida nesta entrada.
    \cite[p.~313]{HopcroftUllman1979}
\end{frame}

\begin{frame}
    \frametitle{Definição de Goldreich}
    Problema: máquinas que agem de acordo com esta definição \\
    podem ser facilmente invertidas;\\
    portanto, se alguma máquina não-determinística resolvesse $\SAT$ \\
    em tempo polinomial (de acordo com esta definição) \\
    poderíamos simplesmente inverter a saída, \\
    provando que $\NP = \coNP$.
\end{frame}

\begin{frame}
    \frametitle{Axiomas de Blum}
    \begin{definition}
        Dada uma enumeração aceitável $\phi$, \\
        uma \emph{medida de complexidade} é uma função \\
        $\Phi: \{0, 1\}^* \times \{0, 1\}^* \to \mathbb N$ tal que
        \begin{itemize}
            \item $\Phi(w, x)$ está definido se e só se $\phi_w(x)$ o está; e \\
            \item O predicado ``$\Phi(w, x) = k$'' é decidível.
        \end{itemize}
    \end{definition}
    \cite[p.~324]{Blum1967}

    (Classes de complexidade podem ser generalizadas diretamente)
\end{frame}

\begin{frame}
    \frametitle{Axiomas de Blum --- Exemplos}
    \begin{itemize}
        \item Tempo
        \item Espaço
        \item Número de mudanças de estado
        \item Número de vezes que uma célula da fita é alterada
        \item Definição de Goldreich \\
            (Problema: Classe análoga à $\NP$ não contém análogo de $\SAT$, \\
            a menos que $\NP = \coNP$.)
    \end{itemize}
\end{frame}

\begin{frame}
    \frametitle{Definição proposta}
    $\SAT$ pode ser reconhecido não-deterministicamente \\
    ``chutando'' uma atribuição de valores verdade. \\
    Altere este algoritmo para, em vez de aceitar/rejeitar, \\
    retornar $1$ e $0$, respectivamente.
    \begin{itemize}
        \item Fórmulas tautológicas devolve apenas $1$
        \item Fórmulas contraditórias devolve apenas $0$
        \item Fórmulas satisfazíveis, mas não tautológicas, \\
            devolverão tanto $0$ quanto $1$.
    \end{itemize}
    A resposta certa exatamente ao máximo dentre os valores devolvidos
\end{frame}

\begin{frame}
    \frametitle{Definição proposta}
    \begin{definition}
        Se todos os ramos de computação de $M$ em $w$ param, \\
        o valor da função é o máximo dentre todos os valores devolvidos. \\
        Se algum ramo não para, deixe a função de $M$ indefinida em $w$.
    \end{definition}
\end{frame}

\begin{frame}
    \frametitle{Definições de Krentel e Valiant}
    \begin{itemize}
        \item Aplicar operadores associativos à lista dos valores devolvidos
        \item Classe $\OptP$, de Krentel: Máximo/Mínimo \cite{Krentel1988}
        \item Classe $\#\P$, de Valiant: Cardinalidade \cite{Valiant1979}
    \end{itemize}
\end{frame}

\begin{frame}
    \frametitle{Considerações finais}
    \begin{itemize}
        \item Algumas definições de cálculo não-determinístico de funções
            que não são ``razoáveis'' ou ``eficientes''
        \item É possível formalizar a exigência de ``razoabilidade''
            e ``eficiência''
        \item É possível definir como uma máquina não-determinística
            pode calcular uma função, satisfazendo estas exigências.
    \end{itemize}
\end{frame}

\begin{frame}[allowframebreaks]
    \frametitle{Bibliografia}
    \bibliographystyle{plain}
    \bibliography{bib/bibliography}
\end{frame}

\begin{frame}
    \titlepage
\end{frame}

\end{document}
