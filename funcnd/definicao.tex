\section{DEFINIÇÃO}

Se $M$ é uma máquina de Turing determinística que computa a função $f$,
para descobrir o valor de $f(x)$,
simplesmente rodamos $M$ em $x$.
Existe apenas uma sequência de passos possível antes de $M$ parar;
o valor que $M$ deixar na fita é $f(x)$.

Para máquinas não"=determinísticas,
ao rodar $M$ numa entrada $x$ arbitrária,
podemos ver vários ramos de computação,
e cada ramo pode encerrar com um valor diferente na fita.
Então, em vez de termos um único valor para $M(x)$,
existe um conjunto de valores possíveis.
Não queremos um conjunto, e sim, um único valor.
Adotaremos uma regra arbitrária para extrair um valor deste conjunto,
que será justificada posteriormente.

\begin{definition}[Função não"=determinística\footnotemark]
    \label{def:nondeterministic_function}
    \footnotetext{
        O termo ``função não"=determinística''
        está tecnicamente incorreto.
        Funções não são determinísticas ou não"=determinísticas;
        o que é determinístico ou não
        são os dispositivos computacionais que as calculam.
        Toleraremos este abuso de nomenclatura
        para encurtar o texto.
    }
    Seja $M$
    uma máquina de Turing não"=determinística
    e $x$ uma palavra.
    Se a árvore de computação que $M$ gera ao processar $x$ for finita
    (isto é, nenhuma sequência de transições de $M$ leva a um loop infinito),
    definiremos $M(x)$ como sendo o maior valor que a fita atinge
    nas folhas da árvore.
    Se a árvore não for finita, deixaremos $M(x)$ indefinido.

    Caso a máquina gere palavras em vez de números,
    consideraremos o ordenamento lexicográfico,
    tratando sempre palavras menores como precedendo palavras maiores.
    Desta forma, a palavra vazia $\epsilon$
    é lexicograficamente menor do que todas as outras.
    Esta suposição facilitará o projeto de linguagens adiante.
\end{definition}

A ideia de pegar o máximo do conjunto vem da analogia com funções booleanas.
Podemos interpretar um decisor para uma linguagem $L \subseteq \Sigma^*$
como uma máquina que computa a função característica de $L$.
Isto é,
se $M$ decide a linguagem $L$,
é como se $M$ computasse a função $f : \Sigma^* \rightarrow \{0, 1\}$
definida por
\begin{equation*}
    f(x) =
    \begin{cases}
        1,& \text{ se } x \in L; \\
        0,& \text{ se } x \notin L.
    \end{cases}
\end{equation*}
De fato, podemos reescrever a máquina $M$
para que escreva $1$ na fita antes de aceitar a entrada
e $0$ antes de rejeitá"=la.

Uma máquina não"=determinística para $\SAT$,
por exemplo,
pode chutar uma atribuição de valores"=verdade
e escrever $1$ na fita se aquela atribuição satisfaz à fórmula
e $0$ caso contrário.
Intuitivamente, os ramos que escreveram $1$ são aqueles em que
a máquina obteve sucesso em provar que a instância pertence à linguagem,
e os ramos que escreveram $0$ são aqueles em que a máquina fracassou.

Uma instância insatisfazível resultaria no conjunto $\{0\}$;
uma instância tautológica retornaria o conjunto $\{1\}$,
e as instâncias satisfazíveis, mas não tautológicas
ficam no meio"=termo: $\{0, 1\}$.
A função característica de $\SAT$ retorna $1$ nos dois últimos casos
e $0$ no primeiro;
corresponde, exatamente,
ao maior valor destes conjuntos.

A exigência de a árvore de computação ter tamanho finito é importante.
Caso apenas exigíssemos que o conjunto de folhas fosse finito, por exemplo,
poderíamos construir uma máquina não"=determinística
que compute a função característica do problema da parada:

Basta que a máquina assuma dois ramos de computação.
No o primeiro ramo, a máquina sempre escreve $0$ na fita e para.
No segundo ramo,
a máquina age como uma máquina universal e simula a entrada.
Se a máquina simulada parar, escreva $1$ na fita e pare também.

Quando alimentada com o par $\langle M, x \rangle$ na entrada,
caso $M$ pare ao computar $x$,
o ``conjunto retornado'' por nossa máquina não"=determinística é $\{0, 1\}$;
$0$ por causa do primeiro ramo (que sempre retorna $0$)
e $1$ por causa do segundo ramo.
Caso $M$ nunca pare, apenas o primeiro ramo retornará,
resultando no conjunto $\{0\}$.
Como o resultado da função é o máximo destes conjuntos,
temos, exatamente,
a função característica do problema da parada.

Entretanto, como exigimos que a árvore de computação seja finita,
podemos demonstrar equivalência entre os dois modelos,
assim como pudemos mostrar equivalência entre determinismo e não"=determinismo
para decisores.%
\footnote{
    O termo ``equivalência'' é utilizado aqui no sentido de poder computacional,
    não eficiência;
    isto é, a classe de funções computadas por estes dispositivos
    (máquinas determinísticas e não"=determinísticas)
    é o mesmo,
    embora as máquinas não"=determinísticas sejam (aparentemente)
    muito mais rápidas.
}

\begin{theorem}
    Toda função calculada por uma máquina não"=determinística
    é calculada por uma máquina determinística
    e vice"=versa.
\end{theorem}
\begin{proof}
    Suponha que $M$ é não"=determinística e calcula certa função.
    Para calcular a mesma função usando uma máquina determinística,
    enumere todas as sequências finitas de transições de $M$,
    ordenando por tamanho;
    isto é,
    primeiro todas as sequências de tamanho $0$,
    depois todas as de tamanho $1$,
    depois todas as de tamanho $2$,
    e assim por diante.

    Para cada sequência de transições,
    rode $M$ na entrada de acordo com a sequência.
    Se $M$ parar,
    compare o que aquele ramo retornou
    com o maior valor obtido até agora
    --- caso seja o primeiro valor retornado,
    será este o maior valor
    --- e mantenha apenas o maior dentre estes valores.

    Quando todas as sequências de transições de um mesmo tamanho retornarem,
    teremos chegado ao fim da árvore;
    limpe a fita, mantendo apenas o maior valor encontrado no percurso.
    Este valor corresponde ao maior valor que qualquer ramo da árvore retorna,
    o que corresponde ao valor da função computada por $M$,
    caso esteja definido.

    Se a função não está definida na entrada atual,
    há dois casos.
    Se a árvore é infinita,
    nossa máquina determinística jamais parará de enumerar transições,
    pois existe ao menos uma sequência de transições que nunca para.
    Se a árvore é finita,
    mas nenhum ramo finaliza a computação
    (por exemplo, faz uma transição inválida),
    o ``maior valor que qualquer ramo retorna''
    estará indefinido assim que chegarmos ao fim da árvore.
    Basta fazer uma transição inválida também neste caso.

    A volta é trivial.
\end{proof}

Usando as mesmas técnicas usadas para máquinas determinísticas,
podemos construir uma máquina não"=determinística universal
e o teorema $S_{mn}$;
portanto, acabamos de provar o seguinte teorema
(e, efetivamente,
concluir o exemplo~\ref{ex:nondeterministic_complexity}).

\begin{ucorollary}
    Seja $\phi$ a função que associa a cada $w = \langle M \rangle$
    (em que $M$ é não"=determinística)
    a função $\phi_w$ dada por
    \begin{equation*}
        \phi_w(x) \simeq M(x).
    \end{equation*}
    Então, $\phi$ é uma enumeração de Gödel aceitável.
\end{ucorollary}
