\section{Definição}

Se $M$ é uma máquina de Turing determinística que computa a função $f$,
para descobrir o valor de $f(x)$,
simplesmente rodamos $M$ em $x$.
Existe apenas uma sequência de passos possível antes de $M$ parar;
o valor que $M$ deixar na fita é $f(x)$.

Para máquinas não"=determinísticas,
ao rodar $M$ numa entrada $x$ arbitrária,
podemos ver vários ramos de computação,
e cada ramo pode encerrar com um valor diferente na fita.
Então, em vez de termos um único valor para $M(x)$,
existe um conjunto de valores possíveis.
Não queremos um conjunto, e sim, um único valor.
Adotaremos uma regra arbitrária para extrair um valor deste conjunto,
que será justificada posteriormente.

\begin{definition}[Função não"=determinística]
    \footnote{
        O termo ``função não"=determinística''
        está tecnicamente incorreto.
        Funções não são determinísticas ou não"=determinísticas;
        o que é determinístico ou não
        é o dispositivo computacional que a calcula.

        Toleraremos este abuso de nomenclatura
        para encurtar o texto.
    }
    Seja $M$
    uma máquina de Turing não"=determinística
    e $x$ uma palavra.
    Se a árvore de computação que $M$ gera ao processar $x$ for finita
    (isto é, nenhuma sequência de transições de $M$ leva a um loop infinito),
    definiremos $M(x)$ como sendo o maior valor que a fita atinge
    nas folhas da árvore.
    Caso a árvore não seja finita, deixaremos $M(x)$ indefinido.
\end{definition}

A ideia de pegar o máximo do conjunto vem da analogia com funções booleanas.
Podemos interpretar um decisor para uma linguagem $L \subseteq \Sigma^*$
como uma máquina que computa a função característica de $L$.
Isto é,
se $M$ decide a linguagem $L$,
é como se $M$ computasse a função $f : \Sigma^* \rightarrow \{0, 1\}$
definida por
\begin{equation*}
    f(x) =
    \begin{cases}
        1,& \text{ se } x \in L; \\
        0,& \text{ se } x \notin L.
    \end{cases}
\end{equation*}
De fato, podemos reescrever a máquina $M$
para que escreva $1$ na fita antes de aceitar a entrada
e $0$ antes de rejeitá"=la.

Uma máquina não"=determinística para $\SAT$,
por exemplo,
pode chutar uma atribuição de valores"=verdade
e escrever $1$ na fita se aquela atribuição satisfaz à fórmula
e $0$ caso contrário.
Intuitivame, os ramos que escreveram $1$ são aqueles em que
a máquina obteve sucesso em provar que a instância pertence â linguagem,
e os ramos que escreveram $0$ são aqueles em que a máquina fracassou.

Uma instância insatisfazível resultaria no conjunto $\{0\}$;
uma instância tautológica retornaria o conjunto $\{1\}$,
e as instâncias satisfazíveis, mas não tautológicas
ficam no meio"=termo: $\{0, 1\}$.
A função característica de $\SAT$ retorna $1$ nos dois últimos casos
e $0$ no primeiro;
corresponde, exatamente,
ao maior valor destes conjuntos.
