\section{Composição de Funções}

Trabalhar com funções computadas por máquinas de Turing determinísticas
nos permite implementar a composição destas funções de maneira muito simples:
basta ``concatenar'' as duas máquinas.
Assim que a primeira máquina parar,
o cabeçote de leitura é movido para o começo da fita
e a segunda máquina assume o controle.
Nesta seção investigaremos a composição de funções não"=determinísticas.

\begin{definition}[Fecho sob composição]
    \simbolo{$\mathcal F^\circ$}{Fecho compositivo de $\mathcal F$}
    Dado um conjunto $\mathcal F$ de funções de $\{0, 1\}^*$ em $\{0, 1\}^*$,
    o \emph{fecho compositivo} de $\mathcal F$,
    denotado por $\mathcal F^\circ$,
    é o conjunto de todas as funções que podem ser obtidas
    através da composição de um número finito de funções de $\mathcal F$.
    Isto é,
    \begin{equation*}
        \mathcal F^\circ = \{g \mid
            \text{Existem funções $f_1, f_2, \dots, f_n$ em $\mathcal F$
                tais que $g = f_1, f_2, \dots, f_n$
            }
        \}.
    \end{equation*}
\end{definition}
