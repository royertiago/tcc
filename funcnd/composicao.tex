\section{Composição de Funções}

Trabalhar com funções computadas por máquinas de Turing determinísticas
nos permite implementar a composição destas funções de maneira muito simples:
basta ``concatenar'' as duas máquinas.
Assim que a primeira máquina parar,
o cabeçote de leitura é movido para o começo da fita
e a segunda máquina assume o controle.
Nesta seção investigaremos a composição de funções não"=determinísticas.

\begin{definition}[Fecho sob composição]
    \simbolo{$\mathcal F^\circ$}{Fecho compositivo de $\mathcal F$}
    Dado um conjunto $\mathcal F$ de funções de $\{0, 1\}^*$ em $\{0, 1\}^*$,
    o \emph{fecho compositivo} de $\mathcal F$,
    denotado por $\mathcal F^\circ$,
    é o conjunto de todas as funções que podem ser obtidas
    através da composição de um número finito de funções de $\mathcal F$.
    Isto é,
    \begin{equation*}
        \mathcal F^\circ = \{g \mid
            \text{Existem funções $f_1, f_2, \dots, f_n$ em $\mathcal F$
                tais que $g = f_1 \circ f_2 \circ \dots \circ f_n$
            }
        \}.
    \end{equation*}
\end{definition}

Se entendermos que a composição de zero funções resulta
na função identidade,
podemos construir esta operação da mesma forma como construímos o fecho de Kleene:

Defina $\mathcal F^{\circ 0} = \{I\}$,
em que $I$ é a função de identidade.
Defina
\begin{equation*}
    \mathcal F^{\circ n+1} = \{ f \circ g \mid
        f \in \mathcal F, g \in \mathcal F^{\circ n}
    \}.
\end{equation*}

Assim, temos
\begin{equation*}
    \mathcal F^\circ = \bigcup_{n \in \mathbb N} \mathcal F^{\circ n}.
\end{equation*}

Esta operação também possui propriedades análogas às do fecho de Kleene.

\begin{proposition}
    Se $\mathcal F$ e $\mathcal G$ são conjuntos de funções, então
    \begin{align*}
        (\mathcal F^\circ)^\circ &= \mathcal F^\circ \\
        \mathcal F &\subseteq \mathcal F^\circ \\
        (\mathcal F^\circ \cup \mathcal G^\circ)^\circ &=
            (\mathcal F \cup \mathcal G)^\circ.
    \end{align*}
    Além disso, se $\mathcal F \subseteq \mathcal G$,
    então
    \begin{equation*}
        \mathcal F^\circ \subseteq \mathcal G^\circ.
    \end{equation*}
\end{proposition}

\begin{proof}
    Todas as funções de $(\mathcal F^\circ)^\circ$
    são composições de funções de $\mathcal F^\circ$,
    que, por sua vez,
    são composições de funções de $\mathcal F$;
    portanto, as funções de $(\mathcal F^\circ)^\circ$
    são composições de funções de $\mathcal F$,
    o que prova a primeira fórmula.

    Para a segunda fórmula,
    basta escolher $n = 1$ na definição de fecho compositivo.

    Para a terceira,
    observe que as funções de $(\mathcal F^\circ \cup \mathcal G^\circ)^\circ$
    são composições de funções de $\mathcal F^\circ$ e $\mathcal G^\circ$,
    possivelmente intercaladas;
    portanto, usando o mesmo raciocínio do primeiro parágrafo,
    concluímos que elas são composições de funções de
    $\mathcal F \cup \mathcal G$.

    E, para a última,
    note que todas as funções de $\mathcal F^\circ$
    são composições de funções de $\mathcal F$,
    portanto também de $\mathcal G$;
    isto é, todas as funções de $\mathcal F^\circ$
    também estão em $\mathcal G$.
\end{proof}

Ao compor duas funções calculadas por máquinas de Turing determinísticas,
o tempo de execução da máquina resultante
será proporcional à soma dos tempos de execução das máquinas originais.
Portanto,
caso ambas as funções estejam em $\FP$,
o tempo de computação resultante também será limitado por um polinômio,
resultando em outra função de $\FP$.
Portanto,
a classe $\FP$ é fechada sob composição
($\FP^\circ = \FP$).

Ao equipar as máquinas com oráculos,
encontramos um pequeno problema na hora de concatenar a máquina.
Caso os oráculos sejam iguais,
a concatenação descrita anteriormente funciona,
pois a máquina resultante usará o mesmo oráculo das máquinas originais.
Caso os oráculos sejam diferentes,
não podemos simplesmente emendar as duas máquinas
pois esta nova máquina precisaria de dois oráculos;
mas podemos fazer a coisa funcionar sob as condições certas.

\begin{theorem}
    Se $\mathcal A$ possui um problema polinomialmente completo,
    então $\FP^\mathcal A$ é fechado sob composição.
\end{theorem}

\begin{proof}
    Basta unificar os oráculos usando o problema completo de $\mathcal A$.
    Seja $A$ o problema completo de $\mathcal A$,
    e suponha que $g$ é a composição
    \begin{equation*}
        f_1 \circ f_2 \circ \dots \circ f_n,
    \end{equation*}
    para $f_i \in \FP^\mathcal A$.

    Tome máquinas $M_1, \dots, M_n$ que computam,
    respectivamente, $f_1, \dots, f_n$,
    e substitua seus oráculos pelo problema $A$;
    sabemos que isso é possível pois $A$ é polinomialmente completo para $\mathcal A$.
    Agora,
    basta concatenar as máquinas.
    A substituição do oráculo manteve todas as máquinas $M_i$
    operando em tempo polinomial,
    portanto a máquina resultante também operará em tempo polinomial;
    isso mostra que $g \in \FP^\mathcal A$,
    o que prova que $\FP^\mathcal A$ é fechado sob composição.
\end{proof}

\begin{ucorollary}
    $\Delta_n^f$ é fechado sob composição.
\end{ucorollary}
