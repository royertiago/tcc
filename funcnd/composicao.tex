\section{COMPOSIÇÃO DE FUNÇÕES}
\label{sec:function_composition}

Trabalhar com funções computadas por máquinas de Turing determinísticas
nos permite implementar a composição destas funções de maneira muito simples:
basta ``concatenar'' as duas máquinas.
Assim que a primeira máquina parar,
o cabeçote de leitura é movido para o começo da fita
e a segunda máquina assume o controle.
Nesta seção investigaremos a composição de funções não"=determinísticas.

\begin{definition}[Fecho sob composição]
    \simbolo{$\mathcal F^\circ$}{Fecho compositivo de $\mathcal F$}
    Dado um conjunto $\mathcal F$ de funções de $\{0, 1\}^*$ em $\{0, 1\}^*$,
    o \emph{fecho compositivo} de $\mathcal F$,
    denotado por $\mathcal F^\circ$,
    é o conjunto de todas as funções que podem ser obtidas
    através da composição de um número finito de funções de $\mathcal F$.
    Isto é,
    \begin{equation*}
        \mathcal F^\circ = \{g \mid
            \text{Existem funções $f_1, f_2, \dots, f_n$ em $\mathcal F$
                tais que $g = f_1 \circ f_2 \circ \dots \circ f_n$
            }
        \}.
    \end{equation*}
\end{definition}

Se entendermos que a composição de zero funções resulta
na função identidade,
podemos construir esta operação da mesma forma como construímos o fecho de Kleene:

Defina $\mathcal F^{\circ 0} = \{I\}$,
em que $I$ é a função de identidade.
Defina
\begin{equation*}
    \mathcal F^{\circ n+1} = \{ f \circ g \mid
        f \in \mathcal F, g \in \mathcal F^{\circ n}
    \}.
\end{equation*}

Assim, temos
\begin{equation*}
    \mathcal F^\circ = \bigcup_{n \in \mathbb N} \mathcal F^{\circ n}.
\end{equation*}

Esta operação também possui propriedades análogas às do fecho de Kleene.

\begin{proposition}
    Se $\mathcal F$ e $\mathcal G$ são conjuntos de funções, então
    \begin{align*}
        (\mathcal F^\circ)^\circ &= \mathcal F^\circ \\
        \mathcal F &\subseteq \mathcal F^\circ \\
        (\mathcal F^\circ \cup \mathcal G^\circ)^\circ &=
            (\mathcal F \cup \mathcal G)^\circ.
    \end{align*}
    Além disso, se $\mathcal F \subseteq \mathcal G$,
    então
    \begin{equation*}
        \mathcal F^\circ \subseteq \mathcal G^\circ.
    \end{equation*}
    \label{thm:compositive_closure_properties}
\end{proposition}

\begin{proof}
    Todas as funções de $(\mathcal F^\circ)^\circ$
    são composições de funções de $\mathcal F^\circ$,
    que, por sua vez,
    são composições de funções de $\mathcal F$;
    portanto, as funções de $(\mathcal F^\circ)^\circ$
    são composições de funções de $\mathcal F$,
    o que prova a primeira fórmula.

    Para a segunda fórmula,
    basta escolher $n = 1$ na definição de fecho compositivo.

    Para a terceira,
    observe que as funções de $(\mathcal F^\circ \cup \mathcal G^\circ)^\circ$
    são composições de funções de $\mathcal F^\circ$ e $\mathcal G^\circ$,
    possivelmente intercaladas;
    portanto, usando o mesmo raciocínio do primeiro parágrafo,
    concluímos que elas são composições de funções de
    $\mathcal F \cup \mathcal G$.

    E, para a última,
    note que todas as funções de $\mathcal F^\circ$
    são composições de funções de $\mathcal F$,
    portanto também de $\mathcal G$;
    isto é, todas as funções de $\mathcal F^\circ$
    também estão em $\mathcal G$.
\end{proof}

Ao compor duas funções calculadas por máquinas de Turing determinísticas,
o tempo de execução da máquina resultante
será proporcional à soma dos tempos de execução das máquinas originais.
Portanto,
caso ambas as funções estejam em $\FP$,
o tempo de computação resultante também será limitado por um polinômio,
resultando em outra função de $\FP$.
Portanto,
a classe $\FP$ é fechada sob composição
($\FP^\circ = \FP$).

Ao equipar as máquinas com oráculos,
encontramos um pequeno problema na hora de concatenar a máquina.
Caso os oráculos sejam iguais,
a concatenação descrita anteriormente funciona,
pois a máquina resultante usará o mesmo oráculo das máquinas originais.
Caso os oráculos sejam diferentes,
não podemos simplesmente emendar as duas máquinas
pois esta nova máquina precisaria de dois oráculos;
mas podemos fazer a coisa funcionar sob as condições certas.

\begin{theorem}
    Se $\mathcal A$ possui um problema polinomialmente completo,
    então $\FP^\mathcal A$ é fechado sob composição.
    \label{thm:fp_closed_under_composition}
\end{theorem}

\begin{proof}
    Basta unificar os oráculos usando o problema completo de $\mathcal A$.
    Seja $A$ o problema completo de $\mathcal A$,
    e suponha que $g$ é a composição
    \begin{equation*}
        f_1 \circ f_2 \circ \dots \circ f_n,
    \end{equation*}
    para $f_i \in \FP^\mathcal A$.

    Tome máquinas $M_1, \dots, M_n$ que computam,
    respectivamente, $f_1, \dots, f_n$,
    e substitua seus oráculos pelo problema $A$;
    sabemos que isso é possível pois $A$ é polinomialmente completo para $\mathcal A$.
    Agora,
    basta concatenar as máquinas.
    A substituição do oráculo manteve todas as máquinas $M_i$
    operando em tempo polinomial,
    portanto a máquina resultante também operará em tempo polinomial;
    isso mostra que $g \in \FP^\mathcal A$,
    o que prova que $\FP^\mathcal A$ é fechado sob composição.
\end{proof}

\begin{ucorollary}
    $\Delta_n^f$ é fechado sob composição.
\end{ucorollary}

Para funções calculadas por máquinas não"=determinísticas,
não podemos meramente emendar as duas máquinas
pois,
em alguns ramos de computação,
a segunda máquina pode estar operando com algum valor menor do que máximo
--- que é o valor real da função.

Para $\FNP$, por exemplo,
podemos calcular as funções de $\FNP^\circ$
usando o poder computacional de $\FP^\FNP$:
como $\FNP$ possui problemas completos,
nós podemos emendar máquinas determinísticas
que somente chamam seus oráculos de $\FNP$
para calcular a função composta.
Portanto,
\begin{equation*}
    \FNP^\circ \subseteq \FP^\FNP.
\end{equation*}

De fato,
podemos fortalecer este ``$\subseteq$'' para um ``$=$''.

\begin{theorem}
    Se $\mathcal A$ possui problemas polinomialmente completos,
    então
    \begin{equation*}
        (\FNP^\mathcal A)^\circ = \FP^{\FNP^\mathcal A}
    \end{equation*}
    \label{thm:compositive_closure}
\end{theorem}

\begin{proof}
    Provar que o primeiro conjunto está contido no segundo é fácil.
    Pelo teorema~\ref{thm:polinomially_complete_problems},
    o conjunto $\FNP^\mathcal A$
    possui problemas polinomialmente completos;
    portanto,
    pelo teorema~\ref{thm:fp_closed_under_composition},
    $\FP^{\FNP^\mathcal A}$ é fechado sob composição.
    Agora, como
    \begin{equation*}
        \FNP^\mathcal A \subseteq \FP^{\FNP^\mathcal A},
    \end{equation*}
    pela quarta propriedade da proposição~\ref{thm:compositive_closure_properties},
    \begin{equation*}
        (\FNP^\mathcal A)^\circ \subseteq
            \left(\FP^{\FNP^\mathcal A}\right)^\circ
            = \FP^{\FNP^\mathcal A}.
    \end{equation*}

    Para a inclusão contrária,
    usaremos um caminho indireto.
    Demonstraremos primeiro que algum problema completo de
    $\FP^{\FNP^\mathcal A}$
    pode ser escrito como a composição de dois problemas de $\FNP^\mathcal A$.
    Chame de $f$ este problema.
    Como $f$ é polinomialmente completo para $\FP^{\FNP^\mathcal A}$,
    para toda função $g \in \FP^{\FNP^\mathcal A}$ existe uma função $h \in FP$
    tal que $g = f \circ h$.
    Como $h \in \FNP^\mathcal A$ e $f \in \left(\FNP^\mathcal A\right)^\circ$,
    a composição $f \circ h = g$
    também estará em $\left(\FNP^\mathcal A\right)^\circ$.

    Tiraremos o problema $f$
    da demonstração do teorema~\ref{thm:polinomially_complete_problems}.
    Escolha
    \newcommand{\HaltFP}{{\lang{HaltFP}}}
    \begin{equation*}
        f = \HaltFP^{\FNP^\mathcal A},
    \end{equation*}
    em que $\HaltFP^{\FNP^\mathcal A}$ é definida por
    \begin{equation*}
        \HaltFP^{\FNP^\mathcal A}(\langle M, x, n \rangle) = t(M^A, x, n),
    \end{equation*}
    em que $A$ é um problema completo para $\FNP^\mathcal A$.
    Observe que $f$ é,
    nos termos do teorema~\ref{thm:polinomially_complete_problems},
    $\HaltFP^\mathcal B$
    para $\mathcal B = \FNP^\mathcal A$.
    $A$ é um problema polinomialmente completo para $\mathcal B$,
    portanto,
    pelo mesmo teorema,
    $f$ é polinomialmente completo para $\FP^\mathcal B = \FP^{\FNP^\mathcal A}$.

    De posse do problema completo $f$,
    o próximo passo é escrevê"=lo como uma composição
    de duas funções de $\FNP^\mathcal A$.
    Estas duas funções serão computadas por máquinas
    que usam o problema $A$,
    que é completo para $\FNP^\mathcal A$.

    Dada a entrada $\langle M, x, n \rangle$,
    o problema $f$ consiste em simular a máquina $M^A$ por $n$ estados,
    na entrada $x$.
    $M$ é determinística, portanto a mera simulação é fácil de ser feita,
    usando recursos de $\FNP^\mathcal A$.
    O problema são as chamadas ao oráculo $A$,
    que é polinomialmente completo para $\FNP^\mathcal A$.
    Como este oráculo representa uma função não"=determinística,
    não podemos simplesmente computar $A$
    e prosseguir a simulação com o que quer que tenhamos obtido,
    pois o valor obtido neste ramo
    pode não ser o valor correto;
    usando apenas recursos de $\FNP^\mathcal A$,
    só temos acesso à maximização global
    no final de todas as etapas de computação.
    Então, precisaremos usar esta ``pós"=maximização''
    para obter esta maximização intermediária.

    Seja $p$ um polinômio que limita o tamanho de $A(x)$;
    isto é, $|A(x)| \leq p(|x|)$ para todo $x$.
    Sabemos que tal polinômio existe
    pois $A \in \FNP^\mathcal A$,
    e uma máquina de $\FNP^\mathcal A$ só tem ``tempo''
    de escrever uma quantidade polinomial de bits na saída.

    Se a máquina $M$ faz $k$ invocações ao oráculo $A$ na entrada $x$
    nos primeiros $n$ movimentos,
    enumere os valores chamados por $y_1, y_2, \dots, y_k$.
    Defina
    \begin{equation*}
        Y(M, x, i, n) = 0^{|A(y_i)|} 1 A(y_i)\ 0^{2*p(n)+1 - |A(y_i)|};
    \end{equation*}
    ou seja, $Y(M, x, i)$ é a resposta da $i$"=ésima invocação ao oráculo ($A(y_i)$),
    mas codificada num formato especial:
    a resposta $A(y_i)$ é sucedida por uma cadeia de zeros de tamanho variável,
    fazendo com que todos os $Y(M, x, i, n)$ sejam de tamanho $2*p(n) + 1$;
    além disso, a resposta está precedida de $0^{|A(y_i)|}1$,
    de forma que podemos recuperar $A(y_i)$
    usando a função $d$ de decodificação.
    (Observe que todas as perguntas feitas ao oráculo
    terão tamanho menor ou igual a $n$,
    portanto todas as respostas terão tamanho menor ou igual a $p(n)$.
    O $p(n)+1$ adicional é para englobar o prefixo inicial para a função $d$.)
    Defina $Y(M, x, i, n) = 0^{2*p(|x|)+1}$ caso $i > k$.

    É importante notar que esta codificação
    preserva a ordem lexicográfica de todas as palavras
    de tamanho menor ou igual a $p(n)$.

    Nossa primeira função $F$,
    dentre as duas que serão compostas para formar $f$,
    é definida por
    \begin{equation*}
        F(\langle M, x, n \rangle) = \langle M, x, n \rangle
            Y(M, x, 1, n) Y(M, x, 2, n) \cdots Y(M, x, n, n).
    \end{equation*}
    Isto é,
    $F$ retorna as respostas a todas as perguntas feitas pela máquina $M$ ao oráculo $A$
    ao rodar $n$ passos sobre $x$
    --- além da entrada inicial $\langle M, x, n \rangle$, é claro.
    Mostraremos que $F \in \FNP^\mathcal A$.
    (A entrada inicial $\langle M, x, n \rangle$
    é incluída pois precisaremos propagar esta informação para a segunda função
    que comporemos com $F$,
    mas, por ora,
    ignoraremos a existência deste ``pedacinho'' de $F$ na argumentação.)

    A máquina não"=determinística que calcula esta função
    usará exatamente a ``pós"=maximização'' global,
    que ela possui por calcular uma função não"=determinística,
    para conseguir maximizações locais nas perguntas ao oráculo $A$.
    Entretanto,
    como $F$ e $A$ estão no mesmo ``nível de complexidade''
    (isto é, ambas as funções estão em $\FNP^\mathcal A$),
    não podemos usar $A$ como oráculo;
    aí entrará a notação de $Y$.

    Chame de $M_A$ a máquina não"=determinística que calcula $A$,
    e $M_F$ a máquina que calcula $F$
    (construiremos $M_F$ agora).
    $M_F$ simplesmente simula a execução de $M$ em $x$.
    Quando $M$ faz uma pergunta ao oráculo,
    $M_F$ age como $M_A$,
    gerando vários ramos de computação.
    (Alguns levarão à resposta correta $A(y_1)$,
    mas outros não.)
    Independente de $M_F$ ter acertado o valor de $M_A$ ou não,
    $M_F$ escreverá a resposta que acabou de achar na fita de saída,
    deixando"=a na formatação dos $Y(M, x, i, n)$.
    Então, $M_F$ retorna o controle a $M$,
    imaginando que a resposta ao oráculo está no ramo atual da fita,
    e prossegue a computação.
    Quando a quantidade de passos passar de $n$,
    basta preencher os $Y$ faltantes com zeros.

    Em outras palavras,
    cada ramo de computação de $M_F$
    retornará uma lista de palavras $w_1 w_2 \dots w_n$,
    todas elas com tamanho $2*p(n)+1$.
    $d(w_1)$ é um candidato a $A(y_i)$;
    De fato, $d(w_1)$ é uma das folhas da computação de $M_A$ em $y_1$.
    Caso $d(w_1)$ seja a resposta certa,
    a simulação feita por $M_F$
    coincidirá com a computação de $M$ em $x$,
    e a próxima pergunta de $M$ ao oráculo será $y_2$.
    Então, $d(w_2)$ será um candidato a $A(y_2)$,
    novamente aparecendo numa das folhas da computação de $M_A$ em $y_2$,
    e assim por diante.

    Entretanto, caso $w_1$ não seja a resposta certa,
    $M_F$ prosseguirá a simulação com a resposta errada,
    e a próxima pergunta feita por $M$ ao oráculo
    (neste ramo de computação)
    pode não ser $y_2$;
    neste caso,
    a simulação é ``inválida'',
    no sentido de não corresponder exatamente à computação de $M$ em $x$.
    Argumentaremos que,
    no máximo lexicográfico destas listas de palavras,
    todos os $w_i$ correspondem às respostas certas do oráculo $A$;
    isto é, estes erros não ocorrem
    e $M_F$ corretamente computa $F$.

    Primeiro,
    note que todos os ramos de computação retornarão palavras de mesmo tamanho.
    Dessa forma,
    dadas duas listas, $w_1 w_2 \dots w_n$ e $v_1 v_2 \dots v_n$,
    se $w_1 > v_1$ lexicograficamente,
    a primeira lista tem precedência sobre a segunda.
    Portanto,
    o máximo dentre os ramos de computação
    também terá de maximizar, localmente,
    o candidato à primeira resposta ao oráculo.
    Isto é,
    se o máximo lexicográfico é
    \begin{equation*}
        \omega_1 \omega_2 \dots \omega_n,
    \end{equation*}
    sabemos com certeza que $d(\omega_1) = A(y_1)$,
    pois algum ramo de computação gerou $A(y_1)$ como candidato à primeira resposta,
    e nenhum outro ramo pode ter gerado um valor maior.

    Agora, como acertamos a primeira resposta do oráculo,
    a simulação de $M_F$ corresponderá à computação de $M$ em $x$,
    e a segunda pergunta feita na simulação será $y_2$.
    Usando um raciocínio similar,
    concluímos que $d(\omega_2) = A(y_2)$,
    e assim por diante,
    até concluir que $d(\omega_i) = A(y_i)$ para todo $i$.

    Isso prova que $F \in \FNP^\mathcal A$.
    Agora,
    construiremos a segunda função, $G$,
    que será composta com $F$ para gerar $f$.
    (A função $G$ usará aquele $\langle M, x, n \rangle$
    que da definição da função $F$ que ignoramos até agora
    --- note que a argumentação continua válido
    pois este valor age como uma constante presente em todos os ramos de computação.)

    A máquina $M_G$ que calculará $G$
    pegará a saída de $M_F$ e descobrirá o valor de $t(M, x, n)$.
    Para isso,
    basta simular $M$ em $x$ por $n$ rodadas;
    as respostas à todas as consultas a $A$ já foram descobertas por $M_F$.
    Portanto,
    $M_G$ pode ser determinística sem oráculos.

    Assim, por construção, $G(F(\langle M, x, n \rangle)) = t(M, x, n)$;
    portanto,
    \begin{equation*}
        G \circ F = \HaltFP^{\FNP^\mathcal A} = f.
    \end{equation*}

    Como $G$ e $F$ pertencem a $\FNP^\mathcal A$,
    mostramos, finalmente, que
    \begin{equation*}
        f \in \left(\FNP^\mathcal A\right) ^ \circ.
    \end{equation*}
    Agora, podemos repetir a argumentação dada no início da demonstração.

    Se $g \in \FP^{\FNP^\mathcal A}$,
    então existe alguma função $h \in \FP$
    tal que $f \circ h = g$
    (pois $f$ é completa para $\FP^{\FNP^\mathcal A}$).
    Portanto,
    como $f \in \left( \FNP^\mathcal A \right) ^ \circ$
    e $\FP \subseteq \FNP^\mathcal A$,
    temos $g \in \left( \FNP^\mathcal A \right) ^ \circ$,
    o que prova que
    \begin{equation*}
        \FP^{\FNP^\mathcal A} \subseteq \left( \FNP^\mathcal A \right) ^ \circ.
    \end{equation*}
\end{proof}

\begin{ucorollary}
    \begin{equation*}
        (\Sigma_n^f) ^ \circ = \Delta_{n+1}^f.
    \end{equation*}
\end{ucorollary}

Com este corolário,
atingimos um dos objetivos deste trabalho;
nós encontramos uma relação na hierarquia polinomial
através da composição de funções não"=determinísticas.

Analisando mais cuidadosamente a demonstração,
podemos fortalecer levemente o teorema.
A função $G$ é calculada por uma máquina determinística sem oráculos;
portanto, na verdade,
temos $G \in \FP$.
Portanto, toda função $f \in \FP^{\FNP^\mathcal A}$
pode ser escrita como
\begin{equation*}
    G \circ F \circ h,
\end{equation*}
em que $G$ e $h$ pertencem a $\FP$ e $F$ pertence a $\FNP^\mathcal A$.

Mas, como $h$ é uma função determinística que opera em tempo polinomial,
podemos ``emendar'' a máquina que computa $h$
no início da máquina que computa $F$ sem prejuízos.
Ou seja,
a função $H = F \circ h$ pertence a $\FNP^\mathcal A$;
provamos, portanto,
o seguinte teorema.

\begin{ucorollary}
    Se $\mathcal A$ possui problemas polinomialmente completos,
    então todo problema $f \in \FP^{\FNP^\mathcal A}$
    pode ser escrito como
    \begin{equation*}
        G \circ H,
    \end{equation*}
    em que $G \in \FP$ e $H \in \FNP^\mathcal A$.
    Além disso, $G$ não depende de $f$.
    \label{thm:strong_compositive_closure}
\end{ucorollary}

Em outras palavras,
compondo apenas duas funções de $\Sigma_n^f$,
obtemos todo o poder computacional
das repetidas chamadas a oráculos de $\Delta_{n+1}^f$.
Isso sugere que talvez tenhamos ``exagerado na dose''
ao definir funções não"=determinísticas;
isto é,
ao definir o valor da função
como sendo o máximo global dentre todos os ramos de computação,
estamos dando à máquina quase tanto poder computacional
quanto fornecer um oráculo para a sua própria classe de complexidade.
