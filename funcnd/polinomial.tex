\section{HIERARQUIA POLINOMIAL}

Munidos da definição de oráculo funcional,
da seção~\ref{sec:functional_oracles},
podemos construir um análogo à hierarquia polinomial,
mas usando funções.

\begin{definition}[Hierarquia Polinomial Funcional]
    \simbolo{$\Sigma_n^f$}{Versão funcional de $\Sigma_n^p$}
    \simbolo{$\Delta_n^f$}{Versão funcional de $\Delta_n^p$}
    \simbolo{$\Pi_n^f$}{Versão funcional de $\Pi_n^p$}
    \begin{align*}
        \Sigma_0^f &= \Delta_0^f = \Pi_0^f = \FP, \\
        \Sigma_{n+1}^f &= \FNP^{\Sigma_n^f}, \\
        \Delta_{n+1}^f &= \FP^{\Sigma_n^f}, \\
        \Pi_{n+1}^f &= \coFNP^{\Sigma_n^f}.
    \end{align*}
\end{definition}

$\Sigma_n^f$ é a generalização funcional de $\Sigma_n^p$,
$\Delta_n^f$ é a de $\Delta_n^p$
e $\Pi_n^f$ a de $\Pi_n^p$.
Alguns casos particulares desta hierarquia já foram discutidos;
por exemplo,
\begin{align*}
    \Sigma_1^f = \FNP, \\
    \Delta_1^f = \FP, \\
    \Pi_1^f = \coFNP, \\
    \Delta_2^f = \FP^\FNP.
\end{align*}

Assim como na hierarquia polinomial,
$\Delta_n^f$ está contido tanto em $\Sigma_n^f$ quanto em $\Pi_n^f$.
O fato de $\Sigma_n^f$ estar contido em $\Delta_{n+1}^f$
é simples de ser verificado,
pois $\Delta_{n+1}^f$ possui acesso a oráculos
que resolvem problemas de $\Sigma_n^f$.
Para provar que $\Delta_{n+1}^f$ também inclui $\Pi_n^f$,
precisaremos, de alguma forma,
inverter o ordenamento das cadeias de $\{0, 1\}^*$.

\begin{proposition}
    Para todo $n$,
    \begin{equation*}
        \Pi_n^f \subseteq \Delta_{n+1}^f.
    \end{equation*}
    \label{thm:pi_f_subseteq_delta_f}
\end{proposition}

\begin{proof}
    Uma função $f \in \Pi_n^f$
    é computada por uma máquina de Turing não"=determinística $M$
    com acesso a um oráculo de $\Sigma_{n-1}^f$,
    tomando"=se o mínimo de seus ramos.
    Construiremos uma máquina não"=determinística $M'$,
    usando o mesmo oráculo,
    de forma que a função em $\Sigma_n^f$ que $M'$ computa
    precisará apenas de um pós"=processamento
    para resultar no valor de $f$.
    Este pós"=processamento será feito por uma máquina determinística,
    que usará $M'$ como oráculo;
    esta máquina determinística,
    portanto, computará $f$ usando os recursos de $\Delta_{n+1}^f$.
    (A ideia é que $M'$ inverta o ordenamento das cadeias
    produzidas por $M$.
    No pós"=processamento, iremos ``desfazer'' a inversão.)

    Usaremos a função $d$ de decodificação definida anteriormente.
    Na entrada $x$, $M'$ começará executando exatamente como $M$
    até $M$ parar, deixando (digamos) $y$ na fita neste ramo de computação.
    (Podemos fazer esta computação pois o oráculo de $M$
    está em $\Sigma_{n-1}^f$,
    portanto também podemos usá"=lo como oráculo.)
    Como $M$ opera em tempo polinomial,
    $|y|$ está limitado superiormente por $p(|x|)$ para algum polinômio $p$.
    Então,
    antes de encerrar a computação,
    $M'$ substituirá o valor $y$ por
    \begin{equation*}
        0^{|y|}1 y 0^{2*p(|x|) - 2*|y|}.
    \end{equation*}
    A primeira cadeia de zeros garante que o valor de $d(M'(x))$
    corresponda a algum dos ramos da computação de $M$.
    A última garante que todos os ramos de computação de $M'$
    resultem numa palavra de tamanho $2*p(|x|)+1$.

    Chame de $e(y)$ a cadeia resultante da substituição acima,
    de forma que $d(e(y)) = y$
    para todo $y$ de tamanho menor que $p(|x|)$.
    Toda esta demonstração se baseia na observação de que,
    se $y_1 < y_2$, então $e(y_1) > e(y_2)$,
    para $|y_1|, |y_2| \leq p(|x|)$.
    Portanto,
    ao aplicar a transformação $e$,
    efetivamente invertemos localmente a ordenação das cadeias de $\{0, 1\}^*$.

    Dessa forma, se tanto $M$ quanto $M'$ tomam menos de $k$ passos de computação
    ao processar $x$,
    \begin{align*}
        M'(x) &= \max T(M', x, k) \\
              &= \max \{e(y) \mid y \in T(M, x, k) \} \\
              &= e \left( \min \{y \mid y \in T(M, x, k) \} \right) \\
              &= e \left( \min T(M, x, k) \right) \\
              &= e( f(x) ).
    \end{align*}
    Como consequência,
    acabamos de demonstrar que a função $e \circ f$ pertence a $\Sigma_n^f$.
    Agora,
    basta usar uma terceira máquina $M''$,
    determinística,
    que usa $e \circ f$ como oráculo,
    e calcular $d$ na resposta para obter o valor original de $f$.

    Portanto, $f \in \Delta_{n+1}^f$.
\end{proof}

O teorema~\ref{thm:fp_np_equals_fp_fnp}
pode ser refraseado como
\begin{equation*}
    \Delta_2^f = \FP^{\Sigma_1^p}.
\end{equation*}
Observe que o oráculo utilizado é $\Sigma_1^p$, não $\Sigma_1^f$.
Podemos generalizar esta relação
para todos os níveis de nossa hierarquia polinomial funcional.

\begin{theorem}
    Para todo $n$,
    \begin{align*}
        \Delta_{n+1}^f &= \FP^{\Sigma_n^p}, \\
        \Sigma_{n+1}^f &= \FNP^{\Sigma_n^p}, \\
        \Pi_{n+1}^f &= \coFNP^{\Sigma_n^p}. \\
    \end{align*}
\end{theorem}

\begin{proof}
    Faremos por indução.
    Observe que,
    como $\Pi_n^p$ contém os complementos dos conjuntos de $\Sigma_n^p$,
    \begin{equation*}
        \coFNP^{\Sigma_n^p} = \coFNP^{\Pi_n^p};
    \end{equation*}
    esta observação ajudará a resolver o caso de $\Pi_{n+1}^f$.

    Para $n = 0$, é simples, pois $\Sigma_0^p = \P$.
    \begin{align*}
        \Delta_1^f = \FP^\FP = \FP &= \FP^\P = \FP^{\Sigma_0^p}; \\
        \Sigma_1^f = \FNP^\FP = \FNP &= \FNP^\P = \FNP^{\Sigma_0^p}; \\
        \Pi_1^f = \coFNP^\FP = \coFNP &= \coFNP^\P = \coFNP^{\Sigma_0^p}.
    \end{align*}

    Para $n \geq 1$, usaremos a mesma técnica usada
    para demonstrar o teorema~\ref{thm:fp_np_equals_fp_fnp}.
    Em essência,
    demonstraremos que uma chamada a um oráculos de $\Sigma_n^f$
    podem ser substituídas por várias chamadas a um oráculo de $\Sigma_n^p$
    (e algum processamento intermediário).
    Como este número de chamadas se manterá polinomial,
    toda a computação em si continuará sendo em tempo polinomial.
    Além disso,
    tomaremos cuidado para que a sequência de passos de computação
    que simulará a chamada a $\Sigma_n^f$
    seja feita de maneira determinística;
    assim,
    a mesma demonstração pode ser repetida para $\FP$, $\FNP$ e $\coFNP$.

    Fixe um oráculo $f \in \Sigma_n^f$.
    Construiremos um oráculo $L \in \Sigma_n^p$
    e mostrar como substituir uma única chamada a $f$
    por uma sequência de chamadas a $L$.
    Defina
    \begin{equation*}
        L = \{\langle x, y\rangle \mid f(x) \geq y\}.
    \end{equation*}
    (Observe que $L$ é $L_f$, nos termos do teorema~\ref{thm:fnp_to_np_conversion}.)
    Primeiro, mostraremos que $L \in \Sigma_n^p$.

    O ponto crucial desta ``subdemonstração'' é perceber que,
    como $f$ pertence a $\Sigma_n^f$,
    $f$ pertence a $\FNP^{\Sigma_{n-1}^p}$,
    pela hipótese indutiva.
    De resto, procederemos como no teorema~\ref{thm:fnp_to_np_conversion}.

    Seja $M$ uma máquina que calcula $f$,
    usando $L' \in \Sigma_{n-1}^p$ como oráculo
    --- a hipótese indutiva garante que tais $M$ e $L'$ existem.
    Construa uma máquina $M'$ que,
    na entrada $\langle x, y \rangle$,
    agirá como $M$ em $x$ e calculará um dos ramos da computação de $M$ em $x$.
    Se o resultado na fita for maior ou igual a $y$,
    aceite;
    caso contrário, rejeite.

    Observe que $M'$ reconhece uma linguagem de $\Sigma_n^p$.
    Argumentaremos que $L(M') = L$.
    De fato, se $f(x) > y$,
    algum dos ramos de computação de $M$ produzirá um valor maior ou igual a $y$,
    portanto $M'$ aceitará a entrada.
    Caso contrário,
    todos os ramos geram valores menores que $y$,
    portanto $M'$ rejeitará a entrada.
    Assim, $L(M') = L$,
    o que prova que $L \in \Sigma_n^p$.

    Agora,
    mostraremos como calcular $f(x)$ usando apenas chamadas a $L$:
    basta fazer busca binária assim como na demonstração
    do teorema~\ref{thm:fp_np_equals_fp_fnp}.
    Primeiro,
    encontre o menor $n$ tal que $f(x) < 0^n$,
    exclua um $0$,
    e vá fazendo chamadas sucessivas a $L$ para descobrir os demais bits de $f(x)$.
    Desta forma, com $2n+1$ chamadas a $L$,
    calculamos $f(x)$.

    Em outras palavras,
    \begin{equation*}
        f \in \FP^{\Sigma_n^p}.
    \end{equation*}

    Portanto, se $g \in \FP^f$,
    por exemplo,
    temos
    \begin{align*}
        g &\in \FP^f \\
          &\subseteq \FP^{\FP^{\Sigma_n^p}} \\
          &= \FP^{\Sigma_n^p}.
    \end{align*}

    Como esse processo pode ser feito para qualquer $f \in \Sigma_n^f$,
    acabamos de mostrar que
    \begin{equation*}
        \Delta_{n+1}^f \subseteq \FP^{\Sigma_n^p}.
    \end{equation*}

    Analogamente,
    \begin{align*}
        \Sigma_{n+1}^f &\subseteq \FNP^{\Sigma_n^p},\\
        \Pi_{n+1}^f &\subseteq \coFNP^{\Sigma_n^p}.
    \end{align*}

    As inclusões contrárias são triviais.
\end{proof}
