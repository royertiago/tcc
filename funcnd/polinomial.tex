\section{Hierarquia Polinomial}

Munidos da definição de oráculo funcional,
da seção~\ref{sec:functional_oracles},
podemos construir um análogo à hierarquia polinomial,
mas usando funções.

\begin{definition}[Hierarquia Polinomial Funcional]
    \simbolo{$\Sigma_n^f$}{Versão funcional de $\Sigma_n^p$}
    \simbolo{$\Delta_n^f$}{Versão funcional de $\Delta_n^p$}
    \simbolo{$\Pi_n^f$}{Versão funcional de $\Pi_n^p$}
    \begin{align*}
        \Sigma_1^f &= \Delta_1^f = \Pi_1^f = \FP, \\
        \Sigma_{n+1}^f &= \FNP^{\Sigma_n^f}, \\
        \Delta_{n+1}^f &= \FP^{\Sigma_n^f}, \\
        \Pi_{n+1}^f &= \coFNP^{\Sigma_n^f}.
    \end{align*}
\end{definition}

$\Sigma_n^f$ é a generalização funcional de $\Sigma_n^p$,
$\Delta_n^f$ é a de $\Delta_n^p$
e $\Pi_n^f$ a de $\Pi_n^p$.
