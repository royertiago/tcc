\section{Oráculos Funcionais}
\label{sec:funcional_oracles}

Na seção anterior,
definimos a função $\CallbackSAT$,
que incorpora na definição da função
o desejo de fazer algum tipo de processamento adicional com a saída.
Agora,
definiremos a noção de \emph{oráculo funcional},
que expande bastante esta capacidade de pós"=processamento.

\begin{definition}[Oráculo funcional]
    Seja $f : \Sigma^* \rightarrow \Sigma^*$.
    Uma máquina que usa $f$ como oráculo
    possui uma fita especial para se ``comunicar'' com $f$.
    A máquina escreve uma palavra nesta fita,
    digamos $x$,
    e consulta o oráculo
    (isto é, transita para $q_?$).
    Caso $f(x)$ exista,
    a máquina transitará para $q_y$
    e o conteúdo da ``fita de comunicação'' é alterado para $f(x)$.
    Caso $f(x)$ não exista,
    a máquina transita para $q_n$,
    com a fita inalterada.
\end{definition}

Os estados $q_y$ e $q_n$ são apenas um preciosismo
para que a definição permita qualquer função como oráculo,
mesmo que não seja computável.
No caso de máquinas de Turing,
as funções computadas só não estarão definidas
nas entradas em que a máquina não parar.
Como estamos lidando com máquinas que possuem limites de tempo,
a máquina sempre para
(portanto a função sempre está definida),
tornando o estado $q_n$ redundante.

\begin{definition}
    Seja $\mathcal A$ um conjunto de funções.
    $\FP^\mathcal A$ é o conjunto das funções
    que podem ser calculadas em tempo polinomial
    por alguma máquinas de Turing que usa alguma função de $\mathcal A$ como oráculo.
    Caso $\mathcal A$ seja um conjunto de linguagens,
    interpretaremos a máquina como usando a função característica correspondente.
    $\FNP^\mathcal A$ e $\coFNP^\mathcal A$
    são definidas de maneira análoga.
    \simbolo{$\FP^\mathcal A$}{$\FP$ usando um oráculo de $\mathcal A$}
    \simbolo{$\FNP^\mathcal A$}{$\FNP$ usando um oráculo de $\mathcal A$}
    \simbolo{$\coFNP^\mathcal A$}{$\coFNP$ usando um oráculo de $\mathcal A$}
\end{definition}

$\FP^\mathcal A$, $\FNP^\mathcal A$ e $\coFNP^\mathcal A$
são os análogos funcionais de
$\P^\mathcal A$, $\NP^\mathcal A$ e $\coNP^\mathcal A$,
respectivamente.

Agora,
remetendo ao teorema~\ref{thm:fnp_to_np_conversion},
podemos mostrar que,
de certa forma,
não ganhamos muito poder computacional
ao utilizar a noção de ``função não"=determinística'' em oráculos.

\begin{theorem}
    \begin{equation*}
        \FP^\NP = \FP^\FNP
    \end{equation*}
\end{theorem}

\begin{proof}
    Como as funções características de $\NP$ estão todas em $\FNP$,
    a inclusão $\FP^\NP \subseteq \FP^\FNP$ segue direto da definição.

    Seja $f \in \FP^\FNP$;
    existe uma máquina polinomial determinística $M$
    que faz chamadas a um certo oráculo $g$.
    Usaremos a linguagem $L_g$,
    do teorema~\ref{thm:fnp_to_np_conversion},
    para fazer uma busca binária atrás do valor de $g$.

    Seja $x$ a entrada de $M$.
    Substitua cada chamada a $g$
    pelo seguinte procedimento:

    Primeiro, calcule qual o menor $n$
    tal que $\langle x, 0^n \rangle \in L_g$;
    como a ordenação lexicográfica que impomos
    considera primeiro o tamanho,
    isto nos dará qual é o tamanho de $g(x)$.
    Depois,
    faremos uma busca binária.
    Avance da esquerda para a direita,
    colocando um bit de $0^n$ para $1$ de cada vez.
    Os bits que fazem o par $\langle x, y \rangle$
    permanecerem em $L_g$ são mantidos,
    os demais são colocados de volta para $0$.
    Isto calculará o valor de $g(x)$,
    bit a bit.

    Este procedimento opera em tempo polinomial em $n$.
    Como $n = |g(x)|$ e $g \in \FNP$,
    $n$ precisa ser polinomial em $|x|$;
    portanto, este procedimento opera em tempo polinomial em $|x|$.
    Já que $M$ faz uma quantidade polinomial de chamadas a $g$,
    a máquina resultante também terá tempo de execução polinomial.

    Cada chamada ao oráculo $g$ foi substituída por
    um procedimento que devolve sempre a mesma resposta que $g$;
    portanto, a resposta final desta máquina modificada
    será igual à resposta de $M$,
    o que prova que $f \in \FP^\NP$.
\end{proof}

Podemos definir e construir linguagens completas para os três conjuntos
$\FP^\mathcal A$, $\FNP^\mathcal A$ e $\coFNP^\mathcal A$,
usando as mesmas técnicas usadas para definir e construir problemas $\FNP$"=completos.

\begin{definition}
    Seja $\mathcal A$ um conjunto de funções.
    Uma função $f$ é \emph{$\FP^\mathcal A$"=completa}
    se $f \in \FP^\mathcal A$ e,
    para toda função $g \in \FP^\mathcal A$,
    existir alguma função $h \in \FP$ tal que
    \begin{equation*}
        g(x) = f(h(x))
    \end{equation*}
    para todo $x \in \{0, 1\}^*$.

    $\FNP^\mathcal A$ e $\coFNP^\mathcal A$"=completude
    são definidos analogamente.
\end{definition}

Esta definição é idêntica à definição de $\FNP$"=completude,
mas equipando tudo com oráculos
--- com exceção da função $h$,
que ainda precisa operar em tempo determinístico polinomial sem oráculos.
