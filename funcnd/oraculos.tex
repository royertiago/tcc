\section{Oráculos Funcionais}

\begin{definition}[Oráculo funcional]
    Seja $f : \Sigma^* \rightarrow \Sigma^*$.
    Uma máquina que usa $f$ como oráculo
    possui uma fita especial para se ``comunicar'' com $f$.
    A máquina escreve uma palavra nesta fita,
    digamos $x$,
    e consulta o oráculo
    (isto é, transita para $q_?$).
    Caso $f(x)$ exista,
    a máquina transitará para $q_y$
    e o conteúdo da ``fita de comunicação'' é alterado para $f(x)$.
    Caso $f(x)$ não exista,
    a máquina transita para $q_n$,
    com a fita inalterada.
\end{definition}

Os estados $q_y$ e $q_n$ são apenas um preciosismo
para que a definição permita qualquer função como oráculo,
mesmo que não seja computável.
No caso de máquinas de Turing,
as funções computadas só não estarão definidas
nas entradas em que a máquina não parar.
Como estamos lidando com máquinas que possuem limites de tempo,
a máquina sempre para
(portanto a função sempre está definida),
tornando o estado $q_n$ redundante.

\begin{definition}
    Seja $\mathcal A$ um conjunto de funções.
    $\FP^\mathcal A$ é o conjunto das funções
    que podem ser calculadas em tempo polinomial
    por alguma máquinas de Turing que usa alguma função de $\mathcal A$ como oráculo.
    Caso $\mathcal A$ seja um conjunto de linguagens,
    interpretaremos a máquina como usando a função característica correspondente.
    $\FNP^\mathcal A$ e $\coFNP^\mathcal A$
    são definidas de maneira análoga.
    \simbolo{$\FP^\mathcal A$}{$\FP$ usando um oráculo de $\mathcal A$}
    \simbolo{$\FNP^\mathcal A$}{$\FNP$ usando um oráculo de $\mathcal A$}
    \simbolo{$\coFNP^\mathcal A$}{$\coFNP$ usando um oráculo de $\mathcal A$}
\end{definition}

$\FP^\mathcal A$, $\FNP^\mathcal A$ e $\coFNP^\mathcal A$
são os análogos funcionais de
$\P^\mathcal A$, $\NP^\mathcal A$ e $\coNP^\mathcal A$,
respectivamente.
