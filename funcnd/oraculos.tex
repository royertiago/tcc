\section{Oráculos Funcionais}

Da mesma forma como definimos problemas $\NP$"=completos,
queremos, de alguma forma,
representar completude dentro da classe $\FNP$.
Entretanto,
os contradomínios de duas funções em $\FNP$ pode diferir
--- em contrapartida,
o contradomínio de todas as ``funções'' em $\NP$ é o conjunto $\{0, 1\}$.
Precisaríamos fazer processamento adicional
nos valores ``retornados'' pelas funções.
Portanto,
parece mais natural lidar diretamente com oráculos.

\begin{definition}
    Seja $\mathcal A$ um conjunto de linguagens.
    $\FP^\mathcal A$ é o conjunto das funções
    que podem ser calculadas da mesma forma que as funções de $\FP$,
    mas agora permitiremos chamadas ``gratuitas'' a um oráculo de $\mathcal A$.
    $\FNP^\mathcal A$ e $\coFNP^\mathcal A$
    são definidas de maneira análoga.
    \simbolo{$\FP^\mathcal A$}{$\FP$ usando um oráculo de $\mathcal A$}
    \simbolo{$\FNP^\mathcal A$}{$\FNP$ usando um oráculo de $\mathcal A$}
    \simbolo{$\coFNP^\mathcal A$}{$\coFNP$ usando um oráculo de $\mathcal A$}
\end{definition}

$\FP^\mathcal A$, $\FNP^\mathcal A$ e $\coFNP^\mathcal A$
são os análogos funcionais de
$\P^\mathcal A$, $\NP^\mathcal A$ e $\coNP^\mathcal A$,
respectivamente.

Observe que apenas definimos $\FP^\mathcal A$
quando $\mathcal A$ for um conjunto de linguagens.
Ainda não podemos falar, por exemplo, de $\FP^\FP$.
Precisamos antes definir o que é um oráculo funcional.

\begin{definition}[Oráculo funcional]
    Seja $f : \Sigma^* \rightarrow \Sigma^*$.
    Uma máquina que usa $f$ como oráculo
    possui uma fita especial para se ``comunicar'' com $f$.
    A máquina escreve uma palavra nesta fita,
    digamos $x$,
    e consulta o oráculo
    (isto é, transita para $q_?$).
    Caso $f(x)$ exista,
    a máquina transitará para $q_y$
    e o conteúdo da ``fita de comunitação'' é alterado para $f(x)$.
    Caso $f(x)$ não exista,
    a máquina transita para $q_n$,
    com a fita inalterada.
\end{definition}

Usaremos este conceito mais tarde.

Podemos estabelecer algumas relações entre estes conjuntos.

\begin{theorem}
    \begin{equation*}
        \FNP \subseteq \FP^\NP
    \end{equation*}
\end{theorem}
\begin{proof}
    Dada uma função $f \in \FNP$,
    mostraremos como computá"=la usando recursos de $\FP^\NP$.

    Em essência, faremos uma busca binária.
    Escolha a linguagem $L$ como
    \begin{equation*}
        L = \{ (x, k) \mid f(x) \geq k \}.
    \end{equation*}
    Podemos computar $L$ usando uma máquina não"=determinística em tempo polinomial.
    Basta rodar o mesmo algoritmo de $f$,
    e verificar se o valor obtido no ramo atual
    é maior que $k$ ou não.

    Se $f(x) \geq k$,
    algum dos ramos aceitará a entrada.
    Se $f(x) < k$,
    todos os ramos rejeitarão a entrada.
    Portanto, $L \in \NP$.
    Usaremos $L$ como oráculo.

    Simplesmente implemente a busca binária.
    Como $f \in \FNP$,
    existe um polinômio $p$ que limita a quantidade de bits que $f$ pode retornar.
    Portanto,
    em uma quantidade polinomial de iterações,
    convergiremos para o valor correto.
\end{proof}

\begin{ucorollary}
    \begin{equation*}
        \coFNP \subseteq \FP^\coNP
    \end{equation*}
\end{ucorollary}
