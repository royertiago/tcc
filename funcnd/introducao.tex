Na seção~\ref{axiomas_blum},
definimos dois axiomas que compreendem toda a informação necessária
para definir uma medida de complexidade abstrata.
As máquinas de Turing foram interpretadas como
representações de funções de inteiros
(isto é, funções cujo domínio e contradomínio são os números inteiros);
neste contexto,
a medidas de complexidade
se referiam a funções, não a linguagens.

As duas medidas mais importantes,
complexidade de tempo e espaço determinísticas,
foram definidos nos exemplos
\ref{complexidade_tempo} e~\ref{complexidade_espaco},
respectivamente.
O exemplo~\ref{complexidade_nao_deterministica},
porém,
ficou incompleto;
o que exatamente é uma ``função não"=determinística''?
O objetivo deste capítulo é prover uma definição para este termo
e interpretá"=lo em termos da hierarquia polinomial.
