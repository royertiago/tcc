\section{Motivação}

No contexto de problemas de decisão,
embora não"=determinismo não acrescente poder computacional
--- isto é,
todo problema que uma máquina de Turing não"=determinística resolve
também pode ser resolvido usando máquinas determinísticas
---
não"=determinismo aparenta reduzir a complexidade de um problema.
Em particular,
máquinas determinísticas aparentam ser exponencialmente mais lentas
do que máquinas não"=determinísticas
para problemas como $\SAT$.

Entretanto,
problemas de decisão apenas retornam uma resposta booleana:
descobrimos se a instância em questão possui solução ou não.
Em problemas do mundo real,
geralmente queremos \emph{a solução},
não apenas saber se ela existe.
Por exemplo,
em vez de apenas descobrir que certa fórmula lógica não é uma tautologia,
queremos saber exatamente qual atribuição de valores"=verdade
torna a fórmula falsa
--- por exemplo,
esta fórmula pode ser gerada por um verificador formal de software,
e a atribuição de valores"=verdade que invalida a fórmula
codifica uma entrada para o programa
em que ele não faz o que deveria.
Conhecer esta instância é importante para corrigir o software sendo verificado.

Nós já medimos a complexidade de uma função
calculada por uma máquina de Turing determinística.
Entretanto,
para que possamos aferir a complexidade de uma função não"=determinística,
precisamos especificar como uma máquina não"=determinística pode computar uma função.
