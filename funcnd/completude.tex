\section{Funções $\FNP$"=completas}

Da mesma forma como definimos problemas $\NP$"=completos,
queremos, de alguma forma,
representar completude dentro da classe $\FNP$.
No caso de $\NP$"=completude,
a redução de $L$ para $L'$ (por exemplo)
é feita por uma função $f$ de forma que
as ``respostas'' às questões ``$x \in L$?'' e ``$f(x) \in L'$?'' são as mesmas.
Há uma barreira trivial ao definir ``$\FNP$"=completude'' de maneira análoga,
que é a possibilidade de os contradomínios diferirem:
para toda função $f: \Sigma^* \to \Sigma^*$ em $\FNP$,
podemos construir uma função $g$ tal que $g(x) = \alpha$ para todo $x$,
para algum $\alpha \notin \Sigma$.
Isto é, para qualquer $x$, $g$ retorna a palavra com o único símbolo $\alpha$.
Como $\alpha \notin \Sigma$,
$f$ é inerentemente incapaz de produzir $\alpha$ na saída;
portanto, qualquer que seja a função de redução $h$,
$g(x)$ e $f(h(x))$ serão sempre diferentes.

Entretanto, intuitivamente,
lidar com $\Sigma$ diretamente não é estritamente necessário,
pois sempre podemos codificar os caracteres de $\Sigma$ usando binário.
Portanto,
para possibilitar uma definição direta de completude de funções,
assumiremos que o domínio e o contradomínio das funções de $\FNP$
é $\{0, 1\}^*$.

\begin{definition}
    Uma função $f$ é \emph{$\FNP$"=completa}
    se $f \in \FNP$ e,
    para toda função $g \in \FNP$,
    existir alguma função $h \in \FP$ tal que
    \begin{equation*}
        g(x) = f(h(x))
    \end{equation*}
    para todo $x \in \{0, 1\}^*$.
\end{definition}

Na definição de $\NP$"=completude,
a função $h$ que faz a redução é obrigada a ter a seguinte propriedade:
$x \in L$ se e somente se $f(x) \in L'$.
Ou seja, os valores booleanos das proposições
``$x \in L$'' e ``$h(x) \in L'$'' são os mesmos.
Se $g$ e $f$ são, respectivamente,
as funções características de $L$ e $L'$,
podemos expressar esta restrição como
$g(x) = f(h(x))$,
que é exatamente a expressão usada na definição de $\FNP$"=completude.
Portanto, de certa forma,
nossa definição de $\FNP$"=completude
é uma generalização da noção de $\NP$"=completude.

Podemos fabricar um problema $\FNP$"=completo usando o problema da parada,
mas impondo uma limitação no tempo de execução.

Dada uma máquina não"=determinística $M$,
uma entrada $x$
e uma limitação de tempo $n$,
defina $T(M, x, n)$
como sendo o conjunto de todas as palavras deixadas na fita,
após exatamente $n$ passos de computações não"=determinísticas de $M$ em $x$.
(Inclua também os resultados de ramos que pararam antes de $n$ passos.)
Caso todos os ramos de computação de $M$ encerrem em menos de $n$ etapas,
então $T(M, x, n)$ conterá todas as palavras das folhas da árvore de computação;
neste caso, a maior palavra (lexicograficamente) de $T(M, x, n)$
é $M(x)$.
Se existirem ramos de computação que não se encerram em $n$ etapas,
$T(M, x, n)$ conterá computações parciais,
que podem ser lexicograficamente maiores que $M(x)$;
portanto, neste caso, nada podemos afirmar sobre o máximo de $T(M, x, n)$.
\simbolo{$T(M, x, n)$}{``Respostas'' da computação de $M$ em $x$ por $n$ passos.}

Usaremos esta função $T$ para definir nossa função $\FNP$"=completa.

\newcommand{\HaltFNP}{{\lang{HaltFNP}}}
\simbolo{$\HaltFNP$}{Versão de $\FNP$ do problema da parada}
\begin{theorem}
    Defina a função $\HaltFNP: \{0,1\}^* \to \{0, 1\}^*$ por
    \begin{equation*}
        \HaltFNP(\langle M, x, n \rangle) = \max(T(M, x, n)),
    \end{equation*}
    em que $T$ é o conjunto definido acima
    e $\max$ é o máximo lexicográfico dentre as palavras do conjunto.
    Então $\HaltFNP$ é $\FNP$"=completa.
    \label{thm:halt_fnp}
\end{theorem}

\begin{proof}
    Seja $g$ uma função de $\FNP$.
    Pela definição de $\FNP$, existe alguma máquina $M$
    e um polinômio $p$ tal que,
    ao rodar $M$ em $x$,
    todos os ramos de computação são mais curtos que $p(|x|)$,
    e $M(x) = g(x)$.

    Defina a função $h: \{0, 1\}^* \to \{0, 1\}^*$
    por
    \begin{equation*}
        h(x) = \langle M, x, 1^{p(|x|)} \rangle.
    \end{equation*}
    Como $M$ roda em tempo menor que $p(|x|)$,
    o maior valor (lexicograficamente) de $T(M, x, p(|x|))$ é $M(x)$,
    que é igual a $g(x)$ pois $M$ computa $g$.
    Portanto,
    \begin{equation*}
        \HaltFNP(h(x)) = M(x) = g(x).
    \end{equation*}

    O polinômio $p$ pode ser computado em tempo $O(p)$,
    e os outros dois termos da tripla $\langle M, x, 1^{p(|x|)} \rangle$
    só precisam ser copiados. Portanto, $h \in \FP$.

    Isso mostra que toda função de $\FNP$ é redutível a $\HaltFNP$;
    agora, falta mostrar que $\HaltFNP \in \FNP$.

    Para isso, simplesmente rode $M$ em $x$,
    usando o próprio não"=determinismo para simular o não"=determinismo de $M$
    e um contador para que o tempo de simulação não extrapole $n$.
    Então, apague todo o conteúdo da fita,
    deixando apenas o resultado da simulação de $M$ em $x$;
    depois encerre a computação.

    Desta forma, todas as folhas desta árvore de computação
    corresponderão exatamente aos valores de $T(M, x, n)$;
    portanto, o valor da função calculado por esta máquina não"=determinística
    é $\HaltFNP(\langle M, x, n \rangle)$,
    o que prova que $\HaltFNP \in \FNP$.
\end{proof}

Temos nossa primeira função $\FNP$"=completa.
Entretanto, esta função é um tanto artificial.
Idealmente, gostaríamos que alguma função mais natural
(como, por exemplo, alguma generalização de $\SAT$)
fosse $\FNP$"=completa.

Podemos tentar seguir justamente este caminho
(generalizar $\SAT$),
tentando reduzir da função $\HaltFNP$ definida acima.
A generalização natural de $\SAT$ para $\FNP$
é a função que retorna alguma atribuição que satisfaz a fórmula
(a maior atribuição, no caso),
ou $\varepsilon$ caso a fórmula não seja satisfazível.
Sabemos da demonstração de $\NP$"=completude de $\SAT$
que é possível codificar estados de computação em fórmulas lógicas;
como esta codificação impõe,
na fórmula lógica,
o estado final da fita,
podemos recuperar qual é a palavra deixada na fita pela computação codificada na fórmula,
usando uma atribuição que satisfaz aquela fórmula.

Entretanto,
não podemos simplesmente retornar uma atribuição para a fórmula gerada,
pois a atribuição contém muito mais informação do que precisamos.
Ela contém todas as etapas da computação,
incluindo todos os estados intermediários
e todas as escolhas não"=determinísticas feitas pela máquina sendo reduzida.
Nós só precisamos da fita deixada pela máquina,
que corresponde a alguns bits da atribuição.

Podemos tentar a seguinte estratégia:
passamos à máquina que calcula nossa generalização de $\SAT$
não apenas a fórmula lógica,
mas também um número $n$,
e a máquina retornará apenas os últimos $n$
bits das atribuições que satisfazem a fórmula.
Entretanto, isto assume que todos os tamanhos de fita serão iguais,
o que pode não ser verdade.
Portanto, precisamos de alguma forma codificar na própria fórmula lógica
um meio de computar este $n$.

Ignoremos por um momento a interpretação de uma palavra sobre $\{0, 1\}^*$
como uma atribuição de valores"=verdade.
Queremos codificar, de maneira variável e não ambígua,
uma instrução de como extrair uma subpalavra de uma palavra dada.
Para codificar a subpalavra $w$,
coloque, no início da palavra que conterá $w$,
a cadeia $0^{|w|}1$, seguida de $w$.
Por exemplo,
a palavra
\begin{equation*}
    0000\ 1\ 0110\ 10101010
\end{equation*}
``esconde'' a subpalavra
\begin{equation*}
    0110
\end{equation*}
pois a palavra original começa com o prefixo $0^4 1$,
indicando que os quatro caracteres seguintes codificam a palavra a ser extraída.
(Poderíamos interpretar que a subpalavra também ``esconde''
a ``subsubpalavra'' $1$, mas não precisamos ir tão longe.)

Definiremos uma função $d: \{0, 1\}^* \to \{0, 1\}^*$ que fará a decodificação.
\simbolo{$d$}{Função de decodificação de subpalavras}
$d(0^n1\omega)$ será definido como sendo os $n$ primeiros dígitos de $\omega$.
Por completude, caso $|\omega| < n$, defina $d(0^n1\omega) = \omega$,
e $d(x) = \varepsilon$ se $x$ não contiver um caractere $1$.
\begin{example}
    \begin{align*}
        d(0000\ 1\ 0110\ 10101010) &= 0110 \\
        d(0110) &= 1 \\
        d(1) &= \varepsilon \\
        d(11010) &= \varepsilon \\
        d(0000) &= \varepsilon \\
        d(0001\ 10) &= 10
    \end{align*}
\end{example}

Agora podemos construir uma função $\FNP$"=completa um pouco mais natural.
\newcommand{\FdSAT}{{\lang{F}d\lang{SAT}}}
\simbolo{$\FdSAT$}{$\SAT$ funcional ``composto'' com $d$}
\begin{definition}
    A função $\FdSAT: \{0, 1\}^* \to \{0, 1\}^*$ é definida por
    \begin{equation*}
        \FdSAT(\varphi) = \max\{d(w) \mid
            \text{$w$ codifica uma atribuição que satisfaz $\varphi$}
        \}
    \end{equation*}
    Por completude, se $x$ não for uma fórmula lógica
    (ou não for satisfazível),
    defina $\FdSAT(x) = \varepsilon$.
\end{definition}

\begin{theorem}
    A função $\FdSAT$ definida acima é $\FNP$"=completa.
\end{theorem}

\begin{proof}
    Basta seguir as ideias anunciadas acima.

    Reduza da função $\HaltFNP$ do teorema~\ref{thm:halt_fnp}.
    Para uma entrada $\langle M, x, n \rangle$,
    utilize a mesma técnica usada na demonstração da $\NP$"=completude de $\SAT$
    para construir um circuito lógico
    que codifique os $n$ primeiros movimentos de $M$ em $x$.
    Então,
    codifique um circuito lógico que ``prepara'' a atribuição que satisfaz a fórmula
    para ``passar'' pela função $d$ e retornar apenas a fita.

    Este circuito tem como entrada todas as variáveis que codificam a fita de $M$
    (que podem ter espaços em branco tanto à esquerda quanto à direita),
    num total de $n + |x|$ variáveis de entrada
    (pois este é o tamanho máximo da fita resultante),
    e tem $2(n + |x|) + 1$ variáveis de saída,
    codificando uma cadeia $w$ tal que
    $d(w)$ seja a fita que $M$ produz após $n$ rodadas computando sobre $x$.

    O circuito pode ser projetado como tendo $n + |x|$ camadas
    para deslocar a fita para a esquerda,
    depois uma camada para transformar a fita no formato $0^{|w|}1w$
    (com o $0^{|w|}1$ sendo introduzido à esquerda do $w$),
    e depois mais $n + |x|$ camadas para deslocar esta cadeia resultante
    novamente para a esquerda.
    Preencha o resto com zeros, é irrelevante.

    Reordenando as variáveis de forma que
    a saída deste circuito combinatorial fiquem no começo,
    temos nossa transformação $h$,
    que pode ser efetuada em tempo polinomial,
    pois tanto a codificação das etapas de computação
    quanto a codificação do circuito podem ser feitas em tempo polinomial.

    Agora, mostraremos que $\FdSAT(h(y)) = \HaltFNP(y)$,
    para $y = \langle M, x, n \rangle$.
    Pela construção da função $h$ acima,
    se $\omega$ codifica uma atribuição de valores"=verdade para $h(y)$,
    então $d(\omega)$ é a fita que $M$ deixa após $n$ rodadas sobre $x$.
    Além disso, para toda computação de $M$ em $x$ por $n$ rodadas,
    existe alguma atribuição $\omega'$ que satisfaz a fórmula lógica $h(y)$.
    Portanto, o conjunto
    \begin{equation*}
        \{d(\omega) \mid \text{$w$ codifica uma atribuição que satisfaz $\varphi$} \}
    \end{equation*}
    é exatamente $T(M, x, n)$.
    Ou seja, o máximo deste conjunto é $\HaltFNP(y)$,
    concluindo que $\FdSAT(h(y)) = \HaltFNP(y)$.

    Portanto, reduzimos $\HaltFNP$ a $\FdSAT$ e,
    portanto,
    todas as funções de $\FNP$ para $\FdSAT$.
    Mostrar que $\FdSAT \in \FNP$ é deixado para o leitor.
\end{proof}

Observe que, de certa forma,
a cadeia inicial $0^n1$ ``explica'' para $d$
como esta função deve extrair a resposta do restante da palavra.
Podemos ir além e,
em vez de apenas prover um parâmetro para uma função fixa $d$,
prover uma \emph{função} que extrairá a ``parte que interessa'' da cadeia.
Representaremos a função por uma máquina de Turing,
associada a um limite de tempo.

\newcommand{\CallbackSAT}{{\lang{CallbackSAT}}}
\simbolo{$\CallbackSAT$}{Versão funcional de $\SAT$ com um callback}
\begin{definition}
    Se $M$ for uma máquina de Turing determinística,
    denote por $t(M, x, n)$ o único elemento de $T(M, x, n)$.

    Defina a função $\CallbackSAT: \{0, 1\}^* \to \{0, 1\}^*$ por
    \begin{equation*}
        \CallbackSAT(\langle \varphi, M, n \rangle) = \max\{
            t(M, \omega, n) \mid \text{
                $\omega$ codifica uma atribuição que satisfaz $\varphi$.%
            }
        \}
    \end{equation*}
    Por completude,
    se $\varphi$ não for uma fórmula lógica
    (ou não for satisfazível),
    defina $\CallbackSAT(\langle \varphi, M, n \rangle) = \varepsilon$.
\end{definition}
Em outras palavras,
as folhas da árvore de computação de $\CallbackSAT$
serão as atribuições que satisfazem $\varphi$,
mas depois de serem pós"=processadas por $M$ por $n$ rodadas.%
\footnote{
    Este mecanismo é semelhante à noção de \emph{callback} em linguagens de programação.
}

\begin{theorem}
    $\CallbackSAT$ é $\FNP$"=completa.
\end{theorem}

\begin{proof}
    Basta reduzir de $\FdSAT$.
    Escolha $M$ como sendo uma máquina que computa $d$ em tempo polinomial.
    Se $p$ é um polinômio que limita o tempo de computação de $M$,
    escolha $n = p(k)$ quando a fórmula de entrada possuir $k$ variáveis,
    e passe a fórmula $\varphi$ intocada para $\CallbackSAT$.

    Como $n$ é grande o bastante para que a máquina $M$ sempre encerre a execução,
    $t(M, x, n) = d(x)$.
    Portanto, o conjunto que $\CallbackSAT(\langle \varphi, M, n \rangle)$ maximiza é
    \begin{equation*}
        \{d(\omega) \mid \text{
            $\omega$ codifica uma atribuição que satisfaz $\varphi$.%
        }\},
    \end{equation*}
    que é exatamente o conjunto maximizado por $\FdSAT(\varphi)$.
    Portanto,
    \begin{equation*}
        \FdSAT(\varphi) = \CallbackSAT(\langle \varphi, M, n \rangle).
    \end{equation*}

    E, para executar $\CallbackSAT$ usando recursos de $\FNP$,
    basta chutar uma atribuição de valores"=verdade para $\varphi$
    e simular $M$ por $n$ rodadas nas atribuições que satisfizerem a fórmula.
\end{proof}
