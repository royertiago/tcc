% Resumo, em inglês.

\textAbstract{
    Normally, the concepts of time and space complexity
    for deterministic and nondeterministic machines
    are defined separately,
    although they share several similar theorems.
    \citeonline[p.~324]{Blum1967} define a ``computational resource''
    using two axioms;
    this allows for an axiomatic treatment of the computational complexity,
    enabling us to provide a single proof for these analogous theorems.

    However, the Blum axioms are not defined in terms of deciders,
    but for integer function computers
    (that is, Turing machines that compute functions of the form
    $f : \mathbb N \to \mathbb N$ \cite[p.~151]{HopcroftUllman1979}).
    Therefore,
    to define $\P$ and $\NP$ via Blum axioms,
    we need to see a decider as an integer function computer.
    The case where the machine is deterministic is easy,
    because it have a single sequence of computation.
    The nondeterministic case, which is more complex,
    is studied in this text.

    In this text,
    we define the concept of ``nondeterministic function''
    in a way that the composition of nondeterministic functions
    correspond to the composition of the analogous deterministic functions.
}
\keywords{
    Computational complexity,
    nondeterministic functions,
    function composition.
}

\paginaabstract
