\numerodemembrosnabanca{3}
\bancaMembroA{
    Profa. Dra. Karina Girardi Roggia \\
    Universidade do Estado de Santa Catarina
}
\bancaMembroB{
    Prof. Dr. Ricardo Azambuja Silveira \\
    Universidade Federal de Santa Catarina
}
\bancaMembroC{
    Prof. Dr. Rosvelter João Coelho da Costa \\
    Universidade Federal de Santa Catarina
}

\def\ABNTorientadordata{Orientadora \\ Universidade Federal de Santa Catarina}
\folhaaprovacao

% Gambiarra.
% Cada "personagem" que precisa assinar alguma coisa
% (membro da banca, orientador, coorientador, coordenador)
% é impresso da mesma forma: primeiro a linha da assinatura,
% depois o nome, armazenados nos comandos \ABNTorientadornome, \bancanameA, etc.,
% e por fim uma variável nomeada "\ABNTorientadordata", ou "\bancadataA", etc.,
% com "dados adicionais".
% No caso da banca, por exemplo, os "dados adicionais" jamais são definidos;
% no caso do orientador, é definido para "Orientador" (ou "Orientadora" no meu caso).
%
% Para incluir a universidade da minha orientadora abaixo do nome,
% eu redefini \ABNTorientadordata para o texto que aparecerá abaixo;
% assim, evita-se de gerar uma nova página apenas para a banca.
% (No meu caso, a ordem fica certa pois "Jerusa" precede lexicograficamente
% "Karina", "Ricardo" e "Rosvelter"; não sei como corrigir isso
% sem alterar \folhaaprovacao para lidar com todos os casos.
% Ou o orientador sempre vem antes dos outros membros da banca?)
