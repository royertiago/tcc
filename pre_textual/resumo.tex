% Resumo

\textoResumo{
    Os conceitos de complexidade de tempo e espaço
    para máquinas de Turing determinísticas e não"=determinísticas,
    embora compartilhem muitos teoremas,
    costumam ser definidos individualmente.
    \citeonline[p.~324]{Blum1967} define um ``recurso computacional''
    como sendo algo que satisfaz dois axiomas;
    desta forma, podemos tratar da complexidade computacional
    de maneira axiomática
    e unificar os teoremas que costumam ser demonstrados separadamente.

    Entretanto, \citeonline[p.~324]{Blum1967}
    não define a noção de recurso computacional para decisores,
    mas sim para computadores de funções de inteiros
    (isto é, máquinas de Turing que computam funções
    $f : \mathbb N \to \mathbb N$ \cite[p.~151]{HopcroftUllman1979}).
    Portanto,
    para definir as classes $\P$ e $\NP$ em termos dos axiomas de Blum,
    precisamos interpretar um decisor como um computador de uma função de inteiros.
    Para o caso determinístico é simples,
    pois a máquina determinística possui apenas uma sequência de computação.
    O caso não"=determinístico é mais complexo,
    e é o objeto de estudo deste TCC.

    Pretendo definir o conceito de ``função não"=determinística''
    de forma que a composição destas funções corresponda,
    de alguma forma,
    à noção de composição de funções determinísticas.
}
\palavrasChave{
    Complexidade computacional,
    funções não"=determinísticas,
    composição de funções.
}

\paginaresumo
