\documentclass[utf8,notheorems]{beamer}
\usetheme[compress]{Singapore}

\usepackage[brazil]{babel}
\usepackage{tikz}
\usetikzlibrary{calc}

\newtheorem*{theorem}{Teorema}
\newtheorem*{corollary}{Corolário}
\theoremstyle{definition}
\newtheorem*{definition}{Definição}

\let\C\someundefinedcommand
\let\G\someundefinedcommand
% These two commands are defined by hyperref in file puenc.def,
% and conflict with complexity - which also defines them.
% \C encodes some unicode character (U+030f), \G is textdoublegrave.
% Seems safe to simply undefine them.
\usepackage{complexity}

\begin{document}

\author{Tiago Royer}
\title{
    Máquinas de Turing não-determinísticas \\
    como computadores de funções
}
\subtitle{Defesa do TCC}
\date{12 de novembro de 2015}
\institute{UFSC}
\begin{frame}
    \titlepage
\end{frame}

\section{Axiomas de Blum}

\subsection{Enumerações de Gödel}
\begin{frame}
    \frametitle{Enumerações de Gödel aceitáveis das funções recursivas}
    \begin{definition}
        Seja $\mathcal P$ o conjunto das funções recursivas parciais.
        Uma \emph{enumeração de Gödel aceitável}
        é uma função sobrejetora $\phi: \{0, 1\}^* \to \mathcal P$,
        tal que exista uma função recursiva parcial $g$ tal que
        \begin{equation*}
            g(x, y) = \phi_x(y)
        \end{equation*}
        e uma função recursiva total $\sigma$ tal que
        \begin{equation*}
            \phi_{\sigma(x, y)}(z) = \phi_x(y, z).
        \end{equation*}
    \end{definition}
\end{frame}

\subsection{Axiomas de Blum}
\begin{frame}
    \frametitle{Axiomas de Blum}
    \begin{definition}
        Uma \emph{medida de complexidade}
        é uma função $\Phi:\{0, 1\}^* \times \{0, 1\}^* \to \mathbb N$
        tal que
        \begin{itemize}
            \item $\Phi(w, x)$ está definido se, e somente se, $\phi_w(x)$ está definido;
            \item O predicado ``$\Phi(w, x) = n?$'' é recursivo.
        \end{itemize}
    \end{definition}
\end{frame}

\subsection{Programador Incompetente}
\begin{frame}
    \frametitle{``Teorema do programador incompetente''}
    \begin{theorem}
        Seja $\phi$ uma numeração de Gödel aceitável,
        $\Phi$ uma medida de complexidade para $\phi$,
        e $f: \{0, 1\}^* \to \{0, 1\}^*$ e $g: \mathbb N \to \mathbb N$
        funções recursivas totais.
        Então existe um índice $w$ para a função $f$ tal que
        \begin{equation*}
            \Phi(w, x) > g(|x|)
        \end{equation*}
        para todo $x$.
    \end{theorem}

    \pause
    Em outras palavras: código ruim pode ser feito em qualquer linguagem.
\end{frame}

\subsection{Máquinas não-determinísticas}
\begin{frame}
    \frametitle{E para máquinas não-determinísticas?}

    \begin{itemize}
        \item Para definir complexidade, precisamos de uma enumeração de Gõdel
        \item Para definir a enumeração, precisamos calcular funções
        \item Como uma máquina não-determinística calcula uma função?
    \end{itemize}
\end{frame}

\section{Funções não-determinísticas}

\subsection{Definição}
\begin{frame}
    \frametitle{Problemas de decisão vs. problemas de busca}
    \begin{itemize}
        \item Relação com funções booleanas
        \item Instância $\SAT$ tautológica retorna $\{1\}$
        \item Instância $\SAT$ contraditória retorna $\{0\}$
        \item Instância $\SAT$ satisfazível, mas não tautológica retorna $\{0, 1\}$
    \end{itemize}
\end{frame}
\begin{frame}
    \frametitle{Definição}
    Seja $M$ uma máquina de Turing não-determinística
    e $x$ uma entrada.

    Se todos os ramos da computação de $M$ em $x$ pararem,
    chame de $M(x)$ o maior valor
    (lexicograficamente)
    dentre todos os ramos de computação.

    Se algum ramo não parar,
    deixe $M(x)$ indefinido.
\end{frame}

\subsection{Definições alternativas}
\begin{frame}
    \frametitle{Outras definições}
    \begin{itemize}
        \item Linguagens polinomialmente equilibradas
        \item Conjunto potência como contradomínio
        \item Definição de Hopcroft e Ullman
        \item Classe $\OptP$
    \end{itemize}
\end{frame}

\subsection{$\NP$-completude funcional}
\begin{frame}
    \frametitle{$\NP$-completude funcional}
    \begin{definition}
        Uma função $f \in \FNP$ é $\FNP$-completa se,
        para toda função $g \in \FNP$
        existir alguma função $h \in \FP$ tal que
        \begin{equation*}
            g(x) = f(h(x)).
        \end{equation*}
    \end{definition}
\end{frame}

\section{Hierarquia polinomial}

\subsection{Hierarquia polinomial}
\begin{frame}
    \frametitle{Hierarquia polinomial}
    \centering
    \begin{tikzpicture}
        \draw let \n1 = {0}, \n2 = {8}, \n3 = {1.5}, \n4 = {0.5} in
            (\n1, -\n4) -- (\n2, -\n4)
            (\n1, 0) -- (\n2, \n3)
            (\n1, \n3) -- (\n2, 0)
            (\n1, \n3) -- (\n2, 2*\n3)
            (\n1, 2*\n3) -- (\n2, \n3)
            (\n1, 2*\n3) -- (\n2, 3*\n3)
            (\n1, 3*\n3) -- (\n2, 2*\n3)
            (\n1, 4*\n3) -- (\n2, 4*\n3)
            (\n1, -\n4) -- (\n1, 4*\n3)
            (\n2, -\n4) -- (\n2, 4*\n3)
            node (a) at ({(\n1+\n2)/2}, 0) {
                $\Sigma_0^p = \Delta_0^p = \Pi_0^p = p$
            }
            node (a) at (\n1+2*\n4, \n3/2) {$\Sigma_1^p = \NP$}
            node (a) at (\n2-2*\n4, \n3/2) {$\Pi_1^p = \coNP$}
            node (a) at ({\n1+\n2/2}, \n3) {$\Delta_2^p = \P^\NP$}
            node (a) at (\n1+\n4, 3*\n3/2) {$\Sigma_2^p$}
            node (a) at (\n2-\n4, 3*\n3/2) {$\Pi_2^p$}
            node (a) at ({\n1+\n2/2}, 2*\n3) {$\Delta_3^p$}
            node (a) at (\n1+\n4, 5*\n3/2) {$\Sigma_3^p$}
            node (a) at (\n2-\n4, 5*\n3/2) {$\Pi_3^p$}
            node (a) at ({\n1+\n2/2}, 3*\n3) {$\vdots$}
            node (a) at ({\n1+\n2/2}, 7*\n3/2) {$\PH \subseteq \PSPACE$};
    \end{tikzpicture}
\end{frame}

\newlang\coFNP{coFNP}
\subsection{Hierarquia polinomial funcional}
\begin{frame}
    \frametitle{Hierarquia polinomial funcional}
    \begin{columns}
        \begin{column}{0.2\textwidth}
            \begin{align*}
                \Sigma_0^p &= \Delta_0^p = \Pi_0^p = \P \\
                \Sigma_{n+1}^p &= \NP^{\Sigma_n^p} \\
                \Delta_{n+1}^p &= \P^{\Sigma_n^p} \\
                \Pi_{n+1}^p &= \coNP^{\Sigma_n^p}
            \end{align*}
        \end{column}
        \begin{column}{0.2\textwidth}
            \begin{align*}
                \Sigma_0^f &= \Delta_0^f = \Pi_0^f = \FP \\
                \Sigma_{n+1}^f &= \FNP^{\Sigma_n^f} \\
                \Delta_{n+1}^f &= \FP^{\Sigma_n^f} \\
                \Pi_{n+1}^f &= \coFNP^{\Sigma_n^f}
            \end{align*}
        \end{column}
    \end{columns}
\end{frame}

\subsection{Fecho compositivo}
\begin{frame}
    \frametitle{Fecho compositivo}

    Se $\mathcal F$ é um conjunto de funções,
    chame de $\mathcal F^\circ$
    o \emph{fecho compositivo} de $\mathcal F$;
    isto é, o conjunto de todas as funções na forma
    \begin{equation*}
        f_1 \circ f_2 \circ \dots \circ f_n
    \end{equation*}
    para $f_1, \dots, f_n \in \mathcal F$.

    \begin{theorem}
        \begin{equation*}
            (\Sigma_n^f)^\circ = \Delta_{n+1}^f
        \end{equation*}
    \end{theorem}
\end{frame}

\section{Trabalhos futuros}
\begin{frame}
    \frametitle{Trabalhos futuros}
    \begin{itemize}
        \item Problemas completos
        \item Talvez a definição tenha ficado forte demais
        \item Relação com a classe $\OptP$, de Krentel
    \end{itemize}
\end{frame}

\end{document}
