\documentclass[utf8,notheorems]{beamer}
\usetheme[compress]{Singapore}

\usepackage[brazil]{babel}
\usepackage{tikz}
\usetikzlibrary{calc}

\newtheorem*{theorem}{Teorema}
\newtheorem*{corollary}{Corolário}
\theoremstyle{definition}
\newtheorem*{definition}{Definição}

\let\C\someundefinedcommand
\let\G\someundefinedcommand
% These two commands are defined by hyperref in file puenc.def,
% and conflict with complexity - which also defines them.
% \C encodes some unicode character (U+030f), \G is textdoublegrave.
% Seems safe to simply undefine them.
\usepackage{complexity}

\begin{document}

\author{Tiago Royer}
\title{
    Máquinas de Turing não-determinísticas \\
    como computadores de funções
}
\subtitle{Defesa do TCC}
\date{12 de novembro de 2015}
\institute{UFSC}
\begin{frame}
    \titlepage
\end{frame}

\section{Axiomas de Blum}

\subsection{Enumerações de Gödel}
\begin{frame}
    \frametitle{Enumerações de Gödel aceitáveis das funções recursivas}
    \begin{definition}
        Seja $\mathcal P$ o conjunto das funções recursivas parciais.
        Uma \emph{enumeração de Gödel aceitável}
        é uma função sobrejetora $\phi: \{0, 1\}^* \to \mathcal P$,
        tal que exista uma função recursiva parcial $g$ tal que
        \begin{equation*}
            g(x, y) = \phi_x(y)
        \end{equation*}
        e uma função recursiva total $\sigma$ tal que
        \begin{equation*}
            \phi_{\sigma(x, y)}(z) = \phi_x(y, z).
        \end{equation*}
    \end{definition}
\end{frame}

\subsection{Axiomas de Blum}
\begin{frame}
    \frametitle{Axiomas de Blum}
    \begin{definition}
        Uma \emph{medida de complexidade}
        é uma função $\Phi:\{0, 1\}^* \times \{0, 1\}^* \to \mathbb N$
        tal que
        \begin{itemize}
            \item $\Phi(w, x)$ está definido se, e somente se, $\phi_w(x)$ está definido;
            \item O predicado ``$\Phi(w, x) = n?$'' é recursivo.
        \end{itemize}
    \end{definition}
\end{frame}

\subsection{Programador Incompetente}
\begin{frame}
    \frametitle{``Teorema do programador incompetente''}
    \begin{theorem}
        Seja $\phi$ uma numeração de Gödel aceitável,
        $\Phi$ uma medida de complexidade para $\phi$,
        e $f: \{0, 1\}^* \to \{0, 1\}^*$ e $g: \mathbb N \to \mathbb N$
        funções recursivas totais.
        Então existe um índice $w$ para a função $f$ tal que
        \begin{equation*}
            \Phi(w, x) > g(|x|)
        \end{equation*}
        para todo $x$.
    \end{theorem}
\end{frame}

\subsection{Máquinas não-determinísticas}
\begin{frame}
    \frametitle{E para máquinas não-determinísticas?}

    \begin{itemize}
        \item Para definir complexidade, precisamos de uma enumeração de Gõdel
        \item Para definir a enumeração, precisamos calcular funções
        \item Como uma máquina não-determinística calcula uma função?
    \end{itemize}
\end{frame}

\section{Funções não-determinísticas}

\subsection{Definição}
\begin{frame}
    \frametitle{Problemas de decisão vs. problemas de busca}
\end{frame}
\begin{frame}
    \frametitle{Definição}
\end{frame}

\subsection{Definições alternativas}
\begin{frame}
    \frametitle{Outras definições}
\end{frame}

\subsection{$\NP$-completude funcional}
\begin{frame}
    \frametitle{$\NP$-completude funcional}
\end{frame}

\section{Hierarquia polinomial}

\subsection{Hierarquia polinomial}
\begin{frame}
    \frametitle{Hierarquia polinomial}
\end{frame}

\subsection{Hierarquia polinomial funcional}
\begin{frame}
    \frametitle{Hierarquia polinomial funcional}
\end{frame}

\subsection{Fecho compositivo}
\begin{frame}
    \frametitle{Fecho compositivo}
\end{frame}

\section{Trabalhos futuros}
\begin{frame}
    \frametitle{Trabalhos futuros}
\end{frame}

\end{document}
