\documentclass[utf8,notheorems]{beamer}
\usetheme[compress]{Singapore}

\usepackage[brazil]{babel}
\usepackage{tikz}
\usetikzlibrary{calc}

\newtheorem*{theorem}{Teorema}
\newtheorem*{corollary}{Corolário}
\theoremstyle{definition}
\newtheorem*{definition}{Definição}

\let\C\someundefinedcommand
\let\G\someundefinedcommand
% These two commands are defined by hyperref in file puenc.def,
% and conflict with complexity - which also defines them.
% \C encodes some unicode character (U+030f), \G is textdoublegrave.
% Seems safe to simply undefine them.
\usepackage{complexity}

\begin{document}

\author{Tiago Royer}
\title{
    Máquinas de Turing não-determinísticas \\
    como computadores de funções
}
\subtitle{Defesa do TCC}
\date{12 de novembro de 2015}
\institute{UFSC}
\begin{frame}
    \titlepage
\end{frame}

\section{Axiomas de Blum}

\subsection{Enumerações de Gödel}
\begin{frame}
    \frametitle{Enumerações de Gödel aceitáveis das funções recursivas}
\end{frame}

\subsection{Axiomas de Blum}
\begin{frame}
    \frametitle{Axiomas de Blum}
\end{frame}

\subsection{Programador Incompetente}
\begin{frame}
    \frametitle{``Teorema do programador incompetente''}
\end{frame}

\subsection{Máquinas não-determinísticas}
\begin{frame}
    \frametitle{E para máquinas não-determinísticas?}
\end{frame}

\section{Funções não-determinísticas}

\subsection{Definição}
\begin{frame}
    \frametitle{Problemas de decisão vs. problemas de busca}
\end{frame}
\begin{frame}
    \frametitle{Definição}
\end{frame}

\subsection{Definições alternativas}
\begin{frame}
    \frametitle{Outras definições}
\end{frame}

\subsection{$\NP$-completude funcional}
\begin{frame}
    \frametitle{$\NP$-completude funcional}
\end{frame}

\section{Hierarquia polinomial}

\subsection{Hierarquia polinomial}
\begin{frame}
    \frametitle{Hierarquia polinomial}
\end{frame}

\subsection{Hierarquia polinomial funcional}
\begin{frame}
    \frametitle{Hierarquia polinomial funcional}
\end{frame}

\subsection{Fecho compositivo}
\begin{frame}
    \frametitle{Fecho compositivo}
\end{frame}

\section{Trabalhos futuros}
\begin{frame}
    \frametitle{Trabalhos futuros}
\end{frame}

\end{document}
