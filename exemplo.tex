%%%%%%%%%%%%%%%%%%%%%%%%%%%%%%%%%%%%%%%%%%%%%%%%%%%%%%%%%%%%%%%%%%%%%%%
% Universidade Federal de Santa Catarina
% Biblioteca Universitária
%----------------------------------------------------------------------
% Exemplo de utilização da documentclass ufscThesis
%----------------------------------------------------------------------
% (c)2013 Roberto Simoni (roberto.emc@gmail.com)
%         Carlos R Rocha (cticarlo@gmail.com)
%         Rafael M Casali (rafaelmcasali@yahoo.com.br)
%%%%%%%%%%%%%%%%%%%%%%%%%%%%%%%%%%%%%%%%%%%%%%%%%%%%%%%%%%%%%%%%%%%%%%%
\documentclass{ufscThesis/ufscThesis}

% Pacotes usados especificamente neste documento
\usepackage{tikz}
\usepackage{listings}
% The package listings is unable to deal correctly
% with multibyte strings.
% A workaround is to replace every "offending character"
% with its respective TeX escape codes.
%
% This is acomplished with the 'literate programming'
% feature of the listings package.
\lstset{
literate=
    {á}{{\' a}}1
    {Á}{{\' A}}1
    {à}{{\` a}}1
    {À}{{\` A}}1
    {â}{{\^ a}}1
    {Â}{{\^ A}}1
    {ã}{{\~ a}}1
    {Ã}{{\~ A}}1
    {ç}{{\c c}}1
    {Ç}{{\c C}}1
    {é}{{\' e}}1
    {É}{{\' E}}1
    {ê}{{\^ e}}1
    {Ê}{{\^ E}}1
    {í}{{\' \i}}1
    {Í}{{\' I}}1
    {ó}{{\' o}}1
    {Ó}{{\' O}}1
    {ô}{{\^ o}}1
    {Ô}{{\^ O}}1
    {õ}{{\~ o}}1
    {Õ}{{\~ O}}1
    {ú}{{\' u}}1
    {Ú}{{\' U}}1
}


\begin{document}

% Capa e folha de rosto

\instituicao[a]{Universidade Federal de Santa Catarina}
\departamento[o]{Informática e Estatística}
\curso[o]{Programa de Graduação em Ciência da Computação}
\documento[o]{{Trabalho de Conclusão de Curso}}
\titulo{Funções não"=determinísticas}
\autor{Tiago Royer}
\grau{Bacharel em Ciência da Computação}
\local{Florianópolis}
\data{01}{novembro}{2015}
\orientador[Orientadora]{Profa. Dra. Jerusa Marchi}
\coordenador{Prof. Dr. Mário Antônio Ribeiro Dantas}

\capa
\folhaderosto

% Informações sobre a banca

\numerodemembrosnabanca{4} % Isso decide se haverá uma folha adicional
\orientadornabanca{nao} % Se faz parte da banca definir como sim
\coorientadornabanca{sim} % Se faz parte da banca definir como sim
\bancaMembroA{Primeiro membro\\Universidade ...} %Nome do presidente da banca
\bancaMembroB{Segundo membro\\Universidade ...} % Nome do membro da Banca
\bancaMembroC{Terceiro membro\\Universidade ...} % Nome do membro da Banca
\bancaMembroD{Quarto membro\\Universidade ...} % Nome do membro da Banca
%\bancaMembroE{Quinto membro\\Universidade ...} % Nome do membro da Banca
%\bancaMembroF{Sexto membro\\Universidade ...} % Nome do membro da Banca
%\bancaMembroG{Sétimo membro\\Universidade ...} % Nome do membro da Banca

\folhaaprovacao

% Dedicatória

\dedicatoria{
    À minha mãe, Venida
}

\paginadedicatoria

% Agradecimentos

\agradecimento{
    Agradeço muito minha família, em especial à minha mãe,
    por sempre terem me apoiado, guiado e auxiliado
    ao longo dos últimos 20 anos.

    Agradeço minha orientadora, Jerusa,
    que me acolheu logo no início do curso,
    por todas as conversas e orientações,
    que muitas vezes extrapolaram os limites da Academia.

    Agradeço aos professores do Programa Avançado de Matemática
    (Melissa, Gilles, Fernando),
    que me ajudaram a construir a maturidade matemática
    que me foi extremamente útil nos quatro anos de graduação.

    Agradeço as discussões com Santana, Schultz, Gabriel e Abouhatem
    sobre os algoritmos da Maratona de Programação.

    Agradeço todas as discussões produtivas (e improdutivas)
    com o pessoal do laboratório IATE.

    E, por último, agradeço o companheirismo dos meus amigos
    ao longo desta jornada.
}

\paginaagradecimento

% Epígrafe

\epigrafe{
    Even if one works basically all week trying to prove $\P \neq \NP$,
    one should set aside Friday afternoon for trying to prove $\P = \NP$.
}
{
    (John Edward Hopcroft)
}
\paginaepigrafe

% Resumo

\textoResumo{
    Complexidade de Circuitos é uma sub"=área da Teoria da Computação
    que estuda a taxa de crescimento de circuitos booleanos.
    Em particular, um dos objetivos desta área é demostrar que certas
    funções booleanas exigem circuitos que crescem exponencialmente
    conforme cresce o tamanho das palavras --- e, como consequências
    disso, concluir que $\P \neq \NP$ \cite{Sipser2006}.
}
\palavrasChave{
    Complexidade de Circuitos,
    Complexidade Computacional
}

\paginaresumo

% Resumo, em inglês.

\textAbstract{
    Normally, the concepts of time and space complexity
    for deterministic and nondeterministic machines
    are defined separately,
    although they share several similar theorems.
    \citeonline[p.~324]{Blum1967} define a ``computational resource''
    using two axioms;
    this allows for an axiomatic treatment of the computational complexity,
    enabling us to provide a single proof for these analogous theorems.

    However, the Blum axioms are not defined in terms of deciders,
    but for integer function computers
    (that is, Turing machines that compute functions of the form
    $f : \mathbb N \to \mathbb N$ \cite[p.~151]{HopcroftUllman1979}).
    Therefore,
    to define $\P$ and $\NP$ via Blum axioms,
    we need to see a decider as an integer function computer.
    The case where the machine is deterministic is easy,
    because it have a single sequence of computation.
    The nondeterministic case, which is more complex,
    is studied in this text.

    In this text,
    we define the concept of ``nondeterministic function''
    in a way that the composition of nondeterministic functions
    correspond to the composition of the analogous deterministic functions.
}
\keywords{
    Computational complexity,
    nondeterministic functions,
    function composition.
}

\paginaabstract

% Listas usadas no documento.

%\listadefiguras
%\listadetabelas
%\listadeabreviaturas
\listadesimbolos


O conceito de ``oráculos computacional'' é um mecanismo bastante poderoso
para definir novas classes de complexidade computacional.
Com ele,
podemos formalizar perguntas como
"Que linguagens conseguiríamos reconhecer
se desse para resolver $\SAT$ gratuitamente?",
ou "E se o problema da parada fosse decidível?".

Na seção~\ref{sec:oracle_definition} será apresentada a definição formal
de ``oráculo computacional''.
Neste primeiro momento,
trabalharemos apenas com linguagens como oráculos;
na seção~\ref{sec:functional_oracles}
utilizaremos funções em vez de linguagens.

A seção~\ref{sec:oracle_equivalence}
discute a interpretação de utilizar um problema indecidível
(como o problema da parada)
como oráculo.
Ainda no contexto da indecidibilidade,
esta seção define o que significa dois oráculos serem equivalentes,
e a seção~\ref{sec:arithmetical_hierarchy}
constrói uma hierarquia de problemas indecidíveis.
A seção~\ref{sec:oracles_and_reductions}
interpreta o conceito de redução em termos de oráculos.
Por fim, a seção~\ref{sec:polynomial_hierarchy}
utiliza o conceito de oráculos para construir a hierarquia polinomial.

% Desenvolvimento
\chapter{EXPOSIÇÃO DO TEMA OU MATÉRIA}

É a parte principal e mais extensa do trabalho.
Deve apresentar a fundamentação teórica,
a metodologia,
os resultados
e a discussão.
Divide-se em seções e subseções
conforme a NBR 6024~\cite{abnt14724}.
Quanto a sua estrutura,
segue as recomendações da norma
para preparação de trabalhos acadêmicos,
a NBR 14724 de 2011~\cite{abnt14724}.
Quanto à Formatação,
segue o modelo adotado pela UFSC, o formato A5.

\section{FORMATAÇÃO DO TEXTO}

No que diz respeito à estrutura do trabalho,
o novo modelo para dissertações e teses
adotado pela UFSC
segue a NBR 14724 (2011).
Porém, em relação à formatação,
a UFSC adotou o tamanho A5,
que corresponde à metade do A4.
Por esta razão,
foi necessário uma adequação no tamanho da fonte,
espaçamento entrelinhas,
margens,
etc,
conforme exposto no quadro abaixo.

O texto deve ser justificado,
digitado em cor preta,
podendo utilizar outras cores
somente para as ilustrações.
Utilizar papel branco.
Os elementos pré-textuais
devem iniciar no anverso da folha,
com exceção da ficha catalográfica.
Os elementos textuais e pós-textuais
devem ser digitados no anverso e verso das folhas,
com espaçamento simples (1).

\subsection{As ilustrações}

Independente do tipo de ilustração 
(quadro, desenho, figura, fotografia, mapa, entre outros)
sua identificação aparece na parte superior,
precedida da palavra designativa.

A indicação da fonte consultada
deve aparecer na parte inferior,
elemento obrigatório mesmo que seja produção do próprio autor.
A ilustração deve ser citada no texto
e inserida o mais próximo possível do texto
a que se refere~\cite{abnt14724}.

A Figura \ref{fig:a} mostra o logo da BU
\begin{figure}[!htb]
   \centering
   \caption{Logo da BU.}\label{fig:a}
   \includegraphics[width=0.3\textwidth]{figuras/brasaoBU.jpg}
\end{figure}

A Tabela~\ref{tab:a} mostra mais informações do template BU.

\begin{table}[!htb]
\begin{center}
    \caption{Formatação do texto.}\label{tab:a}
    \begin{tabular}{ p{3cm} | p{6cm} }
        \hline
        Cor & Branco \\
        \hline
        Formato do papel & A5 \\
        \hline
        Gramatura & 75 \\
        \hline
        Impressão & Frente e verso \\
        \hline
        Margens & Espelhadas: superior 2, Inferior: 1,5, Externa 1,5 e Externa: 2. \\
        \hline
        Cabeçalho & 0,7 \\
        \hline
        Rodapé & 0,7 \\
        \hline
        Paginação & Externa \\
        \hline
        Alinhamento vertical & Superior \\
        \hline
        Alinhamento do texto & Justificado \\
        \hline
        Fonte sugerida & Times New Roman  \\
        \hline
        Tamanho da fonte & 10,5 para o texto
        incluindo os títulos das seções e subseções.
        As citações com mais de três linhas
        as legendas das ilustrações e tabelas, fonte 9,5. \\
        \hline
        Espaçamento entre linhas & Um (1) simples \\
        \hline
        Espaçamento entre parágrafos & Anterior 0,0; Posterior 0,0 \\
        \hline
        Numeração da seção & As seções primárias devem começar
        sempre em páginas ímpares.
        Deixar um espaço (simples) entre o título da seção
        e o texto e entre o texto e o título da subseção. \\
        \hline
    \end{tabular}
\end{center}
Fonte: Universidade Federal de Santa Catarina (2011)
\end{table}

\subsection{Equações e fórmulas}

As equações e fórmulas
devem ser destacadas no texto
para facilitar a leitura.
Para numerá-las,
deve-se usar algarismos arábicos entre parênteses
e alinhados à direita.
Pode-se usar uma entrelinha maior do que a usada no texto~\cite{abnt14724}.

Exemplo: A equação \ref{eq:a}
\begin{equation}
    x^2 + y^2 = z^2
    \label{eq:a}
\end{equation}
 e a equação \ref{eq:b}
\begin{equation}
    x^2 + y^2 = n
    \label{eq:b}
\end{equation}

\section{EXEMPLO DE COMO GERAR A LISTA DE SÍMBOLOS E ABREVIATURAS}

Para gerar a lista de símbolos e abreviaturas use os comandos

\abreviatura{ABNT}{Associação Brasileira de Normas Técnicas}
\abreviatura{IBGE}{Instituto Brasileiro de Geografia e Estatística}
\simbolo{$\int$}{Integral}
\simbolo{$\prod$}{Produtório}

\begin{lstlisting}
\simbolo{símbolo}{descrição}
\end{lstlisting}

\begin{lstlisting}
\abreviatura{abreviatura}{descrição}
\end{lstlisting}

\subsection{Exemplo de citações no \LaTeX}

Segundo \citeonline{alves_2001} ...

...no final da frase \cite{abnt14724,BU_formatoA5}

\nocite{alves_2001,abnt10520,abnt6024,abnt14724}
 % Aqui vêm uma sequência de capítulos
\chapter{Comparação com outros trabalhos}

Este capítulo discute brevemente outros trabalhos que,
de certa forma,
tentam capturar a noção de ``função não"=determinística''.

\section{Problemas de busca vs problemas de decisão (Papadimitriou)}
\label{sec:papadimitriou_comparison}

\citeonline[p.~229]{Papadimitriou1994}
define $\FNP$ diferente de nossa definição
(na seção~\ref{sec:functional_complexity}).

\begin{definition}
    Uma linguagem $L$ é \emph{polinomialmente equilibrada}
    se existir algum polinômio $p$ tal que
    todos os elementos de $L$ são pares ordenados da forma $(x, y)$
    em que $|y| \leq p(|x|)$.
\end{definition}
Isto é, $L$ é constituída de pares,
de forma que o segundo elemento do par não é muito maior que o primeiro elemento.

Linguagens de $\NP$ e linguagens polinomialmente equilibradas
estão relacionadas por certificados de pertinência.
Por exemplo,
para a linguagem $\SAT$,
podemos provar eficientemente
(isto é, em tempo polinomial)
que determinada instância $\varphi$ é satisfazível
se fornecermos uma atribuição de valores"=verdade que satisfaz a instância;
esta atribuição é uma \emph{testemunha} ou \emph{certificado}
para a pertinência de $\varphi$ a $\SAT$.

Para cada linguagem $L \in \NP$,
podemos sistematicamente prover certificados de pertinência para $L$:
como sabemos que existe uma máquina de Turing não"=determinística que decide $L$,
podemos fornecer a sequência de transições não"=determinísticas
como um certificado de pertinência.
Como esta máquina opera em tempo polinomial,
para cada $x \in L$,
o par
\begin{equation*}
    (x, y),
\end{equation*}
em que $y$ é esta sequência de transições,
satisfará $|y| \leq p(|x|)$,
para algum polinômio $p$
--- no caso, $p$ é o próprio limite de tempo da máquina não"=determinística
que reconhece $L$.

Dessa forma,
podemos, sistematicamente,
associar uma linguagem $L \in \NP$
a uma linguagem polinomialmente equilibrada $R_L \in \P$.
Assim,
construiremos o conjunto $\FNP$ definido por \citeonline[p.~229]{Papadimitriou1994}.

\begin{definition}
    Se $L \in \NP$, chame de $R_L$
    uma linguagem polinomialmente equilibrada associada com $L$.
    Então defina $\FNP$ por
    \begin{equation*}
        \FNP = \{R_L \mid L \in \NP\}
    \end{equation*}
    \cite[p.~229]{Papadimitriou1994}
\end{definition}

Estes problemas são por vezes chamados de \emph{problemas de busca}
(do inglês \emph{search problem}),
em oposição a \emph{problemas de decisão}.

Do ponto de vista computacional,
a definição de Papadimitriou captura a ideia de
encontrar alguma solução para o problema $L$,
potencialmente descartando uma quantidade exponencial de outras soluções.


\bibliographystyle{ufscThesis/ufsc-alf}
\bibliography{bibliografia}

\apendice
% Exemplo de apêndice

\chapter{EXEMPLIFICANDO UM APÊNDICE}

Texto do Apêndice aqui.

% O comando \apendice deve aparecer apenas uma vez,
% antes de todos os apêndices.

\anexo
% Exemplo de anexo.

\chapter{EXEMPLIFICANDO UM ANEXO}

Texto do anexo aqui.

% O comando \anexo também deve aparecer apenas uma vez.

\end{document}
