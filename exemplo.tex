%%%%%%%%%%%%%%%%%%%%%%%%%%%%%%%%%%%%%%%%%%%%%%%%%%%%%%%%%%%%%%%%%%%%%%%
% Universidade Federal de Santa Catarina             
% Biblioteca Universitária                     
%----------------------------------------------------------------------
% Exemplo de utilização da documentclass ufscThesis
%----------------------------------------------------------------------                                                           
% (c)2013 Roberto Simoni (roberto.emc@gmail.com)
%         Carlos R Rocha (cticarlo@gmail.com)
%         Rafael M Casali (rafaelmcasali@yahoo.com.br)
%%%%%%%%%%%%%%%%%%%%%%%%%%%%%%%%%%%%%%%%%%%%%%%%%%%%%%%%%%%%%%%%%%%%%%%
\documentclass{ufscThesis} % Definicao do documentclass ufscThesis	

%----------------------------------------------------------------------
% Pacotes usados especificamente neste documento
\usepackage{graphicx} % Possibilita o uso de figuras e gráficos
\usepackage{color}    % Possibilita o uso de cores no documento
\usepackage{listings}
%----------------------------------------------------------------------
% Comandos criados pelo usuário
\newcommand{\afazer}[1]{{\color{red}{#1}}} % Para destacar uma parte a ser trabalhada
\newcommand{\ABNTbibliographyname}{REFERÊNCIAS} % Necessário para abnTeX 0.8.2

%----------------------------------------------------------------------
% Identificadores do trabalho
% Usados para preencher os elementos pré-textuais
\instituicao[a]{Universidade Federal de Santa Catarina} % Opcional
\departamento[a]{Biblioteca Universitária}
\curso[o]{Programa de ...}
\documento[a]{Tese} % [o] para dissertação [a] para tese
\titulo{Título}
\subtitulo{Subtítulo (se houver)} % Opcional
\autor{Nome completo do autor}
\grau{...}
\local{Florianópolis} % Opcional (Florianópolis é o padrão)
\data{04}{junho}{2013}
\orientador[Orientador\\Universidade ...]{Prof. Dr.}
\coorientador[Coorientador\\Universidade ...]{Prof. Dr.}
\coordenador[Coordenador\\Universidade ...]{Prof. Dr. }

\numerodemembrosnabanca{4} % Isso decide se haverá uma folha adicional
\orientadornabanca{nao} % Se faz parte da banca definir como sim
\coorientadornabanca{sim} % Se faz parte da banca definir como sim
\bancaMembroA{Primeiro membro\\Universidade ...} %Nome do presidente da banca
\bancaMembroB{Segundo membro\\Universidade ...}      % Nome do membro da Banca
\bancaMembroC{Terceiro membro\\Universidade ...}     % Nome do membro da Banca
\bancaMembroD{Quarto membro\\Universidade ...}       % Nome do membro da Banca
%\bancaMembroE{Quinto membro\\Universidade ...}       % Nome do membro da Banca
%\bancaMembroF{Sexto membro\\Universidade ...}        % Nome do membro da Banca
%\bancaMembroG{Sétimo membro\\Universidade ...}       % Nome do membro da Banca

\dedicatoria{Este trabalho é dedicado aos meus colegas de classe e aos meus queridos pais.}

\agradecimento{Inserir os agradecimentos aos colaboradores à execução do trabalho.}

\epigrafe{Texto da Epígrafe. Citação relativa ao tema do trabalho. É opcional. A epígrafe pode também aparecer na abertura de cada seção ou capítulo.}
{(Autor da epígrafe, ano)}

\textoResumo {O texto do resumo deve ser digitado, em um único bloco, sem espaço de parágrafo. O resumo deve ser significativo, composto de uma sequência de frases concisas, afirmativas e não de uma enumeração de tópicos. Não deve conter citações. Deve usar o verbo na voz passiva. Abaixo do resumo, deve-se informar as palavras-chave (palavras ou expressões significativas retiradas do texto) ou, termos retirados de thesaurus da área.}
\palavrasChave {Palavra-chave 1. Palavra-chave 2.  Palavra-chave 3. }
 
\textAbstract {Resumo traduzido para outros idiomas, neste caso, inglês. Segue o formato do resumo feito na língua vernácula. As palavras-chave traduzidas, versão em língua estrangeira, são colocadas abaixo do texto precedidas pela expressão ``Keywords'', separadas por ponto.}
\keywords {Keyword 1. Keyword 2. Keyword 3.}

%----------------------------------------------------------------------
% Início do documento                                
\begin{document}
%--------------------------------------------------------
% Elementos pré-textuais
%\capa  
\folhaderosto[comficha] % Se nao quiser imprimir a ficha, é só não usar o parâmetro
\folhaaprovacao
\paginadedicatoria
\paginaagradecimento
\paginaepigrafe
\paginaresumo
\paginaabstract
%\pretextuais % Substitui todos os elementos pre-textuais acima
\listadefiguras % as listas dependem da necessidade do usuário
\listadetabelas 
\listadeabreviaturas
\listadesimbolos
\sumario
%--------------------------------------------------------
% Elementos textuais

\chapter{Introdução}
As orientações aqui apresentadas são baseadas em um conjunto de normas elaboradas pela ABNT. Além das normas técnicas a Biblioteca também elaborou  uma série de tutoriais e guias que estão disponíveis na sua Homepage. \url{http://portalbu.ufsc.br/normalizacao-de-trabalhos-2/}. 


\section{OBJETIVOS}

Descrição...

\subsection{Objetivo Geral}

Descrição...

\subsection{Objetivos Específicos}

Descrição...

\chapter{DESENVOLVIMENTO}

\section{EXPOSIÇÃO DO TEMA OU MATÉRIA}

É a parte principal e mais extensa do trabalho. Deve apresentar a fundamentação teórica, a metodologia, os resultados e a discussão. Divide-se em seções e subseções conforme a NBR 6024~\cite{abnt14724}. Quanto a sua estrutura, segue as recomendações da norma para preparação de trabalhos acadêmicos, a NBR 14724 de 2011~\cite{abnt14724}. Quanto à Formatação, segue o modelo adotado pela UFSC, o formato A5.


\subsection{Formatação do texto}

No que diz respeito à estrutura do trabalho, o novo modelo para dissertações e teses adotado pela UFSC segue a NBR 14724 (2011). Porém, em relação à formatação, a UFSC adotou o tamanho A5, que corresponde à metade do A4. Por esta razão, foi necessário uma adequação no tamanho da fonte, espaçamento entrelinhas, margens, etc, conforme exposto no quadro abaixo.

O texto deve ser justificado, digitado em cor preta, podendo utilizar outras cores somente para as ilustrações. Utilizar papel branco. Os elementos pré-textuais devem iniciar no anverso da folha, com  exceção da ficha catalográfica. Os elementos textuais e pós-textuais devem ser digitados no anverso e verso das folhas, com espaçamento simples (1). 



\subsubsection{As ilustrações}

Independente do tipo de ilustração (quadro, desenho, figura, fotografia, mapa, entre outros) sua identificação aparece na parte superior, precedida da palavra designativa. 



A indicação da fonte consultada deve aparecer na parte inferior, elemento obrigatório mesmo que seja produção do próprio autor. A ilustração deve ser citada no texto e inserida o mais próximo possível do texto a que se refere~\cite{abnt14724}. 

A Figura \ref{fig:a} mostra o logo da BU
\begin{figure}[!htb]
   \centering
   \caption{Logo da BU.}\label{fig:a}
   \includegraphics[width=0.3\textwidth]{figuras/brasaoBU.jpg}
\end{figure}

A Tabela~\ref{tab:a} mostra mais informações do template BU.

\begin{table}[!htb]
\begin{center}
    \caption{Formatação do texto.}\label{tab:a}
  \begin{tabular}{ p{3cm} | p{6cm} }
    \hline
Cor & Branco\\ \hline
Formato do papel & A5\\ \hline
Gramatura & 75\\ \hline
Impressão & Frente e verso\\ \hline
Margens & Espelhadas: superior 2, Inferior: 1,5, Externa 1,5 e Externa: 2.\\ \hline
Cabeçalho & 0,7\\ \hline
Rodapé & 0,7\\ \hline
Paginação & Externa\\ \hline
Alinhamento vertical & Superior\\ \hline
Alinhamento do texto & Justificado\\ \hline
Fonte sugerida & Times New Roman \\ \hline
Tamanho da fonte & 10,5 para o texto incluindo os títulos das seções e subseções. As citações com mais de três linhas as legendas das ilustrações e tabelas, fonte 9,5.\\ \hline
Espaçamento entre linhas & Um (1) simples\\ \hline
Espaçamento entre parágrafos & Anterior 0,0; Posterior 0,0\\ \hline
Numeração da seção & As seções  primárias devem  começar  sempre em páginas ímpares. Deixar um espaço (simples) entre o título da seção e o texto e  entre o texto e o título da subseção. \\  \hline
  \end{tabular}
\end{center}
Fonte: Universidade Federal de Santa Catarina (2011)
\end{table}



\subsubsection{Equações e fórmulas}

As equações e fórmulas devem ser destacadas no texto para facilitar a leitura.  Para numerá-las, deve-se usar algarismos arábicos entre parênteses e alinhados à direita. Pode-se usar uma entrelinha maior do que a usada no texto~\cite{abnt14724}.

Exemplo: A equação \ref{eq:a}
\begin{equation}
 x^2 + y^2 = z^2
 \label{eq:a}
\end{equation}
 e a equação  \ref{eq:b}
\begin{equation}
 x^2 + y^2 = n
\label{eq:b}
\end{equation}

\subsection{Exemplo de como gerar a lista de símbolos e abreviaturas}

Para gerar a lista de símbolos e abreviaturas use os comandos

\abreviatura{ABNT}{Associação Brasileira de Normas Técnicas}
\abreviatura{IBGE}{Instituto Brasileiro de Geografia e Estatística}
\simbolo{$\int$}{Integral}
\simbolo{$\prod$}{Produtório}

\begin{lstlisting}
\simbolo{símbolo}{descrição}
\end{lstlisting}

\begin{lstlisting}
\abreviatura{abreviatura}{descrição}
\end{lstlisting}

\subsubsection{Exemplo de citações no \LaTeX}

Segundo \citeonline{alves_2001} ...

...no final da frase \cite{abnt14724,BU_formatoA5}


\nocite{alves_2001,abnt10520,abnt6024,abnt14724}



\chapter{CONCLUSÃO}

As conclusões devem responder às questões da pesquisa, em relação aos objetivos e hipóteses. Devem ser breves podendo apresentar recomendações e sugestões para trabalhos futuros.

\bibliographystyle{ufscThesis/ufsc-alf}
\bibliography{bibliografia}

%--------------------------------------------------------
% Elementos pós-textuais
\apendice
\chapter{Exemplificando um Apêndice}
Texto do Apêndice aqui. 

\anexo
\chapter{Exemplificando um Anexo}
Texto do anexo aqui.
\end{document}
