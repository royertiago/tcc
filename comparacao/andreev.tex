\section{CONJUNTO POTÊNCIA COMO CONTRADOMÍNIO (ANDREEV)}

\citeonline[p.~3]{Andreev1994}
apresenta uma abordagem diferente.
Em vez de tentar amarrar algum dos ramos de computação,
a função retornará todos eles.

\begin{definition}
    Sejam $\mathcal A$ e $\mathcal B$ conjuntos finitos,
    e $2^\mathcal B$ o conjunto potência de $\mathcal B$.
    Uma \emph{função não"=determinística de $\mathcal A$ em $\mathcal B$}
    é uma função da forma
    \begin{equation*}
        f : \mathcal A \to 2^\mathcal B.
    \end{equation*}
    \cite[p.~3]{Andreev1994}
\end{definition}

Os conjuntos $\mathcal A$ e $\mathcal B$ usados por \citeonline[p.~4]{Andreev1994}
são da forma $\{0, 1\}^k$.
A ideia é que $\mathcal A$ represente as possíveis entradas para um circuito booleano,
e que $\mathcal B$ represente as saídas desse circuito.
Dado um circuito computacional $S$,
com $k$ entradas e $l$ saídas,
dizemos que $S$ computa a função não"=determinística $F$ de $\{0, 1\}^k$ em $\{0, 1\}^l$
se,
para todo $a \in \{0, 1\}^k$,
$S(a) \in F(a)$.
Isto é,
todas as saídas possíveis de $S$ estão previstas em $F$.
Esta definição é similar à definição de \citeonline[p.~229]{Papadimitriou1994},
no sentido de que basta retornar um valor de $F(a)$.
\citeonline[p.~4]{Andreev1994}
define, então,
a \emph{complexidade} de uma função não"=determinística $F$
como sendo o tamanho do menor circuito que computa $F$.

O fato de Andreev se restringir a domínio e contradomínio finitos
mostra que esta definição não ``combina'' corretamente
a noção de função determinística com máquinas não"=determinísticas,
pois as funções operam sobre o domínio $\{0, 1\}^*$.

Além disso,
como o domínio e o contradomínio das funções não"=determinísticas de Andreev
são diferentes,
seria necessário um malabarismo adicional
para desenvolver a noção de composição de funções,
assim como na definição de Papadimitriou.
