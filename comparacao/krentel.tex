\section{PROBLEMAS DE OTIMIZAÇÃO (KRENTEL)}

Dentre os trabalhos avaliados,
o artigo de \citeonline[p.~493]{Krentel1988}
é o que mais se aproxima do trabalho desenvolvido neste TCC.

Krentel define a classe $\OptP$,
\simbolo{$\OptP$}{Classe polinomial de problemas de otimização}
os problemas de otimização que rodam em tempo polinomial,
como sendo,
na terminologia da seção~\ref{sec:functional_complexity},
\begin{equation*}
    \OptP = \FNP \cup \coFNP.
\end{equation*}

Entretanto,
a principal diferença é na forma como é definida completude.
\begin{definition}
    Uma função $f \in \OptP$ é completa para $\OptP$
    se, para toda função $g \in \OptP$,
    existir um par de funções $T_1, T_2 \in \P$
    tais que
    \begin{equation*}
        g(x) = T_2( x, f(T_1(x)) ).
    \end{equation*}
\end{definition}

Nossa definição de $\FNP$"=completo
buscava generalizar diretamente a noção de $\NP$"=completude.
Conforme observado na demonstração do teorema~\ref{thm:pi_f_subseteq_delta_f},
podemos, em tempo polinomial determinístico,
inverter a ordenação das palavras, se elas tiverem um tamanho fixo.
Portanto,
usando este pós"=processamento,
podemos tornar uma função que maximiza seus ``valores não"=determinísticos''
(isto é, sua árvore de computação não"=determinística)
numa função que os minimiza.
Como Krentel já embutiu os problemas de maximização e minimização em $\OptP$,
não há problema em permitir o pós"=processamento feito por $T_2$.

Entretanto,
graças a $T_2$,
a classe $\OptP$ acaba ``funcionando'' como nossa classe $\Delta_n^f$,
pois,
como observado no teorema~\ref{thm:strong_compositive_closure},
podemos resolver todos os problemas de $\Delta_n^f$
apenas pós"=processando um resultado de $\FNP$.
Portanto não é possível construir uma ``hierarquia polinomial funcional''
com base em $\OptP$.
