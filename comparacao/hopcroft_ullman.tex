\section{DEFINIÇÃO DE HOPCROFT E ULLMAN}

A definição de \citeonline[p.~313]{HopcroftUllman1979},
mencionada na introdução,
é a que motivou este trabalho.
\begin{definition}
    Seja $M$ uma máquina de Turing não"=determinística.
    Dizemos que $M(x) = y$ se
    algum dos ramos de computação de $M$ em $x$ produz $y$,
    e nenhum outro ramo de computação que se encerre
    produz um valor diferente de $y$.
    \cite[p.~313]{HopcroftUllman1979}
\end{definition}

O problema é que esta definição não enumera corretamente as funções recursivas.
O grande vilão está na permissão que damos à máquina $M$
não parar em todos os ramos de computação.

Por exemplo,
construa uma máquina $N$ que,
na entrada $\langle M, x \rangle$,
assuma dois ramos de computação:
no primeiro,
$N$ incondicionalmente escreve $1$ na fita de saída e para.
No outro, $N$ simula a máquina determinística $M$ em $x$ e,
caso $M$ pare ao computar $x$,
$N$ escreve $0$ na fita.

Se $M$ não para em $x$,
então $N$ possui um único ramo de computação que para,
e este ramo produz $1$ na fita;
portanto, $N(\langle M, x \rangle) = 1$.
Caso $M$ pare ao computar $x$,
$N(\langle M, x \rangle)$ estará indefinido,
pois dois ramos de computação de $N$ escrevem coisas diferentes na fita.

Ou seja,
esta máquina computa
\begin{equation*}
    f(\langle M, x \rangle) = 1, \text{ se $M$ não parar ao computar $x$,}
\end{equation*}
que é exatamente o complemento do problema da parada.
Portanto,
pela definição de Hopcroft e Ullman,
conseguimos enumerar funções não"=recursivas.

Mas,
se impormos à definição de Hopcroft e Ullman a restrição adicional
de que todos os ramos de $M$ devem parar para que $M(x)$ exista,
então não há a necessidade de haver vários ramos de computação,
pois todos eles retornam o mesmo resultado.
Basta fixar uma das possíveis transições não"=determinísticas,
tornando a máquina \emph{determinística}.
Neste caso,
perdemos o aparente ganho de tempo exponencial ao usar não"=determinismo.
