\section{PROBLEMAS DE BUSCA VS PROBLEMAS DE DECISÃO (PAPADIMITRIOU)}
\label{sec:papadimitriou_comparison}

\citeonline[p.~229]{Papadimitriou1994}
define $\FNP$ diferente de nossa definição
(na seção~\ref{sec:functional_complexity}).

\begin{definition}
    Uma linguagem $L$ é \emph{polinomialmente equilibrada}
    se existir algum polinômio $p$ tal que
    todos os elementos de $L$ são pares ordenados da forma $(x, y)$
    em que $|y| \leq p(|x|)$.
\end{definition}
Isto é, $L$ é constituída de pares,
de forma que o segundo elemento do par não é muito maior que o primeiro elemento.

Linguagens de $\NP$ e linguagens polinomialmente equilibradas
estão relacionadas por certificados de pertinência.
Por exemplo,
para a linguagem $\SAT$,
podemos provar eficientemente
(isto é, em tempo polinomial)
que determinada instância $\varphi$ é satisfazível
se fornecermos uma atribuição de valores"=verdade que satisfaz a instância;
esta atribuição é uma \emph{testemunha} ou \emph{certificado}
para a pertinência de $\varphi$ a $\SAT$.

Para cada linguagem $L \in \NP$,
podemos sistematicamente prover certificados de pertinência para $L$:
como sabemos que existe uma máquina de Turing não"=determinística que decide $L$,
podemos fornecer a sequência de transições não"=determinísticas
como um certificado de pertinência.
Como esta máquina opera em tempo polinomial,
para cada $x \in L$,
o par
\begin{equation*}
    (x, y),
\end{equation*}
em que $y$ é esta sequência de transições,
satisfará $|y| \leq p(|x|)$,
para algum polinômio $p$
--- no caso, $p$ é o próprio limite de tempo da máquina não"=determinística
que reconhece $L$.

Dessa forma,
podemos, sistematicamente,
associar uma linguagem $L \in \NP$
a uma linguagem polinomialmente equilibrada $R_L \in \P$.
Assim,
construiremos o conjunto $\FNP$ definido por \citeonline[p.~229]{Papadimitriou1994}.

\begin{definition}
    Se $L \in \NP$, chame de $R_L$
    uma linguagem polinomialmente equilibrada associada com $L$.
    Então defina $\FNP$ por
    \begin{equation*}
        \FNP = \{R_L \mid L \in \NP\}
    \end{equation*}
    \cite[p.~229]{Papadimitriou1994}
\end{definition}

Estes problemas são por vezes chamados de \emph{problemas de busca}
(do inglês \emph{search problem}),
em oposição a \emph{problemas de decisão}.

Do ponto de vista computacional,
a definição de Papadimitriou captura a ideia de
encontrar alguma solução para o problema $L$,
potencialmente descartando uma quantidade exponencial de outras soluções.

Esta definição contorna o problema de definir ``função não"=determinística''
trabalhando diretamente com os certificados de pertinência.
Entretanto,
a ausência do conceito de função
dificulta a formalização do conceito de oráculo
(pois não há mais uma única resposta ``certa'',
mas sim várias)
e impossibilita a análise do efeito da composição de funções,
como a feita na seção~\ref{sec:function_composition}.
