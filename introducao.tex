\chapter{INTRODUÇÃO}

O mais famoso problema da área de complexidade computacional
é o $P$ versus $NP$.
Os problemas da classe $P$
são os que podem ser resolvidos em tempo polinomial
por uma Máquina de Turing determinística,
enquanto que a classe $NP$
contempla os problemas que podem ser resolvidos em tempo polinomial
por Máquinas não-determinísticas.
Os problemas considerados são problemas de decisão;
isto é,
determinar se uma palavra está ou não
numa determinada linguagem.
$P$ e $NP$ são conjuntos de linguagens
que correspondem a problemas de decisão.
Sabemos que $P \subseteq NP$;
o problema $P$ versus $NP$
pergunta se esta inclusão é estrita ou não \cite{Sipser2006}.
A maior parte dos pesquisadores acredita que $P \neq NP$,
embora ainda não tenhamos demonstrado nada.

Uma das ``frentes de ataque''
consiste em tentar ``enfraquecer''
a máquina de Turing.
A ideia é trabalhar com um modelo de computação muito mais restritivo,
provar um equivalente a $P \neq NP$
para este modelo de computação,
e transpôr esta demonstração para o modelo de Turing.
Circuitos booleanos
são exemplos de modelos de computação mais simples.
Como eles não têm ``partes móveis''
ou múltiplos estados,
intuitivamente uma demonstração de que $P \neq NP$
deve ser mais simples de se obter
neste dispositivo computacional. \cite{Hastad1987}.

As duas principais medidas
de complexidade de circuitos
são a quantidade de nodos
(quantidade de portas lógicas)
e a profundidade do circuito
(maior caminho entre uma entrada e uma saída).
Existem várias classes de complexidades de circuitos
que exploram estas duas métricas;
neste trabalho,
pretende-se relacionar estas classes de complexidade
com as classes de complexidade computacional.

\section{Objetivos}

Estabelecer relações entre classes de complexidade computacional
e complexidade de circuitos.

\subsection{Objetivos Especificos}

\begin{enumerate}
    \item Estudar as classes de complexidade computacional
        e complexidade de circuitos.
    \item Demonstrar a relação entre complexidade de circuitos
        e o problema $P$ versus $NP$.
    \item Correlacionar as principais classes
        de complexidade computacional
        às correspondentes classes de complexidade de circuitos
    \item Estabelecer limites inferiores no tamanho
        e profundidade dos circuitos
        que computam certas funções recursivas.
\end{enumerate}
