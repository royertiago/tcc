\subsection{Funções Recursivas}

Usualmente,
``máquina de Turing'' é definida como uma tupla
na qual um de seus elementos,
$\Sigma$,
denota o alfabeto de entrada.
Por simplicidade,
assumiremos que $\Sigma = \{0, 1\}$,
em todo o TCC.
Observe que não há perda de generalidade nesta definição,
pois outros alfabetos podem ser codificados de forma não"=ambígua
usando apenas este alfabeto binário.

\begin{notation}
    Seja $M$ uma máquina de Turing determinística
    e $x \in \{0, 1\}^*$ uma palavra.
    Suponha que $M$ pare ao computar $x$.
    Se o conteúdo deixado na fita por $M$,
    após remover os caracteres brancos à direita e à esquerda,
    contiver apenas os símbolos $0$ e $1$
    (isto é, não houverem brancos ou outros símbolos de fita),
    este conteúdo forma uma palavra de $\{0, 1\}^*$.
    Denotaremos esta palavra por $M(x)$.
    Caso haja outros símbolos na fita
    ou $M$ não parar ao computar $x$,
    deixaremos $M(x)$ indefinido.
\end{notation}

Observe que a notação $M(x)$ coincide com a notação de chamada de função.
É exatamente esta a analogia que pretendemos fazer;

\begin{definition}
    \label{def:recursive_function}
    Seja $M$ uma máquina de Turing.
    A \emph{função computada por $M$}
    é a função parcial $f: \{0, 1\}^* \to \{0, 1\}^*$ tal que
    \begin{equation*}
        f(x) \simeq M(x)
    \end{equation*}
    para todo $x$.

    Uma função $f$ é \emph{recursiva parcial}
    se for computada por alguma máquina $M$.
    Se $f$ estiver definida para todo $x \in \{0, 1\}^*$,
    dizemos que $f$ é \emph{recursiva total}.
    \footnote{
        Não confundir com o conceito de \emph{recursão} em linguagens de programação.
    }
\end{definition}
(Esta definição é uma variação da definição de \citeonline[p.~151]{HopcroftUllman1979};
em particular, utilizaremos funções que mapeiam palavras para palavras
em vez de funções que mapeiam números naturais para números naturais.)

\subsubsection{Codificação de objetos}
\label{sec:binary_encoding}

Conforme observado por \citeonline[p.~2]{AroraBarak2009},
não perdemos generalidade restringir funções a palavras de $\{0, 1\}^*$,
pois podemos codificar objetos matemáticos complexos usando o sistema binário
--- de fato,
é exatamente isso o que fazemos nos computadores modernos.
Podemos, por exemplo,
codificar inteiros usando sua representação binária,
e grafos podem ser representados por sua matriz de adjacência
ou pelos pares de nodos adjacentes.
\citeonline[p.~10]{GareyJohnson1979} argumentam que
não é necessário se ater aos detalhes da representação dos objetos,
pois quaisquer duas codificações ``razoáveis'' destes objetos
irão diferir polinomialmente de tamanho.

Neste trabalho,
dois objetos aparecerão mais frequentemente:
máquinas de Turing e tuplas.
Uma máquina de Turing pode ser representada em binário
listando todos elementos da função de transição
\cite[p.~182]{HopcroftUllman1979};
os detalhes estão no apêndice~\ref{app:turing_machines}.
Denotaremos por $\langle M \rangle$ a representação em binário da máquina $M$.

Já uma tupla $(x_1, x_2, \dots, x_n)$ pode ser codificada
combinando as representações em binário de $x_1, \dots, x_n$.
A ideia é trocar, nas representações em binário,
$0$ por $00$ e $1$ por $01$;
então, a cadeia $1$ faz o papel de separador.
Denotaremos por $\langle x_1, x_2, \dots, x_n \rangle$
a representação binária da tupla $(x_1, x_2, \dots, x_n)$.
Observe que esta codificação é não"=ambígua;
podemos recuperar a representação em binário de cada um dos $x_i$
a partir de $\langle x_1, \dots, x_n \rangle$.

\begin{example}
    Podemos representar o número $2$ por $10$ e o número $5$ por $101$,
    em binário.
    A transformação mencionada acima troca $10$ por $0100$
    e $101$ por $010001$;
    portanto, a tripla ordenado $(2, 5, 2)$ é representada por
    \begin{equation*}
        \langle 2, 5, 2 \rangle = 0100\ 1\ 010001\ 1\ 0100.
    \end{equation*}
\end{example}

\simbolo{$\langle x \rangle$}{Codificação em binário do objeto matemático $x$}
(Em geral, $\langle x \rangle$ denotará a representação em binário
do objeto matemático $x$.
\citeonline[p.~2]{AroraBarak2009} usa
$\sideset{_\llcorner}{_\lrcorner}{\mathop{\makebox[0.5ex]{$x$}}}$
para o mesmo fim;
eu escolhi a notação $\langle x \rangle$
por conformidade com a notação de \citeonline[p.~182]{HopcroftUllman1979}.)

% Explicando o comando acima:
% \sideset permite adicionar símbolos matemáticos "orbitando" o símbolo subsequente
% nos quatro cantos possíveis. (O primeiro argumento diz o que vai à esquerda
% e o segundo argumento diz o que vai à direita.)
% \sideset exige que o terceiro argumento seja um operador matemático,
% que eu obtenho usando \mathop.
% Mas se for usado "\mathop x" diretamente, as duas marcas sob x
% ficarão muito espaçadas; então, a largura do x é alterada artificialmente
% para 0.5ex usando o \makebox.

É importante notar que codificaremos números naturais
sempre usando sua expansão binária.
\citeonline[p.~151]{HopcroftUllman1979},
por exemplo,
utilizam uma representação unária;
isto é, o número $n$ é representado por $0^n$,
Entretanto,
esta codificação faz com que o tamanho da palavra resultante
seja proporcional a $n$.
Já a expansão binária de $n$ possui tamanho $\lfloor \log_2 n \rfloor + 1$;
ou seja, o tamanho da expansão binária é proporcional ao logaritmo do número.

Em outras palavras,
a representação em unário é exponencialmente maior que a representação em binário.
Isto será importante ao considerarmos classes de complexidade;
portanto,
seguiremos a convenção de \citeonline[p.~2]{AroraBarak2009}
de usar sempre a expansão binária para representar números naturais.

\subsubsection{Funções de várias variáveis}

Observe que o termo ``função recursiva''
foi definido apenas para funções de uma variável.
Entretanto,
em alguns pontos do texto,
precisaremos nos referir a funções de mais de uma variável.

É possível estender as máquinas de Turing
para trabalhar diretamente com com funções de várias variáveis.
Por exemplo, \citeonline[p.~151]{HopcroftUllman1979}
fazem a extensão no contexto de funções de inteiros.
Dada uma máquina $M$,
para calcular o valor de $M(i)$,
executamos $M$ na entrada $0^i$.
Os vários argumentos são separados por caracteres $1$;
isto é, para calcular $M(i_1, i_2, \dots, i_k)$,
executamos $M$ na entrada $0^{i_1} 1 0^{i_2} 1 \dots 1 0^{i_k}$.

Para não precisar alterar a definição de máquinas de Turing,
recorreremos à codificação $\langle \cdot \rangle$,
definida na seção~\ref{sec:binary_encoding}.

\begin{definition}
    \label{def:multi_valued_recursive_function}
    Uma função parcial $f: (\{0, 1\}^*)^k \to \{0, 1\}^*$ é \emph{recursiva parcial}
    se existir alguma função recursiva parcial $g: \{0, 1\}^* \to \{0, 1\}^*$
    tal que
    \begin{equation*}
        g(\langle i_1, i_2, \dots, i_k \rangle) \simeq f(i_1, i_2, \dots, i_k).
    \end{equation*}
    Funções recursivas totais são definidas analogamente.
\end{definition}

Observe que esta definição não é autorreferencial
pois a recursividade de $g$ é estabelecida na definição~\ref{def:recursive_function}.

\subsubsection{Funções de inteiros}

Embora na maior parte deste TCC usaremos funções da forma
\begin{equation*}
    f : \{0, 1\}^* \to \{0, 1\}^*,
\end{equation*}
em alguns pontos do texto precisaremos usar números naturais para contar coisas
--- por exemplo,
quantos passos uma máquina de Turing fez ao computar uma função.
Além disso,
queremos usar máquinas de Turing para manipular estas funções;
portanto,
é importante definir o que significa para uma função dessas ser ``recursiva''.

\begin{definition}
    Uma função parcial $f: \mathbb N \to \mathbb N$ é \emph{recursiva}
    se existir uma função recursiva parcial $g: \{0, 1\}^* \to \{0, 1\}^*$
    tal que, para todo $x$,
    \begin{equation*}
        g(\langle n \rangle) \simeq \langle f(n) \rangle.
    \end{equation*}
    O caso em que o domínio ou o contradomínio é $\{0, 1\}^*$,
    ou $f$ for uma função de várias variáveis,
    é definido analogamente.
\end{definition}

Em outras palavras,
$f$ é recursiva quando existir uma função $g$
que calcula $f$ usando a representação em binário dos números.

\begin{example}
    A função de soma,
    \begin{equation*}
        +: \mathbb N \times \mathbb N \to \mathbb N,
    \end{equation*}
    é recursiva.
    Podemos construir uma máquina de Turing $M$ que,
    ao receber na entrada o valor $\langle m, n \rangle$
    --- isto é, a representação em binário de $m$ e $n$ ---
    soma estes dois números
    (usando adição com \emph{carry}, o ``vai um'')
    e escreve a soma na fita de saída.
    Esta máquina satisfaz
    \begin{equation*}
        M(\langle m, n \rangle) = \langle m+n \rangle,
    \end{equation*}
    portanto, a operação de soma é recursiva.
\end{example}

Todas as operações matemáticas que são realizáveis num computador
correspondem a funções recursivas;
portanto,
nos limitarmos a funções de inteiros recursivas não é uma restrição significativa.
