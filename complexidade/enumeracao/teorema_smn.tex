\subsection{Teorema $S_{mn}$}
\label{sec:s_m_n_theorem}

\begin{theorem}
    Existe uma função recursiva total $\sigma$ de duas variáveis tal que,
    para todo $x, y, z \in \{0, 1\}^*$,
    \begin{equation*}
        f_{\sigma(x, y)}(z) \simeq f_x(\langle y, z \rangle).
    \end{equation*}
\end{theorem}

Isto é,
dado um índice $x$ para a função de duas variáveis $f_x$
e uma palavra $y$ qualquer,
a função $\sigma$ devolve uma máquina de Turing $M$
(quer dizer, $\sigma(x, y) = \langle M \rangle$)
que, na entrada $z$,
calcula o valor $f_x(\langle y, z \rangle)$.

Em outras palavras,
a função $\sigma$ deixa o primeiro argumento da função $f_x$
fixo em $y$.
Esta versão do teorema
(mencionada por \citeonline[p.~324]{Blum1967})
lida apenas com o caso de fixar uma entrada, $y$,
e deixar uma entrada livre, $z$.
A ``versão completa'' do teorema
está presente no livro de \citeonline[p.~23]{Rogers1987};
ele diz como transformar uma função de $m + n$ variáveis
numa função de $n$ variáveis
fixando as $m$ primeiras para valores pré"=estabelecidos.

Entretanto, para nossos propósitos,
esta versão será suficiente.

\begin{proof}
    Dado $x$ e $y$ na entrada,
    $\sigma$ terá de retornar uma representação de uma máquina de Turing.
    A função $\sigma$ é total,
    o que significa que $\sigma(x, y)$ sempre existirá;
    entretanto,
    pode ser que a máquina $\sigma(x, y)$ não pare em todas as entradas.

    Se $M = \sigma(x, y)$,
    a ideia é fazer com que $M$ substitua a entrada $z$ por $\langle y, z \rangle$,
    e execute a máquina representada por $x$ no resultado.
    Para isso,
    basta embutir em $M$ o valor $y$ e as transições de $x$.

    Esta construção pode ser executada para todo $x$ e $y$,
    tornando $\sigma$ uma função recursiva total.
    A máquina $M$ representada por $\sigma(x, y)$ executará,
    na entrada $z$, a máquina $x$ em $\langle y, z \rangle$;
    portanto,
    $M(z)$ estará definido se, e só se, $f_x(\langle y, z \rangle)$ está definido;
    e, neste caso,
    \begin{equation*}
        M(z) = f_x( \langle y, z \rangle ).
    \end{equation*}
    Mas como $\langle M \rangle = \sigma(x, y)$,
    a função calculada por $M$ é, exatamente, $f_{\sigma(x, y)}$,
    o que prova o teorema.
\end{proof}
