\subsection{Relações entre medidas de Complexidade}

Por serem medidas de complexidade específicas,
podemos impor relações mais fortes entre elas
do que as que são fornecidas pelo teorema \ref{relacao_medidas}.

\begin{proposition}
    \begin{align}
        \DTIME(f) &\subseteq \NTIME(f) \label{dtime_in_ntime} \\
        \DSPACE(f) &\subseteq \NSPACE(f) \label{dspace_in_nspace} \\
        \DTIME(f) &\subseteq \DSPACE(f) \label{dtime_in_dspace} \\
        \NTIME(f) &\subseteq \DSPACE(f) \label{ntime_in_dspace}
    \end{align}
\end{proposition}
\begin{proof}
    \ref{dtime_in_ntime} e \ref{dspace_in_nspace} são consequências diretas
    do fato de que toda máquina determinística é,
    em particular, não determinística.

    Para \ref{dtime_in_dspace},
    note que, em $f(n)$ movimentos,
    a máquina pode ler, no máximo,
    $f(n)$ diferentes células
    --- afinal, no máximo uma célula nova pode ser lida a cada movimento.

    Para \ref{ntime_in_dspace},
    usaremos uma máquina com múltiplas fitas.

    Se $M$ é uma máquina que reconhece $L \in \NTIME(f)$,
    existe um limite na quantidade de possíveis transições
    que $M$ pode fazer em cada estado;
    digamos, $t$ transições diferentes.
    Cada cadeia sobre $\{0, \dots, t-1\}$ representa uma possível
    sequência de transições,
    que pode levar à parada ou não.

    Em uma das fitas,
    a nova máquina $M'$
    irá enumerar todas as palavras de $\{0, \dots, t-1\}^*$.
    Para cada palavra enumerada,
    $M'$ simulará $M$ na entrada,
    escolhendo as transições de acordo com a palavra
    que foi enumerada na outra fita.
    No evento de alguma transição ser para um estado final,
    consideraremos tal palavra ``fechada''.

    Se a transição for para um estado de aceitação,
    aceitamos a entrada;
    e, se todas as palavras de um mesmo tamanho $k$
    forem fechadas sem aceitação,
    nenhuma palavra
    codifica uma sequência de transições
    que leva a um estado de aceitação
    --- todas as menores que $k$ já foram analisadas
    e todas as maiores que $k$
    possuem uma palavra de tamanho $k$ que já foi fechada,
    fazendo com que a palavra inteira fique fechada.
    Nesta situação, podemos rejeitar a entrada.

    Como $L \in \NTIME(f)$,
    qualquer sequência de transições leva a algum estado final
    em, no máximo, $f(n)$ transições.
    Portanto,
    sabemos que iremos fechar todas as palavras
    de tamanho $f(n)$,

    Como as fitas de trabalho de $M$
    não ocupam mais do que $f(n)$ células,
    e a fita de enumeração de $M'$
    nunca precisará enumerar uma palavra mais longa que $f(n)$,
    a complexidade de espaço de $M'$ é $f(n)$.
    Concluímos que $L \in \DSPACE(f)$.
\end{proof}

\begin{theorem}
    Suponha que $f(n) \geq \log n$ para todo $n$.
    Se $L \in \DSPACE(f)$,
    então existe uma constante $c$
    tal que $L \in \DTIME(c^f)$.
\end{theorem}

\begin{proof}
    Seja $M$ uma máquina que aceita $L$ em espaço $f$.
    Conforme observado na equação \ref{num_possiveis_configuracoes},
    existem constantes $a$ e $c$ tais que,
    caso $M$ ocupe exatamente $k$ células na fita,
    existirão, no máximo,
    \begin{equation*}
        ac^k
    \end{equation*}
    diferentes configurações na fita.
    Para $k = f(n)$, a equação lê
    \begin{equation*}
        ac^{f(n)}.
    \end{equation*}
    Como $M$ é um decisor,
    $M$ encerra sua computação nesta quantidade de passos.

    $c^{f(n)} \geq n$, pois $f(n) \geq \log n$.
    Portanto, podemos usar o aceleramento linear
    para nos livrar daquela constante $a$,
    provando, assim, o teorema.
\end{proof}

\begin{theorem}
    Suponha que $f(n) \geq \log n$ para todo $n$.
    Se $L \in \NTIME(f)$,
    então existe uma constante $c$
    tal que $L \in \DTIME(c^f)$.
\end{theorem}

\begin{proof}
    Pela equação \ref{ntime_in_dspace},
    $L \in \DSPACE(f)$.
    Combinando com o teorema anterior,
    temos $L \in \DTIME(c^f)$ para algum $c$.
\end{proof}
