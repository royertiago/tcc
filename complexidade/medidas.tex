\section{Medidas de Complexidade Computacional}
\label{medidas_padrao}

\begin{definition}
    \begin{align*}
        \DTIME(f) &= \mathcal C_\PhiDT(f) \\
        \DSPACE(f) &= \mathcal C_\PhiDS(f) \\
        \NTIME(f) &= \mathcal C_\PhiNT(f) \\
        \NSPACE(f) &= \mathcal C_\PhiNS(f)
    \end{align*}
    Costumamos usar $T(n)$ para funções de classes de tempo
    e $S(n)$ para funções de classes de espaço.
\end{definition}

Estas são as principais medias de complexidade
para máquinas de Turing.

Por serem medidas de complexidade específicas,
podemos impôr relações mais fortes entre elas
do que as que são fornecidas pelo teorema \ref{relacao_medidas}.

\begin{proposition}
    \begin{align}
        \DTIME(f) &\subseteq \NTIME(f) \label{dtime_in_ntime} \\
        \DSPACE(f) &\subseteq \NSPACE(f) \label{dspace_in_nspace} \\
        \DTIME(f) &\subseteq \DSPACE(f) \label{dtime_in_dspace} \\
        \NTIME(f) &\subseteq \DSPACE(f) \label{ntime_in_dspace}
    \end{align}
\end{proposition}
\begin{proof}
    \ref{dtime_in_ntime} e \ref{dspace_in_nspace} são consequências diretas
    do fato de que toda máquina determinística é,
    em particular, não determinística.

    Para \ref{dtime_in_dspace},
    note que, em $f(n)$ movimentos,
    a máquina pode ler, no máximo,
    $f(n)$ diferentes células
    --- afinal, no máximo uma célula nova pode ser lida a cada movimento.

    Para \ref{ntime_in_dspace},
    usaremos uma máquina com múltiplas fitas.

    Se $M$ é uma máquina que reconhece $L \in \NTIME(f)$,
    existe um limite na quantidade de possíveis transições
    que $M$ pode fazer em cada estado;
    digamos, $t$ transições diferentes.
    Cada cadeia sobre $\{0, \dots, t-1\}$ representa uma possível
    sequência de transições,
    que pode levar à parada ou não.

    Em uma das fitas,
    a nova máquina $M'$
    irá enumerar todas as palavras de $\{0, \dots, t-1\}^*$.
    Para cada palavra enumerada,
    $M'$ simulará $M$ na entrada,
    escolhendo as transições de acordo com a palavra
    que foi enumerada na outra fita.
    No evento de alguma transição ser para um estado final,
    consideraremos tal palavra ``fechada''.

    Se a transição for para um estado de aceitação,
    aceitamos a entrada;
    e, se todas as palavras de um mesmo tamanho $k$
    forem fechadas sem aceitação,
    nenhuma palavra
    codifica uma sequência de transições
    que leva a um estado de aceitação
    --- todas as menores que $k$ já foram analisadas
    e todas as maiores que $k$
    possuem uma palavra de tamanho $k$ que já foi fechada,
    fazendo com que a palavra inteira fique fechada.
    Nesta situação, podemos rejeitar a entrada.

    Como $L \in \NTIME(f)$,
    qualquer sequência de transições leva a algum estado final
    em, no máximo, $f(n)$ transições.
    Portanto,
    sabemos que iremos fechar todas as palavras
    de tamanho $f(n)$,

    Como as fitas de trabalho de $M$
    não ocupam mais do que $f(n)$ células,
    e a fita de enumeração de $M'$
    nunca precisará enumerar uma palavra mais longa que $f(n)$,
    a complexidade de espaço de $M'$ é $f(n)$.
    Concluímos que $L \in \DSPACE(f)$.
\end{proof}
