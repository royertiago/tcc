\subsection{Principais Classes de Complexidade Computacional}
\label{classes_complexidade}

\begin{definition}
    \begin{align*}
        \P &= \bigcup_{n \geq 1} \DTIME(n^k) \\
        \NP &= \bigcup_{n \geq 1} \NTIME(n^k) \\
        \PSPACE &= \bigcup_{n \geq 1} \DSPACE(n^k) \\
        \NPSPACE &= \bigcup_{n \geq 1} \NSPACE(n^k) \\
    \end{align*}
\end{definition}

De acordo com o teorema da união,
estes conjuntos são classes de complexidade.

As classes $\P$ e $\NP$ são as protagonistas
do problema mais importante da Ciência da Computação teórica.
De um lado, temos a classe $\P$,
as linguagens que podem ser resolvidas em tempo polinomial.
De outro, tempos a classe $\NP$,
as linguagens que podem ser verificadas em tempo polinomial.
Sabemos que $\P \subseteq \NP$;
a inclusão contrária é um problema em aberto
--- apesar dos esforços de cientistas da computação mundo afora,
ainda não sabemos se estas duas classes são iguais ou não.

Pela equação \ref{ntime_in_dspace},
sabemos que a seguinte cadeia de inclusões é verdadeira:
\begin{equation*}
    \P \subseteq \NP \subseteq \PSPACE \subseteq \NPSPACE.
\end{equation*}
O teorema de Savitch nos diz que as duas últimas classes são iguais.
Mas nós sequer sabemos se $\P$ é igual a $\PSPACE$ ou não.
