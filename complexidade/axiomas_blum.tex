\section{Teoria de Complexidade Axiomática: Axiomas de Blum}
\label{axiomas_blum}

(Utilizaremos a definição provida por
\citeonline[p. 156]{Papadimitriou1994}.)

Dada uma máquina de Turing $M$
e uma palavra $x$,
denotaremos por $M(x)$ a ``saída''
de $M$ quando lhe é dado $x$ na entrada.

\begin{definition}
    Uma \emph{medida de complexidade}
    é uma função $\Phi$ que satisfaz aos seguintes axiomas:
    \footnotemark
    \begin{enumerate} [label=\textbf{Axioma \arabic*}, ref=\arabic*, align=left]
        \item
            \label{blum_def}
            $\Phi(M, x)$ está definido
            se, e somente se,
            $M(x)$ está definido.
        \item
            \label{blum_rec}
            Dados $M$, $x$ e $k$,
            é decidível se $\Phi(M, x) = k$.
    \end{enumerate}

    \footnotetext{
        A definição usual dos axiomas de Blum
        (encontrada, por exemplo,
        no texto de \citeonline[p. 313]{HopcroftUllman1979}
        e no próprio artigo original de \citeonline[p. 3]{Blum1967})
        aparece no contexto de computadores de funções de inteiros.
        Seja $M_1, M_2, \dots$ uma enumeração de máquinas de Turing.
        Consideraremos que a máquina $M_i$
        computa a função recursiva parcial $\phi_i$.
        Uma medida de complexidade é uma lista de funções
        $\{\hat \Phi_1, \hat \Phi_2, \dots\}$
        que satisfaz os seguintes axiomas:

        \begin{enumerate} [label=Axioma \arabic*', ref=\arabic*', align=left]
            \item
                \label{blum_def_orig}
                $\hat \Phi_i(n)$ está definido
                se, e somente se,
                $\phi_i(n)$ está definido.
            \item
                \label{blum_rec_orig}
                A função $R(i, n, m)$,
                definida como $1$ se $\hat \Phi_i(n) = m$,
                e $0$ em caso contrário,
                é recursiva.
        \end{enumerate}

        As duas definições são análogas.
        O valor $\hat \Phi_i(n)$ corresponde a $\Phi(M_i, 0^n)$.
        O axioma \ref{blum_def} corresponde ao axioma \ref{blum_def_orig},
        enquanto que a decidibilidade exigida pelo axioma \ref{blum_rec}
        é expressada pela função $R$ no axioma \ref{blum_rec_orig}.
    }
\end{definition}

O axioma \ref{blum_rec}
nos dá um semialgoritmo para calcular $\Phi(M, x)$.
Entretanto, pelo axioma \ref{blum_def},
não podemos ir muito além disso,
pois $M(x)$ não está definido para todo $M$ e $x$.
De fato, sequer podemos decidir se $\Phi(M, x)$ existe.

\begin{example}
    \simbolo{$\PhiDT$}{Complexidade de Tempo}
    A \emph{complexidade de tempo},
    que denotaremos por $\PhiDT$,
    é a função que diz quantos movimentos
    uma máquina de Turing faz até retornar uma resposta.
    Isto é,
    \begin{equation*}
        \PhiDT(M, x) = \begin{cases}
            k, & \text{
                \parbox{0.6\textwidth}{
                    se $M$ executa exatamente $k$ passos em $x$ antes de parar.
                }
            } \\
            \text{indefinido}, & \text{
                \parbox{0.6\textwidth}{
                    caso $M$ nunca pare de computar $x$.
                }
            }
        \end{cases}
    \end{equation*}
    Para determinar se $\PhiDT(M, x) = k$,
    execute a máquina $M$ por $k$ passos
    e veja se é a primeira vez que
    $M$ atinge um estado aceitador.
    E, como $\PhiDT(M, x)$ só está definido se $M$ para ao computar $x$,
    $\PhiDT$ satisfaz aos dois axiomas de Blum.
\end{example}

\begin{example}
    \label{complexidade_espaco}
    \simbolo{$\PhiDS$}{Complexidade de Espaço}
    Para a \emph{complexidade de espaço},
    que denotaremos por $\PhiDS$,
    iremos assumir que $M$ possui uma fita somente"=leitura
    específica para a entrada.
    \begin{equation*}
        \PhiDS(M, x) = \begin{cases}
            k, & \text{
                \parbox{0.6\textwidth}{%
                    se $M$ lê, de alguma de suas fitas,
                    exatamente $k$ células
                    antes de parar.
                }
            } \\
            \text{indefinido}, & \text{
                \parbox{0.6\textwidth}{
                    se $M$ nunca parar ao computar $x$.
                }
            }
        \end{cases}
    \end{equation*}
    É fácil ver que o axioma \ref{blum_def} é satisfeito.
    Para o axioma \ref{blum_rec},
    o algoritmo é um pouco mais complicado.

    Comece executando $M$ em $x$.
    Caso $M$ extrapole $k$ células lidas
    em alguma de suas fitas,
    podemos retornar rejeitar.
    Caso contrário,
    existirá um número finito de configurações da máquina.
    Existem $|\Gamma|^k$ possíveis fitas com $k$ termos;
    $k+1$ possíveis posições da cabeça de leitura;
    $|Q|$ possíveis estados da máquina;
    e $l$ diferentes fitas.
    No total, existem, no máximo,
    \begin{equation*}
        (k+1) l |Q||\Gamma|^k
    \end{equation*}
    possíveis configurações.
    Portanto, se a máquina executar
    mais movimentos do que este número,
    significa que ela entrou em loop.
    Podemos retornar rejeitar.

    E, por último,
    caso $M$ pare,
    precisamos nos assegurar que,
    de fato,
    em alguma das fitas $k$ células foram lidas.
\end{example}

\begin{example}
    \simbolo{$\PhiNT$}{Complexidade de Tempo não"=determinística}
    \simbolo{$\PhiNS$}{Complexidade de Espaço não"=determinística}
    Podemos adaptar $\PhiDT$ e $\PhiDS$
    para máquinas de Turing não"=determinísticas.

    Para a complexidade de tempo não"=determinística,
    que denotaremos por $\PhiNT$,
    definiremos $\PhiNT(M, x)$
    como sendo a maior quantidade de movimentos
    tomadas por $M$ ao computar $x$
    dentre todas as escolhas de transições possíveis.

    Analogamente,
    para a complexidade de espaço não"=determinística,
    que denotaremos por $\PhiNS$,
    definiremos $\PhiNS(M, x)$
    como sendo a maior quantidade de células lidas
    em qualquer dos ramos da computação de $M$ em $x$.
    Aqui, precisamos tomar o mesmo cuidado que tomamos
    com $\PhiDS$ para demonstrar o axioma \ref{blum_rec}.
\end{example}

\begin{example}
    Escolher $\Phi(M, x) = 0$ para todo $M$ e $x$
    satisfaz ao axioma \ref{blum_rec},
    mas não ao axioma \ref{blum_def},
    pois $\Phi(M, x)$ está definida mesmo quando $M(x)$ não está.
    Já definir $\Phi(M, x) = |M(x)|$
    satisfaz ao axioma \ref{blum_def},
    mas não ao axioma \ref{blum_rec},
    pois poderíamos resolver o problema da parada:
    dada uma máquina $M$, podemos modificá"=la
    para apagar sua fita logo antes de parar.
    Então, para esta $M'$,
    $\Phi(M', x) = 0$ se, e somente se,
    a $M$ original para ao computar $x$.
    Estes dois exemplos mostram que os axiomas são independentes
    \cite[p. 3]{Blum1967}.
\end{example}

Podemos ver que as medidas $\PhiDT$ e $\PhiDS$ estão relacionadas.
Para ler uma posição da fita,
é necessário gastar ao menos uma unidade de tempo.
Ou seja,
\begin{equation*}
    \PhiDS(M, x) \leq \PhiDT(M, x).
\end{equation*}
E, de acordo com o raciocínio do exemplo \ref{complexidade_espaco},
para todo $M$ existe algum $c$ que
\begin{equation*}
    \PhiDT(M, x) \leq c^{\PhiDS(M, x)}.
\end{equation*}
De fato, podemos relacionar quaisquer duas medidas de complexidade.

\begin{theorem}
    \label{relacao_medidas}
    Dadas duas medidas de complexidade $\Phi$ e $\hat \Phi$,
    existe uma função recursiva $r$ tal que
    \begin{equation*}
        \Phi(M, x) \leq r( x, \hat \Phi(M, x))
    \end{equation*}
    para todo $M$ e quase todo $x$.
    \footnote{
        ``verdadeiro para quase todo $n$''
        significa que o predicado em questão
        é falso para apenas uma quantidade finida de números $n$.
    }
\end{theorem}

\begin{proof}
    Defina
    \begin{equation*}
        r( x, k ) = \max \{ \Phi(M, x) \mid
            \text{A desrição de $M$ é mais curta que $|x|$}
            \text{ e }
            \hat \Phi(M, x) = k
        \}
    \end{equation*}
    Fixado $x$, existe um número finito de máquinas de Turing
    cuja descrição é menor que $|x|$.
    O conjunto na definição acima é um subconjunto desta lista.
    O predicado $\hat \Phi(M, x) = k$ é recursivo.
    Quando este predicado é verdadeiro,
    $M(x)$ está definido, pelo axioma \ref{blum_def},
    portanto $\Phi(M, x)$ também está definido.
    Concluímos que $r$ é recursiva.

    Agora, para todos os $x$ que são mais longos que a descrição de $M$,
    $\Phi(M, x)$ será um dos elementos do conjunto acima
    para $r( x, \hat \Phi(M, x)))$,
    portanto é menor ou igual a este número.
\end{proof}

\citeonline[p. 4]{Blum1967} demonstra uma versão ligeiramente mais forte
deste teorema.
Ele prova que $r$ pode ser tal que,
simultaneamente,
\begin{equation*}
    \Phi(M, x) \leq r( x, \hat \Phi(M, x))
\end{equation*}
e
\begin{equation*}
    \hat \Phi(M, x) \leq r( x, \Phi(M, x))
\end{equation*}
Podemos construir uma função dessas
pegando o máximo de duas funções obtidas
usando o teorema \ref{relacao_medidas}.

Observe que o teorema,
assim como provamos,
não pode ser fortalecido
para que $r$ seja uma função de apenas uma variável.
Considere $A$ uma máquina de Turing
que opere como um autômato finito.
$\PhiDT(A, x) = |x|$ para toda palavra $x$,
enquanto que $\PhiDS(A, x) = 1$ para toda palavra $x$.
\footnote{
    A complexidade de espaço não é $0$
    pois $A$ é obrigada a ler
    ao menos a célula inical da sua fita de trabalho,
    embora a máquina não use aquela célula.
}
Se $r$ pudesse depender apenas da segunda variável,
isto é, $r(x, m) = r'(m)$ para alguma função $r'$,
teríamos
\begin{align*}
    |x| &= \PhiDT(A, x) \\
        &\leq r(x, \PhiDS(A, x)) \\
        &= r(x, 1) \\
        &= r'(1)
\end{align*}
que é falso para todo $x$ suficientemente comprido.

Caso $r$ pudesse depender apenas da primeira variável,
isto é, $r(x, m) = r''(x)$ para alguma função $r''$,
teríamos, para todas as máquinas de Turing,
\begin{equation*}
    \PhiDT(M, x) \leq r''(x).
\end{equation*}
Mas, como $r''$ é recursiva
(pois $r$ o é),
podemos construir uma máquina que calcula $r''(x)$,
disperdiça $r''(x)$ movimentos,
e aceita a entrada.
Para esta $M'$,
\begin{equation*}
    \PhiDT(M', x) > r''(x),
\end{equation*}
contradizendo a equação anterior.

No parágrafo anterior,
construímos uma máquina de Turing
que deliberadamente desperdiça tempo
ao computar determinada função.
\citeonline[p. 4]{Blum1967} demonstrou que
é sempre possível desperdiçar recursos computacionais,
quaisquer que sejam estes recursos.
Precisamos de um lema,
vindo direto da teoria das funções recursivas.

\begin{lemma}[Teorema da Recursão]
    Denote por $M_m$ a máquina de Turing
    representada pela palavra $x$.

    Para qualquer função recursiva total $\sigma$,
    existe um valor $m$ tal que
    \begin{equation*}
        M_m(x) = M_{\sigma(m)}(x)
    \end{equation*}
    para todo $x$.
    (Tal valor é chamado de \emph{ponto fixo} para $\sigma$.)
\end{lemma}

\begin{proof}
    Construa a máquina $N$ para,
    ao receber $x$ na entrada,
    devolver uma máquina de Turing $N(x)$
    que irá calcular $M_x(x)$
    (simulando uma máquina de Turing universal,
    se necessário)
    e roda o resultado na entrada.
    Como resultado,
    sempre que $M_x(x)$ estiver definido,
    \begin{equation*}
        M_{N(x)}(y) = M_{M_x(x)}(y)
        \label{diagonal_N}
    \end{equation*}
    para toda cadeia $y$.

    É importante notar que $N$ sempre pára;
    $M_x(x)$ pode não parar ao computar $x$,
    mas isso significa, apenas,
    que $N(x)$ nunca parará ao computar qualquer coisa
    --- o valor $N(x)$ existirá.

    Escolha $k$ como sendo uma palavra
    que represente uma máquina
    que compute $\sigma(N(x))$ ao receber $x$ na entrada.
    Afirmamos que $m = N(k)$
    satisfaz às exigências do teorema.

    De fato,
    \begin{align*}
        M_m(x) &= M_{N(k)}(x) \\
               &= M_{M_k(k)}(x) && \text{Pela observação \ref{diagonal_N}}\\
               &= M_{\sigma(N(k))}(x) && \text{Pois $k$ computa $\sigma(N(x))$}\\
               &= M_{\sigma(m)}(x) && \text{Pela definição de $m$.}
        \qedhere
    \end{align*}
\end{proof}

\begin{proposition}
    Sejam $f$ e $g$ duas funções recursivas totais,
    e $\Phi$ uma medida de complexidade.
    Então existe uma máquina de Turing $M$ que computa $f$
    tal que
    \begin{equation*}
        \Phi(M, x) > g(|x|)
    \end{equation*}
    para todo $x$.
\end{proposition}

\begin{proof}
    Defina a função $h$, de duas variáveis, por
    \begin{equation*}
        h(m, x) = \begin{cases}
            M_m(x) a & \text{se $\Phi(M_m, x) \leq g(|x|)$} \\
            f(x) & \text{caso contrário.}
        \end{cases}
    \end{equation*}
    ($M_m(x)a$ é o valor de $M_m(x)$ concatenado com a letra $a$.)

    Observe que $h$ é uma função computável,
    pois $\Phi$, $g$ e $f$ o são,
    e, caso $\Phi(M_m, x)$ esteja definido,
    $M_m(x)$ também estará.

    Construa a função $\sigma$ que,
    ao receber $m$ na entrada,
    altere"=a para que,
    quando ela receber $x$ na entrada,
    compute $h(m, x)$.
    Ou seja,
    \begin{equation*}
        M_{\sigma(m)}(x) = h(m, x).
    \end{equation*}
    $\sigma$ pode, por exemplo,
    codificar $m$ nos estados de $\sigma(m)$;
    esta máquina irá imprimir $m$ na fita,
    ao lado do $x$,
    e executar uma subrotina que computa $h$ no restante.

    Pelo teorema da recursão,
    $\sigma$ possui um ponto fixo $m_0$.
    Demonstraremos que $m_0$ satisfaz às exigências do terorema.

    Caso a função $h$, ao computar o valor de $h(m_0, x)$
    para algum $x$,
    tenha escolhido a primeira cláusula,
    a saída final de $\sigma(m_0)$ teria sido
    $M_{m_0} (x)$ concatenado com $a$,
    que é diferente de apenas $M_{m_0}(x)$.
    Portanto, $m_0$ não seria um ponto fixo de $\sigma$,
    contradizendo o teorema da recursão.

    Portanto, $h$ nunca seleciona a primeira cláusula
    ao computar $h(m_0, x)$, para qualquer $x$.
    Isto significa que $\Phi(m_0, x) > g(|x|)$ para todo $x$,
    o que garante a exigência de complexidade,
    e que
    \begin{equation*}
        M_{\sigma(m_0)}(x) = f(x).
    \end{equation*}
    Mas, como $m_0$ é um ponto fixo de $\sigma$,
    a própria $m_0$ já computava $f$ antes de passar por $\sigma$,
    o que prova a exigência da função.
\end{proof}

Em outras palavras,
código ruim pode ser feito em qualquer linguagem.
