\emph{Complexidade} é a quantidade de recursos
que uma máquina de Turing gasta
para computar determinada função
ou para decidir pertinência a uma linguagem
\cite[p.~285]{HopcroftUllman1979}.
Os recursos mais importantes para a teoria de complexidade computacional
são o espaço e o tempo,
em máquinas de Turing determinísticas e não"=determinísticas.

Os axiomas de Blum definem a noção de complexidade computacional
não apenas para máquinas de Turing,
mas sim para qualquer modelo de computação que satisfaça algumas restrições.

A seção~\ref{sec:enumeration_of_recursive_functions}
contém a maquinaria matemática necessária para expressar estas restrições.
Os axiomas de Blum aparecem na seção~\ref{sec:blum_axioms}.
Por fim, a seção~\ref{default_measures}
define as medidas de complexidade padrão.
