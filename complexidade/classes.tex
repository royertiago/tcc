\subsection{Classes de Complexidade}

\begin{definition}\cite[p.~242]{Kozen2006}
    Dada uma enumeração de Gödel aceitável $\phi$,
    uma medida de complexidade $\Phi$ para $\phi$,
    e uma função recursiva total
    $f: \mathbb N \rightarrow \mathbb N$,
    a \emph{classe de complexidade $f$} com relação a $\Phi$
    é o conjunto
    \begin{equation*}
        \mathcal C_\Phi(f) = \{ \phi_w \mid \Phi(w, x) \leq f(|x|)
            \text{ para quase todos os $x$}
        \}.
    \end{equation*}
\end{definition}
Permitiremos que a complexidade de $\phi_w$
possa ser maior que $f$ para um número finito de elementos
para generalizar algumas demonstrações.%
\footnote{
    Por razões didáticas, \citeonline[p.~285]{HopcroftUllman1979}
    definem classes de complexidade individualmente para cada medida de complexidade.
    Para tempo e espaço,
    que são as medidas levadas em consideração por estes autores,
    não precisamos nos preocupar com este número finito de elementos
    que extrapola o limite de complexidade,
    pois sempre podemos embutir suas respostas no controle finito da máquina.
    Desta forma,
    a complexidade resultante desta nova máquina satisfaz à restrição.

    Entretanto, como estamos trabalhando com medidas de complexidade arbitrárias,
    não há garantia de que possamos fazer esta modificação
    --- a medida de complexidade poderia, por exemplo,
    depender também da quantidade de estados da máquina.
    Portanto, para que estes resultados possam ser generalizados,
    embutimos esta liberdade adicional
    na própria definição de classe de complexidade
    \cite[p.~80]{McCreightMeyer1969}.
}
Como esta permissão existe apenas para um número finito de elementos,
para todo $w$, o valor de $\Phi(w, x)$ eventualmente sempre estará definido;
quer dizer, existe algum $k$ (dependente de $w$)
tal que $\Phi(w, x)$ está definido para todo $x$ com $|x| > k$.
Portanto,
todas as funções $\phi_w$ que pertençam a alguma classe de complexidade
estão definidas para todo $x$ suficientemente longo
(o quão longo é depende de $w$).
Isto é, o domínio destas funções é \emph{cofinito}.

Embora faça sentido definir $\mathcal C_\Phi(f)$
para funções $f$ arbitrárias,
a exigência de $f$ ser recursiva total
torna as classes de complexidade
suscetíveis a argumentos por diagonalização.

Por exemplo,
podemos mostrar que
nenhuma classe de complexidade contém todas as linguagens recursivas.
Precisamos de um lema.%
\footnote{
    A maior parte dos teoremas não precisou ser alterada significativamente
    para ser adaptada a esta noção de classe de complexidade.
    Este teorema é a exceção,
    pois a diagonalização agora precisa provocar uma quantidade infinita de falhas.
}

\begin{lemma}
    Existe uma função total recursiva $h: \{0, 1\}^* \to \{0, 1\}^*$
    tal que, para todos $x$,
    existem infinitos $y$ tais que $h(y) = x$.
    Isto é, as imagens inversas por $h$ são sempre infinitas.
\end{lemma}

\begin{proof}
    Defina $h$ por
    \begin{equation*}
        h( \langle x_1, x_2, \dots, x_n \rangle ) = x_1;
    \end{equation*}
    se $x$ não representa uma tupla,
    conforme definido na seção~\ref{sec:binary_encoding},
    defina $h(x) = \varepsilon$
    (ou qualquer outra cadeia fixa).

    A função $h$ assim definida é recursiva total;
    além disso, para todo $x$ e todo $z$,
    $h(\langle x, z \rangle) = y$,
    portanto todos os $x$ são $g(y)$ para infinitos $y$.
\end{proof}

\begin{theorem}
    Seja $\mathcal C_\Phi(f)$ uma classe de complexidade
    com relação à $\Phi$.
    Então existe uma função total recursiva $g$
    que não pertence à esta classe de complexidade.%
    \footnote{
        Observe que este teorema possui uma afirmação mais forte
        do que o teorema~\ref{thm:resource_waste};
        lá,
        nós encontramos um índice (para uma função)
        que desperdiça recursos ao computá"=la.
        Aqui,
        construiremos uma função que,
        não importa que técnica seja usada,
        a quantidade de recursos gastos eventualmente superará $f$.
    }
    Além disso, $g$ pode ser construída de forma que
    sua imagem seja o conjunto $\{0, 1\}$.
    \label{thm:outside_class}
\end{theorem}

\begin{proof}
    Construiremos uma função $g$
    que discorda de todas as funções
    que gastam menos de $f(n)$ recursos ao computar uma palavra de tamanho $n$.

    Usaremos a função $h$ do lema anterior.
    Defina a função $g$ por
    \begin{equation*}
        g(x) = \begin{cases}
            1, & \text{
                    se $\phi_{h(x)}(x)$ existe,
                    $\Phi(h(x), x) \leq f(|x|)$
                    e $\phi_{h(x)}(x) = 0$
                } \\
            0, & \text{ caso contrário.}
        \end{cases}
    \end{equation*}

    Observe que $g$ é uma função total.
    Para computar $g$,
    primeiro use o fato de que o predicado $\Phi(w, x) = n$ é decidível
    para descobrir se $\Phi(h(x), x) \leq f(|x|)$.
    Caso isso seja falso,
    então independente de $\phi_{h(x)}(x)$ existir ou não,
    $g(x)$ será igual a $0$.
    Agora, se $\Phi(h(x), x) \leq f(|x|)$,
    sabemos automaticamente que $\phi_{h(x)}(x)$ existe,
    e podemos computar este valor usando a função universal da enumeração $\phi$;
    então, usamos o resultado desta computação para calcular $g(x)$.
    Concluímos que $g$ é recursiva.

    Agora,
    tome $\phi_w$ uma função de $\mathcal C_\Phi(f)$.
    Mostraremos que $g \neq \phi_w$.

    Como $\phi_w \in \mathcal C_\Phi(f)$,
    existe algum $n_0$ tal que todos os $x$ com $|x| > n_0$ satisfaz
    $\Phi(w, x) \leq f(|x|)$.
    Escolha algum $y$ com $|y| > n_0$ tal que $h(y) = w$.
    Sabemos que tal $y$ existe pois a imagem inversa de $w$ por $h$ é infinita.

    Argumentaremos que $\phi_w(y) \neq g(y)$.
    Sabemos que $\Phi(w, y) \leq f(|y|)$,
    pois $|y| > n_0$;
    portanto, $\phi_w(y)$ está definido.
    Mas $w = g(y)$;
    portanto, das três condições do primeiro caso da definição de $g(y)$,
    duas já foram atendidas.
    Se a terceira também for atendida,
    isto é, $\phi_w(y) = 0$,
    o primeiro caso é selecionado e $g(y) = 1 \neq \phi_w(y)$.
    Entretanto,
    se a terceira não for atendida,
    temos $\phi_w(y) \neq 0 = g(y)$,
    pois o segundo caso será escolhido.
    Ou seja, $\phi_w \neq g$.

    Como $g$ é diferente de todas as funções de $\mathcal C_\Phi(f)$,%
    \footnote{
        Este argumento é,
        de certa forma,
        uma versão limitada
        (por $f$)
        do problema da parada.
    }
    \begin{equation*}
        g \notin \mathcal C_\Phi(f).
    \end{equation*}
\end{proof}
