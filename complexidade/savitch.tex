\subsection{Teorema de Savitch}

As demonstrações dos teoremas da seção~\ref{sec:relations_between_measures}
observam que,
se uma máquina de Turing respeita certa limitação de complexidade,
esta mesma máquina respeitará alguma outra limitação.
Por exemplo, no teorema~\ref{thm:dspace_in_dtime},
a mesma máquina que aceita a linguagem em espaço $f(n)$
a aceita em tempo $a c^{f(n)}$.
A única exceção é a equação~\ref{eq:ntime_in_dspace},
na qual fizemos uma máquina determinística simular uma máquina não"=determinística
sem estourar certo limite de espaço.
Mas, observe que, em nenhum ponto daquela demonstração,
a máquina simuladora calcula o valor de $f$;
isto é, os limites de complexidade da máquina simulada estavam implícitos.

Digamos que a máquina simulada $M$ tenha um limite de complexidade $f$.
Caso a máquina que está simulando $M$ possa, de alguma forma,
computar $f$,
ela poderá, por exemplo,
prever o tempo de execução (ou a quantidade de memória gasta) por $M$
antes de fazer a simulação;
esta previsão dá à máquina simuladora flexibilidade adicional,
que permite fazer afirmações mais fortes
a respeito das relações entre as classes de complexidade.
Uma destas afirmações é o teorema de Savitch.

\begin{definition}
    Uma função $S: \mathbb N \to \mathbb N$
    é dita ser \emph{espaço"=construtível}\footnote{
        Do inglês \emph{space"=constructible}
    }
    se existir uma máquina de Turing determinística $M$
    tal que, para todo $x$,
    \begin{equation*}
        M(x) = \langle f(|x|) \rangle
    \end{equation*}
    e
    \begin{equation*}
        \PhiDS(M, x) \leq f(|x|).
    \end{equation*}
    Em outras palavras,
    $M$ calcula $f$
    sem gastar mais espaço do que o próprio valor produzido.
    \cite[p.~79]{AroraBarak2009}
\end{definition}

Todas as funções ``bem"=comportadas'' como polinômios,
exponenciais, fatorial, e a soma, multiplicação e composição destas funções
são espaço"=construtíveis;
portanto,
embora usar apenas funções espaço"=construtíveis limite
as funções que podemos usar,
a maior parte dos algoritmos ``encontrados no dia"=a"=dia''
possuem limites de espaço que são construtíveis.

\begin{utheorem}[Teorema de Savitch]
    Se $f$ é uma função espaço"=construtível, então
    \begin{equation*}
        \NSPACE(f) \subseteq \DSPACE(f^2).
    \end{equation*}
\end{utheorem}

Omitiremos a demonstração do teorema de Savitch
pois tanto o enunciado preciso e a demonstração deste teorema
requerem o desenvolvimento mais detalhado da noção de
``espaço"=construtível''.
Ela pode ser encontrada, por exemplo,
no livro de \citeonline[p.~297, 301]{HopcroftUllman1979}
ou no de \citeonline[p.~79, 86]{AroraBarak2009}.
