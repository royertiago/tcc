\subsection{Teorema da União}

Na seção~\ref{sec:standard_classes},
definiremos algumas classes de complexidade computacional
como sendo a união de infinitas classes de complexidade.
O teorema da união diz que,
sob condições apropriadas,
esta união é uma classe de complexidade
de acordo com nossa definição de classe de complexidade.

\begin{theorem}[Teorema da União]
    \label{thm:union}
    Seja $\{h_1, h_2, h_3, \dots\}$ uma lista infinita de funções computáveis
    tais que, para todo $i$ e $n$,
    \begin{equation*}
        h_i(n) \leq h_{i+1}(n).
    \end{equation*}
    Então existe uma função computável $g$
    tal que
    \begin{equation*}
        \bigcup_i \ \mathcal C_\Phi(h_i) = \mathcal C_\Phi(g).
    \end{equation*}
\end{theorem}

A demonstração foi retirada do texto de \citeonline[p.~310]{HopcroftUllman1979}.

\begin{proof}
    Construiremos uma máquina de Turing
    que enumerará os valores de $g(n)$.

    Observe que
    \begin{equation*}
        \mathcal C_\Phi(h_1) \subseteq
        \mathcal C_\Phi(h_2) \subseteq
        \mathcal C_\Phi(h_3) \subseteq
        \mathcal C_\Phi(h_4) \subseteq
        \dots;
    \end{equation*}
    se alguma função $\phi_w$ pertence a união destas classes de complexidade,
    então $\phi_w \in \mathcal C_\Phi(h_k)$ para algum $k$,
    mas também que $\phi_w \in \mathcal C_\Phi(h_i)$
    para todas as classes de complexidade com $i \geq k$.

    Enumere as palavras de $\{0, 1\}^*$ por $w_1, w_2, \dots, w_n, \dots$,
    ordenando lexicograficamente.
    Para definir o valor de $g(n)$,
    analisaremos as funções $\phi_{w_1}, \dots, \phi_{w_n}$
    e manteremos uma lista de ``chutes''
    da forma $\phi_{w_i} \in \mathcal C_\Phi(h_j)$,
    que denotaremos $\mathcal C(i) = j$.
    Isto é, tentaremos ``adivinhar''
    em qual dos conjuntos a função $\phi_{w_i}$ entra.%
    \footnote{
        Embora estejamos usando os termos
        ``adivinhar'' e ``chute'',
        não há não"=determinismo envolvido.
        O problema é que não sabemos de antemão
        qual é o valor correto $\mathcal C(i)$,
        então precisamos estabelecer algum valor arbitrário
        e reestabelecer este valor
        conforme descobrimos que ele não funciona.

        Observe também que não precisamos
        ``acertar na mosca'';
        basta algum conjunto $\mathcal C_\Phi(h_k)$
        ao qual $\phi_{w_i}$ pertença.
    }

    Ao computar o valor de $g(n)$,
    um chute errado
    é um chute $\mathcal C(i) = j$,%
    \footnote{
        Há um abuso de notação aqui.
        Embora a notação $\mathcal C(i)$ seja de chamada de função,
        o uso de $\mathcal C(i)$ aqui está mais próximo do funcionamento
        de um vetor, indexado por $i$;
        talvez fosse melhor usar a notação $C[i] = j$,
        mas manteremos a notação original
        de \citeonline[p.~311]{HopcroftUllman1979}.
    }
    em que $\Phi(w_i, x) > h_j(|x|)$
    para algum $x$ com $|x| = n$.
    Conforme computamos $g(n)$,
    sempre que descobrirmos que nosso chute,
    digamos, para $\phi_{w_i}$, está errado,
    definiremos $g$ para um valor menor que $\Phi(w_i, x)$
    (para $|x| = n$),
    e faremos um novo chute,
    desta vez ``aumentando as apostas''.

    Defina $g(1)$ para $h_1(1)$
    (ou qualquer valor arbitrário)
    e inicialize a lista com o chute
    $\mathcal C(1) = 1$.
    A lista terá sempre tamanho $n-1$
    ao começar a decidir o valor de $g(n)$.

    Ao enumerar o valor de $g(n)$,
    adicione o chute $\mathcal C_n = n$ à lista.
    Percorra a lista atrás de chutes errados;
    isto é,
    pares $\mathcal C_i = k_i$
    tais que $\Phi(w_i, x) > h_{k_i}(x)$
    para algum $x$ com $|x| = n$.
    Dentre os chutes errados,
    escolha $i$ de forma a minimizar $k_i$.
    Defina, então, $g(n) = h_{k_i}(n)$
    e atualize o chute para $\mathcal C_i = n$.
    Caso não haja chutes errados,
    apenas defina $g(n) = h_n(n)$.

    \begin{algorithm}[h]
        Inicialize a lista \texttt{chutes} com o par $\mathcal C(1) = 1$ \;
        Imprima $g(1) = h_1(1)$ \;
        \For{$n = 2$ \KwTo $\infty$}{
            Adicione $\mathcal C(n) = n$ à lista \texttt{chutes} \;
            \tcc{Valor sentinela:}
            \texttt{chute\_errado} = $\infty$ \;
            \For{$i \leftarrow 1$ \KwTo $n$}{
                \tcc{acharemos o menor chute errado}
                \If{$\Phi(w_i, x) > h_{\mathcal C(i)}(|x|)$
                    para algum $x$ de tamanho $n$
                }{
                    \tcc{Achamos um chute errado.}
                    Substitua \texttt{chute\_errado} por $i$
                    caso isso reduza o valor de
                    $\mathcal C(\texttt{chute\_errado})$ \;
                }
            }
            \If{\texttt{chute\_errado} = $\infty$}{
                \tcc{Todos os chutes estavam certos.}
                Imprima $g(n) = h_n(n)$ \;
            }
            \Else{
                \texttt k $\leftarrow$ $\mathcal C(\texttt{chute\_errado})$ \;
                Imprima $g(n) = h_\texttt k (n)$ \;
                Altere o chute errado para $\mathcal C(\texttt{chute\_errado}) = n$ \;
            }
        }
        \caption{
            Algoritmo que enumera os valores da função $g$,
            cuja existência é afirmada pelo teorema da união.
        }
        \label{algo:union_theorem}
    \end{algorithm}

    Caso $\phi_{w_i}$ esteja em algum $\mathcal C_\Phi(h_k)$,
    assim que redefinirmos o chute para algum valor maior que $k$,
    erraremos, no máximo,
    mais uma quantidade finita de vezes.
    Caso $\phi_{w_i}$ não esteja na união das classes de complexidade,
    erraremos o chute infinitas vezes,
    e $g(|x|)$ será maior que $\Phi(w_i, x)$ para todos esses erros.

    Observe que os novos chutes que adicionamos à lista
    sempre serão maiores que todos os demais chutes existentes;
    e sempre redefinimos um chute errado
    para o maior valor de um chute até agora.
    Portanto, depois que adicionamos o chute
    $\mathcal C_i = i$ à lista
    (evento que ocorreu ao computar $g(i)$),
    ocorrerão, no máximo,
    $i-1$ chutes diferentes
    que resultarão na definição de $g(n)$ como $h_k(n)$
    para algum $k < i$.

    Então, se $\phi_{w_i} \in \mathcal C_\Phi(h_k)$,
    teremos $\Phi(w_i, x) > g(|x|)$
    para, no máximo, $k-1$ valores de $|x|$ maiores que $k$.
    Concluímos que
    \begin{equation*}
        \phi_{w_i} \in \mathcal C_\Phi(g),
    \end{equation*}
    pois $\Phi(M, x) > g(|x|)$ apenas um número finito de vezes.

    Isso prova
    \begin{equation*}
        \bigcup_i \ \mathcal C_\Phi(h_i) \subseteq \mathcal C_\Phi(g).
    \end{equation*}

    Do outro lado,
    pegue $\phi_{w_i}$ fora da união das classes de complexidade.
    Qualquer índice $w_j$ para $L$
    terá, para todo $k$,
    $\Phi(w_j, x) > h_k(|x|)$ para infinitos $x$.
    Em todos esses valores de $|x|$,
    iremos ``errar o chute''
    e definir $g(n)$
    para um valor menor que $\Phi(w_j, x)$,
    para algum $x$ com $|x| = n$.

    Isso prova
    \begin{equation*}
        \overline{\bigcup_i \ \mathcal C_\Phi(h_i)}
        \subseteq \overline{\mathcal C_\Phi(g)}.
    \end{equation*}

    Combinando as duas inclusões, prova"=se o teorema.
\end{proof}

Observe que a hipótese $h_i(n) \leq h_{i+1}(n)$
para todo $i$ e todo $n$
pode ser enfraquecida.
Podemos exigir, apenas,
que a desigualdade seja verdadeira para todo $n$
maior que algum $n_0$,
independente de $i$
--- isto é, eventualmente todos os $h_i$
deixam de se cruzar
---
ou que, para todo $i$,
exista algum $n_i$ em que
$h_i(n) \leq h_j(n)$ para todo $n > n_i$
e todo $j > i$
--- isto é, eventualmente,
cada $h_i$ deixa de cruzar com todos os outros.

Neste último caso
(que abrange o anterior),
apenas teríamos que alterar a quantidade máxima de vezes
em que podem haver chutes errados.
Se $\phi_{w_t} \in \mathcal C_\Phi(h_i)$,
então existem, no máximo,
$i - 1$ definições de $g(n)$
que poderiam ser baseados em $h_k$
com $k < j$.
No teorema original,
esses são os únicos valores que temos de nos preocupar
(são esses os $i-1$ ``chutes errados'' que podem haver
após definir $g(i)$).
Agora, além desses valores,
precisamos levar em conta que
todas as definições até $n_i$
podem ser inferiores a $h_i$
(e, possivelmente, abaixo da complexidade de $\phi_{w_t}$).

Em outras palavras,
é essencial que as funções $h_i$
eventualmente parem de se cruzar.

\begin{counterexample}
    Usaremos a complexidade de tempo para mostrar que
    existem uniões de classes de complexidade
    que não são classes de complexidade.
    Defina
    \begin{align*}
        f_1 & = \begin{cases}
                    n, & \text{se $n$ for par.} \\
                    n^3 + 2n, & \text{se $n$ for ímpar.}
                \end{cases} \\
        f_2 & = \begin{cases}
                    n, & \text{se $n$ for ímpar.} \\
                    n^3 + 2n, & \text{se $n$ for par.}
                \end{cases}
    \end{align*}
    É possível demonstrar que existe uma linguagem $L$ em
    \begin{equation*}
        \mathcal C_\PhiDT(n^3) \backslash \mathcal C_\PhiDT(n^2);
    \end{equation*}
    isto é, $L$ exige mais de $n^2$ de espaço para ser resolvida
    \cite[p.~299]{HopcroftUllman1979}.
    (O símbolo~$\backslash$ denota subtração de conjuntos.)

    Seja $\Sigma$ o alfabeto de $L$,
    tome $\$$ um símbolo fora de $\Sigma$
    e construa as linguagens
    \begin{align*}
        L_1 &= L \cup L\$ \cup ( \Sigma \Sigma )^* \\
        L_2 &= L \cup L\$ \cup \Sigma ( \Sigma \Sigma )^*
    \end{align*}

    $L_1$ contém todas as palavras de tamanho par de $\Sigma^*$,
    todas as palavras de tamanho ímpar de $L$,
    e todas as palavras de tamanho par de $L$ concatenadas com $\$$.
    $L_1$ foi construída de forma que pertencesse
    a $\mathcal C_\PhiDT(f_1)$;
    um algoritmo que resolve $L_1$
    é, primeiro, determinar se a entrada possui tamanho par
    e aceitando neste caso
    (gastando $n$),
    ou então apagar um possível $\$$,
    voltar para o começo
    (gastando mais $n$)
    e executar o algoritmo de $L$
    (gastando mais $n^3$).
    Note que todo algoritmo que opera em tempo $T(n)$
    para $L_1$
    pode ser convertido num algoritmo que opera em tempo
    $T(n) + 2n$ para $L$;
    basta verificar a paridade e emendar um $\$$ no final
    se necessário.
    Isso indica que $f_1$ é a complexidade ``ótima''
    de $L_1$;
    isto é,
    todas as máquinas que resolvem $L_1$
    precisam ter complexidade de tempo
    eventualmente acima de $n^2$,
    para todo $n$ par.

    $L_2$ é análogo, mas para $f_2$.
    Observe que $L$ não pertence nem a $\mathcal C_\PhiDT(f_1)$
    nem a $\mathcal C_\PhiDT(f_2)$.

    Suponha que exista uma função $g$ tal que
    \begin{equation*}
        \mathcal C_\PhiDT(f_1) \cup \mathcal C_\PhiDT(f_2) =
        \mathcal C_\PhiDT(g).
    \end{equation*}
    Considere $L'$ como uma linguagem que exija tempo $n^2$
    para ser resolvida em quase todas as instâncias.
    Note que $L'$ não pertence à união das duas classes de complexidade,
    por exigir tempo $n^2$ mesmo nos casos em que
    $f_1(n) = n$ ou $f_2(n) = n$.
    Mostraremos que $L' \in \mathcal C_\PhiDT(g)$.

    Seja $M_1$ um algoritmo que aceita $L_1$
    tal que $\PhiDT(M_1, x) \leq g(|x|)$
    para quase todo $x$.
    Essencialmente,
    após algum $n_1$,
    para toda palavra maior que este número
    precisamos ter $\PhiDT(M_1, x) \leq g(|x|)$.
    Pelo raciocínio acima,
    $\PhiDT(M_1, x)$,
    para $|x|$ par,
    não pode ser menor que $|x|^2$ sempre;
    portanto, $g(n) > n^2$ quase sempre
    que $n$ for par.

    A mesma técnica,
    se aplicada a $L_2$,
    mostra que
    $g(n) > n^2$ quase sempre que $n$ for ímpar.
    Combinando ambos,
    temos $g(n) > n^2$ quase sempre.
    Mas isso colocaria $L'$ em $\mathcal C_\PhiDT(g)$
    --- contradição.

    Portanto,
    \begin{equation*}
        \mathcal C_\PhiDT(f_1) \cup \mathcal C_\PhiDT(f_2)
    \end{equation*}
    não é uma classe de complexidade,
    é apenas um conjunto.
\end{counterexample}

Observe que a ausência de monotonicidade
não foi o ``grande vilão''
deste contraexemplo.
A prova do teorema da união,
em nenhum momento,
assumiu monotonicidade das funções envolvidas.
