\chapter{Complexidade Computacional}

\emph{Complexidade} é a quantidade de recursos
que uma máquina de Turing gasta
para computar determinada função
ou para decidir pertinência a uma linguagem
\cite[p. 285]{HopcroftUllman1979}.
Os recursos mais importantes para a teoria de complexidade computacional
são o espaço e o tempo,
tanto determinísticos como não-determinísticos.

Embora seja possível trabalhar com estas medidas diretamente,
muitos resultados para uma medida
possuem análogos em outra medida.
Para simplificar a exposição,
escolhemos iniciar,
na seção \ref{axiomas_blum},
uma discussão sobre teoria de complexidade axiomática,
e definir as medidas padrão na seção \ref{medidas_padrao}.

\section{Teoria de Complexidade Axiomática: Axiomas de Blum}
\label{axiomas_blum}

(Utilizaremos a definição provida por
\citeonline[p. 156]{Papadimitriou1994}.)

Dada uma máquina de Turing $M$
e uma palavra $x$,
denotaremos por $M(x)$ a ``saída''
de $M$ quando lhe é dado $x$ na entrada.

\begin{definition}
    Uma \emph{medida de complexidade}
    é uma função $\Phi$ que satisfaz aos seguintes axiomas:
    \begin{description}
        \item[Axioma 1]
            $\Phi(M, x)$ está definido
            se, e somente se,
            $M(x)$ está definido.
        \item[Axioma 2]
            Dados $M$, $x$ e $k$,
            é decidível se $\Phi(M, x) = k$.
    \end{description}
\end{definition}

\subsection{Classes de Complexidade}

\subsection{Teorema da União}

\section{Medidas de Complexidade Computacional}
\label{medidas_padrao}

% DTIME, DSPACE, NTIME, NSPACE

\subsection{Principais Classes de Complexidade Computacional}
% P, NP, PSPACE, EXP, NEXP, NEXPSPACE

\section{Hierarquias de Complexidade}

\subsection{Lema da Tradução}

\subsection{Hierarquia Polinomial}

\subsubsection{Oráculos}
