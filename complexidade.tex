\chapter{Complexidade Computacional}

\emph{Complexidade} é a quantidade de recursos
que uma máquina de Turing gasta
para computar determinada função
ou para decidir pertinência a uma linguagem
\cite[p. 285]{HopcroftUllman1979}.
Os recursos mais importantes para a teoria de complexidade computacional
são o espaço e o tempo,
tanto determinísticos como não-determinísticos.

Embora seja possível trabalhar com estas medidas diretamente,
muitos resultados para uma medida
possuem análogos em outra medida.
Para simplificar a exposição,
escolhemos iniciar,
na seção \ref{axiomas_blum},
uma discussão sobre teoria de complexidade axiomática,
e definir as medidas padrão na seção \ref{medidas_padrao}.

\section{Teoria de Complexidade Axiomática: Axiomas de Blum}
\label{axiomas_blum}

(Utilizaremos a definição provida por
\citeonline[p. 156]{Papadimitriou1994}.)

Dada uma máquina de Turing $M$
e uma palavra $x$,
denotaremos por $M(x)$ a ``saída''
de $M$ quando lhe é dado $x$ na entrada.

\begin{definition}
    Uma \emph{medida de complexidade}
    é uma função $\Phi$ que satisfaz aos seguintes axiomas:
    \footnotemark
    \begin{enumerate} [label=\textbf{Axioma \arabic*}, ref=\arabic*]
        \item
            \label{blum_def}
            $\Phi(M, x)$ está definido
            se, e somente se,
            $M(x)$ está definido.
        \item
            \label{blum_rec}
            Dados $M$, $x$ e $k$,
            é decidível se $\Phi(M, x) = k$.
    \end{enumerate}
    \footnotetext{
        A definição usual dos axiomas de Blum
        (encontrada, por exemplo,
        no texto de \citeonline[p. 313]{HopcroftUllman1979}
        e no próprio artigo original de \citeonline[p. 3]{Blum1967})
        aparece no contexto de computadores de funções de inteiros.
        Seja $M_1, M_2, \dots$ uma enumeração de máquinas de Turing.
        Consideraremos que a máquina $M_i$
        computa a função recursiva parcial $\phi_i$.
        Uma medida de complexidade é uma lista de funções
        $\{\hat \Phi_1, \hat \Phi_2, \dots\}$
        que satisfaz os seguintes axiomas:

        \begin{enumerate} [label=Axioma \arabic*', ref=\arabic*']
            \item
                \label{blum_def_orig}
                $\hat \Phi_i(n)$ está definido
                se, e somente se,
                $\phi_i(n)$ está definido.
            \item
                \label{blum_rec_orig}
                A função $R(i, n, m)$,
                definida como $1$ se $\hat \Phi_i(n) = m$,
                e $0$ em caso contrário,
                é recursiva.
        \end{enumerate}

        As duas definições são análogas.
        O valor $\hat \Phi_i(n)$ corresponde a $\Phi(M_i, 0^n)$.
        O axioma \ref{blum_def} corresponde ao axioma \ref{blum_def_orig},
        enquanto que a decidibilidade exigida pelo axioma \ref{blum_rec}
        é expressada pela função $R$ no axioma \ref{blum_rec_orig}.
    }
\end{definition}

\subsection{Classes de Complexidade}

\subsection{Teorema da União}

\section{Medidas de Complexidade Computacional}
\label{medidas_padrao}

% DTIME, DSPACE, NTIME, NSPACE

\subsection{Principais Classes de Complexidade Computacional}
% P, NP, PSPACE, EXP, NEXP, NEXPSPACE

\section{Hierarquias de Complexidade}

\subsection{Lema da Tradução}

\subsection{Hierarquia Polinomial}

\subsubsection{Oráculos}
