\section{Oráculos}

\begin{definition}
    Seja $A$ uma linguagem qualquer.
    Uma \emph{máquina de Turing com oráculo $A$}
    é uma máquina de Turing $M^A$ que possui uma fita especial
    e três estados adicionais: $q_?$, $q_y$ e $q_n$.
    Transitar para $q_?$ significa consultar o oráculo;
    ao fazer esta transição,
    caso a palavra nesta fita pertença à linguagem $A$,
    no próximo estado da computação $M$ transitará para $q_y$
    (a resposta foi positiva);
    caso contrário, $M$ transitará para $q_n$.
    \footnote{
        Observe que a cabeça da fita não se mexe durante a consulta.
        Portanto, a máquina pode escrever seu estado atual na fita
        antes de transitar para $q_?$ e recuperá-lo depois.
    }

    A definição de aceitação não é alterada.
    Chamaremos de $L^A(M)$ o conjunto das palavras aceitas por $M^A$.
\end{definition}

Intuitivamente, o oráculo é um dispositivo computacional
acoplado à máquina de Turing $M$.
É como se a máquina delegasse parte da computação
a outra máquina de Turing;
uma ``chamada de função''.

Observe que a única influência que $A$ possui em $M^A$
são as transições após $M$ ir para o estado $q_?$.
Ou seja, $A$ não pertence a $M$;
de fato, podemos ``acoplar'' várias linguagens diferentes
numa mesma máquina de Turing $M$
e obter diferentes $L^A(M)$ com isso%
\footnote{
    É exatamente por causa disso que,
    na notação do conjunto das palavras aceitas por $M^A$,
    o $A$ superscrito está junto de $L$, não de $M$.
}.

É importante ressaltar que esta visão de $A$ como outra máquina de Turing
é puramente intuitiva;
a linguagem $A$ não precisa ser sequer computável
para que ela possa ser usada como oráculo.
De fato, \citeonline[p.~210]{HopcroftUllman1979}
introduzem este conceito no contexto de decidibilidade.
Intuitivamente, se houvesse um algoritmo para o problema da parada,
poderíamos resolver o problema da vacuidade para máquinas de Turing,
por exemplo.
Mas não podemos partir da premissa de que
``existe uma máquina de Turing que resolve o problema da parada'',
pois esta hipótese contradiz o teorema da parada.
Oráculos podem ser entendidos como uma formalização deste ``e se?''.
\footnote{
    \citeauthoronline{HopcroftUllman1979} chamam de $S_1$ a linguagem
    do problema da vacuidade,
    que é equivalente ao problema da parada.
    Eles mostram que a vacuidade de máquinas com $S_1$ como oráculo
    é indecidível para máquinas com $S_1$ como oráculo.
    Desta forma, podemos ver que existem diferentes ``níveis''
    de indecidibilidade,
    da mesma forma que existem diferentes infinito.
}
