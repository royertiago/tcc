\section{Hierarquia polinomial}
\label{hierarquia_polinomial}

A hierarquia polinomial é uma generalização da classe $\NP$,
baseada no conceito de oráculo.

Um problema é $\NP$-completo se
ele está em $\NP$
e se todo problema em $\NP$ pode ser \emph{reduzido em tempo polinomial}
ao problema em questão.

A consequência desejada da definição de $\NP$-completude é que,
se um problema que eu estou tentando resolver é $\NP$-completo,
então achar um algoritmo polinomial para ele
irá, como consequência,
resultar num algoritmo polinomial para \emph{todos} os problemas da classe $\NP$.
Intuitivamente,
os problemas $\NP$-completos são os ``mais difíceis'' da classe $\NP$,
pois resolver qualquer $\NP$-completo em tempo polinomial
significa resolver todos os $\NP$ em tempo polinomial.
\footnote{
    Na vida real,
    após provar que certo problema é $\NP$-completo,
    nós desistimos de achar algoritmos polinomiais para ele
    e tentamos desenvolver heurísticas
    ou usamos técnicas baseadas em inteligência artificial.
}

Podemos interpretar a redução de forma algorítmica,
usando o conceito de oráculo.
Usaremos como exemplo os problemas $\SAT$ e $3\SAT$.

Sabemos que podemos reduzir $\SAT$ para $3\SAT$.
Dada uma instância $x$ de $\SAT$,
execute o algoritmo de redução para obter uma instância $y$ de $3\SAT$.
$x$ é satisfazível se, e somente se, $y$ o for.
Portanto, executar um algoritmo para $\SAT$ em $x$
retornará a mesma resposta
que executar um algoritmo para $3\SAT$ em $y$.
Ou seja, com um oráculo para $3\SAT$,
podemos resolver $\SAT$ em tempo polinomial:
o algoritmo de redução opera em tempo polinomial
e a chamada ao oráculo gasta apenas uma transição
(de $q_?$ para $q_y$ ou $q_n$).
Esta interpretação está sumarizada pelo algoritmo \ref{algoritmo_reducao}.

\begin{algorithm}[h]
    Leia a entrada $x$\;
    Execute o algoritmo de redução $f$ em $x$ para obter $y$\;
    Retorne \texttt{Oráculo}($y$)\;
    \caption{
        Interpretação algorítmica da noção de redução.
    }
    \label{algoritmo_reducao}
\end{algorithm}

Esta interpretação via oráculos revela que
a noção de redução possui uma restrição bastante forte:
nós podemos chamar o oráculo apenas uma vez,
e esta chamada precisa ser a última coisa que o algoritmo faz
--- não podemos nem mesmo alterar o valor retornado pelo oráculo%
\footnote{
    Note que esta descrição é muito semelhante
    às restrições impostas pela recursão caudal
    (\emph{tail recursion}).
}.

Embora esta restrição garanta que
um algoritmo polinomial para o oráculo
permita construir um algoritmo polinomial para a outra linguagem,
ela não é necessária.
Caso a função \texttt{Oráculo}, do algoritmo \ref{algoritmo_reducao},
for implementada em tempo polinomial,
podemos fazer quantas chamadas quisermos no algoritmo externo
e alterar os valores retornados pelas chamadas à vontade;
se o algoritmo externo em si
(ignorando as chamadas ao oráculo)
for polinomial,
o algoritmo resultante também terá complexidade polinomial.
\footnote{
    Este conceito também é válido para as hierarquias indecidíveis.
    Se $A$ é recursiva em $B$
    (isto é, com um oráculo para $B$, conseguimos decidir $A$),
    então, caso $B$ seja decidível,
    $A$ também o será,
    mesmo que não tenhamos uma redução de $A$ para $B$.
}

Isto sugere que reduções impõem restrições arbitrárias
e podem ser substituídas pelo uso de oráculos.
