\section{Hierarquia polinomial}
\label{hierarquia_polinomial}

A hierarquia polinomial é uma generalização da classe $\NP$,
baseada no conceito de oráculo.

Para defini-la,
iremos utilizar o abuso de notação mencionado na equação
\ref{abuso_notacao_oraculos}.
Formalmente:

\begin{definition}
    Seja $\mathcal A$ uma classe computacional
    para a qual há problemas completos (sob reduções polinomiais),
    e $B$ um problema completo para $\mathcal A$.
    Então,
    \begin{itemize}
        \item $\P^\mathcal A$
            é a classe dos problemas que podem ser resolvidos em tempo polinomial
            por uma máquina de Turing determinística
            usando um oráculo para $B$;
        \item $\NP^\mathcal A$
            é a classe dos problemas que podem ser resolvidos em tempo polinomial
            por uma máquina de Turing não-determinística
            usando um oráculo para $B$;
            e
        \item $\coNP^\mathcal A$
            é a classe dos complementos dos problemas em $\NP^\mathcal A$.
    \end{itemize}
\end{definition}

(A única diferença entre $\P^\mathcal A$ e $\NP^\mathcal A$
é o fato de permitirmos não-determinismo em $\NP^\mathcal A$.)

Intuitivamente,
$\P^\mathcal A$ é a classe de problemas que dá para resolver
se pegarmos as máquinas que resolvem os problemas em $\P$
e darmos oráculos em $\mathcal A$ para elas.

Importante ressaltar que esta notação
não está associada semanticamente ao conceito de potência;
de fato,
existem oráculos $A$ e $B$ para os quais
$\P ^ A = \NP ^ A$ e $\P ^ B \neq \NP ^ B$
\cite[p. 362]{HopcroftUllman1979}%
\footnote{
    De fato, \citeonline[p. 362]{HopcroftUllman1979}
    argumentam que este é um dos motivos pelos quais
    a questão $\P = \NP$ é tão difícil de ser resolvida
    --- os métodos que conhecemos
    (como, por exemplo, diagonalização)
    são facilmente traduzíveis para máquinas com oráculos.
    Portanto,
    uma prova de que $\P \neq \NP$ (por exemplo)
    precisaria empregar um método que deixaria de funcionar
    ao equipar as máquinas com o oráculo $A$ citado acima.
}.
