\section{Hierarquia polinomial}
\label{hierarquia_polinomial}

A hierarquia polinomial é uma generalização da classe $\NP$,
baseada no conceito de oráculo.
Para defini"=la,
utilizaremos o abuso de notação mencionado na equação~\ref{abuso_notacao_oraculos}.
Formalmente:

\simbolo{$\P^\mathcal A$}{Classe $\P$ relativa à $\mathcal A$}
\simbolo{$\NP^\mathcal A$}{Classe $\NP$ relativa à $\mathcal A$}
\simbolo{$\coNP^\mathcal A$}{Classe $\coNP$ relativa à $\mathcal A$}
\begin{definition}
    Seja $\mathcal A$ uma classe computacional.
    Então,
    \begin{itemize}
        \item $\P^\mathcal A$
            é a classe dos problemas que podem ser resolvidos em tempo polinomial
            por uma máquina de Turing determinística,
            usando como oráculo alguma linguagem de $\mathcal A$;
        \item $\NP^\mathcal A$
            é a classe dos problemas que podem ser resolvidos em tempo polinomial
            por uma máquina de Turing não"=determinística,
            usando como oráculo alguma linguagem de $\mathcal A$;
            e
        \item $\coNP^\mathcal A$
            é a classe dos complementos dos problemas em $\NP^\mathcal A$.
    \end{itemize}
\end{definition}

(A única diferença entre $\P^\mathcal A$ e $\NP^\mathcal A$
é o fato de permitirmos não"=determinismo em $\NP^\mathcal A$.)

Intuitivamente,
$\P^\mathcal A$ é a classe de problemas que dá para resolver
se pegarmos as máquinas que resolvem os problemas em $\P$
e darmos oráculos em $\mathcal A$ para elas.

Importante ressaltar que esta notação
não está associada semanticamente ao conceito de potência;
de fato,
existem oráculos $A$ e $B$ para os quais
$\P ^ A = \NP ^ A$ e $\P ^ B \neq \NP ^ B$
\cite[p.~362]{HopcroftUllman1979}%
\footnote{
    De fato, \citeonline[p.~362]{HopcroftUllman1979}
    argumentam que este é um dos motivos pelos quais
    a questão $\P = \NP$ é tão difícil de ser resolvida
    --- os métodos que conhecemos
    (como, por exemplo, diagonalização)
    são facilmente traduzíveis para máquinas com oráculos.
    Portanto,
    uma prova de que $\P \neq \NP$ (por exemplo)
    precisaria empregar um método que deixaria de funcionar
    ao equipar as máquinas com o oráculo $A$ citado acima.
},
independente de $\P$ ser igual a $\NP$ ou não.

Valendo-se desta notação,
podemos definir a hierarquia polinomial de maneira muito enxuta.

\simbolo{$\Delta_n^p$}{Generalização de $\P$ na hierarquia polinomial}
\simbolo{$\Sigma_n^p$}{Generalização de $\NP$ na hierarquia polinomial}
\simbolo{$\Pi_n^p$}{Generalização de $\coNP$ na hierarquia polinomial}
\begin{definition}[Hierarquia polinomial\footnotemark]
    \begin{align*}
        \Delta_0^p &= \Sigma_0^p = \Pi_0^p = \P \\
        \Delta_{n+1}^p &= \P^{\Sigma_n^p} \\
        \Sigma_{n+1}^p &= \NP^{\Sigma_n^p} \\
        \Pi_{n+1}^p &= \coNP^{\Sigma_n^p} \\
        \PH &= \bigcup_{n >= 0} \Sigma_n^p
    \end{align*}
\end{definition}
\footnotetext{
    \citeonline[p.~417]{Papadimitriou1994} e \citeonline[p.~104]{AroraBarak2009}
    notam que, caso $\mathcal A$ possua problemas polinomialmente completos
    (isto é, todos os problemas de $\mathcal A$ se reduzem
    em tempo polinomial a algum problema $B \in \mathcal A$ em questão),
    a classe $\P^\mathcal A$
    é a mesma classe que $\P^B$,
    pois qualquer problema de $\mathcal A$
    pode ser resolvido em tempo polinomial;
    basta substituir a chamada ao oráculo de $\mathcal A$
    pelo algoritmo que o resolve usando $B$.
    Embora esta troca possa aumentar o tempo de computação,
    a máquina ainda termina de executar em tempo polinomial.
    Isso justifica a notação $\P^\mathcal A$;
    é como se tivéssemos todos os problemas de $\mathcal A$ à nossa disposição.

    Com base nesta ideia,
    \citeonline[p.~102]{AroraBarak2009}
    mostram uma caracterização da hierarquia polinomial
    usando apenas um oráculo por nível.
    Por exemplo, $\Sigma_n^p$ é equivalente a $\NP^{\Sigma_{i-1}\SAT}$,
    em que $\Sigma_{i-1}\SAT$ é problema completo para $\Sigma_{i-1}$.
}

Os primeiros elementos da hierarquia polinomial podem ser expressados diretamente.

Por exemplo, $\Sigma_1^p = \NP^\P$.
Esta classe corresponde às máquinas de Turing não"=determinísticas
com acesso a oráculos em $\P$ que operam em tempo polinomial.
Mas o próprio oráculo pode ser computado em tempo polinomial;
como há uma quantidade polinomial de chamadas a este oráculo,
ao substituí-lo por um algoritmo ``de verdade'',
a máquina resultante ainda terá complexidade polinomial.
Portanto, todo problema de $\Sigma_1^p$ está em $\NP$.
E vice"=versa: qualquer problema de $\NP$ está em $\NP^\P$
--- basta a máquina não usar seu oráculo.
Portanto, $\Sigma_1^p = \NP$.

O mesmo raciocínio mostra que $\Delta_1^p = \P$
e que $\Pi_1^p = \coNP$.
\footnote{
    Poderiamos ter escolhido qualquer subconjunto de $\P$
    como início da hierarquia polinomial,
    que o restante seria o mesmo.
    De fato, \citeonline[p.~4]{MeyerStockmeyer1972}
    introduziram a hierarquia polinomial ao mundo
    escolhendo $\Delta_0^p = \Sigma_0^p = \Pi_0^p = \emptyset$.
}

Já $\Delta_2^p$ representa a ideia de generalizar reduções%
\footnote{
    Esta é uma interpretação minha da hierarquia polinomial.
    Os criadores originais da hierarquia,
    \citeonline[p.~4]{MeyerStockmeyer1972},
    tinham por objetivo achar problemas ``naturais''
    que fossem ``difíceis''.
    As separações entre as classes de complexidade,
    como, por exemplo, as demonstrações da seção~\ref{teoremas_hierarquia},
    são baseadas em diagonalização,
    e resultam em linguagens artificiais.
    O que \citeauthoronline{MeyerStockmeyer1972} queriam
    era achar problemas que surgem ``naturalmente''
    ao analisar problemas computacionais,
    que possuíssem as mesmas características
    destes resultantes de diagonalizações
    (isto é, que separassem classes de complexidade
    --- fossem ``difíceis'').
}.
Ao contrário de se comportar da maneira limitada,
como no algoritmo~\ref{algoritmo_reducao},
as máquinas polinomiais de $\Delta_2^p$
podem chamar seu oráculo uma quantidade polinomial de vezes,
alterando o valor retornado arbitrariamente.
Com isso, $\Delta_2^p$ engloba tanto $\NP$ quanto $\coNP$
--- acredita"=se que a inclusão seja própria.

$\Sigma_2^p$ pode ser visto como uma ``versão não"=determinística'' de $\Delta_2^p$.
Ela pode ser elegantemente expressada como $\NP^\NP$.
Para o restante da hierarquia,
não há uma interpretação equivalente;
são apenas os ``conceitos anexos'' que surgem por analogia.
Isto é, já que temos $\Sigma_2^p$,
podemos considerar $\co \Sigma_2^p$, que é $\Pi_2^p$
(ou $\coNP^\NP$);
E podemos considerar estas classes como oráculos;
E podemos considerar as classes resultantes como oráculos;
e assim por diante.
