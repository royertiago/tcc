\section{Equivalência entre oráculos}

Na subseção anterior,
mostramos que $L_u^1$ é decidível usando um oráculo para $S_1$
(isto é, $L_u^1$ é \emph{recursiva em} $S_1$).
O oposto também ocorre:

Suponha que dispomos de um oráculo para $L_u^1$.
Na entrada $\langle M \rangle$,
construa uma máquina $M'$ que ignorará sua própria entrada
e enumerar os pares $(i, j)$.
Para cada par enumerado,
$M'$ executará $M$ na $i$"=ésima palavra por $j$ movimentos.
Caso $M$ aceite, $M'$ aceita a entrada, qualquer que seja.
Caso contrário, $M'$ enumera o próximo par, e tente de novo.
De posse desta máquina, peça ao oráculo se ela aceita $\varepsilon$
(ou qualquer palavra).
Se sim, significa que, em algum par $(i, j)$,
a $M$ aceitou a entrada $i$; portanto, $L(M) \neq \emptyset$.
Caso contrário, significa que $M$ nunca aceitou palavra alguma;
portanto, $L(M) = \emptyset$.

\begin{definition}
    Dois oráculos $A$ e $B$ são ditos equivalentes
    se existem máquinas $M$ e $N$ tais que
    $L^A(M) = B$ e $L^B(N) = A$.
\end{definition}

Isto é, usando uma linguagem como oráculo, conseguimos decidir a outra,
e vice"=versa.
O que nós mostramos no parágrafo anterior é que
$L_u^1$ e $S_1$ são equivalentes.

A intuição por trás da equivalência é que,
ao ``programar'' uma máquina de Turing com oráculo $S_1$ (por exemplo),
podemos fingir que também podemos usar $L_u^1$ como oráculo,
e fazer uma ``chamada'' a este oráculo;
como eles são equivalentes,
há um algoritmo que traduz de um para outro,
portanto, podemos usar este algoritmo de tradução
para simular uma chamada a $L_u^1$.
