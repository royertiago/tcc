\section{PROBLEMAS INDECIDÍVEIS E EQUIVALÊNCIA DE ORÁCULOS}
\label{sec:oracle_equivalence}

É importante ressaltar que esta visão do oráculo $A$
como outra máquina de Turing acoplada à máquina principal
é puramente intuitiva;
a linguagem $A$ não precisa ser sequer computável
para que ela possa ser usada como oráculo.
De fato, \citeonline[p.~210]{HopcroftUllman1979}
introduzem este conceito no contexto de decidibilidade.
Intuitivamente, se houvesse um algoritmo para o problema da parada,
poderíamos resolver o problema da vacuidade para máquinas de Turing,
por exemplo.
Mas não podemos partir da premissa de que
``existe uma máquina de Turing que resolve o problema da parada'',
pois esta hipótese contradiz o teorema da parada.
Oráculos podem ser entendidos como uma formalização deste ``e se?''.

Mais precisamente, defina
\simbolo{$L_u^1$}{Problema da parada}
\simbolo{$S_1$}{Problema da vacuidade}
\begin{align*}
    L_u^1 &= \{ \langle M, x \rangle \mid M \text{ aceita } x \} \\
    S_1 &= \{ \langle M \rangle \mid L(M) = \emptyset \}
\end{align*}
$L_u^1$ é o problema da parada; $S_1$ é o problema da vacuidade.%
\footnote{
    A notação $S_n$ é usada também por \citeonline[p.~210]{HopcroftUllman1979}.
    A notação $L_u^n$ é derivada da notação para o problema da parada
    destes mesmos autores \cite[p.~183]{HopcroftUllman1979}.
}

A demonstração padrão de que $S_1$ é indecidível
é uma redução do problema da parada para o problema da vacuidade.
Em essência, ela diz que,
caso houvesse um algoritmo para a vacuidade,
poderíamos usar este algoritmo para resolver o problema da parada.
Como não há algoritmo para a parada,
não pode haver algoritmo para a vacuidade.

Em termos de oráculos,
isso significa que podemos decidir $L_u^1$
utilizando um oráculo para $S_1$.
O oposto também ocorre:

Suponha que dispomos de um oráculo para $L_u^1$.
Na entrada $\langle M \rangle$,
construa uma máquina $M'$ que ignorará sua própria entrada
e enumerar os pares $(i, j)$.
Para cada par enumerado,
$M'$ executará $M$ na $i$"=ésima palavra por $j$ movimentos.
Caso $M$ aceite, $M'$ aceita a entrada, qualquer que seja.
Caso contrário, $M'$ enumera o próximo par, e tente de novo.
De posse desta máquina, peça ao oráculo se ela aceita $\varepsilon$
(ou qualquer palavra).
Se sim, significa que, em algum par $(i, j)$,
a $M$ aceitou a entrada $i$; portanto, $L(M) \neq \emptyset$.
Caso contrário, significa que $M$ nunca aceitou palavra alguma;
portanto, $L(M) = \emptyset$.

\begin{definition}
    Dois oráculos $A$ e $B$ são ditos equivalentes
    se existem máquinas $M$ e $N$ tais que
    $L^A(M) = B$ e $L^B(N) = A$.
\end{definition}

Isto é, usando uma linguagem como oráculo, conseguimos decidir a outra,
e vice"=versa.
O que nós mostramos no parágrafo anterior é que
$L_u^1$ e $S_1$ são equivalentes.

A intuição por trás da equivalência é que,
ao ``programar'' uma máquina de Turing com oráculo $S_1$ (por exemplo),
podemos fingir que também podemos usar $L_u^1$ como oráculo,
e fazer uma ``chamada'' a este oráculo;
como eles são equivalentes,
há um algoritmo que traduz de um para outro,
portanto, podemos usar este algoritmo de tradução
para simular uma chamada a $L_u^1$.
