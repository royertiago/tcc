\section{HIERARQUIA POLINOMIAL}
\label{sec:polynomial_hierarchy}

A hierarquia polinomial é uma generalização da classe $\NP$,
baseada no conceito de oráculo.
Para defini"=la,
utilizaremos o abuso de notação mencionado na equação~\ref{eq:oracle_notation_abuse}.
Formalmente:

\simbolo{$\P^\mathcal A$}{Classe $\P$ relativa à $\mathcal A$}
\simbolo{$\NP^\mathcal A$}{Classe $\NP$ relativa à $\mathcal A$}
\simbolo{$\coNP^\mathcal A$}{Classe $\coNP$ relativa à $\mathcal A$}
\begin{definition}
    Seja $\mathcal A$ uma classe computacional.
    Então,
    \begin{itemize}
        \item $\P^\mathcal A$
            é a classe dos problemas que podem ser resolvidos em tempo polinomial
            por uma máquina de Turing determinística,
            usando como oráculo alguma linguagem de $\mathcal A$;
        \item $\NP^\mathcal A$
            é a classe dos problemas que podem ser resolvidos em tempo polinomial
            por uma máquina de Turing não"=determinística,
            usando como oráculo alguma linguagem de $\mathcal A$;
            e
        \item $\coNP^\mathcal A$
            é a classe dos complementos dos problemas em $\NP^\mathcal A$.
    \end{itemize}
\end{definition}

(A única diferença entre $\P^\mathcal A$ e $\NP^\mathcal A$
é o fato de permitirmos não"=determinismo em $\NP^\mathcal A$.)

Intuitivamente,
$\P^\mathcal A$ é a classe de problemas que podem ser resolvidos
se pegarmos as máquinas que resolvem os problemas em $\P$
e darmos oráculos em $\mathcal A$ para elas.%
\footnote{
    Note que, caso $\mathcal A$ possua um problema polinomialmente completo $B$
    (isto é, todos os problemas de $\mathcal A$ se reduzem
    em tempo polinomial a algum problema $B \in \mathcal A$ em questão),
    a classe $\P^\mathcal A$
    é a mesma classe que $\P^B$,
    pois qualquer problema de $\mathcal A$
    pode ser resolvido em tempo polinomial usando $B$ como oráculo;
    basta substituir a chamada ao oráculo de $\mathcal A$
    pelo algoritmo que o resolve usando $B$.
    Embora esta troca possa aumentar o tempo de computação,
    a máquina ainda termina de executar em tempo polinomial.
    Isso justifica a notação $\P^\mathcal A$;
    é como se tivéssemos todos os problemas de $\mathcal A$ à nossa disposição.
}

É importante ressaltar que esta notação
não está associada semanticamente ao conceito de potência;
de fato,
existem oráculos $A$ e $B$ para os quais
$\P ^ A = \NP ^ A$ e $\P ^ B \neq \NP ^ B$
\cite[p.~362]{HopcroftUllman1979},%
\footnote{
    De fato, \citeonline[p.~362]{HopcroftUllman1979}
    argumentam que este é um dos motivos pelos quais
    a questão $\P = \NP$ é tão difícil de ser resolvida
    --- os métodos que conhecemos
    (como, por exemplo, diagonalização)
    são facilmente traduzíveis para máquinas com oráculos.
    Portanto,
    uma prova de que $\P \neq \NP$ (por exemplo)
    precisaria empregar um método que deixaria de funcionar
    ao equipar as máquinas com o oráculo $A$ citado acima.
}
independente de $\P$ ser igual a $\NP$ ou não.

Valendo"=se desta notação,
podemos definir a hierarquia polinomial de maneira muito enxuta.

\simbolo{$\Delta_n^p$}{Generalização de $\P$ na hierarquia polinomial}
\simbolo{$\Sigma_n^p$}{Generalização de $\NP$ na hierarquia polinomial}
\simbolo{$\Pi_n^p$}{Generalização de $\coNP$ na hierarquia polinomial}
\begin{definition}[Hierarquia polinomial\footnotemark]
    \begin{align*}
        \Delta_0^p &= \Sigma_0^p = \Pi_0^p = \P \\
        \Delta_{n+1}^p &= \P^{\Sigma_n^p} \\
        \Sigma_{n+1}^p &= \NP^{\Sigma_n^p} \\
        \Pi_{n+1}^p &= \coNP^{\Sigma_n^p} \\
        \PH &= \bigcup_{n >= 0} \Sigma_n^p
    \end{align*}
\end{definition}
\footnotetext{
    Através de problemas completos para os níveis anteriores,
    \citeonline[p.~102]{AroraBarak2009}
    mostram uma caracterização da hierarquia polinomial
    usando apenas um oráculo por nível.
    Por exemplo, $\Sigma_n^p$ é equivalente a $\NP^{\Sigma_{n-1}\SAT}$,
    em que $\Sigma_{n-1}\SAT$ é um problema completo para $\Sigma_{n-1}^p$.
}

Os primeiros elementos da hierarquia polinomial podem ser expressados diretamente.

Por exemplo, $\Sigma_1^p = \NP^\P$.
Esta classe corresponde às máquinas de Turing não"=determinísticas
com acesso a oráculos em $\P$ que operam em tempo polinomial.
Mas o próprio oráculo pode ser computado em tempo polinomial;
como há uma quantidade polinomial de chamadas a este oráculo,
ao substituí"=lo por um algoritmo ``de verdade'',
a máquina resultante ainda terá complexidade polinomial.
Portanto, todo problema de $\Sigma_1^p$ está em $\NP$.
E vice"=versa: qualquer problema de $\NP$ está em $\NP^\P$
--- basta a máquina não usar seu oráculo.
Portanto, $\Sigma_1^p = \NP$.

O mesmo raciocínio mostra que $\Delta_1^p = \P$
e que $\Pi_1^p = \coNP$.%
\footnote{
    Poderíamos ter escolhido qualquer subconjunto de $\P$
    como início da hierarquia polinomial,
    que o restante seria o mesmo.
    De fato, \citeonline[p.~128]{MeyerStockmeyer1972}
    introduziram a hierarquia polinomial ao mundo
    escolhendo $\Delta_0^p = \Sigma_0^p = \Pi_0^p = \emptyset$.
}

Já $\Delta_2^p$ representa a ideia de generalizar reduções.%
\footnote{
    Esta é uma interpretação minha da hierarquia polinomial.
    Os criadores originais da hierarquia,
    \citeonline[p.~128]{MeyerStockmeyer1972},
    tinham por objetivo achar problemas ``naturais''
    que fossem ``difíceis''.
    As separações entre as classes de complexidade,
    como, por exemplo, as demonstrações por \citeonline[p.~299]{HopcroftUllman1979}
    são baseadas em diagonalização,
    e resultam em linguagens artificiais.
    O que \citeonline[p.~129]{MeyerStockmeyer1972} queriam
    era achar problemas que surgem ``naturalmente''
    ao analisar problemas computacionais,
    que possuíssem as mesmas características
    destes resultantes de diagonalizações
    (isto é, que separassem classes de complexidade
    --- fossem ``difíceis'').
}
Ao contrário de se comportar da maneira limitada,
como no algoritmo~\ref{algo:reduction},
as máquinas polinomiais de $\Delta_2^p$
podem chamar seu oráculo uma quantidade polinomial de vezes,
alterando o valor retornado arbitrariamente.
Com isso, $\Delta_2^p$ engloba tanto $\NP$ quanto $\coNP$
--- acredita"=se que a inclusão seja própria.

As inclusões dentro da hierarquia polinomial podem ser sumarizadas por
\begin{align*}
    \Pi_n^p \cup \Sigma_n^p & \subseteq \Delta_{n+1}^p, \\
    \Delta_{n+1}^p & \subseteq \Sigma_{n+1}^p, \\
    \Delta_{n+1}^p & \subseteq \Pi_{n+1}^p.
\end{align*}
Observe que a estrutura de inclusões da hierarquia polinomial
é muito parecida com a estrutura de inclusões da hierarquia aritmética
(o que justifica o uso dos mesmos símbolos para ambas as hierarquias),
mas, no caso da hierarquia polinomial,
não conseguimos demonstrar que as inclusões são estritas.
(De fato, uma delas, $\Delta_1^p \subseteq \Sigma_1^p$,
é exatamente o problema $\P$ vs $\NP$.)
Esta estrutura está esquematizada na figura~\ref{fig:polynomial_hierarchy}.

\begin{figure}
    \centering
    \begin{tikzpicture}
        \draw let \n1 = {0}, \n2 = {8}, \n3 = {1.5}, \n4 = {0.5} in
            (\n1, -\n4) -- (\n2, -\n4)
            (\n1, 0) -- (\n2, \n3)
            (\n1, \n3) -- (\n2, 0)
            (\n1, \n3) -- (\n2, 2*\n3)
            (\n1, 2*\n3) -- (\n2, \n3)
            (\n1, 2*\n3) -- (\n2, 3*\n3)
            (\n1, 3*\n3) -- (\n2, 2*\n3)
            (\n1, 4*\n3) -- (\n2, 4*\n3)
            (\n1, -\n4) -- (\n1, 4*\n3)
            (\n2, -\n4) -- (\n2, 4*\n3)
            node (a) at ({(\n1+\n2)/2}, 0) {
                $\Sigma_0^p = \Delta_0^p = \Pi_0^p = \P$
            }
            node (a) at (\n1+2*\n4, \n3/2) {$\Sigma_1^p = \NP$}
            node (a) at (\n2-2*\n4, \n3/2) {$\Pi_1^p = \coNP$}
            node (a) at ({\n1+\n2/2}, \n3) {$\Delta_2^p = \P^\NP$}
            node (a) at (\n1+\n4, 3*\n3/2) {$\Sigma_2^p$}
            node (a) at (\n2-\n4, 3*\n3/2) {$\Pi_2^p$}
            node (a) at ({\n1+\n2/2}, 2*\n3) {$\Delta_3^p$}
            node (a) at (\n1+\n4, 5*\n3/2) {$\Sigma_3^p$}
            node (a) at (\n2-\n4, 5*\n3/2) {$\Pi_3^p$}
            node (a) at ({\n1+\n2/2}, 3*\n3) {$\vdots$}
            node (a) at ({\n1+\n2/2}, 7*\n3/2) {$\PH \subseteq \PSPACE$};
    \end{tikzpicture}
    \caption[Estrutura de inclusões da hierarquia polinomial.]{
        Estrutura de inclusões da hierarquia polinomial.

        Um conjunto estar listado abaixo do outro
        (mesmo que na diagonal) denota que o conjunto de baixo
        é um subconjunto do conjunto de cima.
    }
    \label{fig:polynomial_hierarchy}
\end{figure}
