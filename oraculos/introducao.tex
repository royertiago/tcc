O conceito de ``oráculos computacional'' é um mecanismo bastante poderoso
para definir novas classes de complexidade computacional.
Com ele,
podemos formalizar perguntas como
``Que linguagens conseguiríamos reconhecer
se desse para resolver $\SAT$ gratuitamente?'',
ou ``E se o problema da parada fosse decidível?''.

Na seção~\ref{sec:oracle_definition} será apresentada a definição formal
de ``oráculo computacional''.
Neste primeiro momento,
trabalharemos apenas com linguagens como oráculos;
na seção~\ref{sec:functional_oracles}
utilizaremos funções em vez de linguagens.

A seção~\ref{sec:oracle_equivalence}
discute a interpretação de utilizar um problema indecidível
(como o problema da parada)
como oráculo.
Ainda no contexto da indecidibilidade,
esta seção define o que significa dois oráculos serem equivalentes,
e a seção~\ref{sec:arithmetical_hierarchy}
constrói uma hierarquia de problemas indecidíveis.
A seção~\ref{sec:oracles_and_reductions}
interpreta o conceito de redução em termos de oráculos.
Por fim, a seção~\ref{sec:polynomial_hierarchy}
utiliza o conceito de oráculos para construir a hierarquia polinomial.
