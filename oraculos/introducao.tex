Oráculos computacionais são um mecanismo bastante poderoso
para definir novas classes de complexidade computacional.
Com ele,
podemos formalizar perguntas como
"Que linguagens conseguiríamos reconhecer
se desse para resolver $\SAT$ gratuitamente?",
ou "E se o problema da parada fosse decidível?".

Na seção~\ref{sec:oracle_definition} será apresentada a definição formal
de ``oráculo computacional''.
A seção~\ref{sec:undecidable_problems_as_oracles}
discute a interpretação de utilizar um problema indecidível
(como o problema da parada)
como oráculo.
Ainda no contexto da indecidibilidade,
a seção~\ref{sec:oracle_equivalence} define o que significa
dois oráculos serem equivalentes,
e a seção~\ref{sec:undecidable_hierarchy}
constroi uma hierarquia de problemas indecidíveis.
Por fim,
a seção~\ref{sec:oracles_and_reductions}
interpreta o conceito de redução em termos de oráculos.
