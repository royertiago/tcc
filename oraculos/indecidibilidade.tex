\section{Problemas indecidíveis como oráculos}

É importante ressaltar que esta visão do oráculo $A$
como outra máquina de Turing acoplada à máquina principal
é puramente intuitiva;
a linguagem $A$ não precisa ser sequer computável
para que ela possa ser usada como oráculo.
De fato, \citeonline[p.~210]{HopcroftUllman1979}
introduzem este conceito no contexto de decidibilidade.
Intuitivamente, se houvesse um algoritmo para o problema da parada,
poderíamos resolver o problema da vacuidade para máquinas de Turing,
por exemplo.
Mas não podemos partir da premissa de que
``existe uma máquina de Turing que resolve o problema da parada'',
pois esta hipótese contradiz o teorema da parada.
Oráculos podem ser entendidos como uma formalização deste ``e se?''.

Mais precisamente, defina
\simbolo{$L_u^1$}{Problema da parada}
\simbolo{$S_1$}{Problema da vacuidade}
\begin{align*}
    L_u^1 &= \{ \langle M, x \rangle \mid M \text{ aceita } x \} \\
    S_1 &= \{ \langle M \rangle \mid L(M) = \emptyset \}
\end{align*}
$L_u^1$ é o problema da parada; $S_1$ é o problema da vacuidade.%
\footnote{
    A notação $S_n$ é usada também por \citeonline[p.~210]{HopcroftUllman1979}.
    A notação $L_u^n$ é derivada da notação para o problema da parada
    destes mesmos autores \cite[p.~183]{HopcroftUllman1979}.
}

A demonstração padrão de que $S_1$ é indecidível
é uma redução do problema da parada para o problema da vacuidade.
Em essência, ela diz que,
caso houvesse um algoritmo para a vacuidade,
poderíamos usar este algoritmo para resolver o problema da parada.
Como não há algoritmo para a parada,
não pode haver algoritmo para a vacuidade.

Em termos de oráculos,
isso significa que podemos decidir $L_u^1$
utilizando um oráculo para $S_1$.
