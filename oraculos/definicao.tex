\section{Definição}
\label{sec:oracle_definition}

\simbolo{$M^A$}{Máquina de Turing $M$ com oráculo $A$}
\simbolo{$L^A(M)$}{Linguagem da máquina $M$, quando equipada com o oráculo $A$}
\begin{definition}
    Seja $A$ uma linguagem qualquer.
    Uma \emph{máquina de Turing com oráculo $A$}
    é uma máquina de Turing $M^A$ que possui uma fita especial
    e três estados adicionais: $q_?$, $q_y$ e $q_n$.
    Transitar para $q_?$ significa consultar o oráculo;
    ao fazer esta transição,
    caso a palavra nesta fita pertença à linguagem $A$,
    no próximo estado da computação $M$ transitará para $q_y$
    (a resposta foi positiva);
    caso contrário, $M$ transitará para $q_n$.%
    \footnote{
        Observe que a cabeça da fita não se mexe durante a consulta.
        Portanto, a máquina pode escrever seu estado atual na fita
        antes de transitar para $q_?$ e recuperá"=lo depois.
    }

    A definição de aceitação não é alterada.
    Chamaremos de $L^A(M)$ o conjunto das palavras aceitas por $M^A$.
\end{definition}

Intuitivamente, o oráculo é um dispositivo computacional
acoplado à máquina de Turing $M$.
É como se a máquina delegasse parte da computação
a outra máquina de Turing;
uma ``chamada de função''.

Observe que a única influência que $A$ possui em $M^A$
são as transições após $M$ ir para o estado $q_?$.
Ou seja, $A$ não faz parte de $M$;
de fato, podemos ``acoplar'' várias linguagens diferentes
numa mesma máquina de Turing $M$
e obter diferentes $L^A(M)$ com isso.%
\footnote{
    É exatamente por causa disso que,
    na notação do conjunto das palavras aceitas por $M^A$,
    o $A$ sobrescrito está junto de $L$, não de $M$.
}

Utilizaremos oráculos para definir a hierarquia polinomial,
uma generalização das classes $\P$ e $\NP$,
na seção~\ref{sec:polynomial_hierarchy}.
