\section{Definição}
\label{sec:oracle_definition}

\simbolo{$M^A$}{Máquina de Turing $M$ com oráculo $A$}
\simbolo{$L^A(M)$}{Linguagem da máquina $M$, quando equipada com o oráculo $A$}
\begin{definition}
    Seja $A$ uma linguagem qualquer.
    Uma \emph{máquina de Turing com oráculo $A$}
    é uma máquina de Turing $M^A$ que possui uma fita especial
    e três estados adicionais: $q_?$, $q_y$ e $q_n$.
    Transitar para $q_?$ significa consultar o oráculo;
    ao fazer esta transição,
    caso a palavra nesta fita pertença à linguagem $A$,
    no próximo estado da computação $M$ transitará para $q_y$
    (a resposta foi positiva);
    caso contrário, $M$ transitará para $q_n$.%
    \footnote{
        Observe que a cabeça da fita não se mexe durante a consulta.
        Portanto, a máquina pode escrever seu estado atual na fita
        antes de transitar para $q_?$ e recuperá"=lo depois.
    }

    A definição de aceitação não é alterada.
    Chamaremos de $L^A(M)$ o conjunto das palavras aceitas por $M^A$.
\end{definition}

Intuitivamente, o oráculo é um dispositivo computacional
acoplado à máquina de Turing $M$.
É como se a máquina delegasse parte da computação
a outra máquina de Turing;
uma ``chamada de função''.

Observe que a única influência que $A$ possui em $M^A$
são as transições após $M$ ir para o estado $q_?$.
Ou seja, $A$ não faz parte de $M$;
de fato, podemos ``acoplar'' várias linguagens diferentes
numa mesma máquina de Turing $M$
e obter diferentes $L^A(M)$ com isso.%
\footnote{
    É exatamente por causa disso que,
    na notação do conjunto das palavras aceitas por $M^A$,
    o $A$ sobrescrito está junto de $L$, não de $M$.
}

Como a linguagem $A$ não faz parte da máquina $M^A$,
do ponto de vista formal,
$M^A$ continua sendo, essencialmente,
uma tabela de transições;
portanto,
podemos estender de maneira natural a codificação $\langle \cdot \rangle$
para máquinas com oráculos.
Da mesma forma, conceitos como $M^A(x)$ e função associada
podem ser definidos de maneira análoga.

Assim,
podemos definir uma enumeração de Gödel aceitável usando oráculos.

\begin{definition}
    \simbolo{$\phi^A$}{Enumeração das funções recursivas em $A$}
    Seja $A$ uma linguagem.
    A enumeração de funções recursivas em $A$,
    denotada por $\phi^A$,
    é o mapeamento
    \begin{equation*}
        \langle M^A \rangle \mapsto \phi^A_{\langle M \rangle},
    \end{equation*}
    em que $\phi^A_{\langle M \rangle}$
    é a função definida por
    \begin{equation*}
        \phi^A_{\langle M \rangle}(x) \simeq M^A(x).
    \end{equation*}
\end{definition}

Nem sempre $\phi^A$ será uma enumeração de Gödel aceitável.
O próximo teorema delimita as situações em que isso não acontece.

\begin{theorem}
    $\phi^A$ é uma enumeração de Gödel aceitável se,
    e somente se,
    $A$ for uma linguagem decidível.
\end{theorem}

\begin{proof}
    Suponha que $A$ não seja decidível.
    Defina a função $f: \{0, 1\}^* \to \{0, 1\}^*$ por
    \begin{equation*}
        f(x) =
        \begin{cases}
            1,& \text{ se } x \in A; \\
            0,& \text{ se } x \notin A.
        \end{cases}
    \end{equation*}
    Observe que $f$ não é uma função recursiva
    (pois $A$ não é decidível),
    mas ela pode ser computada facilmente usando $A$ como oráculo.
    Portanto, $\phi^A$ enumera uma função que não é recursiva,
    fazendo com que não seja uma enumeração de Gödel.

    Agora, suponha que $A$ seja decidível.
    Primeiro,
    observe que qualquer máquina de Turing sem oráculos $M$
    pode ser transformada numa máquina de Turing com oráculo $M^A$,
    que simplesmente ignora seu próprio oráculo.
    Portanto,
    todas as funções recursivas parciais são enumeradas por $\phi^A$.

    Agora,
    como $A$ é uma linguagem decidível,
    existe uma máquina de Turing que sempre para que decide pertinência a $A$.
    Dada uma máquina $M^A$,
    podemos transformá"=la numa máquina sem oráculo $M$
    equivalente --- isto é,
    \begin{equation*}
        M^A(x) \simeq M(x);
    \end{equation*}
    basta trocar as chamadas ao oráculo $A$
    pelo algoritmo que a decide.
    Portanto, $\phi^A$ só enumera funções recursivas parciais.

    Concluímos que $\phi^A$ é uma função sobrejetora
    cuja imagem é o conjunto das funções recursivas parciais.
    A construção da máquina universal
    e do teorema $S_{mn}$ é análoga às máquinas sem oráculos.%
    \footnote{
        Observe que construir a máquina universal
        e o teorema $S_{mn}$ pode ser feito mesmo que $A$ não seja decidível;
        o problema é, justamente,
        a possibilidade de enumerar funções que não são recursivas.
    }
\end{proof}
