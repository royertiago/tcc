\section{Reduções e oráculos}
\label{sec:oracles_and_reductions}

\begin{definition}[$\NP$"=completude]
    Uma linguagem $L$ é $\NP$"=completa se
    \begin{itemize}
        \item $L \in \NP$; e
        \item Para toda linguagem $L' \in \NP$ existe alguma função $f$,
            computável em tempo polinomial,
            tal que
            \begin{equation*}
                x \in L' \iff f(x) \in L.
            \end{equation*}
    \end{itemize}
\end{definition}

Esta é a definição usual de $\NP$"=completude,
encontrada, por exemplo,
nos livros didáticos de \citeonline[p.~288]{Sipser2006}
e \citeonline[p.~42]{AroraBarak2009}.
\citeonline[p.~324]{HopcroftUllman1979} e \citeonline[p.~174]{Papadimitriou1994}
apresentam uma definição um pouco mais restritiva:
a função $f$ precisa ser computada em espaço logarítmico
(o que implica tempo polinomial).

O termo ``$\NP$"=completude'' tenta formalizar a noção de
``ser um problema difícil da classe $\NP$''.
Com a definição acima,
podemos provar que,
se acharmos algum algoritmo polinomial para um problema $\NP$"=completo,
podemos resolver \emph{todos} os problemas da classe $\NP$ em tempo polinomial.%
\footnote{
    Na vida real,
    após provar que certo problema é $\NP$"=completo,
    nós desistimos de achar algoritmos polinomiais para ele
    e tentamos desenvolver heurísticas
    ou usamos técnicas baseadas em inteligência artificial.
}
Isto é,
resolver um significa resolver todos;
portanto,
intuitivamente,
estes problemas devem ser os ``mais difíceis'' da classe $\NP$.
Por exemplo,
deve ser mais fácil achar um algoritmo polinomial para fatoração de inteiros,
que ainda não foi nem provado estar em $\P$
nem ser $\NP$"=completo \cite[p.~120]{DuKo2014},
do que achar um algoritmo polinomial para $\SAT$,
o primeiro problema natural provado $\NP$"=completo \cite[p.~80]{DuKo2014}.

Podemos interpretar a redução de forma algorítmica,
usando o conceito de oráculo.
Usaremos como exemplo os problemas $\SAT$ e $3\SAT$.

Sabemos que podemos reduzir $\SAT$ para $3\SAT$.
Dada uma instância $x$ de $\SAT$,
execute o algoritmo de redução para obter uma instância $y$ de $3\SAT$.
$x$ é satisfazível se, e somente se, $y$ o for.
Portanto, executar um algoritmo para $\SAT$ em $x$
retornará a mesma resposta
que executar um algoritmo para $3\SAT$ em $y$.
Ou seja, com um oráculo para $3\SAT$,
podemos resolver $\SAT$ em tempo polinomial:
o algoritmo de redução opera em tempo polinomial
e a chamada ao oráculo gasta apenas uma transição
(de $q_?$ para $q_y$ ou $q_n$).
Esta interpretação está sumarizada pelo algoritmo~\ref{algo:reduction}.

\begin{algorithm}[h]
    Leia a entrada $x$\;
    Execute o algoritmo de redução $f$ em $x$ para obter $y$\;
    Retorne \texttt{Oráculo}($y$)\;
    \caption{
        Interpretação algorítmica da noção de redução.
    }
    \label{algo:reduction}
\end{algorithm}

Esta interpretação via oráculos revela que
a noção de redução possui uma restrição bastante forte:
nós podemos chamar o oráculo apenas uma vez,
e esta chamada precisa ser a última coisa que o algoritmo faz
--- não podemos nem mesmo alterar o valor retornado pelo oráculo.%
\footnote{
    Note que esta descrição é muito semelhante
    às restrições impostas pela recursão caudal
    (\emph{tail recursion}).
}

Embora esta restrição garanta que
um algoritmo polinomial para o oráculo
permita construir um algoritmo polinomial para a outra linguagem,
ela não é necessária.
Caso a função \texttt{Oráculo}, do algoritmo~\ref{algo:reduction},
for implementada em tempo polinomial,
podemos fazer quantas chamadas quisermos no algoritmo externo
e alterar os valores retornados pelas chamadas à vontade;
se o algoritmo externo em si
(ignorando as chamadas ao oráculo)
for polinomial,
o algoritmo resultante também terá complexidade polinomial.%
\footnote{
    Este conceito também é válido para as hierarquias indecidíveis.
    Se $A$ é recursiva em $B$
    (isto é, com um oráculo para $B$, conseguimos decidir $A$),
    então, caso $B$ seja decidível,
    $A$ também o será,
    mesmo que não tenhamos uma redução de $A$ para $B$.
}

Isto sugere que reduções impõem restrições arbitrárias
e podem ser substituídas pelo uso de oráculos.
