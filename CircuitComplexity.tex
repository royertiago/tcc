\documentclass{article}

\usepackage[utf8]{inputenc}
\usepackage[T1]{fontenc}
\usepackage{amsmath}
\usepackage{amsthm}

\newtheorem{theorem}{Teorema}
\newtheorem{lemma}[theorem]{Lema}
\theoremstyle{definition}
\newtheorem{definition}[theorem]{Definição}

\begin{document}

\author{Tiago Royer}
\title{Complexidade de Circuitos}

\maketitle

\section{Introdução}

    O mais famoso problema de complexidade computacional é o $P$ versus $NP$.
    Em essência, ele pergunta se,
    para uma classe específica de problemas
    (os problemas $NP$),
    é mesmo exponencialmente mais difícil verificar
    se uma resposta está certa
    do que descobrir qual é a resposta certa.

    A maior parte dos pesquisadores acredita que $P \neq NP$,
    entretanto nada foi demonstrado.
    Uma das ``frentes de ataque'' é tentar demonstrar limites inferiores
    em modelos computacionais mais restritivos com a esperança de,
    no futuro,
    demonstrar estes mesmos limites inferiores para Máquinas de Turing \cite{Hastad1987}.
    Um destes modelos computacionais é o de circuitos booleanos.

\bibliographystyle{plain}
\bibliography{bib/bibliography}

\end{document}
