\documentclass[12pt]{letter}
\usepackage[a4paper,margin=2cm]{geometry}

\usepackage[T1]{fontenc}
\usepackage[utf8]{inputenc}

\usepackage{mathptmx}
\usepackage{tabularx}
\usepackage{multirow}

\begin{document}

\pagestyle{empty}

\begin{centering}

    \textbf{DEPARTAMENTO DE INFORMÁTICA E ESTATÍSTICA -- CTC -- UFSC}

    \textbf{RATIFICAÇÃO DE PLANO DE TRABALHO DO SEMESTRE \\ PARA DESENVOLVIMENTO DE TCC}

\end{centering}


\vspace{1em}
\setlength\extrarowheight{5pt}
\begin{tabular}{l l}
    \textbf{Disciplina:} & TCC 2 \\
    \textbf{Curso:}      & Ciência da Computação \\
    \textbf{Autor:}      & Tiago Royer \\
    \textbf{Título:}     & Máquinas de Turing não-determinísticas
                           como computadores de funções \\
    \textbf{Professora responsável:} & Jerusa Marchi \\
\end{tabular}


\vspace{1em}
{\large \textbf{Objetivos}}
\\

\textbf{Objetivo geral:}
Desenvolver uma nova definição de funções não-determinísticas,
de modo a ge\-ne\-ra\-li\-zar naturalmente o aparente aceleramento exponencial
que obtemos ao conceder não-determinismo à máquinas de Turing.

\textbf{Objetivos específicos:}
\begin{enumerate}
    \item Manter compatibilidade com os axiomas de Blum.
    \item Comparar a definição dada com o trabalho de outros autores.
    \item Desenvolver a noção de composição de funções não-determinísticas.
    \item Relacionar a composição de funções não-determinísticas
        com a hierarquia polinomial.
\end{enumerate}


\vspace{1em}
{\large \textbf{Cronograma}}

\begin{tabularx}{\linewidth}{|X|*{4}{c|}}
    \hline
    \multicolumn{1}{|c|}{\multirow{2}{*}{Etapas}} & \multicolumn{4}{|c|}{Meses}\\

    \cline{2-5}
    & ago & set & out & nov \\ \hline

    Estudar problemas completos para a hierarquia polinomial
    &  x  &  x  &     &     \\ \hline

    Usar a nova noção de funções não-determinísticas para impor relações &&&&\\
    entre as classes da hierarquia polinomial.
    &  x  &  x  &  x  &   x \\ \hline

    Escrever o artigo
    &     &     &  x  &  x  \\ \hline

    Comparar meu trabalho a outras definições de ``função não-determinística''
    &     &  x  &  x  &  x  \\ \hline

    Defesa
    &     &     &     &  x  \\ \hline
\end{tabularx}

\begin{centering}

    \fbox{\begin{minipage}[c][6em][c]{0.7\textwidth}
        {\center \textbf{Preenchimento pelo Professor responsável pelo TCC}\\[1em]}

        \qquad $(\quad)$ \ Ciente e de acordo.

        \qquad Data: \_\_ / \_\_ / \_\_
    \end{minipage}}

\end{centering}
\end{document}
