\documentclass[12pt]{article}
\usepackage{sbc/template}

\usepackage[utf8]{inputenc}

\usepackage{complexity}

\sloppy

\title{Blum axioms and nondeterministic computation of functions}

\author{Tiago Royer\inst{1}}

\address{
    Departamento de Informática e Estatística
    --- Universidade Federal de Santa Catarina
    \email{royertiago@gmail.com}
}

\begin{document}

\maketitle

\begin{abstract}
    In his doctoral thesis,
    Manuel Blum proposed two axioms for complexity measures
    that allows us to talk about complexity in an axiomatic manner.
    His axioms does not even specify the machine model
    --- it just requires it to satisfy some properties.
    Blum axioms, however,
    are defined in the context of function computation.
    This restriction is easy to implement with deterministic machines,
    since there is only one output for a given input,
    but how can a nondeterministic Turing machine compute a function?
    This paper surveys techniques to associate
    nondeterministic machines with functions
    and analyze how they interact with computational complexity.
\end{abstract}

\section{Introduction}

\section{Gödel numberings}

\subsection{Acceptable Gödel numberings}

\subsection{Blum axioms}

\section{Nondeterministic computation of functions}

\subsection{Hopcroft-Ullman definition}
\subsection{Oded Goldreich definition}
\subsection{Krentel's $\OptP$ class}
\subsection{Valiant's $\#\P$ class}
\subsection{Alternative approaches for nondeterministic functions}

\end{document}
