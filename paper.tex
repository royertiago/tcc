\documentclass[12pt]{article}
\usepackage{sbc/template}

\usepackage[utf8]{inputenc}
\usepackage[T1]{fontenc}

\usepackage{amsmath}
\usepackage{amssymb}

\usepackage{amsthm}
\theoremstyle{definition}
\newtheorem{definition}{Definition}
\newtheorem{example}[definition]{Example}
\newtheorem{proposition}[definition]{Proposition}

\usepackage{complexity}

\sloppy

\title{Blum axioms and nondeterministic computation of functions}

\author{Tiago Royer\inst{1}, Jerusa Marchi\inst{1}}

\address{
    Universidade Federal de Santa Catarina --- Departamento de Informática e Estatística \\
    \email{royertiago@gmail.com, jerusa.marchi@ufsc.br}
}

\begin{document}

\maketitle

\begin{abstract}
    In his doctoral thesis,
    Manuel Blum proposed two axioms for complexity measures
    that allows us to talk about complexity in an axiomatic manner.
    His axioms does not even specify the machine model
    --- it just requires it to satisfy some properties.
    Blum axioms, however,
    are defined in the context of function computation.
    This restriction is easy to implement with deterministic machines,
    since there is only one output for a given input,
    but how can a nondeterministic Turing machine compute a function?
    This paper surveys techniques to associate
    nondeterministic machines with functions
    and analyze how they interact with computational complexity.
\end{abstract}

\section{Introduction}

In Theory of Computation,
we usually use languages
to mathematically model problems in the real world.
Decision problems (``yes/no'') are mapped to languages in a very natural way,
by just putting every ``yes'' instance in the language,
and leaving the rest out.
Search problems usually are rewritten as a decision problem,
and then this problem is converted to a language.
For instance,
the task of finding a satisfying assignment for a Boolean formula
is reinterpreted as the task of deciding whether such an assignment exists,
and this task is then converted to a language
--- in this example we have $\SAT$, the Boolean satisfiability problem.
This does the trick when it comes to proving that something is hard;
for instance, if we show that the decision problem is $\NP$-hard or undecidable,
then intuitively the corresponding search problem must be at least as hard.
Therefore,
concepts like ``decidable'', ``$\NP$-complete'', ``polynomial-time decidable''
arise naturally in the context of decision problems.

However, even from the theoretical standpoint,
it is useful to extend these concepts to functions.
For instance,
the concept of polynomial-time computable functions
are required to define Karp reductions \cite[p.~42]{AroraBarak2009}.
For single-tape deterministic Turing machines,
this definition is easy to extend:
A function $f : \{0, 1\}^* \to \{0, 1\}^*$
is said to be \emph{polynomial-time computable}
if there is a Turing machine $M$ and a polynomial $p$ such that,
when given the string $x$ as input,
$M$ halts with $f(x)$ in its tape within~$p(|x|)$~steps.

Most extensions to this basic model,
like the use of several tracks, or several tapes, or multidimensional tapes,
can be easily incorporated in the definition of polynomial-time computable.
Except nondeterminism.
We hit a wall right in the start:
how does a nondeterministic machine computes a function in the first place?

This paper presents several attempts to establish
how a nondeterministic Turing machine could compute a function,
and to extend the concept of ``polynomial-time computable'' under each definition.
Section~\ref{sec:quality-standards} sets reasonableness criteria
to both the definition of function computation and complexity of the computation.
Section~\ref{sec:nondeterministic_computation_of_functions},
which contains the several definitions,
presents and criticizes Hopcroft and Ullman's and Goldreich's definitions
(sections \ref{sec:hopcroft-ullman} and~\ref{sec:goldreich}),
proposes one possible definition that meet the quality standards
(section~\ref{sec:proposed}),
and shows other two definitions,
by Krentel (sec~\ref{sec:krentel}) and Valiant (sec~\ref{sec:valiant}),
that, altough were not proposed in the context
of general nondeterministic computation of functions,
also meet the quality standards and sidesteps the problems with the proposed definition
(presented in section~\ref{sec:np-completeness-extension}).
The development of this project is mentioned in section~\ref{sec:development},
after the concluding remarks (section~\ref{sec:conclusion}).

\section{Our approach}
\label{sec:our-approach}

The mathematical definition of function
requires it to return a value for all elements of its domain,
and to return a single value in all cases.
In this paper,
we will try to associate nondeterministic computation
with \emph{partial} recursive functions;
that is,
functions that are only defined for some elements in its domain.
(Strictly speaking,
these objects are not ``functions'';
but we will allow for this abuse of nomenclature.
If the partial function is defined everywhere,
we will call it a \emph{total} function, for contrast.)
In general,
we cannot demand a function resulting from a computation to be total
because the computation may not halt.

Since nondeterministic computation returns several values,
to associate it with functions we need to either restrict what things define a function
or to somehow extract a single value from all these branches of computation.
It is important to note that this is not the only approach available in the literature.
Complexity classes like $\ComplexityFont{NPVM}$
(see, for example, the paper of Selman~\cite[p.~359]{Selman1994})
are defined using \emph{multivalued functions},
which are allowed to return several values for a single input.
Another approach is to use \emph{function problems} associated to problems in $\NP$,
where the machine is required to return any certificate for the given instance,
or to reject the input if it is not in the language;
this yields the class $\FNP$~\cite[p.~229]{Papadimitriou1994}.
We will not consider these approaches here;
in particular, we will define $\FNP$ differently below.

\section{Quality Standards}
\label{sec:quality-standards}

We are trying to make a definition of non-deterministic computation of functions.
We want our definition to be reasonable,
in the sense of not falling in the realm of undecidable problems,
nor ``forgetting'' some computable function;
and we want our definition to be efficient,
in the sense to preserve the apparent\footnote{
    It is apparent in the sense that,
    if proven would show $\P \neq \NP$,
    and if disproved (showing a polynomial slowdown is the best we can do)
    would show $\P = \NP$.
    Although most researchers expect the former to be the case~\cite[p.~54]{Gasarch2012},
    as Papadimitriou noted~\cite[p.~412]{Papadimitriou1994},
    in the absence of a proof that $\P \neq \NP$
    we should not be too emphatic in stating the simulation \emph{require}
    (as opposed to \emph{seemingly require}) exponential slowdown.
}
exponential speed-up when solving,
for instance, the Boolean satisfiability problem
--- that is, there should be an exponential speed-up
when computing the characteristic function\footnotemark{} of $\SAT$.
\footnotetext{
    The characteristic function of a set $A$
    is the function $1_A$ defined to be $1$ for $x \in A$
    and $0$ for $x \notin A$.
}

Sections \ref{sec:godel-numberings} and~\ref{sec:blum-axioms}
formalize these reasonableness and efficiency requirements,
respectively.

\subsection{Reasonableness criterion: acceptable Gödel numberings}
\label{sec:godel-numberings}

One of the most important theoretical results concerning Turing machines
is the existence of undecidable problems.
Namely, the \emph{halting problem}
(the task of deciding whether a given Turing machine will halt on a given input)
cannot be solved by Turing machines \cite[p.~23]{AroraBarak2009}.
The formalization (and proof) of this fact
requires the definition of some sort of \emph{encoding};
since Turing machines can only reason about strings,
we need somehow to encode Turing machines into strings,
to be able to pose the halting problem as a language question.

Each Turing machine can be associated
to the corresponding partial recursive function it computes.
There are several ways to encode Turing machines as strings,
but what is most important about them is that they allow us
to manipulate these partial recursive functions \emph{indirectly}
--- partial recursive functions are (potentially) infinite objects,
so we cannot write them down on a Turing machine tape,
but we \emph{can} write the encoding of a Turing machine that compute these functions.

Therefore,
these encodings provide a way to associate a string
(which is a finite, manipulable object)
with a partial recursive function
(which is an infinite, mathematical, ``untouchable'' object).
Encodings are \emph{enumerations} of all partial recursive functions.

One important feature about the standard encodings of deterministic Turing machines
is the Universal Turing Machine Theorem:
the existence of a partial recursive function
$U: \{0, 1\}^* \times \{0, 1\}^* \to \{0, 1\}^*$
such that, if $w$ encodes the machine $M$,
then $U(m, x)$ is the result of running the machine $M$ on $x$.
A machines that compute $U$ is called \emph{universal Turing machine}.

The concept of \emph{acceptable Gödel numbering}
(\cite[p.~41]{Rogers1987},~\cite[p.~324]{Blum1967})
encompasses the existence of universal machines and a little more.
We will use it as our reasonableness criteria to our definitions.

\vspace{6pt}
\begin{definition}
    Let $\mathcal P$ be the set of all partial recursive functions.
    An \emph{acceptable Gödel numbering}
    is a function $\phi : \{0, 1\}^* \to \mathcal P$,
    that associates each string (or program\footnotemark) $w \in \{0, 1\}^*$
    \footnotetext{
        In texts like Rogers'~\cite{Rogers1987},
        the partial recursive functions have the naturals as domain and codomain
        (that is, they are of the form $f: \mathbb N \to \mathbb N$),
        and Gödel numberings associates natural numbers with recursive functions.
        In this paper we will work with binary strings instead of numbers,
        which will simplify the definition of complexity classes
        and allows us to think the string $w$ as a \emph{program} for $\phi_w$
        (see example~\ref{c}).
    }
    to a function $\phi_w \in \mathcal P$,
    that satisfies
    \begin{enumerate}
        \item $\phi$ is surjective;
            that is, every partial recursive function $f \in \mathcal P$
            has a program $w \in \{0, 1\}^*$ such that $\phi_w = f$;
            \label{surjectiveness}
        \item There is a partial recursive function
            $U:\{0, 1\}^* \times \{0, 1\}^* \to \{0, 1\}^*$ such that,
            for every $w$ and $x$,
            $U(w, x)$ is defined if and only if $\phi_w(x)$ is defined,
            and, in this case,
            \begin{equation*}
                U(w, x) = \phi_w(x);
            \end{equation*}
            \label{universal-tm}
            \vspace{-18pt}
        \item There is a total recursive function
            $\sigma:\{0, 1\}^* \times \{0, 1\}^* \to \{0, 1\}^*$ such that,
            for every $y$,
            $\phi_w(x, y)$ is defined if and only if $\phi_{\sigma(w, x)}(y)$ is defined,
            and, in this case,
            \begin{equation*}
                \phi_w(x, y) = \phi_{\sigma(w, x)}(y).
            \end{equation*}
            \label{smn-theorem}
            \vspace{-18pt}
    \end{enumerate}
\end{definition}

This concept captures the intuitive notion of
``well-behaved numbering''.

Condition~\ref{surjectiveness}
guarantees that the image of~$\phi$ is all of~$\mathcal P$,
so that the numbering neither ``forgets'' a partial recursive function
nor generate a function that is not partial recursive.
Condition~\ref{universal-tm} is the universal Turing machine theorem.

Condition~\ref{smn-theorem} is the ``little more'' we mentioned earlier.
It is known as the $S_{mn}$ theorem \cite[p.~24]{Rogers1987}.
Essentially,
given a partial recursive function of two variables,
we can obtain a partial recursive function of one variable
by fixing the first argument.
The function $\sigma$ provides a systematic way of doing this:
given a description $w$ of a partial recursive function of two variables
and the value $x$ to be fixed as the first variable,
$\sigma(w, x)$ is a machine that computes this new partial recursive function.

\vspace{6pt}
\begin{example}\label{standard-numbering}
    Any encoding of deterministic Turing machines as a binary string
    yields an acceptable numbering of the recursive functions.
\end{example}

\vspace{6pt}
\begin{example}\label{c}
    Any programming language can be understood as an acceptable Gödel numbering.
    For example,
    if we restrict a \texttt C program
    to perform input and output only using the standard input and standard output
    (that is, forbid interactions with the user, file reading, GUIs,
    access to system clock, etc.),
    the resulting program will map a binary input to a binary output,
    characterizing a partial recursive function.
    Thus, we can see a \texttt C compiler
    as an implementation of an acceptable Gödel numbering.\footnote{
        Note we are ignoring here issues like compilation and run-time errors.
        These can be dealt with as in the case of Turing machines:
        any invalid program will signify some fixed partial recursive function
        (say, the function that is defined nowhere);
        and any invalid step in computation
        makes the function to be not defined on that input.
    }
\end{example}

One important theorem that can be proven using acceptable Gödel numberings alone
is the recursion theorem
\cite[p.~181]{Rogers1987}.
It states that, if $\phi$ is any acceptable Gödel numbering
and $f$ is any total recursive function,
then there is some program $w_0$ such that $\phi_{w_0} = \phi_{f(w_0)}$.
That is,
if $f$ is any systematic transformation on programs,
there is a program $w$ (a fixed point for $f$) whose meaning under $\phi$ is unchanged.
The recursion theorem can be used,
for example, to show the existence of quines
(programs whose output are their own source codes)
in any Turing-complete programming language:
choose $f$ to be the function that, given a program $w$,
returns another program $f(w)$ that prints the string $w$ when run.
(For Turing machines, for instance,
we can encode the bits of $w$ in the transition table of $w$.)
Then, by the recursion theorem,
there is some program $w_0$ that is equivalent to its transformed version $f(w_0)$;
thus, $w_0$ already writes the string $w_0$, its own source code.
Therefore, any programming language has quines \cite[p.~227]{Kozen2006}.

\subsection{Efficiency criterion: Blum axioms}
\label{sec:blum-axioms}

The \emph{complexity} of a computation is how much of a resource
that is invested in that computation~\cite[p.~285]{HopcroftUllman1979}.
For each model of computation and each resource under that model,
we can establish a \emph{complexity measure} concerning that resource.
This section is devoted to formalizing this notion.
As we did with the machine encodings,
we will impose some restrictions on what can be a complexity measure
to be able to manipulate it (at least indirectly).
In our case,
we will use Blum axioms~\cite[p.~324]{Blum1967}.

\vspace{12pt}
\begin{definition}
    Given an acceptable Gödel numbering $\phi$,
    a \emph{complexity measure} for $\phi$
    is a function $\Phi:\Sigma^* \times \Sigma^* \to \mathbb N$ of two variables
    that satisfies \cite[p.~324]{Blum1967}:
    \begin{enumerate}
        \item For every $w$ and $x$,
            $\phi_w(x)$ exists if and only if $\Phi(w, x)$ exists; and
        \item For every string $w$, $x$ and every natural number $k$,
            the predicate ``$\Phi(w, x) = k$?'' is decidable.
    \end{enumerate}
\end{definition}

$\Phi(w, x)$ is the complexity of computing $\phi_w(x)$
using the program $w$.
The first axiom says that it only makes sense
to talk about the complexity of a computation that ends.
The second axiom gives minimum tools to manipulate $\Phi$ indirectly,
in the same manner we require $\phi$ to be an \emph{acceptable} Gödel numbering.

\vspace{12pt}
\begin{example}
    The standard measures of time and space can be constructed
    over the acceptable numbering of example~\ref{standard-numbering}.
    They are, respectively,
    the number of moves and tape cells scanned
    before halting.
    Leave the complexity undefined if the machine does not halt
    --- this satisfies the first axiom.
    The predicate of the second axiom is simple for time complexity;
    for space complexity,
    we must keep the whole history of computation
    to make sure the machine does not loop in a limited amount of space
    (because the complexity is not defined in this case).
\end{example}

\vspace{12pt}
\begin{definition}
    Given a complexity measure $\Phi$ for an acceptable Gödel numbering $\phi$
    and a total recursive function $f : \mathbb N \to \mathbb N$,
    define the \emph{complexity class} $\mathcal C_\Phi(f)$ by
    (\cite[p.~232]{Kozen2006})
    \begin{equation*}
        \mathcal C_\Phi(f) = \{
            \phi_w \mid \Phi(w, x) \leq f(|x|)\ \text{almost everywhere\footnotemark}
        \}.
    \end{equation*}
    \footnotetext{
        \emph{Almost everywhere} means that the inequality $\Phi_i(x) \leq f(|x|)$
        holds for all but a finite number of different $x$.
    }
\end{definition}

This formalizes the notion of complexity class.
There are some important theorems concerning complexity classes
under the Blum axioms;
we mention only the Union Theorem \cite[p.~234]{Kozen2006}.
It states that,
if $\{f_i\}$ is any recursive list of increasing functions
such that $f_i(n) \leq f_{i+1}(n)$ for all $x$
(that is, each $f_i$ is greater than the previous),
then there is some function $g$ such that
\begin{equation*}
    \mathcal C_\Phi(g) = \bigcup_{i \in \mathbb N} \mathcal C_\Phi(f_i).
\end{equation*}
That is, the class $\mathcal C_\Phi(g)$ contains \emph{exactly}
all functions present in the classes $\mathcal C_\Phi(f_i)$.
Choosing $\Phi$ as the time complexity and $f_i(n) = n^i$
(the polynomial functions),
the union in the right is exactly the class $\P$,
the problems solvable in polynomial time.
By the Union Theorem, $\P$ is $\mathcal C_\Phi(g)$
for some recursive function $g$,
so, even though $\P$ does not have an easily specifiable bounding function,
such function exists nevertheless.

\section{Nondeterministic computation of functions}
\label{sec:nondeterministic_computation_of_functions}

This section surveys several approaches
for defining nondeterministic computation of functions.

Sections \ref{sec:godel-numberings} and~\ref{sec:blum-axioms}
introduced the concept of acceptable Gödel numbering,
the Blum axioms,
and its complexity classes.
As we want to regard these concepts as ``quality requirements''
for the definitions,
we will analyze each of them to see whether they define acceptable Gödel numberings,
the analogous time complexity satisfies Blum axioms,
and that the characteristic function of the Boolean satisfiability problem
can be solved in ``polynomial time'' according to that complexity measure.

\subsection{Hopcroft-Ullman's definition}
\label{sec:hopcroft-ullman}

\begin{definition}[Hopcroft and Ullman's definition\footnotemark]
    \footnotetext{
        Hopcroft and Ullman's original definition \cite[p.~313]{HopcroftUllman1979}
        was defined in the context of computation of integer functions.
        We are rephrasing here in terms of strings,
        but keeping the relation they imposed on the execution branches.
    }
    If $w$ is an encoding for the Turing machine $M$,
    we say that $\phi_w(x) = y$ if and only if,
    when processing $x$,
    there is some branch of $M$ that halts with $y$ in the tape,
    and there is no branch that halts with some $z \neq x$ in the tape.
\end{definition}

The problem with their definition is that $\phi_w(x)$ is allowed to be defined,
even if some branch of computation does not halt.
This allows us to solve the complement of the halting problem.

\vspace{6pt}
\begin{proposition}
    Define the partial function $f: \Sigma^* \times \Sigma^* \to \{0, 1\}$ by
    \begin{equation*}
        f(w, x) = \begin{cases}
            1, & \text{if the machine $w$ does not halt on $x$.} \\
            \text{undefined}, &\text{if $w$ halts on $x$.}
        \end{cases}
    \end{equation*}
    This function can be computed by a nondeterministic Turing machine under
    Hopcroft and Ullman's definition.
\end{proposition}
\vspace{-3pt}

\begin{proof}
    On input $(w, x)$,
    create two branches of computation.
    On the first, write $1$ on the tape and halt.
    On the second, simulate the universal Turing machine $U$ in the input,
    and if $U$ halts, write $0$ on the tape and halt too.

    If $w$ halts on $x$,
    there will be two halting branches of computation,
    each writing a different value in the tape,
    so, by definition, the function is not defined on this input.
    If $w$ never halts,
    then only the first branch will halt
    (and with $1$ written on the tape),
    so the function is defined on this input and its value is $1$.
\end{proof}

Therefore,
using Hopcroft and Ullman's definition,
we can compute some noncomputable functions,
violating the requirement~\ref{surjectiveness} of Gödel numberings.

We can try to fix this definition by forcing all branches to halt;
but then, as all branches are required to return the same value,
the machine will be (almost) deterministic.
Therefore, if there is a machine $M$ that computes the characteristic function
of the satisfiability problem in nondeterministic polynomial time,
we could simulate $M$ on a given input,
choosing (say) always the first option when confronted with nondeterminism.
If this branch of computation returns $1$,
then every branch returns $1$ and the input is satisfiable;
if this branch returns $0$, every branch returns $0$ and the input is unsatisfiable.
We thus could solve $\SAT$ in \emph{deterministic} polynomial time.

So, with this restriction,
we lose the apparent exponential speed-up in computation time
we have when using nondeterminism.

\subsection{Goldreich's definition}
\label{sec:goldreich}

\begin{definition}[Goldreich's definition]
    Let $\bot$ be some special symbol not in $\{0, 1\}^*$.
    (This symbol will represents ``don't know''.)
    A nondeterministic machine $M$ computes the function $f$ if,
    when processing the input $x$,
    both the following conditions hold \cite[p.~168]{Goldreich2008}:
    \begin{itemize}
        \item Every branch of $M$ halts
            and outputs either $f(x)$ or $\bot$.
        \item At least one branch of $M$ halts with $f(x)$ on the tape.
    \end{itemize}
\end{definition}
The extra symbol $\bot$ sidesteps the problems of a branch looping forever.
This allows us to mechanically convert nondeterministic machines
under Goldreich's definition
to deterministic machines via simulation
--- the deterministic machine just need to simulate all branches until completion,
to actually be sure every branch halts;
if some branch do not halt,
then the simulating machine will not halt either,
but the function is not defined in this case,
so this behavior is correct.

Thus, we only enumerate computable functions.
To show the universal machine theorem and the $S_{mn}$,
we can simply first convert the machine in question to a deterministic machine
and use their theorems;
thus, this definition yields an acceptable Gödel numbering.
And, by counting the number of steps of the deepest branch,
we have a Blum complexity measure.

So, Goldreich's definition defines an acceptable Gödel numbering
and we can form a complexity measure that satisfies Blum axioms.
But, again,
we have trouble with the ``exponential speed-up'' requirement.
For instance,
a machine trying to solve the satisfiability problem
would correctly return $1$ for a satisfiable instance,
but no branch can write a $0$ alone
because it cannot be sure that instance is unsatisfiable
--- so, the function would be undefined for unsatisfiable formulas.

That is, unless $\NP = \coNP$.

\begin{proposition}
    If there is a nondeterministic Turing machine
    that computes in polynomial time, according to Goldreich's definition,
    the characteristic function of the satisfiability problem,
    then $\NP = \coNP$.
\end{proposition}

($\NP = \coNP$ implies that $\NP = \PH$; that is,
the polynomial hierarchy collapses to the first level \cite[p.~280]{Kozen2006}.)

\begin{proof}
    Suppose $M$ is the machine that computes $\SAT$'s characteristic function,
    under Goldreich's definition, in polynomial time.

    The characteristic function of $\SAT$ is a total function,
    and its only outputs are $0$ and $1$.
    Therefore, in every computation of $M$,
    there is at least one branch that writes something different that $\bot$ on the tape,
    and whatever it writes is the correct answer.
    So, if we convert this to a nondeterministic decider and invert the output
    (branches that write $0$ will accept the input and vice-versa;
    branches that write $\bot$ always reject),
    any satisfiable formula will be rejected,
    because no branch of $M$ ever writes $0$ on the tape for these formulas;
    and any unsatisfiable formula will be accepted,
    because at least one branch of $M$ writes $0$ for these formulas.
    Thus, we can decide $\overline \SAT$, the complement of $\SAT$.

    Since $\overline \SAT$ is $\coNP$-complete,
    the existence of such a machine $M$
    would show that $\coNP \subseteq \NP$,
    and this implies that $\coNP = \NP$.\footnote{
        To see why $\coNP \subseteq \NP$ implies $\coNP = \NP$,
        pick a language $L \in \NP$.
        Its complement $\overline L$ is in $\coNP$, by the definition of $\coNP$.
        But by hypothesis, $\overline L$ is in $\NP$,
        so, by the definition of $\coNP$,
        its complement, $\overline{\overline L} = L$ is in $\coNP$,
        thus showing $\coNP = \NP$.
    }
\end{proof}

Therefore, even though Goldreich's definition
yields an acceptable Gödel numbering and a Blum complexity measure,
it also have trouble in transposing the exponential speed-up
we have using nondeterminism.

\subsection{Proposed definition}
\label{sec:proposed}

Analyzing the problems with the first two definitions,
we know the nondeterministic machine must be allowed to return several values
(one for each branch)
and somehow pick only one to be the value of the function.

If $M$ is any deterministic decider for the language $L$,
we can create a machine that computes the characteristic function of $L$
by running $M$ and returning $1$ if $M$ accepted and $0$ if it rejected.

If we apply this transformation to a nondeterministic machine
that recognizes the Boolean satisfiability problem,
then, when running this machine,
we have three possible results.
\begin{itemize}
    \item For tautological formulas,
        the set of possible answers is only $\{1\}$.
    \item For contradictory formulas,
        the set of answers is $\{0\}$.
    \item For satisfiable formulas that are not tautological,
        we have both values: $\{0, 1\}$.
\end{itemize}
We want the return value for all satisfiable formulas to be $1$
and for unsatisfiable formulas to be $0$.
Note that this corresponds exactly to the maximum value of each set;
so,
our definition of nondeterministic computation of functions
will preserve exactly this behavior.

\begin{definition}[Proposed definition]
    Let $M$ be a nondeterministic Turing machine,
    and $x$ an input.
    If every branch of $M$ halts when processing $x$,
    the value of the function computed by $M$ on $x$
    is the lexicographically maximum
    between all strings written in the computation branches.
    If some branch does not halt, leave the function undefined in $x$.
\end{definition}

The same reasoning of Goldreich's definition applies here;
therefore, we have an acceptable Gödel numbering,
and counting steps of the deepest branch
(as in Goldreich's definition)
yields a Blum complexity measure.

And, using exactly the algorithm mentioned above,
we can compute the characteristic function of the satisfiability problem
under linear time;
thus, this definition meets all the requirements
proposed in the beginning of section~\ref{sec:nondeterministic_computation_of_functions}.

\vspace{-12pt}
\subsubsection{Extending $\NP$-completeness}
\label{sec:np-completeness-extension}

The last definition provides an extension of the class $\NP$ to function computation,
so the next step is extend the concept of $\NP$-completeness.

Call $\FNP$ the class of all functions computable in polynomial time,
under our definition\footnote{
    As noted in section~\ref{sec:our-approach},
    this definition is different
    from the one given by Papadimitriou~\cite[p~229]{Papadimitriou1994}.
}.
A language $L$ is $\NP$-complete if both $L \in \NP$
and every language in $\NP$ reduces to $L$;
that is, for every language $L' \in \NP$,
there is a polynomial-time computable function $f$
such that, for every $x$,
\begin{equation*}
    x \in L' \text{ if and only if } f(x) \in L.
\end{equation*}
(We are using Karp reductions here \cite[p.~42]{AroraBarak2009}.)
If we rephrase in terms of the characteristic functions $1_L$ and $1_{L'}$,
of $L$ and $L'$, respectively,
we have $1_{L'}(x) = 1_L(f(x))$ for all $x$.
We will generalize specifically this equation to define $\FNP$-completeness.

\vspace{6pt}
\begin{definition}
    A function $f$ is $\FNP$-complete if $f \in \FNP$ and,
    for every function $g$ in $\FNP$,
    there is a polynomial-time computable function $h$ such that
    \begin{equation*}
        g(x) = f(h(x))
    \end{equation*}
    for every $x \in \{0, 1\}^*$.
\end{definition}

We can construct a $\FNP$-complete function based on the halting problem:
define $f : \{0, 1\}^* \times \mathbb N \to \{0, 1\}^*$
such that $f(w, n)$ is the lexicographically greatest value written by any branch of $w$
after running for $n$ steps.
Such functions are $\FNP$-complete
because they simulate Turing machines directly.

However, this definition is very rigid
and allow for few functions to be $\FNP$-complete.
For example, if $g$ is a function that returns
(say) a largest clique in a graph,
any candidate function for $\FNP$-completeness
must be able to output representation of graphs.
This restrict the class of $\FNP$-complete functions
to functions like the one mentioned above.

\subsection{Krentel's $\OptP$ class}
\label{sec:krentel}

The next two authors were not specifically concerned
with nondeterministic computation of functions,
but rather in generalizing the $\NP$ class for functions.
Therefore, the corresponding notion of $\NP$-completeness
behave better than our proposed generalization.

Krentel's definition, in particular,
are very similar to ours.

\begin{definition}
    A function $f: \{0, 1\}^* \to \mathbb N$ is in $\OptP$
    if there is some nondeterministic Turing machine $M$ such that
    \cite[p.~493]{Krentel1988}:
    \begin{itemize}
        \item For every input,
            every branch of $M$ halts within a polynomial number of steps
            and writes in its tape a number in binary; and
        \item Either,
            for all $x$, the largest number written by $M$ on $x$ is $f(x)$,
            or, for all $x$, the smallest number written by $M$ on $x$ is $f(x)$.
    \end{itemize}
\end{definition}

Therefore, Krentel's $\OptP$ class contains the optimization problems
that can be ``solved'' in polynomial time by nondeterministic Turing machines.
(Note we must always take the maximum value, or always take the minimum value.)
The definition of $\OptP$-completeness, however,
is significantly different.

\vspace{6pt}
\begin{definition}
    A function $f$ is $\OptP$-complete if $f \in \OptP$ and,
    for every function $g \in \OptP$,
    there are two functions $T_1 : \{0, 1\}^* \to \{0, 1\}^*$
    and $T_2 : \{0, 1\}^* \times \mathbb N \to \{0, 1\}^*$,
    both computable in deterministic polynomial time,
    such that, for all $x$,
    \begin{equation*}
        g(x) = T_2\big( x, f(T_1(x)) \big).
    \end{equation*}
\end{definition}

That is, besides the preprocessing function $T_1$,
we are allowed to make a post-processing which have access
both to the input $x$ and to the ``reduced output'' $f(T_1(x))$.
Under this definition,
the traveling salesperson problem and $0-1$ integer linear programming
are both $\OptP$-complete \cite[p~495]{Krentel1988}.

\subsection{Valiant's $\#\P$ class}
\label{sec:valiant}

\begin{definition}[Valiant's definition]
    A nondeterministic machine $M$ computes a function $f: \{0, 1\}^* \to \mathbb N$ if,
    for every input $x$,
    $M$ halts on every branch\footnote{
        Valiant did not include the requirement of halting in his definition,
        but we will include it to avoid having the same problems
        of Hopcroft and Ullman's definition.
    }
    and the number of accepting branches of computation is $f(x)$
    \cite[p.~191]{Valiant1979}.
\end{definition}

The class $\#\P$ is the set of functions computed in polynomial time,
under Valiant's definition.
$\#\P$-completeness is defined through oracles;
that is, a function $f$ is $\#\P$-complete if $f \in \#\P$
and $\#\P \subseteq \FP^f$
(that is, every function of $\#\P$ can be computed by a deterministic machine
with access to an oracle that computes $f$)
\cite[p.~191]{Valiant1979}.

Besides problems like counting the number of satisfying assignments
for a given formula,
Valiant also proves less trivial problems are $\#\P$-complete,
like the task of computing the permanent of a matrix
\cite[p.~194]{Valiant1979}.
Therefore,
this definition is also flexible, like Krentel's.

Krentel note that both his and Valiant's definition
arise from applying an associative operator
to all values returned in the branches of computation
--- maximum/minimum in Krentel's case,
and cardinality (counting) in Valiant's case.
Therefore, using other associative operations yield alternative definitions
of nondeterministic function computation \cite[p.~493]{Krentel1988}.

\section{Concluding Remarks}
\label{sec:conclusion}

As we have seen,
it is possible to associate nondeterministic Turing machines
with function computation,
although the association is not so straightforward as with deterministic machines.

Through the use of the notions of acceptable Gödel numberings
and Blum axioms,
we formalized the notion of ``reasonable'' definition
for nondeterministic function computation,
and by consequence we formalized what is a ``generalization of $\NP$ to functions''.
Krentel's and Valiant's definitions,
besides generalizing $\NP$, in particular,
also allow for a flexible generalization of $\NP$-completeness
--- that is, they allow for interesting problems to be ``functionally $\NP$-complete'',
under each definition.

\section{Project Development}
\label{sec:development}

Tiago Royer studied this problem of associating a nondeterministic machine
with a function in his undergraduate thesis\footnote{
    The project was developed while he was an undergraduate student
    at Federal University of Santa Catarina (UFSC).
    Currently, he is pursuing his master's degree
    at the University of São Paulo (USP).
},
under the supervision of Jerusa Marchi.
After finding Hopcroft and Ullman's definition
(section~\ref{sec:hopcroft-ullman})
and noting it yields a computation that is very close to be deterministic,
he devised the definition of section~\ref{sec:proposed}
(using the concept of acceptable Gödel numberings and Blum axioms
to be sure his definition would not bee too unreasonable).
As noted in section~\ref{sec:np-completeness-extension},
his generalization to $\NP$ and $\NP$-completeness are very rigid.

Krentel's and Valiant's definitions
(which were discovered later in the project),
although not created specifically to associate nondeterministic machine with functions,
provides a more elegant generalization of $\NP$ and $\NP$-completeness;
this paper summarize all these findings
under the light of Gödel numberings and Blum axioms.

\bibliographystyle{sbc/sbc}
\bibliography{bib/bibliography}

\end{document}
